% Master File: wift95.tex
\documentstyle[11pt]{article}
\topmargin .2in
\textwidth 5.5in
\textheight 7.75in
\oddsidemargin .65in
\evensidemargin .41in
\sloppy
\title{\bf Tutorial Proposal for WIFT:\\
Mechanically Checked Formal Specification and Verification using PVS}
\author{John Rushby\\
Computer Science Laboratory\\
SRI International\\
Menlo Park CA 94025 USA
\\ \mbox{ }\\
{\tt Rushby@csl.sri.com}
\\Phone: +1 (415) 859-5456\ \ Fax: +1 (415) 859-2844
}

\date{}
\begin{document}
\maketitle

\section{Tutorial Proposal}

We propose a half-day tutorial on mechanized formal methods using PVS
(PVS is introduced below).  The goal of the tutorial is not to teach
PVS to but to give some appreciation for the opportunities conferred
by state-of-the-art mechanization of formal methods.  In our view, the
significance of formal methods is that they allow certain question
about computational systems to be reduced to calculation.  The
role of mechanization is to make it possible to actually undertake the
calculations for questions of real significance at reasonable cost.
The ideas and techniques presented will be useful to users of systems
other than PVS.

The tutorial will mainly be presented as an extended interaction with
PVS\@.  We will bring our own Sparc Laptop and overhead display tablet
and will work through a graduated series of examples that demonstrate
the utility of very strong typechecking, of definitional forms of
expression that guarantee conservative extensions, and of
proof-obligations generated during typechecking.  After explaining the
basics of theorem proving in PVS, we will demonstrate how
specifications may be ``challenged'' by putative theorems and how
flaws may be detected in this way.  We will end by demonstrating the
interactive verification of a hard algorithm (e.g., Byzantine
Agreement), and the highly automated verification of structured
designs (e.g., correctness of a microprocessor pipeline).

Previous tutorials and courses using PVS in this style have been given
at the Formal Methods Europe conference at Odense, Denmark in 1993
(half day), Compass '93 in Washington DC (whole day), Boeing (whole
day), Collins Commercial Avionics (three days), NSA (three days), SRI
(two days and three days), Loral Houston/Johnson Space Center (whole
day), JPL (half day).  Another is upcoming at the 2nd International
Conference on Theorem Provers in Circuit Design, Bad Herrenalb,
Germany in September.  These tutorials have been given by various
(usually pairwise) combinations of Sam Owre, Pat Lincoln, John Rushby,
Shankar, and Srivas, who are all from SRI and are either developers or
major users of PVS\@.  The WIFT tutorial presenter will be selected
from among these; all have advanced degrees from respectable
universities (MS in Math from UCLA, and PhDs in Computer Science from
Stanford, Newcastle, Texas, and MIT, respectively), and numerous
publications in formal methods.

PVS tutorial course material is available by World Wide Web at URL
{\tt http://www.csl.sri.com/sri-csl-pvs.html}, where a general
overview tutorial and one specialized towards hardware verification
may be found.  We will provide a set of tutorial notes based on these
materials and on an elementary tutorial developed by Ricky Butler at
NASA Langley~\cite{Butler:PVS-tut}.  We may also use some material
from an undergraduate course taught using PVS by Professor Ian Wand at
the University of York in England.

As can be seen from the number of times we have presented this and
similar material, the basic presentation is quite mature.  Audience
reaction has generally been very positive; most attendees (especially
in Europe) have previously been unaware of the sheer power of modern
verification systems and theorem provers, and have left with a new
appreciation of the possibilities and benefits of mechanization in
formal methods.

\section{Introducing PVS}

PVS is the latest in a line of verification systems developed at SRI
International over a period of 20 years.  PVS is unusual in that it
combines a very expressive and convenient specification language with
a powerful proof checker.  

\subsection{Language}

The specification language of PVS is a higher order logic with very
rich type system that includes predicate and dependent types.  This
allows division, for example, to be typed so that the divisor is
constrained to be nonzero---thereby providing a treatment for
partially-defined functions without the usual difficulties of logics
of partial functions.  The PVS typechecker generates proof obligations
to ensure that functions are applied appropriately.  The PVS type
system allows many of the constraints on a specification to be stated
in the types; the typechecker can then provide very strong checks on
consistency.  For example, the injections can be defined as the
subtype of the functions associated with the predicate $f(x)=f(y)
\supset x=y$.  Any attempt to define an injection will then cause the
typechecker to generate the proof obligation that ensures this
predicate is satisfied for the definition concerned.

The base types of PVS include built-in types such as the booleans,
integers, reals, and the ordinals up to $\epsilon_0$, and uninterpreted
types that may be introduced by the user; the type-constructors
include functions, sets, tuples, records, enumerations, and
recursively-defined abstract data types.  

PVS specifications are organized into parameterized theories that may
contain assumptions, definitions, axioms, and theorems.  Definitions
are guaranteed to provide conservative extension; to ensure this,
recursive function definitions generate proof obligations.  PVS
expressions provide the usual arithmetic and logical operators,
function application, lambda abstraction, and quantifiers, within a
natural syntax.  Names may be freely overloaded, including those of the
built-in operators such as AND and +.

\subsection{Proof Checker}

The PVS theorem prover provides a collection of powerful primitive
inference procedures that are applied interactively under user
guidance within a sequent calculus framework.  The primitive inferences
include quantifier rules, BDD-based propositional simplification,
induction, rewriting, simplification of abstract datatype expressions,
and decision procedures for linear arithmetic (over both reals and
integers).  User-defined procedures can combine these primitive
inferences to yield higher-level proof strategies.  Proofs yield
scripts that can be edited, attached to additional formulas, and
rerun.  This allows many similar theorems to be proved efficiently,
permits proofs to be adjusted economically to follow changes in
requirements or design, and encourages the development of readable
proofs.

The combination of powerful automation for the primitive proof steps
with user-written strategies and interactive guidance provides a proof
checker whose efficiency matches or exceeds that of other theorem
provers and proof checkers over a wide class of applications.  For
example, using a single strategy, PVS is able to verify microcode
correctness for the Gordon's Tamarack computer, pipeline correctness
for Saxe's processor, and correctness of Bundy and Cantu's
12-operation ALU, all completely automatically and within a few
minutes~\cite{Cyrluk94:TPCD}.

\subsection{Applications and Users}

The main applications of PVS have been fault-tolerant algorithms,
problems in real-time, processor verification, and requirements
specification.

While the verification of hard fault-tolerant algorithms has mainly
been done by
us~\cite{Lincoln&Rushby93:CAV,Lincoln&Rushby93:FTCS,Lincoln&Rushby94:FTP,Owre-etal:FME93,Rushby94:icah},
we are very pleased that many of the other applications have been
undertaken jointly with collaborators in government, industry or
academia.

Jens Skakkeb{\ae}k, a student from the Technical University of
Denmark, has collaborated with Shankar to develop a proof assistant
for the Duration Calculus (called PC/DC) on top of
PVS~\cite{Skakkebaek&Shankar94}.  PC/DC has been used to check a
number of pencil and paper specifications and proofs in the Duration
Calculus, leading to the discovery of errors in every
one~\cite{Inal&Jens94}.  Working independently, Professor Jozef Hooman
at Eindhoven University of Technology in The Netherlands has developed
an approach for the top-down development of distributed real-time
systems using an extension to Hoare logic, and has mechanized it using
PVS~\cite{Hooman94}.  Also in The Netherlands, Sreeranga Rajan used
PVS to specify and verify some of the transformations used for high
level synthesis in a VLSI CAD system at Philips Research Laboratories.
PVS is used at a number of other universities in Northern Europe and a
PVS User's meeting was held at Lyngby in April, 1994.

A collaborative project between Collins Commercial Avionics and SRI
has been using PVS to formally specify the macro- and
micro-architectures and to formally verify the microcode of an
avionics processor called AAMP5~\cite{Miller&Srivas95}.  This is
probably the most complex processor verification ever undertaken (the
AAMP5 has 500,000 transistors, 208 instructions, is microcoded, and
has a pipeline and other complicating factors).  The formal
verification detected an error that had been missed by Collins
traditional V\&V process.  Ricky Butler of NASA Langley developed the
extensive PVS bitvector library that was used in this verification.

Ricky Butler, together with Damir Jamsek of ORA, also developed a
finite sets library that has been used in the PVS specification of
requirements for railroad signalling developed by ORA and Union Switch
and Signal.  Other groups making use of PVS include JPL, Loral, NSA,
and several universities in the US and Europe.

\bibliographystyle{modplain}
\begin{thebibliography}{10}

\bibitem{Butler:PVS-tut}
Ricky~W. Butler.
\newblock An elementary tutorial on formal specification and verification using
  {PVS}.
\newblock NASA Technical Memorandum 108991, NASA Langley Research Center,
  Hampton, VA, June 1993.

\bibitem{Cyrluk94:TPCD}
D.~Cyrluk, S.~Rajan, N.~Shankar, and M.~K. Srivas.
\newblock Effective theorem proving for hardware verification.
\newblock In {\em Theorem Provers in Circuit Design}, Bad Herrenalb
  (Blackforest), Germany, September 1994.
\newblock To appear.

\bibitem{Hooman94}
Jozef Hooman.
\newblock Correctness of real time systems by construction.
\newblock In {\em Symposium on Formal Techniques in Real-Time and
  Fault-Tolerant Systems}, L\"{u}beck, Germany, September 1994. Lecture Notes
  in Computer Science, Springer-Verlag.
\newblock (To appear).

\bibitem{Inal&Jens94}
Ricep Inal and Jens~U. Skakkeb{\ae}k.
\newblock Applying a mechanized duration calculus assistant.
\newblock In Hans Rischel, editor, {\em Nordic Seminar on Dependable Computing
  Systems}, pages 69--80, Lyngby, Denmark, August 1994. Technical University of
  Denmark.

\bibitem{Lincoln&Rushby93:CAV}
Patrick Lincoln and John Rushby.
\newblock Formal verification of an algorithm for interactive consistency under
  a hybrid fault model.
\newblock In Costas Courcoubetis, editor, {\em Computer-Aided Verification, CAV
  '93}, pages 292--304, Elounda, Greece, June/July 1993. Volume 697 of {\em
  Lecture Notes in Computer Science}, Springer-Verlag.

\bibitem{Lincoln&Rushby93:FTCS}
Patrick Lincoln and John Rushby.
\newblock A formally verified algorithm for interactive consistency under a
  hybrid fault model.
\newblock In {\em Fault Tolerant Computing Symposium 23}, pages 402--411,
  Toulouse, France, June 1993. IEEE Computer Society.

\bibitem{Lincoln&Rushby94:FTP}
Patrick Lincoln and John Rushby.
\newblock Formal verification of an interactive consistency algorithm for the
  {Draper FTP} architecture under a hybrid fault model.
\newblock In {\em COMPASS '94 (Proceedings of the Ninth Annual Conference on
  Computer Assurance)}, pages 107--120, Gaithersburg, MD, June 1994. IEEE
  Washington Section.

\bibitem{Miller&Srivas95}
Steven~P. Miller and Mandayam Srivas.
\newblock Formal verificatin of the {AAMP5} microprocessor: A case study in the
  industrial use of formal methods.
\newblock Submitted to WIFT '95.

\bibitem{Owre-etal:FME93}
Sam Owre, John Rushby, Natarajan Shankar, and Friedrich von Henke.
\newblock Formal verification for fault-tolerant architectures: Some lessons
  learned.
\newblock In J.~C.~P. Woodcock and P.~G. Larsen, editors, {\em FME '93:
  Industrial-Strength Formal Methods}, pages 482--500, Odense, Denmark, April
  1993. Volume 670 of {\em Lecture Notes in Computer Science}, Springer-Verlag.

\bibitem{Rushby94:icah}
John Rushby.
\newblock A formally verified algorithm for clock synchronization under a
  hybrid fault model.
\newblock In {\em Thirteenth ACM Symposium on Principles of Distributed
  Computing}, pages 304--313, Los Angeles, CA, August 1994. Association for
  Computing Machinery.

\bibitem{Skakkebaek&Shankar94}
Jens~U. Skakkeb{\ae}k and N.~Shankar.
\newblock Towards a duration calculus proof assistant in pvs.
\newblock In {\em Symposium on Formal Techniques for Real-Time and
  Fault-Tolerant Systems}, L\"{u}beck, Germany, September 1994. Lecture Notes
  in Computer Science, Springer-Verlag.
\newblock (To appear).

\end{thebibliography}

\end{document}
% End of wift95.tex -----------------------------------
