% Document Type: LaTeX
% Master File: tutorial.tex
\documentstyle[11pt,relative,alltt,twoside,relative,fancyheadings,boxedminipage,/homes/EHDM/pvs/pvs]{article}
\sloppy

% full list of sections:
%\includeonly{title,intro,informal,pspace,undecide,conclu,ack,rules}

%\pagestyle{myheadings} % page number in upper right corner
%\markboth{Specification and Verification}{}
%\makeindex
\newcommand{\allttinput}[1]{\hozline{\smaller\smaller\smaller\begin{alltt}\input{#1}\end{alltt}}\hozline}
\newenvironment{pvsscript}{\hozline\smaller\smaller\smaller\begin{alltt}}{\end{alltt}\hozline}

\topmargin -10pt
\textheight 8.5in
\textwidth 6.0in
\headheight 15 pt
\columnwidth \textwidth
\oddsidemargin 0.5in
\evensidemargin 0.5in   % fool system for page 0
\setcounter{topnumber}{9}
\renewcommand{\topfraction}{.99}
\setcounter{bottomnumber}{9}
\renewcommand{\bottomfraction}{.99}
\setcounter{totalnumber}{10}
\renewcommand{\textfraction}{.01}
\renewcommand{\floatpagefraction}{.01}
%\newenvironment{smalltt}{\begin{alltt}\small}{\end{alltt}}

\raggedbottom

\font\largett=cmtt10 scaled\magstep2
\font\hugett=cmtt10 scaled\magstep4
\def\opt{{\smaller\sc {\smaller\smaller \&}optional}}
\def\rest{{\smaller\sc {\smaller\smaller \&}rest}}
\def\default#1{[\,{\tt #1}] }
\def\bkt#1{{$\langle$#1$\rangle$}}

\newenvironment{usage}[1]{\item[usage:\hspace*{-0.175in}]#1\begin{description}\setlength{\itemindent}{-0.2in}\setlength{\itemsep}{0.1in}}{\end{description}}

\begin{document}
\pagestyle{empty}
\title{\Large\bf A Tutorial on Specification and Verification Using
PVS\\(Beta Release)}
\author{N.~Shankar and S.~Owre and J.~M.~Rushby\\Computer Science Laboratory\\SRI International\\
Menlo Park CA 94025 USA\\
Phone: (415) 859-5272\\
\{Shankar,Owre,Rushby\}@csl.sri.com}


%\renewcommand{\baselinestretch}{2}
\newenvironment{display}{\begin{alltt}\small\tt\vspace{0.3\baselineskip}}{\vspace{0.3\baselineskip}\end{alltt}}
\newcommand{\choice}{[\!]}
\newcommand{\normtt}[1]{{\obeyspaces {\tt #1 }}}
\newenvironment{pagegroup}{}{}
%\newenvironment{smalltt}{\begin{alltt}\small\tt}{\end{alltt}}
\newenvironment{tdisplay}{\begin{alltt}\footnotesize\tt\vspace{0.3\baselineskip}}{\vspace{0.3\baselineskip}\end{alltt}}
%
%
%    LaTeX  Macro File  /usr2/jcm/tex/macros.tex
%
%
\tracingonline=0 % shorter error messages (on screen)

%\long\def\comment#1{} % mulit-line comments
%\newcommand{\note}[1]{\fbox{#1}}

% font changes  (as function calls, scribe style)

\renewcommand{\i}[1]{{\it #1\/}}       % italics with space correction
%\newcommand{\emex}[1]{\/{\em #1}}   % emphasis in example, exercise, theorem, etc.
\renewcommand{\c}[1]{{\sc #1}}       % small caps (eliminates \c as cedilla)
%\newcommand{\r}[1]{{\rm #1}}            % for roman font in math mode
%\newcommand{\s}[1]{{\scr #1}}          % script (for use with tatex) 
\newcommand{\s}[1]{{\cal #1}}
\renewcommand{\b}[1]{{\bf #1}}           % bold face
\newcommand{\calg}[1]{{\cal #1}}          % caligraphic 

%\newcommand{\q}[1]{``#1''}    % matching quotes for in-line quotation

% numbered environments

%\newcounter{partcounter}
%\setcounter{partcounter}{0}
%\renewcommand{\part}[1]{\newpage \addtocounter{partcounter}{1}
%\noindent{\Large \bf Part \Roman{partcounter}. #1 } \\[1ex]}

\newtheorem{thm}{Theorem}[section]
\newtheorem{theorem}[thm]{Theorem}
\newtheorem{lemma}[thm]{Lemma}
\newtheorem{cor}[thm]{Corollary}
\newtheorem{corollary}[thm]{Corollary}
\newtheorem{claim}[thm]{Claim}
\newtheorem{prop}[thm]{Proposition}
\newtheorem{conj}[thm]{Conjecture}
\newtheorem{definition}[thm]{Definition}
\newtheorem{exercise}[thm]{Exercise}
\newtheorem{example}[thm]{Example}
\newtheorem{remark}[thm]{Remark}
\newtheorem{open}[thm]{Open Problem}

%\newcommand{\proof}{\\{\bf Proof.}\ }
\newenvironment{proof}{{\bf Proof. }}{\thmbox}

% axiom and inference rule (centered, in math mode,  with name at left)

\newcommand{\axiom}[2]
{\[ \hbox to \columnwidth
    { \rlap{$#2$} \hfil {$ #1 $} \hfil }
\]}

\newcommand{\infrule}[4]
{\[ \hbox to \columnwidth %\textwidth
    { \rlap{$#4$} \hfil $
      \frac {\strut\displaystyle #1 } {\strut\displaystyle #2 } \; \rlap{$#3$} \hfil $
      \hfil }
\]}
\newcommand{\infruletw}[4]
{\[ \hbox to \textwidth
    { \rlap{$#4$} \hfil $
      \frac {\strut\displaystyle #1 } {\strut\displaystyle #2 } \; \rlap{$#3$} \hfil $
      \hfil }
\]}

%\newcommand{\infrule}[4]
%{\[ \hbox to \columnwidth
%    { \rlap{$#4$} \hfil $
%      {{\displaystyle\strut #1}\over{\displaystyle\strut #2}}\quad\makebox[0pt][l]{\it #3} $
%      \hfil }
%\]}

% axiom and inference rule macros for use in tables, etc.
% presumed in math mode #1=axiom, #2=side condition
\newcommand{\Axiom}[2]{
{\displaystyle\strut #1}\qquad\makebox[0pt][l]{\it #2}
}
% presumed in math mode #1=top, #2=bottom, #3=side condition
\newcommand{\Infrule}[3]{
{{\displaystyle\strut #1}\over{\displaystyle\strut #2}}\;\mbox{\scriptsize$\bf #3$}
}

% sequence of #1's, numbered up to #2  (e.g., seq{x}{n} for x1, ..., xn   )

\newcommand{\seq}[2]{#1_{1} \ldots #1_{#2}}  
%\newcommand{\ie}{{\it i.e.}}
%\newcommand{\eg}{{\it e.g.}}
%\newcommand{\cf}{{\it c.f.\,}}

% common symbols

\newcommand{\union}{\cup}
\newcommand{\intersect}{\cap}
\newcommand{\subs}{\subseteq}
\newcommand{\el}{\in}
\newcommand{\nel}{\not\in}
\newcommand{\ns}{\emptyset}
\newcommand{\compose}{\circ}
\newcommand{\set}[2]{ \{\, #1 \,\mid\, #2 \,\}  } % set macro
\newcommand{\infinity}{\infty}
\newcommand{\pair}[1]{\langle #1 \rangle}
\newcommand{\tuple}[1]{\langle #1 \rangle}


\newcommand{\fa}{\forall}
\newcommand{\te}{\exists}
\newcommand{\imp}{\supset}
\newcommand{\implies}{\supset}
\newcommand{\ts}{\vdash}
\newcommand{\dts}{\models}

\newcommand{\aro}{\mathord\rightarrow} % see pages 154-155 of TeX manual
\newcommand{\paro}{\rightharpoonup} 
\newcommand{\karo}{\mathop\Rightarrow} % see pages 154-155 of TeX manual
%\newcommand{\cross}{\times}
%\newcommand{\dlb}{\lbrack\!\lbrack}
%\newcommand{\drb}{\rbrack\!\rbrack}
\newcommand{\mean}[1]{\lbrack\!\lbrack #1 \rbrack\!\rbrack}
\newcommand{\lam}{\lambda}
\newcommand{\subst}[2]{{}[#1/#2]}
\renewcommand{\dot}{\mathrel{\bullet}}
\newcommand{\Dinf}{D_{\infty}}
\newcommand{\bottom}{\perp}

\mathcode`:="603A  % treat : as punctuation instead of relation in math mode
\mathchardef\colon="303A	% relation colon

\newcommand{\eqdef}{\mathrel{:=}}
\newcommand{\Dom}{\mathop{\rm dom}}
\newcommand{\Pow}{\mathop{\rm Pow}}

\newcommand{\aequiv}{\equiv_\alpha}
\newcommand{\baro}{\buildrel \beta \over \rightarrow}
\newcommand{\earo}{\buildrel \eta \over \rightarrow}
\newcommand{\red}{\rightarrow\!\!\!\!\rightarrow}
\newcommand{\backred}{\leftarrow\!\!\!\!\leftarrow}
\newcommand{\bred}{\buildrel \beta \over \red}
\newcommand{\ered}{\buildrel \eta \over \red}
\newcommand{\conv}{\leftrightarrow}
\newcommand{\beconv}{\buildrel {\beta, \eta} \over\leftrightarrow}

% lambda calculus abbreviations
\newcommand{\letdec}[3]{\b{let\ } #1 = #2 \b{\ in\ } #3}
\newcommand{\letrec}[3]{\b{letrec\ } #1 = #2 \b{\ in\ } #3}

\newcommand{\thmbox}
   {{\ \hfill\hbox{%
      \vrule width1.0ex height1.0ex
   }\parfillskip 0pt }}
\newcommand{\qed}{\thmbox}


% make single spacing

\newcommand{\singlespace}{\renewcommand{\baselinestretch}{1}\@normalsize}
\newcommand{\etal}{{\em et al.}}

% bycase command for definition by cases (pg 49 of LaTeX)

\newcommand{\bycase}[1]
	{\left\{ \begin{array}{ll}  #1  \end{array} \right. }


% make \cite put blanks after the comma (use in alpha style)

%\def\@citex[#1]#2{\if@filesw\immediate\write\@auxout{\string\citation{#2}}\fi
%  \def\@citea{}\@cite{\@for\@citeb:=#2\do
%    {\@citea\def\@citea{,\penalty100\hskip2.5pt plus1.5pt minus.8pt}%
%       \@ifundefined{b@\@citeb}{{\bf ?}\@warning
%       {Citation `\@citeb' on page \thepage \space undefined}}
%\hbox{\csname b@\@citeb\endcsname}}}{#1}}

 % zero-width inference rule markers
 % (if they have width, they throw off centering)
\newcommand{\zwb}[1]{\makebox[0cm][l]{{\scriptsize $#1$}}}

\newcommand{\mall}{{\sc mall}}
\newcommand{\ol}{\overline}
\newcommand{\ul}{\underline}
\newcommand{\lan}{\langle}
\newcommand{\ran}{\rangle}

 % Linear Arrow
\newcommand{\la}{\mbox{$-\!\circ$}} 

 % Backwards Linear Arrow
\newcommand{\bla}{\mbox{$\circ\!-$}} 

 % Par
%\newcommand{\Par}{\,{\parfont P}\,}
\newcommand{\Par}{\mathrel{\wp}}

 % With (Linear And) (&)
\newcommand{\with}{\,{\&}\,}

% Old par
% \newcommand{\Par}{\mbox{\raisebox{.4ex}{$\wp$}}} 

 % PDL's prettier bounded reuse operator
\newcommand{\bang}[1]{\stackrel{\scriptstyle \dagger}{\scriptstyle #1}}
 
 % Implies in Relevant Implication
\newcommand{\ra}{\mbox{$\rightarrow_{R_{\rightarrow}}$}}

 % Relevant Implication (Logic)
\newcommand{\rara}{\mbox{${R_{\rightarrow}}$}}

 % Multiplicative Linear Logic  
\newcommand{\lala}{\mbox{$LL_{\la, \otimes, !}$}}

 % (Propositional) Circular Logic
\newcommand{\cl}{\mbox{ CL }}

 % (Propositional) Multiplicative Circular Logic
\newcommand{\mcl}{\mbox{ MCL }}

 % Definition
\newcommand{\define}{\mbox{${\stackrel{\Delta}{=}}$}}

 % Derived From
\newcommand{\derive}{\mbox{${\stackrel{\vdots}{\vdash}}$}}

 % Semi-Thue Reduction
\newcommand{\stred}[1]{\mbox{$\Longrightarrow$}}

\newcommand{\stredstar}[1]{\mbox{$\Longrightarrow^*$}}

\newcommand{\deltared}[1]{\mbox{$\, \stackrel{\delta_{#1}}
                                             {\longrightarrow} \, $}}

\newcommand{\deltaredp}[1]{\mbox{$\, \stackrel{\delta'_{#1}}
                                             {\longrightarrow} \, $}}


\newenvironment{commentary}{\begin{quote}\small{\bf Commentary:}}{{\bf
End of Commentary}\normalsize\end{quote}} 
%\newenvironment{proof}{{\bf Proof:}}{$\Box$\\}
%\newtheorem{theorem}{Theorem}
%\newtheorem{lemma}{Lemma}
%\newtheorem{proposition}{Proposition}
%\newtheorem{definition}{Definition}
%\newtheorem{corollary}{Corollary}
%\newtheorem{example}{Example}
\newenvironment{mytheorem}{\begin{theorem}\rm}{\end{theorem}}
%\newcommand{\qed}[0] {\rule{1ex}{1ex} \vspace{2ex}}
\newcommand{\defdef}{\mbox{$\stackrel{\rm def}{=}$}}
\def\defn{\mathrel{\defdef}}
%\newcommand{\defn}{\mbox{$\stackrel{\rm def}{=}$}}
\newcommand{\intro}[1]{\begin{itemize}\item #1 \end{itemize}}
\newcommand{\simeqrel}[1]{\stackrel{#1}{\simeq}}
\newcommand{\eqrel}[1]{\stackrel{#1}{\approx}}
\newcommand{\simrel}[1]{\stackrel{#1}{\sim}}
\newcommand{\congrel}[1]{\stackrel{#1}{\cong}}
\def\maj{\mathop{\rm maj}}
\def\min{\mathop{\rm min}}
\def\rem{\mathop{\rm rem}}
\def\div{\mathop{\rm div}}
\newcommand{\rmand}{\mbox{\bf \ and }}
\newcommand{\rmif}{\mbox{\bf if\ }}
\newcommand{\rmiff}{\mbox{\bf \ iff \ }}
\newcommand{\rmthen}{\mbox{\bf \ then }}
\newcommand{\rmelse}{\mbox{\bf \ else }}
\newcommand{\rmend}{\mbox{\bf end}}
\newcommand{\rmendif}{\mbox{\bf \ endif}}
\newcommand{\rmotherwise}{\mbox{\bf otherwise}}
\newcommand{\rmwith}{\mbox{\bf \ with\ }}
\newcommand{\aless}{\mathrel{\mbox{\lower.9ex\hbox{$\stackrel{\textstyle <}{\sim}$}}}}
\newcommand{\amore}{\mathrel{\mbox{\lower.9ex\hbox{$\stackrel{\textstyle >}{\sim}$}}}}
\newcommand{\seqarrow}{\mathrel{\mbox{\boldmath $\rightarrow$}}}

%% Derived from John Rushby's prelude.tex, modified for NFSS2
%
% define variants of the \LaTeX macro that avoid using \sc
% for use in headings
%

% Define fonts that work in math or text mode
\def\dwimrm#1{\ifmmode\mathrm{#1}\else\textrm{#1}\fi}
\def\dwimsf#1{\ifmmode\mathsf{#1}\else\textsf{#1}\fi}
\def\dwimtt#1{\ifmmode\mathtt{#1}\else\texttt{#1}\fi}
\def\dwimbf#1{\ifmmode\mathbf{#1}\else\textbf{#1}\fi}
\def\dwimit#1{\ifmmode\mathit{#1}\else\textit{#1}\fi}
\def\dwimnormal#1{\ifmmode\mathnormal{#1}\else\textnormal{#1}\fi}

\def\BigLaTeX{{\rm L\kern-.36em\raise.3ex\hbox{\small\small A}\kern-.15em
    T\kern-.1667em\lower.7ex\hbox{E}\kern-.125emX}}
\def\BoldLaTeX{{\bf L\kern-.36em\raise.3ex\hbox{\small\small\bf A}\kern-.15em
    T\kern-.1667em\lower.7ex\hbox{E}\kern-.125emX}}
%\def\labelitemi{$\bullet$}
\def\labelitemii{$\circ$}
\def\labelitemiii{$\star$}
\def\labelitemiv{$\diamond$}
\newcommand{\tcc}{{\small\small TCC}}
\newcommand{\tccs}{\tcc s}
\newcommand{\emacs}{{Emacs}}
\newcommand{\Emacs}{\emacs}
\newcommand{\ehdm}{{E{\small\small HDM}}}
\newcommand{\Ehdm}{\ehdm}
\newcommand{\tm}{$^{\mbox{\tiny TM}}$}
\newcommand{\hozline}{{\noindent\rule{\textwidth}{0.4mm}}}

\newcommand{\allclear}%
  {\mbox{\boldmath$\stackrel{\raisebox{-.2ex}[0pt][0pt]%
              {$\textstyle\oslash$}}{\displaystyle\bot}$}}

\newenvironment{private}{}{}

\newenvironment{smalltt}{\begin{alltt}\small}{\end{alltt}}

\newlength{\hsbw}

\newenvironment{session}%
  {\begin{flushleft}
   \setlength{\hsbw}{\linewidth}
   \addtolength{\hsbw}{-\arrayrulewidth}
   \addtolength{\hsbw}{-\tabcolsep}
   \begin{tabular}{@{}|c@{}|@{}}\hline 
   \begin{minipage}[b]{\hsbw}
   \begingroup\small\mbox{ }\\[-1.8\baselineskip]\begin{alltt}}
  {\end{alltt}\endgroup\end{minipage}\\ \hline 
   \end{tabular}
   \end{flushleft}}

\newenvironment{smallsession}%
  {\begin{flushleft}
   \setlength{\hsbw}{\linewidth}
   \addtolength{\hsbw}{-\arrayrulewidth}
   \addtolength{\hsbw}{-\tabcolsep}
   \begin{tabular}{@{}|c@{}|@{}}\hline 
   \begin{minipage}[b]{\hsbw}
   \begingroup\footnotesize\mbox{ }\\[-1.8\baselineskip]\begin{alltt}}%
  {\end{alltt}\endgroup\end{minipage}\\ \hline 
   \end{tabular}
   \end{flushleft}}

\newenvironment{spec}%
  {\begin{flushleft}
   \setlength{\hsbw}{\textwidth}
   \addtolength{\hsbw}{-\arrayrulewidth}
   \addtolength{\hsbw}{-\tabcolsep}
   \begin{tabular}{@{}|c@{}|@{}}\hline 
   \begin{minipage}[b]{\hsbw}
   \begingroup\small\mbox{ }\\[-0.2\baselineskip]}%
  {\endgroup\end{minipage}\\ \hline 
   \end{tabular}
   \end{flushleft}}

\newcommand{\memo}[1]%
  {\mbox{}\par\vspace{0.25in}%
   \setlength{\hsbw}{\linewidth}\addtolength{\hsbw}{-1.5ex}%
   \noindent\fbox{\parbox{\hsbw}{{\bf Memo: }#1}}\vspace{0.25in}}

\newcommand{\nb}[1]%
  {\mbox{}\par\vspace{0.25in}%
   \setlength{\hsbw}{\linewidth}\addtolength{\hsbw}{-1.5ex}%
   \noindent\fbox{\parbox{\hsbw}{{\bf Note: }#1}}\vspace{0.25in}}

\newcommand{\comment}[1]{}
\newcommand{\exfootnote}[1]{}
%\newcommand{\ifelse}[2]{#1}
\sloppy
\clubpenalty=100000
\widowpenalty=100000
%\displaywidowpenalty=100000
\setcounter{secnumdepth}{3} 
\setcounter{tocdepth}{3}
\setcounter{topnumber}{9}
\setcounter{bottomnumber}{9}
\setcounter{totalnumber}{9}
\renewcommand{\topfraction}{.99}
\renewcommand{\bottomfraction}{.99}
\renewcommand{\floatpagefraction}{.01}
\renewcommand{\textfraction}{.2}
\font\largett=cmtt10 scaled\magstep1
\font\Largett=cmtt10 scaled\magstep2
\font\hugett=cmtt10 scaled\magstep3

\def\labelitemii{$\circ$}
\def\labelitemiii{$\star$}
\def\labelitemiv{$\diamond$}
\newcommand{\tcc}{{\small\small TCC}}
\newcommand{\tccs}{\tcc s}

%\renewcommand{\memo}[1]{\mbox{}\par\vspace{0.25in}\noindent\fbox{\parbox{.95\linewidth}{{\bf Memo: }#1}}\vspace{0.25in}}

\newcommand{\eg}{{\em e.g.\/},}
\newcommand{\ie}{{\em i.e.\/},}

\newcommand{\pvs}{PVS}

\newcommand{\ch}{\choice}
\newcommand{\rsv}[1]{{\rm\tt #1}}

\newcommand{\lpvstheory}[3]{\figurehead{\hozline\smaller\smaller\begin{alltt}}%
                           \figuretail{\end{alltt}\vspace{-0in}\hozline}%
                           \figurelabel{#3}\figurecap{#2}%
                           \begin{longfigure}\input{#1}\end{longfigure}}

\newcommand{\pvstheory}[3]
  {\begin{figure}[htb]\begin{boxedminipage}{\textwidth}%
   {\smaller\smaller\begin{alltt} \input{#1}\end{alltt}}\end{boxedminipage}%
   \caption{#2}\label{#3}\end{figure}}

\newcommand{\bpvstheory}[3]
  {\begin{figure}[b]\begin{boxedminipage}{\textwidth}%
   {\smaller\smaller\begin{alltt} \input{#1}\end{alltt}}\end{boxedminipage}%
   \caption{#2}\label{#3}\end{figure}}

\newcommand{\spvstheory}[1]
  {\vspace{0.1in}\par\noindent\begin{boxedminipage}{\textwidth}%
   {\smaller\smaller\begin{alltt} \input{#1}\end{alltt}}\end{boxedminipage}\vspace{0.1in}%
   }
%\newenvironment{spvstext}%
%  {\vspace{0.1in}\par\noindent\begin{boxedminipage}{\textwidth}%
%   \smaller\smaller\begin{alltt}}%
%  {\end{alltt}\end{boxedminipage}\vspace{0.1in}%
%   }


%  {\begin{boxedminipage}{\textwidth}{\smaller\smaller\begin{alltt}#1\end{alltt}}\end{boxedminipage}}
%\newenvironment{spvstheory}{\par\noindent\begin{boxedminipage}{\textwidth}\smaller\smaller\begin{alltt}}{\end{alltt}\end{boxedminipage}}

\newenvironment{pvsex}%
  {\setlength{\topsep}{0in}\smaller\begin{alltt}}%
  {\end{alltt}}

\newcommand{\pvsbnf}[2]
  {\begin{figure}[htb]\begin{boxedminipage}{\textwidth}%
   \input{#1}\end{boxedminipage}\caption{#2}\label{#1}\end{figure}}

\newcommand{\spvsbnf}[1]
  {\begin{boxedminipage}{\textwidth}\input{#1}\end{boxedminipage}}

\newcommand{\pidx}[1]{{\rm #1}} % primary index entry
\newcommand{\sidx}[1]{{\rm #1}} % secondary index entry
\newcommand{\cmdindex}[1]{\index{#1@\cmd{#1}}}
\newcommand{\icmd}[1]{\cmd{#1}\cmdindex{#1}}
\newcommand{\iecmd}[1]{\ecmd{#1}\cmdindex{#1}}
\newcommand{\buf}[1]{\texttt{#1}}
\newcommand{\ibuf}[1]{\buf{#1}\index{#1 buffer@\buf{#1} buffer}\index{buffers!\buf{#1}}}

\newenvironment{pvscmds}%
  {\par\noindent\smaller%
   \begin{tabular*}{\textwidth}{|l@{\extracolsep{\fill}}l@{\extracolsep{\fill}}l|}\hline%
     {\it Command} & {\it Aliases} & {\it Function}\\ \hline}%
  {\hline\end{tabular*}\vspace{0.1in}}

\newenvironment{pvscmdsna}%
  {\par\noindent\smaller%
   \begin{tabular*}{\textwidth}{|l@{\extracolsep{\fill}}l|}\hline%
     {\it Command} & {\it \,\,Function}\\ \hline}%
  {\hline\end{tabular*}\vspace{0.1in}}

\newcommand{\cmd}[1]{{\tt #1}}
\newcommand{\ecmd}[1]{{\tt M-x #1}}

\newcommand{\latex}{\LaTeX}                  %  LaTeX
\newcommand{\sun}{{S{\smaller\smaller UN}}}                 %  Sun
\newcommand{\sparc}{{S{\smaller\smaller PARC}}}             %  Sparc
\newcommand{\sunos}{{S{\smaller\smaller UN}OS}}             %  SunOS
\newcommand{\solaris}{{\em Solaris\/}}        %  Solaris
\newcommand{\sunview}{{S{\smaller\smaller UN}V{\smaller\smaller IEW}}} %SunView
\newcommand{\unix}{{U{\smaller\smaller NIX}}}               %  Unix
\newcommand{\lisp} {{\sc Lisp}}              %  Lisp
\newcommand{\gnu}{{Gnu Emacs}}           %  Gnu Emacs
\newcommand{\gnuemacs}{{Gnu Emacs}}      %  Gnu Emacs
\newcommand{\emacsl}{{Emacs-Lisp}}       %  Emacs Lisp
\newcommand{\shell}{{\sc Csh}}               %  C-shell

\newcommand{\update}[3]{#1\{#2\leftarrow #3\}}
\newcommand{\interp}[3]{\cal{M}\dlb {\tt #1 : #2 }\drb #3}
\newcommand{\myforall}[2]{(\forall{#1 .}\ #2)}
\newcommand{\myexists}[2]{(\exists{#1 .}\ #2)}
\newcommand{\mth}[1]{$ #1 $}
\newcommand{\labst}[2]{(\lambda{#1}.\ #2)}
\newcommand{\app}[2]{(#1\ #2)}
\newcommand{\problem}[1]{{\bf Exercise: } {\em #1}}
\newcommand{\rectype}[1]{[\# 1 \#]}
\newcommand{\recttype}[1]{{\tt [\# 1 \#]}}
\newcommand{\dlb}{\lbrack\!\lbrack}
\newcommand{\drb}{\rbrack\!\rbrack}
\newcommand{\cross}{\times}
\newcommand{\key}[1]{{\tt #1}}
\newcommand{\keyindex}[1]{\index{#1@\key{#1}}}
\newcommand{\ikey}[1]{\key{#1}\keyindex{#1}}
\newcommand{\keyword}[1]{{\smaller\texttt{#1}}}

\newenvironment{keybindings}%
  {\begin{center}\begin{tabular}{|l|l|}\hline Key & Function\\ \hline}%
  {\hline\end{tabular}\end{center}}
\def\rmif{\mbox{\bf if\ }}
\def\rmiff{\mbox{\bf \ iff \ }}
\def\rmthen{\mbox{\bf \ then }}
\def\rmelse{\mbox{\bf \ else }}
\def\rmend{\mbox{\bf end}}
\def\rmendif{\mbox{\bf \ endif}}
\def\rmotherwise{\mbox{\bf otherwise}}
\def\rmwith{\mbox{\bf \ with\ }}
\def\mapb{\char"7B\char"7B}
\def\mape{\char"7D\char"7D}
\def\setb{\char"7B}
\def\sete{\char"7D}

% ---------------------------------------------------------------------
% Macros for little PVS sessions displayed in boxes.
%
% Usage: (1) \setcounter{sessioncount}{1} resets the session counter
%
%        (2) \begin{session*}\label{thissession}
%             .
%              < lines from PVS session >
%             .
%            \end{session*}
%
%            typesets the session in a numbered box in ALLTT mode.
%
%  session instead of session* produces unnumbered boxes
%
%  Author: John Rushby
% ---------------------------------------------------------------------
\newlength{\hsbw}
\newenvironment{session}{\begin{flushleft}
 \setlength{\hsbw}{\linewidth}
 \addtolength{\hsbw}{-\arrayrulewidth}
 \addtolength{\hsbw}{-\tabcolsep}
 \begin{tabular}{@{}|c@{}|@{}}\hline 
 \begin{minipage}[b]{\hsbw}
% \begingroup\small\mbox{ }\\[-1.8\baselineskip]\begin{alltt}}{\end{alltt}\endgroup\end{minipage}\\ \hline
 \begingroup\sessionsize\vspace*{1.2ex}\begin{alltt}}{\end{alltt}\endgroup\end{minipage}\\ \hline
 \end{tabular}
 \end{flushleft}}
\newcounter{sessioncount}
\setcounter{sessioncount}{0}
\newenvironment{session*}{\begin{flushleft}
 \refstepcounter{sessioncount}
 \setlength{\hsbw}{\linewidth}
 \addtolength{\hsbw}{-\arrayrulewidth}
 \addtolength{\hsbw}{-\tabcolsep}
 \begin{tabular}{@{}|c@{}|@{}}\hline 
 \begin{minipage}[b]{\hsbw}
 \vspace*{-.5pt}
 \begin{flushright}
 \rule{0.01in}{.15in}\rule{0.3in}{0.01in}\hspace{-0.35in}
 \raisebox{0.04in}{\makebox[0.3in][c]{\footnotesize \thesessioncount}}
 \end{flushright}
 \vspace*{-.57in}
 \begingroup\small\vspace*{1.0ex}\begin{alltt}}{\end{alltt}\endgroup\end{minipage}\\ \hline 
 \end{tabular}
 \end{flushleft}}
\def\sessionsize{\footnotesize}
\def\smallsessionsize{\small}
\newenvironment{smallsession}{\begin{flushleft}
 \setlength{\hsbw}{\linewidth}
 \addtolength{\hsbw}{-\arrayrulewidth}
 \addtolength{\hsbw}{-\tabcolsep}
 \begin{tabular}{@{}|c@{}|@{}}\hline 
 \begin{minipage}[b]{\hsbw}
 \begingroup\smallsessionsize\mbox{ }\\[-1.8\baselineskip]\begin{alltt}}{\end{alltt}\endgroup\end{minipage}\\ \hline 
 \end{tabular}
 \end{flushleft}}
\newenvironment{spec}{\begin{flushleft}
 \setlength{\hsbw}{\textwidth}
 \addtolength{\hsbw}{-\arrayrulewidth}
 \addtolength{\hsbw}{-\tabcolsep}
 \begin{tabular}{@{}|c@{}|@{}}\hline 
 \begin{minipage}[b]{\hsbw}
 \begingroup\small\mbox{
}\\[-0.2\baselineskip]}{\endgroup\end{minipage}\\ \hline 
 \end{tabular}
 \end{flushleft}}
\newcommand{\memo}[1]{\mbox{}\par\vspace{0.25in}%
\setlength{\hsbw}{\linewidth}%
\addtolength{\hsbw}{-2\fboxsep}%
\addtolength{\hsbw}{-2\fboxrule}%
\noindent\fbox{\parbox{\hsbw}{{\bf Memo: }#1}}\vspace{0.25in}}
\newcommand{\nb}[1]{\mbox{}\par\vspace{0.25in}\setlength{\hsbw}{\linewidth}\addtolength{\hsbw}{-1.5ex}\noindent\fbox{\parbox{\hsbw}{{\bf Note: }#1}}\vspace{0.25in}}

%%% Local Variables: 
%%% mode: latex
%%% TeX-master: t
%%% End: 

\def\id#1{\hbox{\tt #1}} %changing ids from roman to tt.
\bibliographystyle{alpha}
%\input{title}
\vspace{4in}
\maketitle
\pagestyle{fancy}
\renewcommand{\sectionmark}[1]{\markboth{{\em #1}}{}\markright{{\em #1}}}
\renewcommand{\subsectionmark}[1]{\markright{\em #1}}
%\lhead[\thepage]{\rightmark}
%\cfoot{\protect\small\bf \fbox{Beta Release}}
%\rhead[\leftmark]{\thepage}
\setcounter{secnumdepth}{2} 
\setcounter{tocdepth}{3}
\begin{abstract}

\newcommand{\PVSlanguage}{S.~Owre, N.~Shankar, and J.~M. Rushby.
{\em The PVS Specification Language (Beta Release)}.
Computer Science Laboratory, SRI International, Menlo Park, CA,
February 1993.}

\newcommand{\PVSuserguide}{S.~Owre, N.~Shankar, and J.~M. Rushby.
{\em User Guide for the PVS Specification and Verification
System (Beta Release)}.   Computer Science Laboratory, SRI
International, Menlo Park, CA, February 1993.}

\newcommand{\PVSprover}{ N.~Shankar, S.~Owre, and J.~M. Rushby.
{\em The PVS Proof Checker: A Reference Manual (Beta Release)}.
Computer Science Laboratory, SRI International, Menlo Park,
CA, February 1993.}

PVS stands for ``Prototype Verification System.'' It consists of a
specification language integrated with support tools and a theorem
prover.  PVS tries to provide the mechanization needed to apply formal
methods both rigorously and productively.

This tutorial serves to introduce PVS.  In the first section, we
briefly sketch the purposes for which PVS is intended and the
rationale behind its design, mention some of the uses that we and
others are making of it, and explain how to get a copy of the system.
In Section 2, we use a simple example to briefly introduce the major
functions of PVS; Sections 3 and 4 then give more detail on the PVS
language and theorem prover, respectively, also using examples.  The
PVS language, system, and theorem prover each have their own reference
manuals,~\footnote{\PVSlanguage}$^{,}$\footnote{\PVSprover}$^{,}$\footnote{\PVSuserguide}
which you will need to study in order to make productive use of the
system.  A pocket reference card, summarizing all the features of the
PVS language, system, and prover is also available.

This tutorial does not introduce the general ideas of formal methods,
nor explain how formal specification and verification can best be
applied to various problem domains; rather, its purpose is to
introduce some of the more unusual and powerful capabilities that are
provided by PVS.  Consequently, this document, and the examples we
use, are somewhat technical and are most suitable for those who
already have some experience with formal methods and wish to
understand how PVS provides mechanized support for some of the more
challenging aspects of formal methods.

\end{abstract}
\pagenumbering{roman}
\newpage
\tableofcontents
\newpage
\pagenumbering{arabic}

% Master File: wift-tutintro.tex

\section{Introducing PVS}

PVS stands for ``Prototype Verification System.''\footnote{A number of
people have contributed significantly to the design and implementation
of PVS.  They include David Cyrluk, Friedrich von~Henke, Pat Lincoln,
Steven Phillips, Sreeranga Rajan, Jens Skakkeb\ae{}k, Mandayam Srivas,
and Carl Witty.  We also thank Mark Moriconi, Director of the SRI
Computer Science Laboratory, for his support and encouragement.} It
consists of a specification language integrated with support tools and
a theorem prover.  PVS tries to provide the mechanization needed to
apply formal methods both rigorously and productively.

%We have tried to make it a very productive system to employ for
%most purposes that require mechanized support for formal methods.

% We think you will find that it is the most productive verification
% system available for many purposes.

The specification language of PVS is a higher-order logic with a rich
type-system, and is quite expressive; we have found that most of the
mathematical and computational concepts we wish to describe can be
formulated very directly and naturally in PVS\@.  Its theorem prover, or
proof checker (we use either term, though the latter is more correct),
is both interactive and highly mechanized: the user chooses each step
that is to be applied and PVS performs it, displays the result, and then
waits for the next command.  PVS differs from most other interactive
theorem provers in the power of its basic steps: these can invoke
decision procedures for arithmetic, automatic rewriting, induction, and
other relatively large units of deduction; it differs from other highly
automated theorem provers in being directly controlled by the user.  We
have been able to perform some significant new verifications quite
economically using PVS; we have also repeated some verifications first
undertaken in other systems and have usually been able to complete them
in a fraction of the original time (of course, these are previously
solved problems, which makes them much easier for us than for the
original developers).

PVS is the most recent in a line of specification languages, theorem
provers, and verification systems developed at SRI, dating back over
20 years.  That line includes the Jovial Verification
System~\cite{jovial:pv}, the Hierarchical Development Methodology
(HDM)~\cite{Robinson&Levitt,HDM:Handbook}, STP~\cite{STP}, and
\ehdm~\cite{Melliar-Smith&Rushby,EHDM:tutorial}.  We call PVS a
``Prototype Verification System,'' because it was built partly as a
lightweight prototype to explore ``next generation'' technology for
\ehdm, our main, heavyweight, verification system.  Another goal for
PVS was that it should be freely available, require no costly
licenses, and be relatively easy to install, maintain, and use.
Development of PVS was funded entirely by SRI International

%\memo{and
%by our unpaid work at nights and weekends.} and it is made available
%free of charge.

%SRI's investment is not entirely altruistic: we
%expect to significantly increase the size of the market for formal
%methods by making a productive verification system readily available.

In the rest of this introduction, we briefly sketch the purposes for
which PVS is intended and the rationale behind its design, mention
some of the uses that we and others are making of it, and explain how
to get a copy of the system.  In Section~2, we use a simple example to
briefly introduce the major functions of PVS; Sections~3 and 4 then
give more detail on the PVS language and theorem prover, respectively,
also using examples.  More realistic examples are provided in
Section~5.  The PVS language, system, and theorem prover each have
their own reference
manuals~\cite{PVS:language,PVS:prover,PVS:userguide}, which you will
need to study in order to make productive use of the system.  A pocket
reference card, summarizing all the features of the PVS language,
system, and prover is also available.

The purpose of this tutorial is not to introduce the general ideas of
formal methods, nor to explain how formal specification and
verification can best be applied to various problem domains; rather,
its purpose is to introduce some of the more unusual and powerful
capabilities that are provided by PVS.  Consequently, this
document, and the examples we use, are somewhat technical and are most
suitable for those who already have some experience with formal
methods and wish to understand how PVS provides mechanized support for
some of the more challenging aspects of formal methods.

\subsection{Design Goals for PVS}

PVS provides mechanized support for Formal Methods in Computer
Science.  ``Formal Methods'' refers to the use of concepts and
techniques from logic and discrete mathematics in the development of
computer systems, and we assume that you already have some
familiarity with this topic.


Formal methods can be undertaken for many different purposes, in many
different ways and styles, and with varying degrees of rigor.  The
earliest formal methods were concerned with proving programs
``correct'': a detailed specification was assumed to be available and
assumed to be correct, and the concern was to show that a program in
some concrete programming language satisfied the specification.  If
this kind of program verification is your interest, then PVS is not
for you.  You will probably be better served by a verification system
built around a programming language, such as Penelope~\cite{Prasad92} (for
Ada), or by some member of the Larch family~\cite{Larch85}.
Similarly, if your interests are gate-level hardware designs, you
will probably do best to consider model-checking and automatic
procedures based on BDDs~\cite{Clarke-etal90}.

\comment{
We have worked chiefly on problems in ``critical systems'' of various
kinds (e.g., digital flight control) and the design of PVS reflects
this background.  These critical systems are developed according to
very exacting procedures, involving extensive testing and review
(see, e.g., the standards for airborne systems for civil
aircraft~\cite{DO178B}).  The evidence is that few undetected faults are
introduced during the later stages of the lifecycle (i.e., detailed
design and coding) of systems constructed according to these
procedures.  Instead, the overwhelming evidence is that the serious
and undetected faults are introduced during the early stages of the
lifecycle---in requirements formulation, interface specification, and
mistaken or inconsistent assumptions about the behavior of the larger
system with which the computer system must interact.  For example, in
203 formal inspections of six projects at the Jet Propulsion
Laboratory, it was found that requirements documents averaged one
major defect every three pages, compared with one every 20 pages for
code.  Two-thirds of the defects in requirements were
omissions~\cite{Kelly-etal92}.  In other JPL data, 197 faults
detected during integration and system testing of the Voyager and
Galileo spacecraft were characterized as having potentially
significant or catastrophic effects (with respect to the spacecraft's
missions)~\cite{Lutz92:rqts}.  Of these 197 faults, 3 were coding
errors.  The remaining 194 faults were divided approximately 3:1
overall between ``function faults'' (faults within a single software
module) and interface faults (interactions with other modules or
system components).  Two thirds of function faults were attributed to
flawed requirements (of which omissions were the most common flaw);
the remaining one third were due to incorrect implementation of
requirements (i.e., faulty design or algorithms), and tended to
involve inherent technical complexity, rather than a failure to
follow the letter of the requirements.  Turning to hardware,
Keutzer~\cite{Keutzer:hol91} reports that over half of all VLSI chips contain
faults that are only detected after first fabrication.  None of these
faults come from the later stages of the development lifecycle
(probably because of the extensive simulation and analysis
performed); {\em all\/} are due to flawed requirements or to errors
introduced in the earliest stages of design.

Thus, in the fields from which our applications come, the later
stages of the lifecycle are considered under control (maybe
expensively and clumsily, but under control nonetheless).  All the
concern is with the early lifecycle: with requirements, overall
architectural design, interfaces, and critical algorithms.  The
latter arise, for instance, in fault-tolerant systems, where
malfunctioning channels and timing anomalies can lead to unexpected
behavior.  For example, in flight testing of the AFTI-F16 (which had
an early digital flight control system), ill-understood interactions
in the redundancy management and fault-tolerance mechanisms became
the {\em primary\/} source of system failure~\cite{Mackall:TR}.  
}

The design of PVS was shaped by our experience in doing or
contemplating early-lifecycle applications of formal methods.  Many of
the larger examples we have done concern algorithms and architectures
for fault-tolerance (see~\cite{Owre95:prolegomena} for an overview).  We
found that many of the published proofs that we attempted to check
were in fact, incorrect, as was one of the important algorithms.  We
have also found that many of our own specifications are subtly flawed
when first written.  For these reasons, PVS is designed to help in the
detection of errors as well as in the confirmation of ``correctness.''
One way it supports early error detection is by having a very rich
type-system and correspondingly rigorous typechecking.  A great deal
of specification can be embedded in PVS types (for example, the
invariant to be maintained by a state-machine can be expressed as a
type constraint), and typechecking can generate proof obligations that
amount to a very strong consistency check on some aspects of the
specification.\footnote{As a way to further strengthen error checking,
we are thinking of adding dimensions and dimensional analysis to the
PVS type system and typechecker.}

Another way PVS helps eliminate certain kinds of errors is by
providing very rich mechanisms for conservative extension---that is,
definitional forms that are guaranteed to preserve consistency.
Axiomatic specifications can very effective for certain kinds of
problem (e.g., for stating assumptions about the environment), but
axioms can also introduce inconsistencies---and our experience has
been that this does happen rather more often than one would wish.
Definitional constructs avoid this problem, but a limited repertoire
of such constructs (e.g., requiring everything to be specified as a
recursive function) can lead to excessively constructive
specifications: specifications that say ``how'' rather than ``what.''
PVS provides both the freedom of axiomatic specifications, and the
safety of a generous collection of definitional and constructive
forms, so that users may choose the style of specification most
appropriate to their problems.\footnote{Unlike \ehdm, PVS does not
provide special facilities for demonstrating the consistency of
axiomatic specifications.  We do expect to provide these in a later
release, but using a different approach than \ehdm.}

The third way that PVS supports error detection is by providing an
effective theorem prover.  Our experience has been that the act of
trying to prove properties about specifications is the most effective
way to truly understand their content and to identify errors.  This
can come about incidentally, while attempting to prove a ``real''
theorem, such as that an algorithm achieves its purpose, or it can be
done deliberately through the process of ``challenging''
specifications as part of a validation process.  A challenge has the
form ``if this specification is right, then the following ought to
follow''---it is a test case posed as a putative theorem; we
``execute'' the specification by proving theorems about
it.\footnote{Directly executable specification languages
(e.g.,~\cite{me-too,Hekmatpour&Ince:prototyping}) support validation
of specifications by running conventional test cases.  We think there
can be merit in this approach, but that it should not compromise the
effectiveness of the specification language as a tool for deductive
analysis; we are considering supporting an executable subset within
PVS.}

\comment{
Some challenges can be constructed systematically (e.g., it is
always useful to demonstrate that a predicate---a boolean valued
function---is capable of yielding both true and false results), but
most require imagination and insight.  Of course, challenges can be
undertaken without mechanized proof checking, but our experience is
that informal proofs are too often influenced by what the author
thinks the specification says (or ought to say); mechanized proof
checking helps the user understand what the specification really does
say.\footnote{For experienced users, the discovery of overlooked
cases is the most common benefit of performing challenges---which is
consistent with the observation that omissions are the most common
flaw in informal requirements specifications.  Inexperienced users
usually find that their specifications are essentially meaningless.}
Our own experience, and that of users of \ehdm\ and PVS, has been
such that we attach very little credibility to formal specifications
that have not been subjected to some degree of mechanized proof
checking.

A theorem prover that is to help in challenging specifications
obviously needs to be effective---so that the user can concentrate on
the substance of the problem, and not on incidental difficulties of
mechanization---but the notion of ``effective'' must comprehend more
than just ``efficiency'' or ``power.''  It will be obvious from what
we have said that many of the ``theorems'' on which the prover is
invoked will not, in fact, be theorems at all.  Most theorem provers
are set up to prove true theorems, and may waste a lot of time
exploring fruitless paths when confronted by a nontheorem.  Even when
they return with ``unproved,'' it may not be clear whether the
theorem is false, or inadequate heuristics were employed.  The PVS
theorem prover, on the other hand, is designed so that the user
understands and shapes the overall argument.  This not only helps
discover errors fairly effectively (typically, the user discovers an
``obviously false'' case due to some missing constraint or other
oversight), but it maximizes the information content and
understanding derived from successful proofs; our goal in PVS is to
shift the focus from theorems (``can you get it proved?'') to proofs
(``what did you learn by proving it?'').  As a side effect, the
approach to theorem proving used in PVS generally seems to allow
proofs of true theorems to be constructed far more quickly than with
any other theorem prover we know.
}

%\memo{Main use of FM should be to discover assumptions, test
%requirements---to do validation as much as verification.}
%
\comment{
%\section{Design Choices in PVS}

The first choice that must be made in design of a specification
language is selection of its logical foundation.  There are three
main traditions in mathematics and logic to draw on: the {\em
logicist\/} approach of Frege and Russell, which uses higher-order
logic, with the constraints of type theory to keep it consistent; the
{\em formalist\/} approach of Hilbert, Zermelo, and Fraenkel, which
uses first-order logic plus axiomatic set theory; and the {\em
intuitionistic\/} approach of Brouwer, which in computer science is
mostly shaped by Martin-L\"{o}f's notions of propositions-as-types.

Our decisions for PVS were largely pragmatic rather than
philosophical.  The intuitionistic approach is a fruitful area of
research in computer science, and may lead to new understandings of
computational processes, but it does not seem likely to lead, in the
short term at least, to the major gains in productivity of formal
methods that are our aim.

Set theory is the foundation of choice for most mathematicians, but
it is important to understand their motivations, and how they differ
from those who undertake formal methods in computer science.
Set theory was designed as a minimalist foundation within which
``all of mathematics'' could {\em in principle\/} be formalized.
The ``in principle'' is important: mathematicians seldom
actually formalize anything.  Formal methods in computer science, on
the other hand, is concerned with formalizing requirements, designs,
algorithms and programs, and with developing formal proofs {\em in
practice\/}.  Set theory requires mathematical knowledge to be built
from the bottom up by encoding concepts in terms of more primitive
concepts.   Proofs in computing are, however, best carried out with some
degree of abstraction that is unencumbered by details of particular
encodings of concepts.  

%Mathematical logic was developed by mathematicians to address matters
%of concern to them.  Initially, those concerns were to provide a
%minimal and self-evident foundation for mathematics; later, technical
%questions about logic itself became important.  For these reasons,
%much of mathematical logic is set up for {\em meta\/}mathematical
%purposes: to support (relatively) simple proofs of properties such as
%soundness and completeness, and to show that certain elementary
%concepts allow some parts of mathematics to be formalized in
%principle.  

The problems with set theory as a foundation for formal methods are
that many of its constructions are metamathematical (i.e., they are
constructions about set theory, not {\em in\/} set theory---although,
of course, they could performed within set theory {\em in
principle\/}) and use inperspicuous encodings (e.g., Kuratowski and
Wiener's ``programming trick'' for representing the ordered pair $(a,
b)$ as the set $\{a, \{a, b\}\}$).  Of course, these difficulties can
be mitigated: for example, we can provide parameterized modules or
``schemas'' in order to admit some metamathematical constructions,
and we can use a set theory with some of the basic data types built
in to avoid the more grotesque constructions.  Despite these
mitigations, however, set theory still has two problems that we
consider fatal.  First, it is an essentially untyped system:
everything is a set and can be freely combined with other sets.
Thus, a function is a set of pairs, and we can take its union with
some other set.  Our experience has been that strong typing is
essential for efficient and early detection of errors in
specifications, and is a useful discipline in its own right (e.g.,
simply writing down the types of the inputs to a function, or of the
components of a state, can be a valuable first step in a
specification).  Strong typing sits uncomfortably on set theory and
is essentially as kludge (since it must be escaped to perform certain
constructions).  Second, functions in set theory are sets of pairs,
and are inherently partial; total functions are a special case.
Efficient theorem proving usually requires that functions be total,
and can therefore be difficult to achieve in classical set theory.
But there is another difficulty to partial functions apart from this:
it is tricky to ascribe a semantics to expressions involving
functions that might be applied outside their domain.  Mathematicians
avoid this problem, by informally ensuring that their function
applications are always well-defined, but in formal methods we have
to find some way to enforce this mechanically.  The best choice is
probably a logic of partial terms~\cite{Beeson86}, in which terms can
be undefined, but expressions retain the familiar two truth values of
classical logic.  A less satisfactory choice is a logic of partial
functions in which ``undefined'' becomes a truth value and the whole
structure of the logic is modified to accommodate
it~\cite{Cheng&Jones90}.  In either case, the appealing simplicity of
set theory is compromised.  As with the other difficulties described
earlier, it is possible to mitigate these ones, but the mitigations
tend to move in the direction of type theory (also known as
higher-order logic), and it might seem best to simply adopt that
approach in the first place.

Higher-order logic\footnote{``Higher-order'' means that functions can
take functions as arguments and return functions as values, and
allows quantification to range over functions.} is the principal
alternative to axiomatic set theory as a classical (i.e.,
nonintuitionistic) foundation for mathematics.  Higher-order logic
needs the discipline of strong typing to keep it
consistent,\footnote{Russell's paradox shows that unrestricted
higher-order logic is inconsistent (as is unrestricted set theory).
Russell developed his theory of types to restore consistency to
higher-order logic.  His original theory had a notion of ``order'' as
well of type and is known as the ``Ramified Theory of Types.''
Ramsey observed that orders were not needed to exclude the
``logical'' paradoxes, and the theory with types but not orders is
called the ``Simple Theory of Types.''  Its modern formulation is due
to Church.  In current terminology ``Simple Type Theory'' is
equivalent to ``Higher-Order Logic.''} but this aligns with our
wishes anyway.  Many mathematicians find this discipline and some of
its associated distinctions irksome.  For example, the empty set is
not a single notion in type theory: there is a different empty set
for each type of elements.  Mathematicians call this ``reduplication
of notions'' ``repugnant''~\cite{Fraenkel-etal84}, but it is
perfectly defensible on linguistic grounds (e.g., is having no money
the same as having no worries?), and no trouble for formal methods in
practice.\footnote{Mathematicians also saddled themselves with opaque
notation for type theory: they reverse the order of the type symbols,
``curry'' all functions, and write applications without parentheses.
Computer Scientists are usually happy to ``declare'' the types of
variables and functions before use, and can take advantage of
computerized analysis that supports notational conveniences such as
name overloading and type-inference.}

One attraction of higher-order logic as a foundation for mechanized
formal methods is that it is very expressive: it is possible to say
a great deal in higher-order logic without metalogical assistance.
A second attraction is that it is inherently a typed system, and a
third is that functions are total, so that the lower levels of
theorem proving can be made relatively efficient.

For the reasons described, we have chosen higher-order logic as the
foundation for the PVS specification language.  Essentially similar
logics also provide the foundations for \ehdm, and for
HOL~\cite{Gordon:HOL88}, and the wide range of examples successfully
undertaken in those systems attest to the utility of higher-order
logic as a foundation for formal methods.  PVS differs from HOL in
supplying built-in interpretations for the {\tt integer} and {\tt
rational} numeric types, and in providing records, enumerations, and
certain tree-like data structures through built-in type constructors
(in addition to functions and tuples).  In addition PVS allows {\em
predicate subtypes\/} and {\em dependent types\/}.  These greatly
increase the ``precision'' with which terms may be typed---to such an
extent that typechecking is no longer a deterministic operation but
can require the assistance of the theorem prover.

As their name suggests, predicate subtypes use a predicate to induce
a subtype on some parent type.   For example, the natural numbers are
specified in PVS as:
\[ {\em nat}: {\bf type} = \{ n: {\em int} | n >= 0 \}. \]
More interestingly, the signature for the division operation (on the
rationals) is specified by
\[ / : [rat, nonzero\_rat -> rat] \]
where
\[{\em nonzero\_rat}: {\bf type} = \{ x : {\em rat} | x /= 0 \}\]
specifies the nonzero rational numbers.   
This constrains division to nonzero divisors, so that a formula
such as
\[ x /= y => (y-x)/(x-y) <0 \]
requires the typechecker to discharge the proof obligation
\[ x /= y => (x-y) /= 0 \]
in order to ensure that the occurrence of division is well-typed.
Proof obligations such as this are called Typecheck Correctness
Conditions, or TCCs; they are sufficient (though not always
necessary) conditions which ensure that the values of logical
expressions do not depend on functions applied outside their domains.
The decision procedures of the PVS theorem prover  can
instantly dispose of simple TCCs similar to this example.   More
complex TCCs are carried along as proof obligations that must
discharged (under control of the user) before analysis of the
specification is considered complete.

As the example of division illustrates, predicate subtypes allow
certain functions that are partial in some other treatments to remain
total (thereby avoiding the need for logics of partial terms or
three-valued logics).  Related constructions allow nice treatments of
errors, such as {\em pop\/}({\em empty\/}) in the theory of stacks.
Here we can type the stack operations as follows:
\begin{alltt}\rm
  {\em stack\/}: {\bf type}
  {\em empty\/}: {\em stack\/}
  {\em nonempty\_stack\/}: {\bf type} = \{({\em s: stack\/}) \(|\) {\em s\/} \(\neq\) {\em empty\/}\}

  {\em push\/}: [{\em elem\/}, {\em stack\/} \(\rightarrow\) {\em nonempty\_stack\/}]
  {\em pop\/}: [{\em nonempty\_stack} \(\rightarrow\) {\em stack\/}]
  {\em top\/}: [{\em nonempty\_stack} \(\rightarrow\) {\em elem\/}]
\end{alltt}
With these signatures, the expression ${\em pop\/}({\em empty\/})$ is
rejected during typechecking (because {\em pop\/} requires a {\em
nonempty\_stack\/} as its argument), and the theorem
\[ {\em push\/}(e, s) \neq {\em empty\/}\]
is an immediate consequence of the type definitions.   More
interestingly, the formula
\[{\em pop\/}({\em pop\/}({\em push\/}(x, {\em push\/} (y, s)))) = s\]
is shown to be well-typed by proving the TCC
\[{\em pop\/}({\em push\/}(x, {\em push\/}(y, s))) /= {\em empty\/},\]
which follows from the usual stack axioms.  

Untrue proof obligations indicate a type-error
in the specification, and have proved a potent method for the early
discovery of specification errors.
For example, the injections are specified as that subtype of the
 functions associated with the one-to-one property:
\[{\em injection}: {\bf type} = \{f: [t_{1} -> t_{2}] \;|
	 \;\forall (i, j: t_{1}): f(i) = f(j) => i = j\}\]
(here $t_{1}$ and $t_{2}$ are uninterpreted types introduced
in the module parameter list).
If we were later to specify the function {\em square\/} as an
injection on the integers by the declaration
\[ {\em square}: {\em injection} = 
	lambda (x: {\em int}): x \times x\]
then the PVS typechecker would require us to show that the body of
{\em square\/} satisfies the {\em injection\/} subtype predicate.
That is, it requires the proof obligation $i^{2} = j^{2} => i = j$ to be
proved in order to establish that the {\em square\/} function is
well-typed.  Since this theorem is untrue (e.g., $2^{2} = (-2)^{2}$
but $2 \neq -2$), we are led to discover a fault in this
specification.

Notice how use of predicate subtypes here has automatically led to
the generation of proof obligations that might require
special-purpose tools in other systems.  Yet another example of the
utility of predicate subtypes arises when modeling a system by means
of a state machine.  In this style of specification, we first
identify the components of the system state; an invariant then
specifies how the components of the system state are related, and
operations are required to preserve this relation.  With predicate
subtypes available, we can use the invariant to induce a subtype on
the state type, and can specify that each operation returns a value
of that subtype.  Typechecking the specification will then
automatically generate the proof obligations necessary to ensure that
the operations preserve the invariant.

Dependent types increase the expressive convenience of the language
still further.   We find them particularly convenient for dealing
with functions that would be partial in simpler type systems.
The standard ``challenge'' for treatments of partial
functions~\cite{Cheng&Jones90} is the function {\em subp\/} on the
integers defined by
\[{\em subp\/}(i, j) = \rmif i = j \rmthen 0 
\rmelse {\em subp\/}(i,j+1)+1 \rmendif.\]
This function is undefined if $i<j$ (when $i \geq j, {\em
subp\/}(i,j)=i-j$) and it is often argued that if a specification
language is to admit this kind of definition, then it must provide a
treatment for partial functions.  Fortunately, examples such as these
do {\em not\/} require partial functions: they can be admitted as
total functions on a very precisely specified domain.  {\em Dependent
types\/}, in which the {\em type\/} of one component of a structure
depends on the {\em value\/} of another, are the key to this.  For
example, in the language of PVS, {\em subp\/} can be specified as
follows.
\begin{alltt}\rm
  {\em subp}({\em i:int}, ({\em j:int} \(|\) \(i\geq j\))): {\bf recursive} {\em int\/} =
       (\rmif \(i=j\) \rmthen 0 \rmelse \({\em subp\/}(i, j+1)+1\) \rmendif)
   {\bf measure} \((\lambda (i:{\em int\/}), (j:{\em int\/} | i\geq j): i-j)\)\footnotemark
\end{alltt}
\footnotetext{The {\bf measure} clause specifies the function to
be used in the termination proof.}
Here, the domain of {\em subp\/} is the dependent tuple-type
\[ [i:{\em int\/}, \{j:{\em int\/} | i >= j\}]\]
(i.e., the pairs of integers in which the first component is greater
than or equal to the second) and the function is total on this domain.

PVS is not unique in providing predicate and dependent types;
Nuprl~\cite{Nuprl-book} and Veritas~\cite{Hanna89:Veritas}, for
example, also support these constructions.  PVS differs from others
in that we support a rich type-system within an entirely classical
framework (Nuprl mechanizes a constructive type theory, Veritas
provides the unusual combination of a Martin-L\"{o}f type system and
a classical higher-order logic).

A unique feature of PVS is the tightness of the integration between
the specification language and its typechecker, and the theorem
prover.  We have just seen examples how willingness to use theorem
proving in typechecking provides simple and sound solutions to
problems that can otherwise require very complex treatments.
Conversely, the PVS theorem prover can exploit information from the
typechecker to guide its search and to decide certain properties.
Furthermore, the prover uses the PVS language-processing tools, such
as the parser, typechecker and prettyprinters, so that its dialog
with the user is conducted entirely in terms of the PVS specification
language (even though much more austere forms are used internally),
and so that the user can modify or add the statement of a lemma or
definition during an ongoing proof.  Much of what happens during a
proof attempt is the discovery of inadequacies, oversights, and
faults in the specification that is intended to support the theorem.
Having to abandon the current proof attempt, correct the problem, and
then get back to the previous position in the proof, can be very time
consuming.  Allowing the underlying specification to be extended and
modified during a proof confers enormous gains in productivity.

The design of the PVS theorem prover was guided by our conviction
that proofs are at least as important as theorems: it is usually not
enough to know that a theorem is true, we need to understand {\em
why\/} it is true---because this understanding will be needed if (or,
more likely, when) we need to modify the specification to accommodate
changed assumptions, requirements, or designs.  The PVS prover is
therefore designed so that the main steps of the proof are given by
the user; the theorem prover automates the bookkeeping and the routine
steps and provides the interactive environment that allows the user
to explore and develop the main argument.  The ideal to which we
aspire is that developing a proof with PVS should be akin to
developing one with a knowledgeable but skeptical human colleague.
This means that the basic inferences that PVS performs automatically
must be rather powerful, so that the ``dialog'' between user and
machine is not interrupted by tedious subcases to establish trivial
facts of arithmetic or to expand and simplify definitions.

For the reasons just explained, the basic inference steps in PVS were
chosen to be powerful in comparison with the simple rules given in
textbook introductions to logic.  Each inference step is flexible, so
it can be used in a variety of related ways, and takes optional
parameters that adjust its behavior.  For example, the beta reduction
rule eliminates all redexes (and for flexibility many things are
regarded as redexes) from a set of formulas specified by a parameter
(the default is all formulas).  PVS also provides a mechanism for
composing basic inference steps into proof ``strategies'' (rather
like subroutines in a programming language, or the tacticals of
LCF-style provers), a facility for rerunning proofs, and another to
check that all secondary proof obligations (such as Type Correctness
Conditions) have been discharged.  Powerful primitive inferences make
the composed inference steps correspondingly more powerful, and allow
the proof to be represented in a manner that can be rerun efficiently
and that is robust in the face of small changes to the specification
or theorem.  A small and carefully chosen set of primitive inferences
also makes the system easier to learn and use.

Interaction between PVS and the user is organized in the manner of
Gentzen's Sequent Calculus.  This is explained in the documents
describing the theorem prover, but the basic idea is that a proof is
developed as a tree of {\em sequents\/}; each sequent can be
considered as a disjunction of ``antecedent'' formulas that implies a
conjunction of ``consequent'' formulas; at any instant, the focus of
attention is one of the leaf sequents in the proof tree; a proof step
(i.e., a primitive inference) either recognizes the current focus
sequent as true and shifts attention to some other leaf sequent in
the proof tree, or else it adds one or more children to the current
focus sequent and shifts attention to one of those children; the goal
is to develop a proof tree whose leaves are all recognized as true.
This is a ``backwards'' approach to proof, in that we start from the
conclusion to be proved and progressively apply inference steps to
generate subgoals until the subgoals are trivially provable.  The
attraction of the sequent calculus is that a sequent is a
very compact and clear way to represent all the information relevant
to the current step of the proof, and the basic steps are very
regular and intuitive.

The various tools and functions of PVS use the GNU Emacs editor
running under Unix as their interface: the theorem prover runs in its
own buffer and its commands are typed directly into that buffer; the
other tools and functions are invoked through extended Emacs
commands.  This means that you do need to learn Emacs in order to use
PVS effectively.  Apart from the extra functionality of the PVS
commands, the PVS Emacs is a perfectly standard Gnu Emacs and that
can be used for editing non-PVS files, reading mail and news, and so
on.  We chose Emacs as the interface for PVS partly for its
portability and economy---the rich functionality of Emacs and of
extensions such as ILISP allowed us to construct many attractive
capabilities rather easily and inexpensively---and partly from
personal preference: Emacs is our interface of choice for everything
else we use our computers for.

PVS specifications are composed of units called {\em theories\/},
which are stored in standard ascii text files with extension {\tt
pvs}.  Each PVS file contains one or more theories, and a collection
of such files stored in one directory make up a specification.  Any
proofs developed for the theories in a given PVS file are saved in a
file with the same name, but extension {\tt prf}.  The {\tt .pvs} and
{\tt .prf} files in a single directory constitute a PVS {\em
context\/} whose state of development is automatically saved and
restored from one PVS session to another.  A typical PVS session
begins by loading and modifying some existing PVS files, or creating
new some new ones, using standard Emacs editing capabilities.
Usually, the commands to parse and typecheck a PVS file will be given
next.  Either or both of these operations may detect errors in the
PVS specification whose correction may require some iteration of
these steps.  Next, a proof may be attempted; this is started by
moving the cursor to the formula to be proved and giving the
appropriate command.  If a proof has already been saved for the
formula concerned, the user is given the option of rerunning it or
developing a new proof from scratch.  If the former option is chosen,
the saved proof may succeed, or it may not (e.g., because the
specification has changed).  In the latter case, control is returned
to the user in the same state as if the saved proof had just been
developed interactively; the user can now choose to undo some proof
steps on certain branches of the proof tree in order to develop a
modified proof suitable to the changed specification.  An alternative
approach, which is especially suitable when a specification changes
in a very regular way (e.g., a function is renamed), is to use the PVS
facilities for editing saved proof scripts---this approach is most
suitable for more advanced users.  An edited proof can be attached to
a different (or additional) formula than the one it came from
originally.  This can be very convenient if several theorems have a
very similar form and should yield to similar proofs.  Another
approach in these cases is to develop a {\em strategy\/}, that is
series of PVS proof steps combined within a control mechanism that
can be used rather like a ``proof subroutine.''

When a specification is modified, all saved proofs associated with
the changed PVS files become ``suspect'': the system will once again
consider their corresponding theorems ``proved'' only when the PVS
prover has successfully rerun them.  PVS provides commands for
rerunning such proofs as a batch, and for discovering the current
status of the theories, files, and proofs that constitute a
specification.

Specifications and proofs generally need to be studied by others than
their original developers.  PVS provides a prettyprinter for
reformatting specifications in a very regular manner, and a rather
versatile \LaTeX-printer that can be used to typeset PVS
specifications.  The \LaTeX-printer can be customized by simple
user-supplied tables in ways that allow it to reproduce standard
mathematical notation.  The same capability can also be used to
typeset a proof transcript.  PVS is also able to generate a
cross-reference to the declarations of identifiers, and has functions
that allow the declaration or uses of an identifier under the cursor
to be viewed or visited.

%\section{What's in PVS?}

In this section we briefly list the capabilities and functions of the
PVS language, prover, and system.  

TBD.

%\section{Comparing PVS to other Verification Systems}

In this section we briefly compare the facilities, capabilities, and
design choices employed in PVS with those of a number of other
systems that you might be familiar with.  The purpose of this section
is not to suggest that PVS is better than these other very fine
systems, but to give you an idea how it differs from them, and
thereby to help you decide whether PVS is likely to provide the
services you need.

% three traditions.

% VDM, Z, and mural etc,
% BM, Otter, Paulson's system
% HOL, Eves, Imps

% tight leash or drag towards conclusion
}

\subsection{Uses of PVS}

PVS has so far been applied to several small demonstration examples,
and a growing number of significant verifications.  The smaller
examples include the specification and verification of ordered binary
tree insertion~\cite{Shankar:ADT}, a compiler for simple arithmetic
expressions~\cite{Rushby95:Movie}, and several small hardware examples
including pipeline and microcode correctness~\cite{Cyrluk94:TPCD}.
Examples of this scale can typically be completed within a day.  More
substantial examples include the correctness of a real-time railroad
crossing controller~\cite{Shankar93:CAV}, an embedding of the Duration
Calculus~\cite{Skakkebaek&Shankar94}, the correctness of some
transformations used in digital syntheses~\cite{Sree94:TR}, and the
correctness of distributed agreement protocols for a hybrid fault
model consisting of Byzantine, symmetric, and crash
faults~\cite{Lincoln&Rushby93:CAV,Lincoln&Rushby93:FTCS,Lincoln&Rushby94:FTP}.
These harder examples can take from several days to several weeks.
Industrial applications of PVS include verification of selected
elements of a commercial avionics microprocessor whose implementation
has 500,000 transistors~\cite{Miller&Srivas95}.
Some of these applications of PVS are summarized
in~\cite{Owre95:prolegomena}, which also
motivates and describes
some of the design decisions underlying PVS\@.
Applications of PVS undertaken independently of SRI
include~\cite{Hooman94,Butler:PVS-tut,Johnson94:TPCD,Miner94:circuit}.

\subsection{Getting and Using PVS}

At the moment, PVS is readily available only for Sun SPARC
workstations running SunOS 4.1.3, although versions of the system have
been run on IBM Risc 6000 (under AIX) and DECSystem 5000 (under
Ultrix).  PVS is implemented in Common Lisp (with CLOS), and has been
ported to Lucid, Allegro, AKCL, CMULISP, and Harlequin Lisps.
Only the Lucid and Allegro versions deliver acceptable performance.
All versions of PVS require \gnuemacs, which must be obtained
separately.  It is not particular about the window system, as long as
it supports \gnuemacs, although some facilities for presenting
graphical representaitons of theory dependencies and proof trees
(implemented in Tcl/TK) do require X-Windows.  In addition, \LaTeX\
and an appropriate viewer are needed to support certain optional
features of \pvs.

PVS is quite large, requiring about 50 megabytes of disk space.  In
addition, any system on which it is to be run should have a minimum of
100 megabytes of swap space and 48 megabytes of real memory (more is
better).  To obtain the \pvs\ system, send a request to {\tt
pvs-request@csl.sri.com}, and we will provide further instructions for
obtaining a tape or for getting the system by FTP\@.  Alternatively,
you may inspect the installation instructions over WWW at URL {\tt
http://www.csl.sri.com/pvs.html}.  All installations of PVS must be
licensed by SRI\@.  The Lucid Lisp version requires that you have a
runtime license for Lucid Lisp.  A nominal distribution fee is charged
for tapes; there is no charge for obtaining PVS by FTP.

% Document Type: LaTeX
% Master File: tutorial.tex
\section{A Brief Tour of \pvs}
\label{system-tutorial}

In this section we introduce the system by developing a theory and
doing a simple proof.  This will introduce the most useful commands
and provide a glimpse into the philosophy behind \pvs.  You will get
the most out of this section if you are sitting in front of a
workstation (or terminal) with \pvs\ installed.  In the following we
assume familiarity with Sun Unix and \gnu.

Start by going to a \unix\ shell window and creating a working
directory (using {\tt mkdir}). Next, connect ({\tt cd}) to that
working directory and start up \pvs\ by typing {\tt pvs}.\footnote{You
may need to include a pathname, depending on where and how \pvs\ is
installed.} This command executes a shell script which runs \gnu,
loads the necessary \pvs\ \emacs\ extensions, and starts the \pvs\
lisp image as a subprocess.\footnote{All the \gnu\ (and X-Windows or
Emacstool) command line flags can be added to the {\tt pvs} command
and passed through as appropriate; the {\tt -q} flag inhibits loading
of the user's {\tt .emacs} initialization file, and should be used if
difficulties are encountered starting \pvs\ or if there appear to be
conflicts in keybindings.  Do {\em not\/} report errors to us unless
they can be reproduced when the {\tt -q} flag is used.} 
After a few moments, you should see the
welcome screen indicating the version of \pvs\ being run, the current
directory, and instructions for getting help.  You may be asked
whether you want to create a new context in the directory; answer {\tt
yes} unless it is the wrong directory or you don't have write
permission there, in which case you should answer {\tt no} and provide
an alternative directory when prompted.

\pvs\ uses \emacs\ as its interface by extending \emacs\ with \pvs\
functions, but all the underlying capabilities of \emacs\ are available.
Thus the user can read mail and news, edit non\pvs\ files, or execute
commands in a shell buffer in the usual way.

In the following, \pvs\ \emacs\ commands are given first in their long
form, followed by an alternative abbreviation and/or key binding in
parentheses.  For example, the command for proving in \pvs\ is given
as \ecmd{prove} (\ecmd{pr}, \key{C-c p}).  This command can be entered
by typing the {\tt Escape} key, then an {\tt x}\footnote{Many
keyboards provide a {\tt Meta} key (hence the {\tt M-} prefix), and
this may be used instead.  On the \sun 3, the {\tt Meta} key is
normally labeled {\tt Left} and on the \sun 4 ({\sc sparc}), it is
labeled $\Diamond$.  The {\tt Meta} key is like the shift key; to use
it simply hold the {\tt Meta} key down while typing another key.}
followed by {\tt prove} (or {\tt pr}) and the {\tt Return} key.
Alternatively, hold the {\tt Control} key down while typing a {\tt c},
then let go and type a {\tt p}.  The {\tt Return} key does not need to
be pressed when giving the key binding form.  In \pvs\ all commands
and abbreviations are preceded by a {\tt M-x}; everything else is a
key-binding.  In later sections we will refer to commands by their
long form name, without the {\tt M-x} prefix.  Some of the commands
prompt for a theory or \pvs\ file name and specify a default; if the
default is the desired theory or file, you can simply type the {\tt
Return} key.  Although the basic keyword commands described here are
preferred by most serious users, \pvs\ commands are also available as
menu selections if you are running under \emacs\ 19.

To begin, type \iecmd{pvs-help} ({\tt C-h p}) for an overview of the
commands available in \pvs\ (type {\tt q} to exit the help buffer).
To exit \pvs, use \iecmd{exit-pvs} (\key{C-x C-c}).

\pvs\ specifications consist of a number of files, each of which
contains one or more theories.  Theories may import other theories;
imported theories must either be part of the prelude (the standard
collection of theories built-in to PVS), or the files containing them
must be in the same directory.\footnote{\pvs\ does support soft links,
thus supporting a limited capability for reusing theories.}
Specification files in \pvs\ all have a {\tt .pvs} extension.  As
specifications are developed, their proofs are kept in files of the
same name with {\tt .prf} extensions.  The specification and proof
files in a given directory constitute a PVS {\em context\/}; \pvs\
maintains the state of a specification between sessions by means of
the {\tt .pvscontext} file.  The {\tt .pvscontext} and {\tt .prf}
files are not meant to be modified by the user.  Other files used or
created by the system will be described as needed.  You may move to a
different context (\ie\ directory) using the \ecmd{change-context}
command, which is analogous to the \unix\ {\tt cd} command.

Now let's develop a small specification:
\begin{pvsex}
sum: THEORY
 BEGIN
  n: VAR nat
  sum(n): RECURSIVE nat =
   (IF n = 0 THEN 0 ELSE n + sum(n - 1) ENDIF)
   MEASURE (LAMBDA n: n)
  closed_form: THEOREM sum(n) = (n * (n + 1))/2
 END sum
\end{pvsex}
%
This is a specification for summation of the first $n$ natural numbers

This simple theory has no parameters and contains three declarations.
The first declares {\tt n} to be a variable of type {\tt nat}, the
built-in type of natural numbers.  The next declaration is a recursive
definition of the function {\tt sum(n)}, whose value is the sum of the
first {\tt n} natural numbers.  Associated with this definition is a
{\em measure\/} function, following the {\tt MEASURE} keyword, which
will be explained below.\footnote{In this case, the measure is the
identity function, which could have been written simply as {\tt MEASURE
n}.} The final declaration is a formula which gives the closed form of
the sum.

\subsection{Creating the Specification}

The {\tt sum} theory may be introduced to the system in a number of
ways, all of which create a file with a {\tt .pvs}
extension,\footnote{The file does not have to be named {\tt sum.pvs}, it
simply needs the {\tt .pvs} extension.} which can be done by
\begin{enumerate}

\item using the {\tt M-x new-pvs-file} command (\ecmd{nf}) to create a new
\pvs\ file, and typing {\tt sum} when prompted.  Then type in the {\tt
sum} specification.

\item Since the file is included on the distribution tape in the {\tt
Examples/tutorial} subdirectory of the main \pvs\ directory, it can be
imported with the {\tt M-x import-pvs-file} command (\ecmd{imf}).  Use
the \ecmd{whereis-pvs} command to find the path of the main \pvs\
directory.

\item Finally, any external means of introducing a file with extension
{\tt .pvs} into the current directory will make it available to the
system; for example, using {\tt vi} to type it in, or {\tt cp} to copy
it from the {\tt Examples/tutorial} subdirectory.

\end{enumerate}
The first two alternatives display the specification in a buffer.
The third option requires an explicit request such as a built-in \gnu\
file command (like {\tt M-x find-file}, {\tt C-x C-f}), or the {\tt M-x
find-pvs-file} ({\tt M-x ff} or {\tt C-c C-f}) command.  The latter is
more useful when there are multiple specification files, as it supports
completion on just the specification files, ignoring other files that
you or the system have created in the directory.

\subsection{Parsing}

Once the {\tt sum} specification is displayed, it can be parsed with the
{\tt M-x parse} ({\tt M-x pa}) command, which creates the internal
abstract representation for the theory described by the specification.
If the system finds an error during parsing, an error window will pop up
with an error message, and the cursor will be placed in the vicinity of
the error.  If you didn't get an error, introduce one (say by
misspelling the {\tt VAR} keyword), then move the cursor somewhere else and
parse the file again (note that the buffer is automatically saved).  Fix
the error and parse once more.  In practice, the parse command is rarely
used, as the system automatically parses the specification when it needs
to.

\subsection{Typechecking}
\index{typecheck|(}

The next step is to typecheck the file by typing \ecmd{typecheck}
(\ecmd{tc}, \key{C-c t}), which checks for semantic errors, such as
undeclared names and ambiguous types.  Typechecking may build new files
or internal structures such as \tccs.  When {\tt sum} has been
typechecked, a message is displayed in the minibuffer indicating that
two \tccs\index{TCCs@\tccs|(} were generated.  These \tccs\ represent
{\em proof obligations\/} that must be discharged before the {\tt sum}
theory can be considered typechecked.  The proofs of the \tccs\
may be postponed indefinitely, though it is a good idea to view them to
see if they are provable.  \tccs\ can be viewed using the \ecmd{show-tccs}
command, the results of which are shown in Figure~\ref{sum-tccs} below.

\pvstheory{sum-tccs}{\tccs\ for Theory {\tt sum}}{sum-tccs}

The first \tcc\ is due to the fact that {\tt sum} takes an argument of
type {\tt nat}, but the type of the argument in the recursive call to
{\tt sum} is integer, since {\tt nat} is not closed under subtraction.
Note that the \tcc\ includes the condition {\tt NOT n = 0}, which holds
in the branch of the {\tt IF-THEN-ELSE} in which the expression
{\tt n - 1} occurs.

The second \tcc\ is needed to ensure that the function {\tt sum} is
total, \ie\ terminates.  \pvs\ does not directly support partial
functions, although its powerful subtyping mechanism allows \pvs\ to
express many operations that are traditionally regarded as partial.  The
measure function is used to show that recursive definitions are total by
requiring the measure to decrease with each recursive call.

These \tccs\ are trivial, and in fact can be discharged automatically
by using the \ecmd{typecheck-prove} (\ecmd{tcp}) command, which attempts
to prove all \tccs\ that have been generated.  (Try it).
\index{TCCs@\tccs|)}\index{typecheck|)}

\subsection{Proving}

We are now ready to try to prove the main theorem.  Place the cursor on
the line containing the {\tt closed\_form} theorem, and type
\ecmd{prove} (\ecmd{pr} or \key{C-c p}).  A new buffer will pop up, the
formula will be displayed, and the cursor will appear at the {\tt Rule?}
prompt, indicating that the user can interact with the prover.  The
commands needed to prove this theorem constitute only a very small
subset of the commands available to the prover; more details can be
found in the prover guide~\cite{PVS:prover}.

First, notice the display (reproduced below), which consists of a
single formula (labeled {\tt \{1\}}) under a dashed line.  This is a
{\em sequent\/}; formulas above the dashed lines are called {\em
antecedents\/} and those below are called {\em succedents\/}.  The
interpretation of a sequent is that the conjunction of the antecedents
implies the disjunction of the succedents.  Either or both of the
antecedents and succedents may be empty.\footnote{An empty antecedent
is equivalent to {\tt true}, and an empty succedent is equivalent to
{\tt false}, so if both are empty the sequent is unprovable.} In our
case, we are trying to prove a single succedent.

The basic objective of the proof is to generate a {\em proof tree\/} in
which all of the leaves are trivially true.  The nodes of the proof tree
are sequents, and while in the prover you will always be looking at an
unproved leaf of the tree.  The {\em current\/} branch of a proof is the
branch leading back to the root from the current sequent.  When a given
branch is complete (\ie\ ends in a true leaf), the prover automatically
moves on to the next unproved branch, or, if there are no more unproven
branches, notifies you that the proof is complete.

Now back to the proof.  We will prove this formula by induction on {\tt
n}.  To do this, type {\tt (induct "n")}.\footnote{\pvs\ expressions are
case-sensitive, and must be put in double quotes when they appear as
arguments in prover commands.} This is not an \emacs\ command, rather it
is typed directly at the prompt, including the parentheses.  This
generates two subgoals; the one displayed is the base case, where {\tt
n} is {\tt 0}.  To see the inductive step, type {\tt (postpone)}, which
postpones the current subgoal and moves on to the next unproved one.
Type {\tt (postpone)} a second time to cycle back to the original
subgoal (labeled {\tt closed\_form.1}).\footnote{Three extremely useful
\emacs\ key sequences to know here are \key{M-p}, \key{M-n}, and
\key{M-s}.  \key{M-p} gets the last input typed to the prover; further
uses of \key{M-p} cycle back in the input history.  \key{M-n} works in
the opposite direction.  To use \key{M-s}, type the beginning of a
command that was previously input, and type \key{M-s}.  This will get
the previous input that matches the partial input; further uses of
\key{M-s} will find earlier matches.  Try these key sequences out; they
are easier to use than to explain.}

To prove the base case, we need to expand the definition of {\tt sum},
which is done by typing {\tt (expand "sum")}.  After expanding the
definition of {\tt sum}, we send the proof to the \pvs\ decision
procedures, which automatically decide certain fragments of
arithmetic, by typing {\tt (assert)}.\footnote{The {\tt
assert} command actually does a lot more than decide arithmetical
formulas, performing three basic tasks:
\begin{itemize}\def\itemsep{0in}
\item it tries to prove the subgoal using the decision procedures.

\item it stores the subgoal information in an underlying database,
allowing automatic use to be made of it later.

\item it simplifies the subgoal, again utilizing the underlying decision
procedures.
\end{itemize}
These arithmetic and equality procedures are the main workhorses to
most \pvs\ proofs.  You should learn to use them effectively in a
proof.} 
This completes the proof of this subgoal, and the system moves on to
the next subgoal, which is the inductive step.

The first thing to do here is to eliminate the {\tt FORALL} quantifier.
This can most easily be done with the {\tt skolem!}\
command\footnote{The exclamation point differentiates this command from
the {\tt skolem} command, where the new constants have to be provided by
the user.}, which provides new constants for the bound variables.  To
invoke this command type {\tt (skolem!)} at the prompt.  The resulting
formula may be simplified by typing {\tt (flatten)}, which will break up
the succedent into a new antecedent and succedent.  The obvious thing to
do now is to expand the definition of {\tt sum} in the succedent.  This
again is done with the {\tt expand} command, but this time we want to
control where it is expanded, as expanding it in the antecedent will not
help.  So we type {\tt (expand "sum" +)}, indicating that we want to
expand {\tt sum} in the succedent.\footnote{We could also have specified
the exact formula number (here {\tt 1}), but including formula numbers
in a proof tends to make it less robust in the face of changes.  There
is more discussion of this in the prover guide~\cite{PVS:prover}.}

The final step is to send the proof to the \pvs\ decision procedures
by typing {\tt (assert)}.  The proof is now complete, the system may
ask whether to save the new proof, and whether to display a brief
printout of the proof.  You should answer {\tt yes} to these questions
just to see how they work.  After responding to these questions, the
buffer from which the \cmd{prove} command was issued is redisplayed if
necessary, and the cursor is placed on the formula that was just
proved.  The entire proof transcript is shown below.  Yours may be
different, depending on your window size and the timings involved.

{\smaller\smaller\smaller\begin{alltt}
     closed_form :  
     
       |-------
     \{1\}   (FORALL (n: nat): sum(n) = (n * (n + 1)) / 2)
     
     Rule? {\bf (induct "n")}
     Inducting on n,
     this yields  2 subgoals: 
     closed_form.1 :  
     
       |-------
     \{1\}   sum(0) = (0 * (0 + 1)) / 2
     
     Rule? {\bf (postpone)}
     Postponing closed_form.1.
     
     closed_form.2 :  
     
       |-------
     \{1\}   (FORALL (j: nat):
              sum(j) = (j * (j + 1)) / 2
                IMPLIES sum(j + 1) = ((j + 1) * (j + 1 + 1)) / 2)
     
     Rule? {\bf (postpone)}
     Postponing closed_form.2.
     
     closed_form.1 :  
     
       |-------
     \{1\}   sum(0) = (0 * (0 + 1)) / 2
     
     Rule? {\bf (expand "sum")}
     (IF 0 = 0 THEN 0 ELSE 0 + sum(0 - 1) ENDIF) 
     simplifies to 0
     Expanding the definition of sum,
     this simplifies to: 
     closed_form.1 :  
     
       |-------
     \{1\}   0 = 0 / 2
     
     Rule? {\bf (assert)}
     Simplifying, rewriting, and recording with decision procedures,
     
     This completes the proof of closed_form.1.
     
     closed_form.2 :  
     
       |-------
     \{1\}   (FORALL (j: nat):
              sum(j) = (j * (j + 1)) / 2
                IMPLIES sum(j + 1) = ((j + 1) * (j + 1 + 1)) / 2)
     
     Rule? {\bf (skolem!)}
     Skolemizing,
     this simplifies to: 
     closed_form.2  |-------
     \{1\}   sum(j!1) = (j!1 * (j!1 + 1)) / 2
             IMPLIES sum(j!1 + 1) = ((j!1 + 1) * (j!1 + 1 + 1)) / 2
     
     Rule? {\bf (flatten)}
     Applying disjunctive simplification to flatten sequent,
     this simplifies to: 
     closed_form.2 :  
     
     \{-1\}   sum(j!1) = (j!1 * (j!1 + 1)) / 2
       |-------
     \{1\}   sum(j!1 + 1) = ((j!1 + 1) * (j!1 + 1 + 1)) / 2
     
     Rule? {\bf (expand "sum" +)}
     (IF j!1 + 1 = 0 THEN 0 ELSE j!1 + 1 + sum(j!1 + 1 - 1) ENDIF) 
     simplifies to 1 + sum(j!1) + j!1
     Expanding the definition of sum,
     this simplifies to: 
     closed_form.2 :  
     
     [-1]   sum(j!1) = (j!1 * (j!1 + 1)) / 2
       |-------
     \{1\}   1 + sum(j!1) + j!1 = (2 + j!1 + (j!1 * j!1 + 2 * j!1)) / 2
     
     Rule? {\bf (assert)}
     Simplifying, rewriting, and recording with decision procedures,
     
     This completes the proof of closed_form.2.
     
     Q.E.D.
     
     
     Run time  = 5.62 secs.
     Real time = 58.95 secs.

\end{alltt}}

Note: The proof presented here is a low-level interactive one chosen
for illustrative purposes.  In practice, trivial theorems such as this
are handled automatically by the higher-level strategies of PVS.  This
particular theorem, for example, is proved automatically by the single
command {\tt (induct-and-simplify "n" :defs T)}.


\subsection{Status}

Now type \iecmd{status-proof-theory} (\ecmd{spt}) and you will see a
buffer which displays the formulas in {\tt sum} (including the \tccs),
along with an indication of their proof status.  This command is
useful to see which formulas and \tccs\ still require proofs.  Another
useful command is \iecmd{status-proofchain} (\ecmd{spc}), which
analyzes a given proof to determine its dependencies.  To use this,
go to the {\tt sum.pvs} buffer, place the cursor on the {\tt
closed\_form} theorem, and enter the command.  A buffer will pop up
indicating whether the proof is complete, and that it depends on the
\tccs\ and the {\tt nat\_induction} axiom.

\subsection[Generating \LaTeX]{Generating \BoldLaTeX}

In order to try out this section, you must have access to \LaTeX\ and a
\TeX\ previewer, such as {\tt vitex} or {\tt dvitool} (for \sunview), or
{\tt xdvi} (for X-windows).  Otherwise this section may be skipped.

Type \iecmd{latex-theory-view} (\ecmd{ltv}).  You will be prompted for
the theory name---type {\tt sum}, or just {\tt Return} if {\tt sum} is
the default.  You will then be prompted for the \TeX\ previewer name.
Either the previewer must be in your path, or the entire pathname must
be given.  This information will only be prompted for once per session,
after that \pvs\ assumes that you want to use the same previewer.

\begin{figure}[ht]
\begin{center}
\begin{boxedminipage}{\textwidth}
{\smaller\smaller% The following substitutions are from the file:
%   /homes/helium/pvs/pvs-tex.sub
\def\membertwofn#1#2{{#1 \in #2}}% How to print function member with arity (2)
 \begin{program} 
 \pvsid{sum} :\mbox{ } \pvskey{THEORY} \\
\zi \pvskey{BEGIN} \\
 \\[-0.2in]
 n :\mbox{ } \pvskey{VAR\mbox{ }} \pvsid{nat} \\
 \\[-0.2in]
 \pvsid{sum}(\ii n) :\mbox{ } \pvskey{RECURSIVE\mbox{ }} \pvsid{nat } \mbox{ }= \\
\oo\zi\zi( \pvskey{IF\mbox{ }} n \mbox{ }=\mbox{ } 0 \pvskey{\mbox{ }THEN\mbox{ }} 0 \pvskey{\mbox{ }ELSE\mbox{ }} n \mbox{ }+\mbox{ } \pvsid{sum}(\ii n \mbox{ }-\mbox{ } 1) \pvskey{\mbox{ }ENDIF}) \\
\oo\zi\zi\zi\zo\zo \pvskey{MEASURE\mbox{ }}( \lambda\mbox{ }\ii n :\mbox{ } n) \\
\oo\zo\zo\zo \\[-0.2in]
 \pvsid{closed\_form} :\mbox{ } \pvskey{THEOREM\mbox{ }} \pvsid{sum}(\ii n) \mbox{ }=\mbox{ }( n \mbox{ }\times\mbox{ }( n \mbox{ }+\mbox{ } 1)) \mbox{ }/\mbox{ } 2 \\
\oo\zi\zi\zo\zo \\[-0.2in]
 \pvskey{END\mbox{ }} \pvsid{sum}
 \end{program}
}
\end{boxedminipage}
\end{center}
\caption{Theory {\tt sum}}\label{sum-plain}
\end{figure}

After a few moments the previewer will pop up displaying the {\tt sum}
theory, as shown in Figure~\ref{sum-plain}.  Note that {\tt LAMBDA}
has been translated as $\lambda$.  This and other translations are
built into \pvs; the user may also specify translations for keywords
and identifiers (and override those built-in) by providing a
substitution file, {\tt pvs-tex.sub}, which contains commands to
customize the \LaTeX\ output.  For example, if the substitution file
contains the three lines

{\smaller\smaller\begin{alltt}
    THEORY key 7 \verb|{\large\bf Theory}|
    sum    1   2 \verb|{\sum_{i = 0}^{#1} i}|
\end{alltt}}
the output will look like Figure~\ref{sum-sub}.

\begin{figure}[ht]
\begin{center}
\begin{boxedminipage}{\textwidth}
{\smaller\smaller% The following substitutions are from the file:
%   /amber/homes/rushby/tadpole/pvs-tex.sub
\def\sumonefn#1{{\sum_{i=0}^{#1} i}}% How to print function sum with arity (1)
\def\removetwofn#1#2{{#2 \setminus \{#1\}}}% How to print function remove with arity (2)
\def\addtwofn#1#2{{\{#1\} \cup #2}}% How to print function add with arity (2)
\def\emptysetoneid#1{{\emptyset_#1}}% How to print name emptyset with (1) actuals
\def\singletononefn#1{{\{ #1 \}}}% How to print function singleton with arity (1)
\def\uniontwofn#1#2{{#1 \cup #2}}% How to print function union with arity (2)
\def\differencetwofn#1#2{{#1 \setminus #2}}% How to print function difference with arity (2)
\setlength{\fboxsep}{1pt} 
% The following substitutions are from the file:
%   /homes/EHDM/pvs/pvs-tex.sub
\def\membertwofn#1#2{{#1 \in #2}}% How to print function member with arity (2)
\setlength{\fboxsep}{1pt} 
 \begin{program} 
 \pvsid{sum} :\mbox{ } \mbox{\large\bf Theory} \pvsnewline{}
\zi\zi \pvskey{BEGIN} \pvsnewline{}
 \pvsnewline{}
 n :\mbox{ } \pvskey{VAR\mbox{ }} \pvsid{nat} \pvsnewline{}
 \pvsnewline{}
 \sumonefn { n } :\mbox{ } \pvskey{RECURSIVE\mbox{ }} \pvsid{nat } \mbox{ }=\mbox{ }(\ii \pvskey{IF\mbox{ }} n \mbox{ }=\mbox{ }\ii 0 \pvskey{\mbox{ }THEN\mbox{ }} 0 \pvskey{\mbox{ }ELSE\mbox{ }} n \mbox{ }+\mbox{ }\ii \sumonefn { n \mbox{ }-\mbox{ } 1 } \pvskey{\mbox{ }ENDIF}) \pvsnewline{}
\\[-\baselineskip]\oo\oo\oo\zi\zi\zi\zi\zi\zo\zo\zo \pvskey{MEASURE\mbox{ }}(\ii \lambda\mbox{ } n :\mbox{ } n) \pvsnewline{}
\\[-\baselineskip]\oo\zi\zo\zo\zo \pvsnewline{}
 \pvsid{closed\_form} :\mbox{ } \pvskey{THEOREM\mbox{ }} \sumonefn { n } \mbox{ }=\mbox{ }\ii(\ii n \mbox{ }\times\mbox{ }\ii(\ii n \mbox{ }+\mbox{ }\ii 1)) \mbox{ }/\mbox{ }\ii 2 \pvsnewline{}
\\[-\baselineskip]\oo\oo\oo\oo\oo\oo\zi\zi\zi\zi\zi\zo\zo\zi\zo\zo\zo\zo \pvsnewline{}
 \pvskey{END\mbox{ }} \pvsid{sum} \pvsnewline{}
\zo \end{program}}
\end{boxedminipage}
\end{center}
\caption{Theory {\tt sum}}\label{sum-sub}
\end{figure}

Finally, using the \iecmd{latex-proof} command, it is possible to
generate a \LaTeX\ file from a proof.  A part of an example is shown
below; details are in the PVS system manual.

\noindent
\begin{boxedminipage}{\linewidth}
\def\sumonefn#1{{\sum_{i = 0}^{#1} i}}

Expanding the definition of sum

{\tt closed\_form.2:}

\vspace*{0.2in}\hspace*{0.2in}
\begin{tabular}{ll}
$\{-1\}$ &\begin{minipage}[t]{6in}{ \begin{program} 
 \sumonefn { j^{\prime} } \mbox{ }=\mbox{ }(\ii j^{\prime} \mbox{ }\times\mbox{ }(\ii j^{\prime} \mbox{ }+\mbox{ } 1)) \mbox{ }/\mbox{ } 2 \\ 
\oo\oo\zi\zi\zi\zi\zo\zo \end{program}}\end{minipage}\\\hline
$\{1\}$ &\begin{minipage}[t]{6in}{ \begin{program} 
(\ii \pvskey{IF\mbox{ }} j^{\prime} \mbox{ }+\mbox{ } 1 \mbox{ }=\mbox{ } 0 \pvskey{\mbox{ }THEN\mbox{ }} 0 \pvskey{\mbox{ }ELSE\mbox{ }} j^{\prime} \mbox{ }+\mbox{ } 1 \mbox{ }+\mbox{ } \sumonefn { j^{\prime} \mbox{ }+\mbox{ } 1 \mbox{ }-\mbox{ } 1 } \pvskey{\mbox{ }ENDIF}) \\
\oo\zi \mbox{ }=\mbox{ }(\ii(\ii j^{\prime} \mbox{ }+\mbox{ } 1) \mbox{ }\times\mbox{ }(\ii j^{\prime} \mbox{ }+\mbox{ } 1 \mbox{ }+\mbox{ } 1)) \mbox{ }/\mbox{ } 2 \\ 
\oo\oo\oo\zi\zi\zi\zo\zo \end{program}}\end{minipage}\\
\end{tabular}
\end{boxedminipage}

% Document Type: LaTeX
% Master File: language.tex
\documentclass[12pt]{book}
\usepackage{alltt}
\usepackage{makeidx}
\usepackage{relsize}
\usepackage{boxedminipage}
\usepackage{url}
\usepackage{../../pvs}
\usepackage{../makebnf}
\usepackage[chapter]{tocbibind}
\usepackage{fancyvrb}
\usepackage[dvipsnames,usenames]{color}

\usepackage{amssymb}
\usepackage{mathpazo}
\usepackage{fontspec}
\setmainfont[Ligatures=TeX]{XITS}
\setmonofont{DejaVu Sans Mono}[Scale=MatchLowercase]
%\setmonofont{Free Mono}[Scale=0.8]
\usepackage[math-style=ISO]{unicode-math}
\renewcommand{\leadsto}{\rightsquigarrow}
%\setmathfont{XITS Math}

\topmargin -10pt
\textheight 8.5in
\textwidth 6.0in
\headheight 15 pt
\columnwidth \textwidth
\oddsidemargin 0.5in
\evensidemargin 0.5in   % fool system for page 0
\setcounter{topnumber}{9}
\renewcommand{\topfraction}{.99}
\setcounter{bottomnumber}{9}
\renewcommand{\bottomfraction}{.99}
\setcounter{totalnumber}{10}
\renewcommand{\textfraction}{.5}
\renewcommand{\floatpagefraction}{.1}
\usepackage{fancyhdr}
\pagestyle{fancy}
\raggedbottom

%\setcounter{secnumdepth}{1}

\index{type correctness condition|see{TCC}}
\makeindex

\usepackage[bookmarks=true,hyperindex=true,colorlinks=true,linkcolor=Brown,citecolor=blue,backref=page,pagebackref=true,plainpages=false,pdfpagelabels]{hyperref}

%% Derived from John Rushby's prelude.tex, modified for NFSS2
%
% define variants of the \LaTeX macro that avoid using \sc
% for use in headings
%

% Define fonts that work in math or text mode
\def\dwimrm#1{\ifmmode\mathrm{#1}\else\textrm{#1}\fi}
\def\dwimsf#1{\ifmmode\mathsf{#1}\else\textsf{#1}\fi}
\def\dwimtt#1{\ifmmode\mathtt{#1}\else\texttt{#1}\fi}
\def\dwimbf#1{\ifmmode\mathbf{#1}\else\textbf{#1}\fi}
\def\dwimit#1{\ifmmode\mathit{#1}\else\textit{#1}\fi}
\def\dwimnormal#1{\ifmmode\mathnormal{#1}\else\textnormal{#1}\fi}

\def\BigLaTeX{{\rm L\kern-.36em\raise.3ex\hbox{\small\small A}\kern-.15em
    T\kern-.1667em\lower.7ex\hbox{E}\kern-.125emX}}
\def\BoldLaTeX{{\bf L\kern-.36em\raise.3ex\hbox{\small\small\bf A}\kern-.15em
    T\kern-.1667em\lower.7ex\hbox{E}\kern-.125emX}}
%\def\labelitemi{$\bullet$}
\def\labelitemii{$\circ$}
\def\labelitemiii{$\star$}
\def\labelitemiv{$\diamond$}
\newcommand{\tcc}{{\small\small TCC}}
\newcommand{\tccs}{\tcc s}
\newcommand{\emacs}{{Emacs}}
\newcommand{\Emacs}{\emacs}
\newcommand{\ehdm}{{E{\small\small HDM}}}
\newcommand{\Ehdm}{\ehdm}
\newcommand{\tm}{$^{\mbox{\tiny TM}}$}
\newcommand{\hozline}{{\noindent\rule{\textwidth}{0.4mm}}}

\newcommand{\allclear}%
  {\mbox{\boldmath$\stackrel{\raisebox{-.2ex}[0pt][0pt]%
              {$\textstyle\oslash$}}{\displaystyle\bot}$}}

\newenvironment{private}{}{}

\newenvironment{smalltt}{\begin{alltt}\small}{\end{alltt}}

\newlength{\hsbw}

\newenvironment{session}%
  {\begin{flushleft}
   \setlength{\hsbw}{\linewidth}
   \addtolength{\hsbw}{-\arrayrulewidth}
   \addtolength{\hsbw}{-\tabcolsep}
   \begin{tabular}{@{}|c@{}|@{}}\hline 
   \begin{minipage}[b]{\hsbw}
   \begingroup\small\mbox{ }\\[-1.8\baselineskip]\begin{alltt}}
  {\end{alltt}\endgroup\end{minipage}\\ \hline 
   \end{tabular}
   \end{flushleft}}

\newenvironment{smallsession}%
  {\begin{flushleft}
   \setlength{\hsbw}{\linewidth}
   \addtolength{\hsbw}{-\arrayrulewidth}
   \addtolength{\hsbw}{-\tabcolsep}
   \begin{tabular}{@{}|c@{}|@{}}\hline 
   \begin{minipage}[b]{\hsbw}
   \begingroup\footnotesize\mbox{ }\\[-1.8\baselineskip]\begin{alltt}}%
  {\end{alltt}\endgroup\end{minipage}\\ \hline 
   \end{tabular}
   \end{flushleft}}

\newenvironment{spec}%
  {\begin{flushleft}
   \setlength{\hsbw}{\textwidth}
   \addtolength{\hsbw}{-\arrayrulewidth}
   \addtolength{\hsbw}{-\tabcolsep}
   \begin{tabular}{@{}|c@{}|@{}}\hline 
   \begin{minipage}[b]{\hsbw}
   \begingroup\small\mbox{ }\\[-0.2\baselineskip]}%
  {\endgroup\end{minipage}\\ \hline 
   \end{tabular}
   \end{flushleft}}

\newcommand{\memo}[1]%
  {\mbox{}\par\vspace{0.25in}%
   \setlength{\hsbw}{\linewidth}\addtolength{\hsbw}{-1.5ex}%
   \noindent\fbox{\parbox{\hsbw}{{\bf Memo: }#1}}\vspace{0.25in}}

\newcommand{\nb}[1]%
  {\mbox{}\par\vspace{0.25in}%
   \setlength{\hsbw}{\linewidth}\addtolength{\hsbw}{-1.5ex}%
   \noindent\fbox{\parbox{\hsbw}{{\bf Note: }#1}}\vspace{0.25in}}

\newcommand{\comment}[1]{}
\newcommand{\exfootnote}[1]{}
%\newcommand{\ifelse}[2]{#1}
\sloppy
\clubpenalty=100000
\widowpenalty=100000
%\displaywidowpenalty=100000
\setcounter{secnumdepth}{3} 
\setcounter{tocdepth}{3}
\setcounter{topnumber}{9}
\setcounter{bottomnumber}{9}
\setcounter{totalnumber}{9}
\renewcommand{\topfraction}{.99}
\renewcommand{\bottomfraction}{.99}
\renewcommand{\floatpagefraction}{.01}
\renewcommand{\textfraction}{.2}
\font\largett=cmtt10 scaled\magstep1
\font\Largett=cmtt10 scaled\magstep2
\font\hugett=cmtt10 scaled\magstep3

\def\labelitemii{$\circ$}
\def\labelitemiii{$\star$}
\def\labelitemiv{$\diamond$}
\newcommand{\tcc}{{\small\small TCC}}
\newcommand{\tccs}{\tcc s}

%\renewcommand{\memo}[1]{\mbox{}\par\vspace{0.25in}\noindent\fbox{\parbox{.95\linewidth}{{\bf Memo: }#1}}\vspace{0.25in}}

\newcommand{\eg}{{\em e.g.\/},}
\newcommand{\ie}{{\em i.e.\/},}

\newcommand{\pvs}{PVS}

\newcommand{\ch}{\choice}
\newcommand{\rsv}[1]{{\rm\tt #1}}

\newcommand{\lpvstheory}[3]{\figurehead{\hozline\smaller\smaller\begin{alltt}}%
                           \figuretail{\end{alltt}\vspace{-0in}\hozline}%
                           \figurelabel{#3}\figurecap{#2}%
                           \begin{longfigure}\input{#1}\end{longfigure}}

\newcommand{\pvstheory}[3]
  {\begin{figure}[htb]\begin{boxedminipage}{\textwidth}%
   {\smaller\smaller\begin{alltt} \input{#1}\end{alltt}}\end{boxedminipage}%
   \caption{#2}\label{#3}\end{figure}}

\newcommand{\bpvstheory}[3]
  {\begin{figure}[b]\begin{boxedminipage}{\textwidth}%
   {\smaller\smaller\begin{alltt} \input{#1}\end{alltt}}\end{boxedminipage}%
   \caption{#2}\label{#3}\end{figure}}

\newcommand{\spvstheory}[1]
  {\vspace{0.1in}\par\noindent\begin{boxedminipage}{\textwidth}%
   {\smaller\smaller\begin{alltt} \input{#1}\end{alltt}}\end{boxedminipage}\vspace{0.1in}%
   }
%\newenvironment{spvstext}%
%  {\vspace{0.1in}\par\noindent\begin{boxedminipage}{\textwidth}%
%   \smaller\smaller\begin{alltt}}%
%  {\end{alltt}\end{boxedminipage}\vspace{0.1in}%
%   }


%  {\begin{boxedminipage}{\textwidth}{\smaller\smaller\begin{alltt}#1\end{alltt}}\end{boxedminipage}}
%\newenvironment{spvstheory}{\par\noindent\begin{boxedminipage}{\textwidth}\smaller\smaller\begin{alltt}}{\end{alltt}\end{boxedminipage}}

\newenvironment{pvsex}%
  {\setlength{\topsep}{0in}\smaller\begin{alltt}}%
  {\end{alltt}}

\newcommand{\pvsbnf}[2]
  {\begin{figure}[htb]\begin{boxedminipage}{\textwidth}%
   \input{#1}\end{boxedminipage}\caption{#2}\label{#1}\end{figure}}

\newcommand{\spvsbnf}[1]
  {\begin{boxedminipage}{\textwidth}\input{#1}\end{boxedminipage}}

\newcommand{\pidx}[1]{{\rm #1}} % primary index entry
\newcommand{\sidx}[1]{{\rm #1}} % secondary index entry
\newcommand{\cmdindex}[1]{\index{#1@\cmd{#1}}}
\newcommand{\icmd}[1]{\cmd{#1}\cmdindex{#1}}
\newcommand{\iecmd}[1]{\ecmd{#1}\cmdindex{#1}}
\newcommand{\buf}[1]{\texttt{#1}}
\newcommand{\ibuf}[1]{\buf{#1}\index{#1 buffer@\buf{#1} buffer}\index{buffers!\buf{#1}}}

\newenvironment{pvscmds}%
  {\par\noindent\smaller%
   \begin{tabular*}{\textwidth}{|l@{\extracolsep{\fill}}l@{\extracolsep{\fill}}l|}\hline%
     {\it Command} & {\it Aliases} & {\it Function}\\ \hline}%
  {\hline\end{tabular*}\vspace{0.1in}}

\newenvironment{pvscmdsna}%
  {\par\noindent\smaller%
   \begin{tabular*}{\textwidth}{|l@{\extracolsep{\fill}}l|}\hline%
     {\it Command} & {\it \,\,Function}\\ \hline}%
  {\hline\end{tabular*}\vspace{0.1in}}

\newcommand{\cmd}[1]{{\tt #1}}
\newcommand{\ecmd}[1]{{\tt M-x #1}}

\newcommand{\latex}{\LaTeX}                  %  LaTeX
\newcommand{\sun}{{S{\smaller\smaller UN}}}                 %  Sun
\newcommand{\sparc}{{S{\smaller\smaller PARC}}}             %  Sparc
\newcommand{\sunos}{{S{\smaller\smaller UN}OS}}             %  SunOS
\newcommand{\solaris}{{\em Solaris\/}}        %  Solaris
\newcommand{\sunview}{{S{\smaller\smaller UN}V{\smaller\smaller IEW}}} %SunView
\newcommand{\unix}{{U{\smaller\smaller NIX}}}               %  Unix
\newcommand{\lisp} {{\sc Lisp}}              %  Lisp
\newcommand{\gnu}{{Gnu Emacs}}           %  Gnu Emacs
\newcommand{\gnuemacs}{{Gnu Emacs}}      %  Gnu Emacs
\newcommand{\emacsl}{{Emacs-Lisp}}       %  Emacs Lisp
\newcommand{\shell}{{\sc Csh}}               %  C-shell

\newcommand{\update}[3]{#1\{#2\leftarrow #3\}}
\newcommand{\interp}[3]{\cal{M}\dlb {\tt #1 : #2 }\drb #3}
\newcommand{\myforall}[2]{(\forall{#1 .}\ #2)}
\newcommand{\myexists}[2]{(\exists{#1 .}\ #2)}
\newcommand{\mth}[1]{$ #1 $}
\newcommand{\labst}[2]{(\lambda{#1}.\ #2)}
\newcommand{\app}[2]{(#1\ #2)}
\newcommand{\problem}[1]{{\bf Exercise: } {\em #1}}
\newcommand{\rectype}[1]{[\# 1 \#]}
\newcommand{\recttype}[1]{{\tt [\# 1 \#]}}
\newcommand{\dlb}{\lbrack\!\lbrack}
\newcommand{\drb}{\rbrack\!\rbrack}
\newcommand{\cross}{\times}
\newcommand{\key}[1]{{\tt #1}}
\newcommand{\keyindex}[1]{\index{#1@\key{#1}}}
\newcommand{\ikey}[1]{\key{#1}\keyindex{#1}}
\newcommand{\keyword}[1]{{\smaller\texttt{#1}}}

\newenvironment{keybindings}%
  {\begin{center}\begin{tabular}{|l|l|}\hline Key & Function\\ \hline}%
  {\hline\end{tabular}\end{center}}
\def\rmif{\mbox{\bf if\ }}
\def\rmiff{\mbox{\bf \ iff \ }}
\def\rmthen{\mbox{\bf \ then }}
\def\rmelse{\mbox{\bf \ else }}
\def\rmend{\mbox{\bf end}}
\def\rmendif{\mbox{\bf \ endif}}
\def\rmotherwise{\mbox{\bf otherwise}}
\def\rmwith{\mbox{\bf \ with\ }}
\def\mapb{\char"7B\char"7B}
\def\mape{\char"7D\char"7D}
\def\setb{\char"7B}
\def\sete{\char"7D}

% ---------------------------------------------------------------------
% Macros for little PVS sessions displayed in boxes.
%
% Usage: (1) \setcounter{sessioncount}{1} resets the session counter
%
%        (2) \begin{session*}\label{thissession}
%             .
%              < lines from PVS session >
%             .
%            \end{session*}
%
%            typesets the session in a numbered box in ALLTT mode.
%
%  session instead of session* produces unnumbered boxes
%
%  Author: John Rushby
% ---------------------------------------------------------------------
\newlength{\hsbw}
\newenvironment{session}{\begin{flushleft}
 \setlength{\hsbw}{\linewidth}
 \addtolength{\hsbw}{-\arrayrulewidth}
 \addtolength{\hsbw}{-\tabcolsep}
 \begin{tabular}{@{}|c@{}|@{}}\hline 
 \begin{minipage}[b]{\hsbw}
% \begingroup\small\mbox{ }\\[-1.8\baselineskip]\begin{alltt}}{\end{alltt}\endgroup\end{minipage}\\ \hline
 \begingroup\sessionsize\vspace*{1.2ex}\begin{alltt}}{\end{alltt}\endgroup\end{minipage}\\ \hline
 \end{tabular}
 \end{flushleft}}
\newcounter{sessioncount}
\setcounter{sessioncount}{0}
\newenvironment{session*}{\begin{flushleft}
 \refstepcounter{sessioncount}
 \setlength{\hsbw}{\linewidth}
 \addtolength{\hsbw}{-\arrayrulewidth}
 \addtolength{\hsbw}{-\tabcolsep}
 \begin{tabular}{@{}|c@{}|@{}}\hline 
 \begin{minipage}[b]{\hsbw}
 \vspace*{-.5pt}
 \begin{flushright}
 \rule{0.01in}{.15in}\rule{0.3in}{0.01in}\hspace{-0.35in}
 \raisebox{0.04in}{\makebox[0.3in][c]{\footnotesize \thesessioncount}}
 \end{flushright}
 \vspace*{-.57in}
 \begingroup\small\vspace*{1.0ex}\begin{alltt}}{\end{alltt}\endgroup\end{minipage}\\ \hline 
 \end{tabular}
 \end{flushleft}}
\def\sessionsize{\footnotesize}
\def\smallsessionsize{\small}
\newenvironment{smallsession}{\begin{flushleft}
 \setlength{\hsbw}{\linewidth}
 \addtolength{\hsbw}{-\arrayrulewidth}
 \addtolength{\hsbw}{-\tabcolsep}
 \begin{tabular}{@{}|c@{}|@{}}\hline 
 \begin{minipage}[b]{\hsbw}
 \begingroup\smallsessionsize\mbox{ }\\[-1.8\baselineskip]\begin{alltt}}{\end{alltt}\endgroup\end{minipage}\\ \hline 
 \end{tabular}
 \end{flushleft}}
\newenvironment{spec}{\begin{flushleft}
 \setlength{\hsbw}{\textwidth}
 \addtolength{\hsbw}{-\arrayrulewidth}
 \addtolength{\hsbw}{-\tabcolsep}
 \begin{tabular}{@{}|c@{}|@{}}\hline 
 \begin{minipage}[b]{\hsbw}
 \begingroup\small\mbox{
}\\[-0.2\baselineskip]}{\endgroup\end{minipage}\\ \hline 
 \end{tabular}
 \end{flushleft}}
\newcommand{\memo}[1]{\mbox{}\par\vspace{0.25in}%
\setlength{\hsbw}{\linewidth}%
\addtolength{\hsbw}{-2\fboxsep}%
\addtolength{\hsbw}{-2\fboxrule}%
\noindent\fbox{\parbox{\hsbw}{{\bf Memo: }#1}}\vspace{0.25in}}
\newcommand{\nb}[1]{\mbox{}\par\vspace{0.25in}\setlength{\hsbw}{\linewidth}\addtolength{\hsbw}{-1.5ex}\noindent\fbox{\parbox{\hsbw}{{\bf Note: }#1}}\vspace{0.25in}}

%%% Local Variables: 
%%% mode: latex
%%% TeX-master: t
%%% End: 


\begin{document}

\begin{titlepage}
\renewcommand{\thepage}{title}
\vspace*{1in}
\noindent
\rule[1pt]{\textwidth}{2pt}
\begin{center}
\newfont{\pvstitle}{cmss17 scaled \magstep4}
\textbf{\pvstitle PVS Language Reference}
\end{center}
\begin{flushright}
{\Large Version 7.1 {\smaller$\bullet$} August 2020}
\end{flushright}
\rule[1in]{\textwidth}{2pt}
\vspace*{2in}
\begin{flushleft}
S.~Owre\\
N.~Shankar\\
J.~M.~Rushby\\
D.~W.~J.~Stringer-Calvert\\
{\smaller\url{{Owre,Shankar,Rushby,Dave_SC}@csl.sri.com}}\\
{\smaller\url{http://pvs.csl.sri.com/}}
\end{flushleft}
\vspace*{1in}
\vbox{\hbox to \textwidth{{\Large SRI International\hfill}}%
\hbox to \textwidth{{\small\sf%
Computer Science Laboratory $\bullet$ 333 Ravenswood Avenue $\bullet$ Menlo Park CA 94025\hfil}}}
\end{titlepage}

\renewcommand{\chaptermark}[1]{\markboth{{\em #1}}{}\markright{{\em #1}}}
\renewcommand{\sectionmark}[1]{\markright{\thesection \em \ #1}}
%\lhead[\thepage]{\rightmark}
%\cfoot{\protect\small\bf \fbox{PVS 2.3 DRAFT}}
%\cfoot{}
%\rhead[\leftmark]{\thepage}
\thispagestyle{empty}

\newpage
\renewcommand{\thepage}{ack}

\noindent\textbf{NOTE:} This manual is in the process of being updated.
Almost everything stated here is still correct, but incomplete due to the
many new features that have been introduced into PVS over the years.  The
release notes should be consulted for the most current information.

\vspace*{6in}\noindent
The initial development of PVS was funded by SRI International.
Subsequent enhancements were partially funded by SRI and by NASA
Contracts NAS1-18969 and NAS1-20334, NRL Contract N00014-96-C-2106,
NSF Grants CCR-9300044, CCR-9509931, and CCR-9712383, AFOSR contract
F49620-95-C0044, and DARPA Orders E276, A721, D431, D855, and E301.
\newpage
\pagenumbering{roman}
\setcounter{page}{1}

\tableofcontents
%\listoffigures

%\chapter{The PVS Specification Language}

%% Master File: language.tex
\addcontentsline{toc}{chapter}{\protect\numberline{}Preface}
\vspace{4in}
{\Huge\bf Preface}\linebreak
\vspace{.75in}

%\chapter{Preface}

This report presents a description of the \pvs\ specification language,
as implemented in Version 1.0 beta of the \pvs\ specification and
verification environment.  It is intended to provide a reference of all
of the features of the language, including the complete grammar, some
examples, and an informal semantics. This report is one of several
needed to effectively use \pvs.  Companion documents are devoted to the
use of the system~\cite{PVS:userguide}, the user of the
prover~\cite{PVS:prover}, a tutorial introduction~\cite{PVS:tutorial},
and a semantics~\cite{PVS:semantics}.

\memo{Give prerequisites to using \pvs.}

The \pvs\ system is the culmination of the effort of a large number of
people over many years, drawing heavily from the research and experience
gained from E{\sc
hdm}~\cite{EHDM:Userguide,EHDM:Language,EHDM:semantics,EHDM:supplement,EHDM:tutorial}.
The primary contributers to E{\sc hdm} in rough chronological order
were Michael Melliar-Smith, Richard Schwartz, Rob Shostak, Judith Crow,
Friedrich von Henke, Stan Jefferson, Rosanna Lee, John Rushby, Mark
Stickel, Natarajan Shankar, Sam Owre, David Cyrluk, Steven Phillips,
and Carl Witty.
%In addition to those named above, valuable contributions were made by
%Dorothy Denning, Brian Fromme, Allen van Gelder, Dwight Hare, Peter
%Ladkin, Sheralyn Listgarten, Jeff Miner, Paul Oppenheimer, Jeff
%Reninger, and Lorna Shinkle.
\pvs\ is primarily the work of John Rushby, Natarajan Shankar, Sam Owre,
Friedrich von Henke, David Cyrluk, and Carl Witty.

The present version of the \pvs\ Language Description was assembled by
Sam Owre, Natarajan Shankar, and John Rushby.



\cleardoublepage
\pagenumbering{arabic}
\setcounter{page}{1}

\setcounter{topnumber}{9}
\renewcommand{\topfraction}{.99}
\setcounter{bottomnumber}{9}
\renewcommand{\bottomfraction}{.99}
\setcounter{totalnumber}{10}
\renewcommand{\textfraction}{.01}
\renewcommand{\floatpagefraction}{.01}

% Document Type: LaTeX
% Master File: intro.tex

\chapter{Introduction}

PVS is a \emph{P}rototype \emph{V}erification \emph{S}ystem for the
development and analysis of formal specifications.  The PVS system
primarily consists of a specification language, a parser, a typechecker, a
prover, specification libraries, and various browsing tools.  This
document describes the specification language and is meant to be used as a
reference manual.  The \emph{PVS System Guide}~\cite{PVS:userguide} is to
be consulted for information on how to use the system to develop
specifications and proofs.  The \emph{PVS Prover Guide}~\cite{PVS:prover}
is a reference manual for the commands used to construct proofs.  The web
site \url{http://pvs.csl.sri.com} provides many useful links, including
various tutorials and examples.

In this section, we provide a brief summary of the PVS specification
language, enumerate the key design principles behind the language, and
discuss a simple \texttt{stacks} example.

\section{Summary of the PVS Language}

A PVS specification consists of a collection of \emph{theories}.
Each theory consists of a \emph{signature} for the type names and
constants introduced in the theory, and the axioms, definitions, and
theorems associated with the signature.  For example, a typical
specification for a queue would introduce the \texttt{queue} type and the
operations of \texttt{enq}, \texttt{deq}, and \texttt{front} with their
associated types.  In such a theory, one can either define a
representation for the \texttt{queue} type and its associated operations in
terms of some more primitive types and operations, or merely axiomatize
their properties.  A theory can build on other theories: for example, a
theory for ordered binary trees could build on the theory for
binary trees.  A theory can be \emph{parametric} in certain specified
types and values: as examples, a theory of queues can be parametric in
the maximum queue length, and a theory of ordered binary trees can be
parametric in the element type as well as the ordering relation.  It is
possible to place constraints, called \emph{assumptions}, on the
parameters of a theory so that, for instance, the ordering relation
parameter of an ordered binary tree can be constrained to be a total
ordering.

The PVS specification language is based on simply typed higher-order
logic.  Within a theory, \emph{types} can be defined starting from
\emph{base} types (Booleans, numbers, etc.) using type constructors such
as function, record, and tuple types.  The \emph{terms} of the language
can be constructed using, for example, function application, lambda
abstraction, and record or tuple constructions.

There are a few significant enhancements to the simply typed language
above that lend considerable power and sophistication to PVS.  New
uninterpreted base types may be introduced.  One can define a
\emph{predicate subtype} of a given type as the subset of individuals in a
type satisfying a given predicate: the subtype of nonzero reals is
written as \texttt{$\{$x:real | x /= 0$\}$}.  One benefit of such
subtyping is that when an operation is not defined on all the elements of
a type, the signature can directly reflect this.  For example, the
division operation on reals is given a type where the denominator is
constrained to be nonzero.  Typechecking then ensures that
division is never applied to a zero denominator.  Since the predicate used
in defining a predicate subtype is arbitrary, typechecking is undecidable
and may lead to proof obligations called \emph{type correctness
conditions} (TCCs).  The user is expected to discharge these proof
obligations with the assistance of the PVS prover.  The PVS type system
also features dependent function, record, and tuple type constructions.
There is also a facility for defining a certain class of abstract datatype
(namely well-founded trees) theories automatically.

\section{PVS Language Design Principles}

There are several basic principles that have motivated the design of
PVS which are explicated in this section.

\paragraph{Specification vs. Programming Languages.}
A specification represents requirements or a design whereas a program
text represents an implementation of a design.  A program can be seen as
a specification, but a specification need not be a program.  Typically,
a specification expresses \emph{what} is being
computed whereas a program expresses \emph{how} it is computed.  A
specification can be incomplete and still be meaningful whereas an
incomplete program will typically not be executable.  A specification
need not be executable; it may use high-level constructs, quantifiers
and the like, that need have no computational meaning.  However, there
are a number of aspects of programming languages that a specification
language should include, such as:
\begin{itemize}

\item the usual basic types: booleans, integers, and rational numbers

\item the familiar datatypes of programming languages such as arrays,
records, lists, sequences, and abstract datatypes

\item the higher-order capabilities provided by modern functional
programming languages so that extremely general-purpose operations can
be defined

\item definition by recursion

\item support for dividing large specifications into parameterized
modules

\end{itemize}

It is clearly not enough to say that a specification language shares some
important features of a programming language but need not be executable.
Any useful formal language must have a clearly defined
semantics\footnote{The PVS semantics are presented in a technical
report~\cite{PVS:semantics}.} and must be capable of being manipulated in
ways that are meaningful relative to the semantics.  A programming
language for example can be given a denotational semantics so that the
execution of the program respects its denotational meaning.  The reason
one writes a specification in a formal language is typically to ensure
that it is sensible, to derive some useful consequences from it, and to
demonstrate that one specification implements another.  All of these
activities require the notion of a justification or a proof based on the
specification, a notion that can only be captured meaningfully within the
framework of logic.

\paragraph{Untyped set theory versus higher-order logic}
\index{set theory}\index{higher-order logic}
Which logic should be chosen?  There is a wide variety of choices:
simple propositional logics, which can be classical or intuitionistic,
equational logics, quantificational logics, modal and temporal logics,
set theory, higher-order logic, etc.  Some propositional and modal
logics are appropriate for dealing with finite state machines where one
is primarily interested in efficiently deciding certain finite state
machine properties.  For a general purpose specification language,
however, only a set theory or a higher-order logic would provide the
needed expressiveness.  Higher-order logic requires strict typing to
avoid inconsistencies whereas set theory restricts the rules for forming
sets.  Set theory is inherently untyped, and grafting a typechecker onto
a language based on set theory is likely to be too strict and arbitrary.
Typechecking, however, is an extremely important and easy way of
checking whether a specification makes semantic sense (although 
for an opposing view, the reader is referred to a report by Lamport
and Paulson~\cite{Lamport&Paulson97}).  Higher-order
logic does admit effective typechecking but at the expense of an
inflexible type system.  Recent advances in type theory have made it
possible to design more flexible type systems for higher-order logic
without losing the benefits of typechecking.  We have therefore chosen
to base PVS on higher-order logic.

\paragraph{Total versus partial functions}
\index{function!total}\index{function!partial} In the PVS higher-order
logic, an individual is either a function, a tuple, a record, or the
member of a base type.  Functions are extremely important in higher-order
logic.  They are \emph{first-class} individuals, i.e., variables can range
over functions.  In general, functions can represent either \emph{total}
or \emph{partial} maps.  A total map from domain $A$ to range $B$ maps
each element of $A$ to some element of $B$, whereas a partial map only
maps some of the elements of $A$ to elements of $B$.  Most traditional
logics build in the assumption that functions represent total maps.
Partial functions arise quite naturally in specifications.  For example,
the division operation is undefined on a zero denominator and the
operation of popping a stack is undefined on an empty stack.

Some recent logics, notably those of VDM~\cite{Jones:VDM},
LUTINS~\cite{Farmer:functions}, RAISE~\cite{RAISE-tutorial},
Beeson~\cite{Beeson:book} and Scott~\cite{Scott79}, admit partial
functions.  In these logics, some terms may be \emph{undefined} by not
denoting any individuals.  Some of these logics have mechanisms for
distinguishing defined and undefined terms, while others allow
``undefined'' to propagate from terms to expressions and therefore must
employ multiple truth values.  In all these cases, the ability to
formalize partially defined functions comes at the cost of complicating
the deductive apparatus, even when the specification does not involve any
partial functions.  Though logics that allow partial functions are
extremely interesting, we have chosen to avoid partial functions in PVS.
We have instead employed the notion of a \emph{predicate
subtype}\index{predicate subtype}, a type that consists of those elements
of a given type satisfying a given predicate.  Using predicate subtypes,
the type of the division operator, for example, can be constrained to
admit only nonzero denominators.  Division then becomes a total operation
on the domain consisting of arbitrary numerators and nonzero denominators.
The domain of a \emph{pop} operation on stacks can be similarly restricted
to nonempty stacks.  PVS thus admits partial functions within the
framework of a logic of total functions by enriching the type system to
include predicate subtypes.  We find this use of predicate subtypes to be
significantly in tune with conventional mathematical practice of being
explicit about the domain over which a function is defined.

\section{An Example: \texttt{stacks}}\label{stacks-example}
\index{stacks example@\texttt{stacks} example}

In this section we discuss a specific example, the theory of
\texttt{stacks}, in order to give a feel for the various aspects of the
PVS language before going into detail.  Apart from the basic notation for
defining a theory, this example illustrates the use of type parameters at
the theory level, the general format of declarations, the use of predicate
subtyping to define the type of nonempty stacks, and the generation of
typechecking obligations.

\pvstheory{stacks-alltt}{{Theory \texttt{stacks}}}{stacks-alltt}

Figure~\ref{stacks-alltt} illustrates a theory for stacks of an arbitrary
type with corresponding stack operations.  Note that this is not the
recommended approach to specifying stacks; a more convenient and complete
specification is provided in Section~\ref{stacks-adt},
page~\pageref{stacks-adt}.

The first line introduces a theory named \texttt{stacks} that is
parameterized by a type \texttt{t} (the \emph{formal parameter} of
\texttt{stacks}).  The keyword \texttt{TYPE+} indicates that \texttt{t} is
a \emph{non-empty} type.  The uninterpreted (nonempty) type \texttt{stack}
is declared, and the constant \texttt{empty} and variable \texttt{s} are
declared to be of type \texttt{stack}.  The defined predicate
\texttt{nonemptystack?}~is then declared on elements of type
\texttt{stack}; it is \texttt{true} for a given \texttt{stack} element
iff\footnote{If and only if.} that element is not equal to
\texttt{empty}.\footnote{The \texttt{bool} type and \texttt{/=} operator
are declared in the \emph{prelude}, which is a large body of theories that
are preloaded into PVS.  This is described in Appendix~/ref{prelude}.}
The functions \texttt{push}, \texttt{pop}, and \texttt{top} are then
declared.  Note that the predicate \texttt{nonemptystack?}~is being used
as a type in specifying the signatures of these functions; any predicate
may be used where a type is expected simply by putting parentheses around
it.

The variables \texttt{x} and \texttt{y} are then declared, followed by the
usual axioms for \texttt{push}, \texttt{pop}, and \texttt{top}, which make
\texttt{push} a stack constructor and \texttt{pop} and \texttt{top} stack
accessors.  Finally, there is the theorem \texttt{pop2push2}, that can
easily be proved by two applications of the \texttt{pop\_push} axiom.

This simple theorem has an additional facet that shows up during
typechecking.  Note that \texttt{pop} expects an element of type
\texttt{(nonemptystack?)} and returns a value of type \texttt{stack}.
This works fine for the inner \texttt{pop} because it is applied to
\texttt{push}, which returns an element of type \texttt{(nonemptystack?)};
but the outer occurrence of \texttt{pop} cannot be seen to be type correct
by such syntactic means.  In cases like these, where a subtype is expected
but not directly provided, the system generates a \emph{type-correctness
condition} (TCC).  In this case, the TCC is
\begin{pvsex}
  pop2push2_TCC1: OBLIGATION
    FORALL (s: stack, x, y: t): nonemptystack?(pop(push(x, push(y, s))));
\end{pvsex}
and is easily proved using the \texttt{pop\_push} axiom.  The system keeps
track of all such obligations and will flag the unproved ones during proof
chain analysis.

Parameterized theories such as \texttt{stacks} introduce theory schemas,
where the type \texttt{t} may be instantiated with any other nonempty
type.  To use the types, constants, and formulas of the \texttt{stacks}
theory from another theory, the \texttt{stacks} theory must be imported,
with \emph{actual parameters} provided for the corresponding theory
parameters.  This is done by means of an \texttt{IMPORTING} clause. For
example, consider the theory \texttt{ustacks}.
\begin{session}
  ustacks : THEORY
   BEGIN
    IMPORTING stacks[int], stacks[stack[int]]

    si : stack[int]
    sos : stack[stack[int]] = push(si, empty)
   END ustacks
\end{session}
It imports stacks of integers and stacks of stacks of integers.  The constant
\texttt{si} is then declared to be a stack of integers, and the constant
\texttt{sos} is a stack of stacks of integers whose top element is
\texttt{si}.  Note that the system is able to determine which instance of
\texttt{push} and \texttt{empty} is meant from the type of the first
argument.  In general, the typechecker infers the type of an expression
from its context.  

The following chapters provide more details on the various features of the
language.  The lexical aspects of the language are explained in
Chapter~\ref{lexical}.  Chapter~\ref{declarations} describes declarations,
Chapters~\ref{types} and~\ref{expressions} describe type expressions and
expressions, and Chapter~\ref{theories} explains theories, theory
parameters, and the importing and exporting of names.  Theory
interpretaions and mappings are described in
Chapter~\ref{interpretations}.  Chapter~\ref{names} describes names and
name resolution, and Chapter~\ref{adts} details the datatype facility of
PVS.  Finally, Appendix~\ref{grammar} provides the grammar of the
language and Appendix~\ref{prelude} gives a brief overview of the theories
of the PVS prelude.

% Document Type: LaTeX
% Master File: language.tex
\chapter{The Lexical Structure}\label{lexical}

PVS specifications are text files, each composed of a sequence of lexical
elements which in turn are made up of characters.  The lexical elements of
PVS are the identifiers, reserved words, special symbols, numbers,
whitespace characters, and comments.

Identifiers\index{identifiers} are composed of letters, digits, and the
characters \texttt{\_} or \texttt{?}; they must begin with a letter.  They
may be arbitrarily long, constrained only by the limits imposed by the
underlying Common Lisp system.  Identifiers are case-sensitive;
\texttt{FOO}, \texttt{Foo}, and \texttt{foo} are different identifiers.
PVS strings contain any ASCII character: to include a \texttt{"} in the
string, use \texttt{\char'134 "} and to include a \texttt{\char'134} use
\texttt{\char'134\char'134}.

\pvsbnf{bnf-lexical}{Lexical Syntax}

The reserved words\index{reserved words} are shown in
Figure~\ref{reserved-words}.  Unlike identifiers, they are not
case-sensitive.  In this document, reserved words are always displayed in
upper case.  Note that identifiers may have reserved words embedded in
them, thus \texttt{ARRAYALL} is a valid identifier and will not be
confused with the two embedded reserved words.  The meaning of the
reserved words are given in the appropriate sections; they are collected
here for reference.

\begin{figure}[tb]
{\smaller\tt
\begin{tabular}{|*{5}{p{1.03in}}|}\hline
\input{keywords}
\hline
\end{tabular}}
\caption{\pvs\ Reserved Words}\label{reserved-words}
\end{figure}

The special symbols\index{special symbols} are listed in
Figure~\ref{special-symbols}.  All of these symbols are separators; they
separate identifiers, numbers, and reserved words.

\begin{figure}[tb]
\begin{center}
{\small\tt
\begin{tabular}{|*{6}{@{\hspace*{.2in}}c@{\extracolsep{.5in}}}@{\hspace*{.25in}}|}\hline
\input{operator-table}
\hline
\end{tabular}}
\end{center}
\caption{\pvs\ Special Symbols}\label{special-symbols}
\end{figure}

The whitespace characters are space, tab, newline, return, and newpage;
they are used to separate other lexical elements.  At least one whitespace
character must separate adjacent identifiers, numbers, and reserved words.

Comments\index{comments} may appear anywhere that a whitespace character
is allowed.  They consist of the `\texttt{\%}'\index{\%@\texttt{\%}} character
followed by any sequence of characters and terminated by a newline.

The \emph{definable} symbols are shown in table~\ref{definable-symbols}.
These keywords and symbols may be given declarations.  Some of them have
declarations given in the prelude.\footnote{In particular,
\texttt{\char38}, \texttt{*}, \texttt{+}, \texttt{-}, \texttt{/},
\texttt{/=}, \texttt{<}, \texttt{<<}, \texttt{<=}, \texttt{<=>},
\texttt{=}, \texttt{=>}, \texttt{>}, \texttt{>=}, \texttt{AND},
\texttt{IFF}, \texttt{IMPLIES}, \texttt{NOT}, \texttt{O}, \texttt{OR},
\texttt{WHEN}, \texttt{XOR}, \texttt{\char94}, and \texttt{\char126} are
declared there.  Note that many of these are overloaded, for example,
\texttt{\char94} has three different definitions.}  Any of these may be
(re)declared any number of times, though this may lead to ambiguities.
Such ambiguities may be resolved by including the theory name, actual
parameters,  and possibly the type as a coercion.

Symbols that are binary infix (\hyperlink{Binop}{\emph{Binop}}), for
example \texttt{AND} and \texttt{+}, may be declared with any number of
arguments.  If they are declared with two arguments then they may
subsequently be used in prefix or infix form.  Otherwise they may only be
used in prefix form.  Similarly for unary operators, and the \texttt{IF}
operator, which may be used in \texttt{IF-THEN-ELSE-ENDIF} form if
declared with three arguments.

Note that when typing the operators \texttt{/\\} or \texttt{\\/} outside
of a specification, the backslash may need to be doubled (or in rare
cases, quadrupled).  This is because it is commonly used as an ``escape''
character, and the character following may be interpreted specially.

The symbol pairs \lit{[|} and \lit{|]}, \lit{(|} and \lit{|)}, and
\lit{$\{$|} and \lit{|$\}$} are available as outfix operators.  They are
declared using \lit{[||]}, \lit{(||)}, and \lit{$\{$||$\}$}, respectively.
For example, with the declaration \texttt{[||]:\ [bool, int -> int]} the
outfix term \texttt{[| TRUE, 0 |]} is equivalent to the prefix form
\texttt{[||](TRUE, 0)}.

\begin{figure}[tb]
\begin{center}
{\small\tt
\begin{tabular}{|*{6}{@{\hspace*{.2in}}c@{\extracolsep{.4in minus .4in}}}@{\hspace*{.2in}}|}\hline
\char35\char35 &  \char47\char47 &  \char60\char124 &  AND &  ORELSE &  \char123\char124\char124\char125\\
\char38 &  \char47\char61 &  \char61 &  ANDTHEN &  TRUE &  \char124\char45\\
\char40\char124\char124\char41 &  \char47\char92\char92 &  \char61\char61 &  FALSE &  WHEN &  \char124\char61\\
\char42 &  \char60 &  \char61\char62 &  IF &  XOR &  \char124\char62\\
\char42\char42 &  \char60\char60 &  \char62 &  IFF &  \char91\char93 &  \char126\\
\char43 &  \char60\char60\char61 &  \char62\char61 &  IMPLIES &  \char91\char124\char124\char93 &  \\
\char43\char43 &  \char60\char61 &  \char62\char62 &  NOT &  \char92\char92\char47 &  \\
\char45 &  \char60\char61\char62 &  \char62\char62\char61 &  O &  \char94 &  \\
\char47 &  \char60\char62 &  \char64\char64 &  OR &  \char94\char94 &  \\


\hline
\end{tabular}}
\end{center}
\caption{\pvs\ Definable Symbols}\label{definable-symbols}
\end{figure}

% Document Type: LaTeX
% Master File: language.tex

\chapter{Declarations}\label{declarations}
\index{declaration|(pidx}

Entities of PVS are introduced by means of \emph{declarations}, which are
the main constituents of PVS specifications.  Declarations are used to
introduce types, variables, constants, formulas, judgements, conversions,
and other entities.  Most declarations have an \emph{identifier} and
belong to a unique theory.  Declarations also have a body which indicates
the \emph{kind} of the declaration and may provide a signature or
definition for the entity.  \emph{Top-level}
declarations\index{declaration!top-level} occur in the formal parameters,
the assertion section and the body of a theory.  \emph{Local}
declarations\index{declaration!local} for variables may be given, in
association with constant and recursive declarations and \emph{binding
expressions} (\eg\ involving \texttt{FORALL} or \texttt{LAMBDA}).
Declarations are ordered within a theory; earlier declarations may not
reference later ones.\footnote{Thus mutual recursion is not directly
supported.  The effect can be achieved with a single recursive function
that has an argument that serves as a switch for selecting between two or
more subexpressions.}

\index{exporting|(}\index{importing|(}
Declarations introduced in one theory may be referenced in another by
means of the \texttt{IMPORTING} and \texttt{EXPORTING} clauses.  The
\texttt{EXPORTING} clause of a theory indicates those entities that may be
referenced from outside the theory.  There is only one such clause for a
given theory.  The \texttt{IMPORTING} clauses provide access to the
entities exported by another theory.  There can be many \texttt{IMPORTING}
clauses in a theory; in general they may appear anywhere a top-level
declaration is allowed.  See Section~\ref{importings} for more details.
\index{importing|)}\index{exporting|)}

PVS allows the overloading\index{overloading} of declaration identifiers.
Thus a theory named \texttt{foo} may declare a constant \texttt{foo} and a
formula \texttt{foo}.  To support this \emph{ad hoc} overloading,
declarations are classified according to kind\index{declaration!kind}; in
PVS the primary kinds are \emph{type}\index{declaration!kind!type},
\emph{prop}\index{declaration!kind!prop},
\emph{expr}\index{declaration!kind!expr}, and
\emph{theory}\index{declaration!kind!theory}.  Type declarations are of
kind \emph{type}, and may be referenced in type declarations, actual
parameters, signatures, and expressions.  Formula declarations are of kind
\emph{prop}, and may be referenced in auto-rewrite declarations
(Section~\ref{auto-rewrite-decls}) or proofs (see the PVS Prover
Guide~\cite{PVS:prover}).  Variable, constant, and recursive definition
declarations are of kind \emph{expr}; these may be referenced in
expressions and actual parameters.  Newly introduced names need only be
unique within a kind, as there is no way, for example, to use an
expression where a type is expected.\footnote{There are a few exceptions,
for example the actual parameters of theories, since theories may be
instantiated with types or expressions.}

\pvsbnf{bnf-decls}{Declarations Syntax}
\index{syntax!declarations}

Declarations generally consist of an
\emph{identifier}\index{declaration!identifier}, an optional list of
\emph{bindings}\index{declaration!binding}, and a
\emph{body}\index{declaration!body}.  The body determines the kind of the
declaration, and the bindings and the body together determine the
signature and definition of the declared entity.  Multiple
declarations\index{declaration!multiple} may be given in compressed form
in which a common body is specified for multiple identifiers; for example
%
\begin{pvsex}
  x, y, z: VAR int
\end{pvsex}
In every case this is treated the same as the expanded form, thus the
above is equivalent to:
\begin{pvsex}
  x: VAR int
  y: VAR int
  z: VAR int
\end{pvsex}

In the rest of this chapter we describe declarations for types, variables,
constants, recursive definitions, macros, inductive and coinductive
definitions, formulas, judgements, conversions, libraries, and
auto-rewrites.  The declarations for theory parameters, importings,
exportings, and theory abbreviations are given in Chapter~\ref{theories}.
Figure~\ref{bnf-decls} gives the syntax for declarations.

\section{Type Declarations}\label{type-declarations}
\index{type declarations|(}

Type declarations are used to introduce new type names to the context.
There are four kinds of type declaration:

\begin{itemize}

\item \emph{uninterpreted type declaration}: \texttt{T:\ TYPE}
\index{uninterpreted type}\index{type!uninterpreted}

\item \emph{uninterpreted subtype declaration}: \texttt{S:\ TYPE FROM T}
\index{uninterpreted subtype}\index{type!uninterpreted subtype}

\item \emph{interpreted type declaration}: \texttt{T:\ TYPE =
int}\index{interpreted type}\index{type!interpreted}

\item \emph{enumeration type declarations}: \texttt{T:\ TYPE = \setb r,
g, b\sete} \index{enumeration types}\index{type!enumeration}

\end{itemize}

These type declarations introduce \emph{type names}\index{type!name}
that may be referenced in type expressions (see Section~\ref{types}).
They are introduced using one of the keywords
\keyword{TYPE}\index{type@\texttt{TYPE}},
\keyword{NONEMPTY\_TYPE}\index{type@\texttt{NONEMPTY\_TYPE}}, or
\keyword{TYPE+}\index{type+@\texttt{TYPE+}}.

\subsection{Uninterpreted Type Declarations}
\index{type!uninterpreted|(}

Uninterpreted types support abstraction by providing a means of
introducing a type with a minimum of assumptions on the type.  An
uninterpreted type imposes almost no constraints on an implementation of
the specification.  The only assumption made on an uninterpreted type
\texttt{T} is that it is disjoint from all other types, except for
subtypes of \texttt{T}.  For example,
\begin{pvsex}
  T1, T2, T3: TYPE
\end{pvsex}
%
introduces three new pairwise disjoint types.  If desired, further
constraints may be put on these types by means of axioms or assumptions
(see Section~\ref{formula-declarations} on
page~\pageref{formula-declarations}).

It should be emphasized that uninterpreted types are important in
providing the right level of abstraction in a specification.  Specifying
the type body may have the undesired effect of restricting the possible
implementations, and cluttering the specification with needless detail.

\index{type!uninterpreted|)}\index{uninterpreted type|)}


\subsection{Uninterpreted Subtype Declarations}
\index{uninterpreted subtype|(}

Uninterpreted subtype declarations are of the form
\begin{pvsex}
  s: TYPE FROM t
\end{pvsex}
\index{FROM@\texttt{FROM}}
This introduces an uninterpreted
\emph{subtype}\index{subtypes}\index{type!subtype} \texttt{s} of
the \emph{supertype}\index{supertype}\index{type!supertype}
\texttt{t}.  This has the same meaning as
\begin{pvsex}
  s_pred: [t -> bool]
  s: TYPE = (s_pred)
\end{pvsex}
%
in which a new predicate is introduced in the first line and the type
\texttt{s} is declared as a \emph{predicate} subtype in the second
line\footnote{This is described in Section~\ref{subtypes}
(page~\pageref{subtypes}).}.  No assumptions are made about uninterpreted
subtypes; in particular, they may or may not be empty, and two different
uninterpreted subtypes of the same supertype may or may not be disjoint.
Of course, if the supertypes themselves are disjoint, then the
uninterpreted subtypes are as well.

\index{uninterpreted subtype|)}

\subsection{Interpreted Type Declarations}
\index{interpreted type declarations|(}\index{type!interpreted|(}

Interpreted type declarations are primarily a means for providing names
for type expressions.  For example,
\begin{pvsex}
  intfun: TYPE = [int -> int]
\end{pvsex}
%
introduces the type name \texttt{intfun} as an abbreviation for the type
of functions with integer domain and range.  Because PVS uses
\emph{structural equivalence}\index{structural equivalence} instead of
\emph{name equivalence}\index{name equivalence}, any type expression
\texttt{T} involving \texttt{intfun} is equivalent to the type expression
obtained by substituting \texttt{[int -> int]} for \texttt{intfun} in
\texttt{T}.  The available type expressions are described in
Chapter~\ref{types} on page~\pageref{types}.

Interpreted type declarations may be given
parameters.\index{parameterized type names} For example, the type of
integer subranges may be given as
\begin{pvsex}
  subrange(m, n: int): TYPE = \setb{}i:int | m <= i AND i <= n\sete
\end{pvsex}
and \texttt{subrange} with two integer parameters may subsequently be used
wherever a type is expected.  Any use of a parameterized type must include
all of the parameters, so currying of the parameters is not allowed.  Note
that \texttt{subrange} may be overloaded to declare a different type in
the same theory without any ambiguity, as long as the number or type of
parameters is different.

\index{type!interpreted|)}\index{interpreted type declarations|)}


\subsection{Enumeration Type Declarations}\label{enum-types}
\index{enumeration types|(}\index{type!enumeration|(}

Enumeration type declarations are of the form
\begin{pvsex}
  enum: TYPE = \setb{}e_1,\ldots, e_n\sete
\end{pvsex}
%
where the \texttt{e\_i} are distinct identifiers which are taken to
completely enumerate the type.  This is actually a shorthand for the
datatype specification
\begin{pvsex}
  enum: DATATYPE
    e_1: e_1?
         \vdots
    e_n: e_n?
  END enum
\end{pvsex}
%
explained in Chapter~\ref{adts}.  Because of this, enumeration types may
only be given as top-level declarations, and are \emph{not} type
expressions.  The advantage of treating them as datatypes is that the
necessary axioms are automatically generated, and the prover has built-in
facilities for handling datatypes.

\index{type!enumeration|)}\index{enumeration types|)}

\index{type declarations|)}


\subsection{Empty versus Nonempty Types}
\label{emptytypes}
\index{nonempty type}
\index{empty type}
\index{type!nonempty|(}\index{type!empty|(}

As noted before, PVS allows empty types, and the term \emph{type} refers
to either empty or nonempty types.  Constants declared to be of a given
type provide elements of the type, so the type must be nonempty or there
is an inconsistency.  Thus whenever a constant is declared, the system
checks whether the type is nonempty, and if it cannot decide that it is
nonempty it generates an \emph{existence TCC}.\index{existence
TCC}\index{TCC!existence} This is the simple explanation, but it is made
somewhat complicated by the considerations of formal parameters,
uninterpreted versus interpreted type declarations, explicit declarations
of nonemptiness, and
\keyword{CONTAINING}\index{CONTAINING@\texttt{CONTAINING}} clauses on type
declarationss, as well as a desire to keep the number of TCCs generated to
a minimum, while guaranteeing soundness.  The details are provided below.

First note that having variables range over an empty type causes no
difficulties,\footnote{If the type \texttt{T} is empty, then the following
two equivalences hold:
\begin{pvsex}
  (FORALL (x: T): p(x)) IFF TRUE \quad \mbox{\textrm{and}} \quad (EXISTS (x: T): p(x)) IFF FALSE
\end{pvsex}}
so variable declarations and variable bindings never trigger the
nonemptiness check.

During typechecking, type declarations may indicate that the type is
nonempty, and constant declarations of a given type require that the type
be nonempty.  When a type is determined to be nonempty, it is marked as
such so that future checks of constants do not trigger more TCCs.  Below
we describe how type declarations are handled first for declarations in the
body of a theory, and then for type declarations that appear in the formal
parameters, as they require special handling.

\paragraph{Theory Body Type Declarations}

\begin{itemize}

\item Uninterpreted type or subtype declarations introduced with the
keyword \keyword{TYPE} may be empty.  Declaring a constant of that type
will lead to a TCC that is unprovable without further axioms.

\item Uninterpreted type declarations introduced with the keyword
\keyword{NONEMPTY\_TYPE}\index{nonempty_type@\keyword{NONEMPTY\_TYPE}}
or \keyword{TYPE+}\index{type+@\texttt{TYPE+}} are assumed to be nonempty.
Thus the type is marked nonempty.

\item Uninterpreted subtype declarations introduced with the keyword
\keyword{NONEMPTY\_TYPE} or \keyword{TYPE+} are assumed to be nonempty, as long as the
supertype is nonempty.  Thus the supertype is checked, and an existence
TCC is generated if the supertype is not known to be nonempty.  Then the
subtype is marked nonempty.

\item The type of an interpreted constant is nonempty, as the definition
provides a witness.

\item Interpreted type declarations introduced with the keyword
\keyword{TYPE} may be nonempty, depending on the type definition.

\item Any interpreted type declaration with a \keyword{CONTAINING} clause
is marked nonempty, and the \keyword{CONTAINING} expression is typechecked
against the specified type.  In this case no existence TCC is generated,
since the \keyword{CONTAINING} expression is a witness to the type.  Of
course, other TCCs may be generated as a result of typechecking the
\keyword{CONTAINING} expression.

\end{itemize}

\paragraph{Formal Type Declarations}

Only uninterpreted (sub)type declarations may appear in the formal
parameters list.

\begin{itemize}

\item Formal type declarations introduced with the \texttt{TYPE} keyword may
be empty.  This is handled according to the occurrences of constant
declarations involving the type.

\item If there is a constant declaration of that type in the formal
parameter list, then no TCCs are generated, since
any instance of the theory will need to provide both the type and a
witness.  The type is marked nonempty in this case.

\item If the type declaration is a formal parameter and a constant is
declared whose type involves the type, but is not the type itself (for
example, if the formal theory parameters are \texttt{[t:\ TYPE, f:\ [t ->
t]]}), then a TCC may be generated, and a comment is added to the TCC
indicating that an assuming clause may be needed in order to discharge the
TCC.  This TCC will be generated only if an earlier constant declaration
hasn't already forced the type to be marked nonempty.  Note that there are
circumstances in which the formal type may be empty but the type
expression involving that type is nonempty.  This is discussed further
below.

\end{itemize}

\subsection{Checking Nonemptiness}\label{nonemptiness-check}
\index{type!nonempty}
The typechecker knows a type to be nonempty under the
following circumstances:
\begin{itemize}

\item The type was declared to be nonempty, using either the
\keyword{NONEMPTY\_TYPE}\index{nonempty_type@\keyword{NONEMPTY\_TYPE}} or
the synonymous \keyword{TYPE+}\index{type+@\texttt{TYPE+}} keyword.  If the
type is uninterpreted, this amounts to an assumption that the type is
nonempty.  If the type has a definition, then an existence TCC is
generated unless the defining type expression is known to be nonempty.

\item The type was declared to have an element using a
\keyword{CONTAINING}\index{CONTAINING@\texttt{CONTAINING}} expression.

\item A constant was declared for the type.  In this case an existence TCC
is generated for the first such constant, after which the type is marked
as nonempty.

\item It was marked as nonempty from an earlier check.

\end{itemize}

Once an unmarked type is determined to be nonempty, it is marked by the
typechecker so that later checks will not generate existence TCCs.  In
addition, the type components are marked as nonempty.  Thus the types that
make up a tuple type, the field types of a record type, and the supertype
of a subtype are all marked.

It is possible for two equivalent types to be marked differently, for
example:
\begin{pvsex}
  t1: TYPE = \setb{}x: int | x > 2\sete
  t2: TYPE = \setb{}x: int | x > 2\sete
  c1: t1
\end{pvsex}
only marks the first type (\texttt{t1}).  Hence, it is best to name your types and
to use those names uniformly.

\index{type!empty|)}
\index{type!nonempty|)}

\section{Variable Declarations}
\index{variables|(}\index{declaration!variables|(}

Variable declarations introduce new variables and associate a type with
them.  These are \emph{logical} variables, not program variables; they
have nothing to do with state---they simply provide a name and associated
type so that binding expressions and formulas can be succinct.
Variables may not be exported.  Variable
declarations also appear in binding expressions such as \texttt{FORALL} and
\texttt{LAMBDA}.  Such local declarations ``shadow'' any earlier
declarations.  For example, in
\begin{pvsex}
  x: VAR bool
  f: FORMULA (FORALL (x: int): (EXISTS (x: nat): p(x)) AND q(x))
\end{pvsex}
%
The occurrence of \texttt{x} as an argument to \texttt{p} is of type
\texttt{nat}, shadowing the one of type \texttt{int}.  Similarly, the
occurrence of \texttt{x} as an argument to \texttt{q} is of type
\texttt{int}, shadowing the one of type \texttt{bool}.

\index{variables|)}\index{declaration!variables|)}

\section{Constant Declarations}\label{constants}
\index{constants|(}\index{declaration!constants|(}

Constant declarations introduce new constants, specifying their type and
optionally providing a value.  Since PVS is a higher order logic, the term
\emph{constant} refers to functions and relations, as well as the usual
(0-ary) constants.  As with types, there are both \emph{uninterpreted} and
\emph{interpreted} \index{constants!interpreted}%
\index{constants!uninterpreted} constants.  Uninterpreted constants make
no assumptions, although they require that the type be nonempty (see
Section~\ref{nonemptiness-check}, page~\pageref{nonemptiness-check}).
Here are some examples of constant declarations:
\begin{pvsex}
  n: int
  c: int = 3
  f: [int -> int] = (lambda (x: int): x + 1)
  g(x: int): int = x + 1
\end{pvsex}
%
The declaration for \texttt{n} simply introduces a new integer constant.
Nothing is known about this constant other than its type, unless further
properties are provided by \texttt{AXIOM}s.  The other three constants are
interpreted.  Each is equivalent to specifying two declarations: \eg\
the third line is equivalent to
\begin{pvsex}
  f: [int -> int]
  f: AXIOM  f = (LAMBDA (x: int): x + 1)
\end{pvsex}
%
except that the definition is guaranteed to form a \emph{conservative
extension}\index{conservative extension} of the theory.  Thus the
theory remains consistent after the declaration is given if it was
consistent before.

The declarations for \texttt{f} and \texttt{g} above are two different ways to
declare the same function.  This extends to more complex arguments, for
example
\begin{pvsex}
  f: [int -> [int, nat -> [int -> int]]] =
     (LAMBDA (x: int): (LAMBDA (y: int), (z: nat): (LAMBDA (w: int):
       x * (y + w) - z)))
\end{pvsex}
%
is equivalent to
\begin{pvsex}
  f(x: int)(y: int, z: nat)(w: int): int = x * (y + w) - z
\end{pvsex}
%
This can be shortened even further if the variables are declared first:
\begin{pvsex}
  x, y, w: VAR int
  z: VAR nat
  f(x)(y,z)(w): int = x * (y + w) - z
\end{pvsex}
%
Finally, a mix of predeclared and locally declared variables is possible:
\begin{pvsex}
  x, y: VAR int
  f(x)(y,(z: nat))(w: int): int = x * (y + w) - z
\end{pvsex}
%
Note the parentheses around \texttt{z:\ nat}; without these, \texttt{y} would
also be treated as if it were declared to be of type \texttt{nat}.

A construct that is frequently encountered when subtypes are involved is
shown by this example
\begin{pvsex}
  f(x: \setb{}x: int | p(x)\sete): int = x + 1
\end{pvsex}
%
There are two useful abbreviations for this expression.  In the first, we
use the fact that the type \texttt{\setb{}x:\ int | p(x)\sete} is equivalent to
the type expression \texttt{(p)} when \texttt{p} has type \texttt{[int ->
bool]}, and we can write
\begin{pvsex}
  f(x: (p)): int = x + 1
\end{pvsex}
%
The second form of abbreviation basically removes the set braces and the
redundant references to the variable, though extra parentheses are
required:
\begin{pvsex}
  f((x: int | p(x))): int = x + 1
\end{pvsex}
%
Which of these forms to use is mostly a matter of taste; in general,
choose the form that is clearest to read for a given declaration.

Note that functions with range type \texttt{bool} are generally referred
to as \emph{predicates}, and can also be regarded as relations or sets.
For example, the set of positive odd numbers can be characterized by a
predicate as follows:
\begin{pvsex}
  odd: [nat -> bool] = (LAMBDA (n: nat): EXISTS (m: nat): n = 2 * m + 1)
\end{pvsex}
%
PVS allows an alternate syntax for predicates that encourages a
set-theoretic interpretation:
\begin{pvsex}
  odd: [nat -> bool] = \setb{}n: nat | EXISTS (m: nat): n = 2 * m + 1\sete
\end{pvsex}

\index{constants|)}

\section{Recursive Definitions}\label{recursive-definitions}
\index{recursive definitions|(}

Recursive definitions are treated as constant declarations, except that
the defining expression is required, and a \emph{measure}\index{measure
function} must be provided, along with an optional well-founded order
relation.\index{well-founded order releation} The same syntax for
arguments is available as for constant declarations; see the preceding
section.

PVS allows a restricted form of recursive definition; mutual
recursion\index{recursion!mutual}\index{mutual recursion} is not allowed,
and the function must be \emph{total},\index{total function} so that the
function is defined for every value of its domain.  In order to ensure
this, recursive functions must be specified with a
\emph{measure}\index{measure}, which is a function whose signature matches
that of the recursive function, but with range type the domain of the
order relation, which defaults to \texttt{<} on \texttt{nat} or
\texttt{ordinal}\index{ordinal}\index{type!ordinal}.  If the order
relation is provided, then it must be a binary relation on the range type
of the measure, and it must be well-founded; a \emph{well-founded} \tcc\
\index{well-founded TCC}\index{TCC!well-founded} is generated if the order
is not declared to be well-founded.

Here is the classic example of the
\texttt{factorial}\index{factorial@\texttt{factorial}} function:
%
\begin{pvsex}
  factorial(x: nat): RECURSIVE nat =
    IF x = 0 THEN 1 ELSE x * factorial(x - 1) ENDIF
    MEASURE (LAMBDA (x: nat): x)
\end{pvsex}
%
The measure is the expression following the \texttt{MEASURE} keyword (the
optional order relation follows a \texttt{BY} keyword after the
measure).  This definition generates a \emph{termination
TCC};\index{TCC!termination}\index{termination TCC} a proof obligation
which must be discharged in order that the function be well-defined.  In
this case the obligation is
%
\begin{pvsex}
  factorial_TCC2: OBLIGATION
    FORALL (x: nat): NOT x = 0 IMPLIES x - 1 < x
\end{pvsex}

It is possible to abbreviate the given \texttt{MEASURE} function by
leaving out the \texttt{LAMBDA} binding.  For example, the measure
function of the factorial definition may be abbreviated to:
\begin{pvsex}
  MEASURE x
\end{pvsex}
The typechecker will automatically insert a lambda binding corresponding
to the arguments to the recursive function if the measure is not already
of the correct type, and will generate a typecheck error if this process
cannot determine an appropriate function from what has been specified.

A termination \tcc\ is generated for each recursive occurrence of the
defined entity within the body of the definition.\footnote{Some of these
may be subsumed by earlier TCCs, and hence will not be displayed with the
\texttt{M-x show-tccs} command.}  It is obtained in one of two ways.  If a
given recursive reference has at least as many arguments provided as
needed by the measure, then the \tcc\ is generated by applying the measure
to the arguments of the recursive call and comparing that to the measure
applied to the original arguments using the order relation.  The
\texttt{factorial} \tcc\ is of this form.  The context of the occurrence
is included in the \tcc; in this case the occurrence is within the
\texttt{ELSE} part of an \texttt{IF-THEN-ELSE} so the negated condition is
an antecedent to the proof obligation.

If the reference does not have enough arguments available, then the
reference is actually given a \emph{recursive signature}\index{recursive
signature} derived from the recursive function as described below.  This
type constrains the domain to satisfy the measure, and the termination
\tcc\ is generated as a \emph{termination-subtype}
\tcc.\index{termination-subtype TCC}\index{TCC!termination-subtype}
Termination-subtype \tccs\ are recognized as such by the occurrence of the
order in the goal of the \tcc.  For example, we could define a
substitution function for terms as follows.
\begin{pvsex}
  term: DATATYPE
  BEGIN
   mk_var(index: nat): var?
   mk_const(index: nat): const?
   mk_apply(fun: term, args: list[term]): apply?
  END term

  subst(x: (var?), y: term)(s: term): RECURSIVE term =
    (CASES s OF
      mk_var(i): (IF index(x) = i THEN y ELSE s ENDIF),
      mk_const(i): s,
      mk_apply(t, ss): mk_apply(subst(x, y)(t), map(subst(x, y))(ss))
     ENDCASES)
  MEASURE s BY <<
\end{pvsex}
Now the first recursive occurrence of \texttt{subst} has all arguments
provided, so the termination TCC is as expected.  The second occurrence
does not have enough arguments.  The recursive signature of that
occurrence is
\begin{pvsex}
  [[(var?), term] -> [\setb{}z1: term | z1 << s\sete -> term]]
\end{pvsex}
Hence the signature of \texttt{subst(x, y)} is \texttt{[\setb{}z1:\ term | z1 <<
s\sete -> term]}, and map is instantiated to \texttt{map[\setb{}z1:\ term | z1 <<
s\sete, term]}, which leads to the TCC
\begin{pvsex}
subst_TCC2: OBLIGATION
  FORALL (ss: list[term], t: term, s: term, x: (var?)):
    s = mk_apply(t, ss) IMPLIES every[term](LAMBDA (z: term): z << s)(ss);
\end{pvsex}
Note that this \texttt{map} instance could be given directly, just don't
make the mistake of providing \texttt{map[term, term]}, as this leads to a
TCC that says every \texttt{term} is \texttt{<<} \texttt{s}.
For the same reason, if the uncurried form of this definition is given,
then a lambda expression will have to be provided and the type will have
to include the measure, for example,
\begin{pvsex}
   subst(x: (var?), y, s: term): RECURSIVE term =
     (CASES s OF
       mk_var(i): (IF index(x) = i THEN y ELSE s ENDIF),
       mk_const(i): s,
       mk_apply(t, ss): mk_apply(subst(x, y, t),
                                 map(LAMBDA (s1: \setb{}z: term|z<<s\sete):
                                       subst(x, y, s1))(ss))
      ENDCASES)
   MEASURE s BY <<
\end{pvsex}

The recursive signature is generated based on the type of the recursive
function and the measure.  For curried functions, it may be that the
measure does not have the entire domain of the recursive function, but
only the first few.  For example, consider the measure for the function
\texttt{f}.
\begin{pvsex}
  f(r: real)(x, y: nat)(b: boolean): RECURSIVE boolean
    = ...
   MEASURE LAMBDA (r: real): LAMBDA (x, y: nat): x
\end{pvsex}
The type of the measure function is \texttt{[real -> [nat, nat -> nat]]},
which is a prefix of the function type.  In deriving the recursive
signature, the last domain type of the measure is constrained (using a
subtype) in the corresponding position of the recursive function type.  In
this case the recursive signature is
\begin{pvsex}
  [real -> [\setb{}z: [nat, nat] | z`1 < x\sete -> [boolean -> boolean]]]
\end{pvsex}
Note that the recursive signature is a dependent type that depends on the
arguments of the recursive function (\texttt{x} in this case), and hence
only applies within the body of the recursive definition.

The formal argument that typechecking the body of a recursive function
using the recursive signature is sound will appear in a future version of
the semantics manual, for now note that simple attempts to subvert this
mechanism do not work, as the following example illustrates.
\begin{pvsex}
  fbad: RECURSIVE [nat -> nat] = fbad
   MEASURE lambda (n: nat): n
\end{pvsex}
This leads an unprovable TCC.
\begin{pvsex}
  fbad_TCC1: OBLIGATION FORALL (x1: nat, x: nat): x < x1;
\end{pvsex}
The TCC results from the comaprison of the expected type \texttt{[nat ->
nat]} to the derived type \texttt{[\setb{}z:\ nat | z < x1\sete -> nat]}.  Remember
that in PVS domains of function types must be equal in order for the
function types to satisfy the subtype relation, and this is exactly what
the TCC states.

\pvstheory{f91-alltt}{Theory \texttt{f91}}{f91-alltt}
\index{f91@{\texttt{f91}}}

When a doubly recursive call is found, the inner recursive calls are
replaced by variables in the termination \tccs\ generated for the outer
calls.  For example, given the theory of Figure~\ref{f91-alltt} the
termination \tcc\ is

\begin{session}
f91_TCC5: OBLIGATION
  FORALL (i: nat,
          v: [i1:
               \setb{}z: nat |
                        (IF z > 101 THEN 0 ELSE 101 - z ENDIF) <
                         (IF i > 101 THEN 0 ELSE 101 - i ENDIF)\sete ->
               \setb{}j: nat | IF i1 > 100 THEN j = i1 - 10 ELSE j = 91 ENDIF\sete]):
    NOT i > 100 IMPLIES
     IF i > 100 THEN v(v(i + 11)) = i - 10 ELSE v(v(i + 11)) = 91 ENDIF;
\end{session}
where the inner calls to \texttt{f91} have been replaced by the
higher-order variable \texttt{v}, with the recursive signature as shown.
Since the obligation forces us to prove the termination condition for all
functions whose type is that of \texttt{f91}, it will also hold for
\texttt{f91}.  This example also illustrates the use of dependent types,
discussed in Section~\ref{dependent-types}.

\pvstheory{ackerman-alltt}{Theory \texttt{ackerman}}{ackerman-alltt}
\index{ackerman@{\texttt{ackerman}}}

In some cases the natural numbers are not a convenient measure; PVS
also provides the \texttt{ordinal}s, which allow recursion with measures up
to $\varepsilon_0$.  This is primarily useful in handling
lexicographical orderings.  For example, in the definition of the
Ackerman function in Figure~\ref{ackerman-alltt},\footnote{There are
ways of specifying \texttt{ackerman} using higher-order functionals, in
which case the measure is again on the natural numbers.} there are two
termination \tccs\ generated (along with a number of subtype \tccs).
The first termination \tcc\ is
\begin{pvsex}
  ack_TCC2:
    OBLIGATION
      (FORALL m, n:
        NOT m = 0 AND n = 0 IMPLIES ackmeas(m - 1, 1) < ackmeas(m, n))
\end{pvsex}
%
and corresponds to the first recursive call of \texttt{ack} in the body of
\texttt{ack}.  In this occurrence, it is known that \texttt{m $\neq$ 0}
and \texttt{n = 0}.  The remaining expression says that the measure
applied to the arguments of the recursive call to \texttt{ack} is less
than the measure applied to the initial arguments of \texttt{ack}.  Note
that the \texttt{<} in this expression is over the \texttt{ordinal}s, not
the \texttt{real}s.

\index{recursive definitions|)}


\section{Macros}\label{macro-declarations}
\index{macros|(}

There are some definitions that are convenient to use, but it's preferable
to have them expanded whenever they are referenced.  To some extent this
can be accomplished using auto-rewrites in the prover, but rewriting is
restricted.  In particular terms in types or actual parameters are not
rewritten; \texttt{typepred} and \texttt{same-name} must be used.  These
both require the terms to be given as arguments, making it difficult to
automate proofs.

The \texttt{MACRO} declaration is used to indicate definitions that are
expanded at typecheck time.  Macro declarations are normal constant
declarations, with the \texttt{MACRO} keyword preceding the
type.\footnote{This is similar to the \texttt{==} form of E\textsc{hdm}.}
For example, after the declaration
\begin{pvsex}
  N: MACRO nat = 100
\end{pvsex}
any reference to \texttt{N} is now automatically replaced by \texttt{100},
including such forms as \texttt{below[N]}.

Macros are not expanded until they have been typechecked.  This is because
the name overloading allowed by PVS precludes expanding during parsing.
TCCs are generated before the definition is expanded.
\index{macros|)}

% Master File: language.tex
\section{Inductive and Coinductive Definitions}
\label{inductive-definitions}
\index{inductive definition|(}

\emph{Inductive} definitions~\cite{Aczel:Handbook} are used frequently in
mathematics.  In general, some rules are given that generate elements of a
set, and the inductively defined set is the smallest set that contains
those elements.  The obvious example of an inductive definition is the
natural numbers, where the rules are given by Peano's axioms, with the
induction scheme ensuring that the natural numbers are the smallest set
containing $1$ and the successor of any natural number.  Language
definitions are another example.  Most logics have a notion of
\emph{formulas}, and these are usually defined inductively.

Paulson~\cite{paulson-fixedpoint} notes that this is simply a \emph{least
fixedpoint} with respect to a given domain of elements and a set of rules,
which is well-defined if the rules are \emph{monotonic}, by the
well known Knaster-Tarski theorem.  From this perspective, the greatest
fixedpoint also exists and corresponds to \emph{coinductive} definitions.
Inductive and coinductive definitions are similar to recursive
definitions, in that they have induction principles, and both must satisfy
additional constraints to guarantee that they are well defined.

We will describe inductive definitions first, as they are more familiar.
The even integers provide a simple example of an inductive
definition:\footnote{This is an alternative to the more traditional
definition of \texttt{even?} in the prelude.}
\begin{pvsex}
  even(n: int): INDUCTIVE bool = n = 0 OR even(n - 2) OR even(n + 2)
\end{pvsex}
With this definition, it is easy to prove, for example, that \texttt{0} or
\texttt{1000} are even, simply by expanding the definition enough
times.\footnote{In the latter case, \texttt{(apply (repeat (then (expand
"even") (flatten) (assert))))} is a good strategy to use, though it should
be used with care since it does not terminate on \texttt{even} applied to
anything other than an even numeral.}  More is needed, however, in proving
general facts, such as if $n$ is even, then $n+1$ is not even.  To deal
with these, we need a means of stating that an integer is even iff it is
so as a result of this definition.  In PVS, this is accomplished by the
automatic creation of two induction schemas, that may be viewed using the
\texttt{M-x~prettyprint-expanded} command:
\begin{session}
  even_weak_induction: AXIOM
    FORALL (P: [int -> boolean]):
      (FORALL (n: int): n = 0 OR P(n - 2) OR P(n + 2) IMPLIES P(n)) IMPLIES
       (FORALL (n: int): even(n) IMPLIES P(n));

  even_induction: AXIOM
    FORALL (P: [int -> boolean]):
      (FORALL (n: int):
         n = 0 OR even(n - 2) AND P(n - 2) OR even(n + 2) AND P(n + 2)
          IMPLIES P(n))
       IMPLIES (FORALL (n: int): even(n) IMPLIES P(n));
\end{session}
The weak induction axiom states that if \texttt{P} is another predicate
that satisfies the \texttt{even} form, then any \texttt{even} number
satisfies \texttt{P}.  Thus \texttt{even} is the smallest such \texttt{P}.
The second (strong) axiom allows the \texttt{even} predicate to be carried
along, which can make proofs easier.  These axioms are used by the
\texttt{rule-induct} strategy described in the Prover
Guide~\cite{PVS:prover}.

Inductive definitions are predicates, hence must be functions with
eventual range type \texttt{boolean}.  For example, in
\begin{session}
  f1(n,m:int) INDUCTIVE int = n
  f2(n,m:int)(x,y:int)(z:int): INDUCTIVE [int,int,int -> bool] =
      LAMBDA (a,b,c:int): n = m IMPLIES f2(n,m)(x,y)(z)(a,b,c)
\end{session}
\texttt{f1} is illegal, while \texttt{f2} returns a boolean value if
applied to enough arguments, hence is valid.

To be monotonic, every occurrence of the definition within the defining
body must be \emph{positive}.\index{positive occurrence} For this we need
to define the parity of an occurrence of a term in an expression $A$: If a
term occurs in $A$ with a given parity, then the occurrence retains its
parity in \texttt{$A$ AND $B$}, \texttt{$A$ OR $B$}, \texttt{$B$ IMPLIES
$A$}, \texttt{FORALL y:$A$}, \texttt{EXISTS y:$A$}, and reverses it in
\texttt{$A$ IMPLIES $B$} and \texttt{NOT $A$}.  Any other occurrence is of
unknown parity.

The parity of the inductive definition in the definition body is checked,
and if some occurrence of the definition is negative, a type error is
generated.  If some occurrence is of unknown parity, then a
\emph{monotonicity TCC}\index{TCC!monotonicity}\index{monotonicity TCC} is
generated.  For example, given the declarations
\begin{session}
  f: [nat, bool -> bool]
  G(n:nat): INDUCTIVE bool =
    n = 0 OR f(n, G(n-1))
\end{session}
the monotonicity TCC has the form
\begin{session}
  (FORALL (P1: [nat -> boolean], P2: [nat -> boolean]):
     (FORALL (x: nat): P1(x) IMPLIES P2(x))
         IMPLIES
       (FORALL (x: nat):
          x = 0 OR f(x, P1(x - 1)) IMPLIES x = 0 OR f(x, P2(x - 1))));
\end{session}

Inductive definitions act as constants for the most part, so they may be
expanded or used as rewrite rules in proofs.  However, they are not usable
as auto-rewrite rules, as there is no easy way to determine when to stop
rewriting.

To provide induction schemes in the most usable form, they are generated
as follows.  First, the variables in the definition are partitioned into
fixed\index{fixed inductive variable} and non-fixed variables.  For
example, in the transitive-reflexive closure
\begin{pvsex}
  TC(R)(x, y) : INDUCTIVE bool =
     R(x, y) OR (EXISTS z: TC(R)(x, z) AND TC(R)(z, y))
\end{pvsex}
\texttt{R} is fixed since every occurrence of \texttt{TC} has \texttt{R}
as an argument in exactly the same position, whereas \texttt{x} and
\texttt{y} are not fixed.  The induction is then over predicates $P$ that
take the non-fixed variables as arguments.  If the inductive definition is
defined for variable $V$ partitioned into fixed variables $F$, and
non-fixed variables $N$, the general form of the (weak) induction scheme
is
\begin{session}
  FORALL (\(F\), \(P\)):
   (FORALL (\(N\)):
     \emph{inductive_body}(\(N\))\([P/\emph{def}]\) IMPLIES \(P\)(\(N\)))
      IMPLIES
     (FORALL (\(N\)): \emph{def}(\(V\)) IMPLIES \(P\)(\(N\)))
\end{session}
In the case of \texttt{TC}, this becomes
\begin{session}
  TC_weak_induction: AXIOM
        (FORALL (R: relation, P: [[T, T] -> boolean]):
           (FORALL (x: T, y: T):
              R(x, y) OR (EXISTS z: (P(x, z) AND P(z, y))) IMPLIES P(x, y))
               IMPLIES (FORALL (x: T, y: T): TC(R)(x, y) IMPLIES P(x, y)));
\end{session}

\index{coinductive definitions(|}
Coinductive definitions have the same form as inductive definitions, but
are introduced with the keyword \texttt{COINDUCTIVE}, and generate the
greatest fix point, rather than the least fix point.  The monotonicity
conditions are the same, but the coinduction axioms reverse some of the
implications.  Thus the general form of the (weak) coinduction scheme is
\begin{session}
  FORALL (\(F\), \(P\)):
   (FORALL (\(N\)):
     \(P\)(\(N\)) IMPLIES \emph{coinductive_body}(\(N\))\([P/\emph{def}]\))
      IMPLIES
     (FORALL (\(N\)): \(P\)(\(N\)) IMPLIES \emph{def}(\(V\)))
\end{session}

As noted earlier, inductive and coinductive definitions are really
fixedpoint definitions.  For example, the theory in
Figure~\ref{inductive-fixpoints} shows that an
inductive definition is a least fixedpoint, a coinductive definition is a
greatest fixpoint, an inductively defined set is a subset of a
coindutively defined set, and, if the universe contains a non-wellfounded
element, then the coinductively defined set is strictly larger.  These
results all build on the definitions in  the \texttt{mucalculus} theory of
the prelude.

{\begin{figure}[htb]\begin{boxedminipage}{\textwidth}%
{\smaller\smaller\begin{alltt}
inductive_fixpoint: THEORY
 BEGIN
  N: TYPE+
  n, m: VAR N
  0: N
  S: [N -> N]
  Sax1: AXIOM 0 /= S(n)
  Sax2: AXIOM S(m) = S(n) => m = n
  % Assume a non-wellfounded element
  nwf_exists: AXIOM EXISTS n: n = S(n)

  Nind(n):     INDUCTIVE bool = n = 0 OR Nind(S(n))
  Ncoind(n): COINDUCTIVE bool = n = 0 OR Ncoind(S(n))

  % NN is the predicate transformer corresponding to the (co)inductive defs
  NN(p: pred[N])(n): bool = n = 0 OR p(S(n))

  % These use the lfp and gfp defs from the prelude mucalculus theory
  ind_lfp: FORMULA Nind = lfp(NN)
  coind_gfp: FORMULA Ncoind = gfp(NN)

  % Repeat Nind_weak_induction, which is proved from lfp_induction
  Nind_weak_induction_repeated: FORMULA 
    FORALL (P: [N -> boolean]):
      (FORALL n: n = 0 OR P(S(n)) IMPLIES P(n)) IMPLIES
       (FORALL n: Nind(n) IMPLIES P(n));

  % Inductive definitions are a subset of coinductive
  ind_sub_co: FORMULA Nind(n) => Ncoind(n)

  % Because there is a non-wellfounded element, we can show that
  % the coinductive set is larger.
  co_has_more: FORMULA EXISTS n: Ncoind(n) & NOT Nind(n)
 END inductive_fixpoint
\end{alltt}}\end{boxedminipage}%
\caption{Inductive definitions and fixpoints}\label{inductive-fixpoints}\end{figure}}

\index{coinductive definition|)}
\index{inductive definition|)}


\section{Formula Declarations}\label{formula-declarations}
\index{formula declarations|(}\index{declaration!formulas|(}

Formula declarations introduce \emph{axioms}\index{axioms},
\emph{assumptions}\index{assumptions}, \emph{theorems}\index{theorems},
and \emph{obligations}\index{obligations}.  The identifier associated with
the declaration may be referenced in auto-rewrite declarations (see
Section~\ref{auto-rewrite-decls} and in proofs (see the \texttt{lemma} command
in the PVS Prover Guide~\cite{PVS:prover}).  The expression that makes up
the body of the formula is a boolean expression.  Axioms, assumptions, and
obligations are introduced with the keywords \texttt{AXIOM},
\texttt{ASSUMPTION}, and \texttt{OBLIGATION}, respectively.  Axioms may
also be introduced using the keyword \texttt{POSTULATE}\index{postulate}.
In the prelude postulates are used to indicate axioms that are provable by
the decision procedures, but not from other axioms.  Theorems may be
introduced with any of the keywords
\texttt{CHALLENGE}\index{claim@{\texttt{CHALLENGE}}},
\texttt{CLAIM}\index{claim@{\texttt{CLAIM}}},
\texttt{CONJECTURE}\index{conjecture@{\texttt{CONJECTURE}}},
\texttt{COROLLARY}\index{corollary@{\texttt{COROLLARY}}},
\texttt{FACT}\index{fact@{\texttt{FACT}}},
\texttt{FORMULA}\index{formula@{\texttt{FORMULA}}},
\texttt{LAW}\index{law@{\texttt{LAW}}},
\texttt{LEMMA}\index{lemma@{\texttt{LEMMA}}},
\texttt{PROPOSITION}\index{proposition@{\texttt{PROPOSITION}}},
\texttt{SUBLEMMA}\index{sublemma@{\texttt{SUBLEMMA}}}, or
\texttt{THEOREM}\index{theorem@{\texttt{THEOREM}}}.

Assumptions are only allowed in assuming clauses (see
Section~\ref{assuming}).  Obligations are generated by the system for
\tccs, and cannot be specified by the user.  Axioms are treated
specially when a proof is analyzed, in that they are not expected to
have an associated proof.  Otherwise they are treated exactly like
theorems.  All the keywords associated with theorems have the same
semantics, they are there simply to allow for greater diversity in
classifying formulas.

Formula declarations may contain free variables\index{free variables}, in
which case they are equivalent to the universal closure\index{universal
closure} of the formula.\footnote{The universal closure of a formula is
obtained by surrounding the formula with a \texttt{FORALL} binding
operator whose bindings are the free variables of the formula.  For
example, the universal closure of \texttt{p(x,y) => q(z)} is
\texttt{(FORALL x,y,z:\ p(x,y) => q(z))} (assuming \texttt{x}, \texttt{y}
and \texttt{z} resolve to variables).} In fact, the prover actually uses
the universal closure when it introduces a formula to a proof.  Formula
declarations are the only declarations in which free variables are
allowed.

\index{declaration!formulas|)}\index{formula declarations|)}

% Document Type: LaTeX
% Master File: language.tex
\section{Judgements}
\label{judgements}\index{judgements|(}

The facility for defining predicate subtypes is one of the most useful
features provided by PVS, but it can lead to a lot of redundant TCCs.
\emph{Judgements}\footnote{We prefer this spelling, though many spell
checkers do not.} provide a means for controlling this by allowing
properties of operators on subtypes to be made available to the
typechecker.  There are two kinds of judgements available in PVS\@.  The
\emph{constant judgement}\index{constant judgement} states that a
particular constant (or number) has a type more specific than its declared
type.  The \emph{subtype judgement}\index{subtype judgement} states that
one type is a subtype of another.

\subsection{Constant Judgements}
\index{constant judgement}

There are two kinds of constant judgements.  The simpler kind 
states that a constant or number belongs to a type different than its
declared type.\footnote{Remember that all numbers are implicitly declared to
be of type \texttt{real}.}  For example, the constant judgement
declaration
\begin{pvsex}
  JUDGEMENT c, 17 HAS_TYPE (prime?)
\end{pvsex}
simply states that the constant \texttt{c} and the number \texttt{17} are
both prime numbers.  This declaration leads to the TCC formulas
\texttt{prime?(c)} and \texttt{prime?(17)}, but in any context in which
this declaration is visible, these TCCs will not be generated.  Thus no
TCCs are generated for the formula \texttt{F} in
\begin{pvsex}
  RP: [(prime?), (prime?) -> bool]
  F: FORMULA RP(c, 17) IMPLIES RP(17, c)
\end{pvsex}

The second kind of constant judgement is for functions; argument types are
provided and the judgement states that when the function is applied to
arguments of the given types, then the result has the type following the
\texttt{HAS\_TYPE} keyword.  Here is an example that illustrates the need
for this kind of judgement:
\begin{pvsex}
  x, y: VAR real
  f(x,y): real = x*x - y*y
  n: int = IF f(1,2) > 0 THEN f(4,3) ELSE f(3,2) ENDIF
\end{pvsex}
This leads to two TCCs:
\begin{pvsex}
  n_TCC1: OBLIGATION
    f(1, 2) > 0 IMPLIES
      rational_pred(f(4, 3)) AND integer_pred(f(4, 3))
  n_TCC2: OBLIGATION
    NOT f(1, 2) > 0 IMPLIES
      rational_pred(f(3, 2)) AND integer_pred(f(3, 2))
\end{pvsex}
The problem here is that although we know that \texttt{f} is closed under
the integers, the typechecker does not.  If \texttt{f} is heavily used,
dealing with these TCCs becomes cumbersome.  We can try the \emph{ad hoc}
solution of adding new overloaded declarations for \texttt{f}:
\begin{pvsex}
  i, j: VAR nat
  f(i, j): int = f(i, j)
\end{pvsex}
But now proofs require an extra definition expansion, and such overloading
leads to confusion.\footnote{This is one of the motivations for providing
the \texttt{M-x~show-expanded-sequent} command.}  A more elegant solution
is to use a judgement declaration:
\begin{pvsex}
  f_int_is_int: JUDGEMENT f(i, j: int) HAS\_TYPE int
\end{pvsex}
This generates the TCC
\begin{pvsex}
  f_int_is_int: FORALL (x:int, y:int):
                  rational_pred(f(x, y)) AND integer_pred(f(x, y))
\end{pvsex}
But now the declaration of \texttt{n} given above generates \emph{no}
TCCs, as the typechecker ``knows'' that \texttt{f} is closed on the
integers.  Note that this is different than the simple judgement
\begin{pvsex}
  f_int: JUDGEMENT f HAS\_TYPE [int, int -> int]
\end{pvsex}
In this case, the TCC generated is unprovable:
\begin{pvsex}
  f_int: OBLIGATION
    ((FORALL (x: real): rational_pred(x) AND integer_pred(x)) AND
      (FORALL (x: real): rational_pred(x) AND integer_pred(x)))
     AND
     (FORALL (x1: [real, real]): rational_pred(f(x1)) AND integer_pred(f(x1)));
\end{pvsex}
A warning is generated when simple constant judgements are declared to be
of a function type.\footnote{Earlier versions of PVS simply interpreted
this form as a closure condition, but this is less flexible.}  In
addition, this judgement will not help with the declaration \texttt{n}
above; it can only be used in higher-order functions, for example:
\begin{pvsex}
  F: [[int, int -> int] -> bool]
  FF: FORMULA F(f)
\end{pvsex}

The arguments for a function judgement follow the syntax for function
declarations; so a curried function may be given multiple judgements:
\begin{pvsex}
  f(x, y: real)(z: real): real
  f_closed: JUDGEMENT f(x, y: nat)(z: int) HAS\_TYPE int
  f2_closed: JUDGEMENT f(x, y: int) HAS\_TYPE [real -> int]
\end{pvsex}

If a constant judgement declaration specifies a name, it must refer to a
unique constant and its type must be compatible with the type expression
following the \texttt{HAS\_TYPE} keyword.  If it is a number, then its
type must be compatible with the \texttt{number} type.

Constant judgements generally lead to TCCs.  If no TCC is generated, then
the judgement is not actually needed, and a warning to this effect is
produced.  Simple (non-functional) constant judgements generate TCCs
indicating that the constant belongs to the specified type.  Constant
function judgements generate TCCs that reflect closure conditions.

The judgement facility cannot be used to remove all redundant TCCs; the
variables used for arguments must be unique, and full expressions may not
be included.  Hence the following are not legal:
\begin{pvsex}
  x: VAR real
  x_times_x_is_nonneg: JUDGEMENT *(x, x) HAS\_TYPE nonneg_real
  c: real
  x_times_c_is_even: JUDGEMENT *(x, c) HAS\_TYPE (even?)
\end{pvsex}


\subsection{Subtype Judgements}
\index{subtype judgement}

The subtype judgement is used to fill in edges of the subtype graph that
otherwise are unknown to the typechecker.  For example, consider the
following declarations:
\begin{pvsex}
  nonzero_real: NONEMPTY_TYPE = \{r: real | r /= 0\} CONTAINING 1
  rational: NONEMPTY_TYPE FROM real
  nonneg_rat: NONEMPTY_TYPE = \{r: rational | r >= 0\} CONTAINING 0
  posrat: NONEMPTY_TYPE = \{r: nonneg_rat | r > 0\}  CONTAINING 1
  /: [real, nonzero_real -> real]
\end{pvsex}
For \texttt{r} of type \texttt{real} and \texttt{q} of type
\texttt{posrat}, the expression \texttt{r / q} leads to the TCC \texttt{q
/= 0}.  One solution, if \texttt{q} is a constant, is to use a constant
judgement as described above.  But if there are many constants involving
the type \texttt{posrat}, this requires a lot of judgement declarations,
and does not help at all for variables or compound expressions.  The
subtype judgement solves this by stating that \texttt{posrat} is a subtype
of \texttt{nzrat}.  Another subtype judgement states that \texttt{nzrat}
is a subtype of \texttt{nzreal}:
\begin{pvsex}
  JUDGEMENT posrat SUBTYPE_OF nzrat
  JUDGEMENT nzrat SUBTYPE_OF nzreal
\end{pvsex}
With these judgements, TCCs will not be generated for any denominator that
is of type \texttt{posrat}.  With the (prelude) judgement declarations
\begin{pvsex}
  nnrat_plus_posrat_is_posrat:   JUDGEMENT +(nnx, py) HAS_TYPE posrat
  posrat_times_posrat_is_posrat: JUDGEMENT *(px, py)  HAS_TYPE posrat
\end{pvsex}
not only are there no TCCs generated for \texttt{r / q}, but none are
generated for \texttt{r / (q + 2)}, \texttt{r / ((q + 2) * q)}, etc.

Given a subtype judgement declaration of the form
\begin{pvsex}
  JUDGEMENT S SUBTYPE_OF T
\end{pvsex}
it is an error if \texttt{S} is already known to be a subtype of
\texttt{T}, or if they are not compatible.  Otherwise, \texttt{T} must be
of the form \texttt{\{x:\ ST | p(x)\}}, where \texttt{ST} is the least
compatible type of \texttt{S} and \texttt{T}, and a TCC will is generated
of the form \texttt{FORALL (x:S): p(x)}.  Remember that subtyping on
functions only works on range types, so the subtype judgement
\begin{pvsex}
  JUDGEMENT [nat -> nat] SUBTYPE_OF [int -> int]
\end{pvsex}
leads to the unprovable TCC
\begin{pvsex}
FORALL (x1:nat, y1:int): y1 >= 0 AND TRUE
\end{pvsex}

\subsection{Judgement Processing}

When a judgement declaration is typechecked, TCCs are generated as
explained above and the judgement is added to the current context for use
in typechecking expressions.  The typechecker typechecks expressions in
two passes; in the first pass it simply collects possible types for
subexpressions, and in the second pass it recursively tries to determine a
unique type based on the expected type, and generates TCCs accordingly;
this is where judgements are used.  If the expression is a constant (name
or number), then all non-functional judgements are collected for that
constant and used to generate a minimal TCC relative to the expected type.

If it is an application whose operator is a name, then functional
judgements of the corresponding arity are collected for the operator, and
those judgements for which the application arguments are all known to be
of the corresponding judgement argument types are extracted, and a minimal
TCC is generated from these.

In addition to inhibiting the generation of TCCs during typechecking,
judgements are also important to the prover; they are used explicitly in
the \texttt{typepred} command, and implicitly in \texttt{assert}, where
the judgement type information is provided to the ground decision
procedures.

Subtype judgements are used in determining when one type is a subtype of
another, which is tested frequently during typechecking and proving,
including in the test on argument types described above.

\index{judgements|)}

% Document Type: LaTeX
% Master File: language.tex
\section{Conversions}
\label{coercion-decls}\index{conversions|(}

Conversions are functions that the typechecker can insert automatically
whenever there is a type mismatch.  They are similar to the implicit
coercions for converting integers to floating point used in many
programming languages.  PVS provides some builtin conversions in the
prelude, but conversions may also be provided by the user using
\emph{conversion declarations}.  A conversion declaration consists of the
keyword \texttt{CONVERSION}, optionally followed by `\texttt{+}' or
`\texttt{-}' and an expression.  \texttt{CONVERSION+} is equivalent to
\texttt{CONVERSION}.  The expression must be of type a (subtype of) a
function type, where the domain and range are not compatible.  This is
because conversions are only triggered when there would otherwise be a
type error, and compatible types may lead to unproveable TCCs, but not to
type errors.  Judgements are the proper way to control the generation of
TCCs, see Section~\ref{judgements} for details.

\subsection{Conversion Examples}
\label{conversion-examples}

Here is a simple example.
\begin{pvsex}
  c: [int -> bool]
  CONVERSION c
  two: FORMULA 2
\end{pvsex}
Here, since formulas must be of type boolean, the typechecker
automatically invokes the conversion and changes the formula to
\texttt{c(2)}.  This is done internally, and is only visible to the user
on explicit command\footnote{The \texttt{M-x~prettyprint-expanded} command.}
and in the proof checker.

A more complex conversion is illustrated in the following example.
\begin{pvsex}
  g: [int -> int]
  F: [[nat -> int] -> bool]
  F_app: FORMULA F(g)
\end{pvsex}
As this stands, \texttt{F\_app} is not type-correct, because a function of
type \texttt{[int -> int]} is supplied where one of type \texttt{[nat ->
int]} is required, and PVS requires equality on domain types for function
types to be compatible.  However it is clear that \texttt{g} naturally
induces a function from \texttt{nat} to \texttt{int} by simply restricting
its domain.  Such a domain restriction is achieved by the
\texttt{restrict} conversion\index{restrict
conversion}\index{conversion!restrict} that is defined in the PVS prelude
as follows:
\begin{session}
  restrict [T: TYPE, S: TYPE FROM T, R: TYPE]: THEORY
   BEGIN
    f: VAR [T -> R]
    s: VAR S
    restrict(f)(s): R = f(s)
    CONVERSION restrict
   END restrict
\end{session}
The construction \texttt{S: TYPE FROM T} specifies that the actual
parameter supplied for \texttt{S} must be a subtype of the one supplied
for \texttt{T}.  The specification states that \texttt{restrict(f)} is a
function from \texttt{S} to \texttt{R} whose values agree with \texttt{f}
(which is defined on the larger domain \texttt{T}).  Using this approach,
a type correct version of \texttt{F\_app} can be written as
\texttt{F(restrict[int,nat,int](g))}.  This provides the convenience of
contravariant subtyping, but without the inherent complexity (in
particular, with contravariant subtyping the type of equality must be
correct in substituting equals for equals, making proofs less
perspicuous).

It is not so obvious how to expand the domain of a function in the general
case, so this approach does not work automatically in the other direction.
It does, however, work well for the important special case of sets (or,
equivalently, predicates): a set on some type \texttt{S} can be extended
naturally to one on a supertype \texttt{T} by assuming that the members of
the type-extended set are just those of the original set.  Thus, if
\texttt{extend(s)} is the type-extended version of the original set
\texttt{s}, we have \texttt{extend(s)(x) = s(x)} if \texttt{x} is in the
subtype \texttt{S}, and \texttt{extend(s)(x) = false} otherwise.  We can
say that \texttt{false} is the ``default'' value for the type-extended
function.  Building on this idea, we arrive at the following specification
for a general type-extension function.
\begin{session}
  extend [T: TYPE, S: TYPE FROM T, R: TYPE, d: R]: THEORY
   BEGIN
    f: VAR [S -> R]
    t: VAR T
    extend(f)(t): R = IF S_pred(t) THEN f(t) ELSE d ENDIF
   END extend
\end{session}
The function \texttt{extend(f)} has type \texttt{[T -> R]} and is
constructed from the function \texttt{f} of type \texttt{ [S -> R]} (where
\texttt{S} is a subtype of \texttt{T}) by supplying the default value
\texttt{d} whenever its argument is not in \texttt{S} (\texttt{S\_pred} is
the {\em recognizer\/} predicate for \texttt{ S}).  Because of the need to
supply the default \texttt{d}, this construction cannot be applied
automatically as a conversion.  However, as noted above, \texttt{false} is
a natural default for functions with range type \texttt{bool} (i.e., sets
and predicates), and the following theory establishes the corresponding
conversion.\index{extend\_bool conversion}\index{conversion!extend\_bool}
\begin{session}
extend_bool [T: TYPE, S: TYPE FROM T]: THEORY
 BEGIN
  CONVERSION extend[T, S, bool, false]
 END extend_bool
\end{session}
In the presence of this conversion, the type-incorrect formula
\texttt{B\_app} in the following specification
\begin{pvsex}
  b: [nat -> bool]
  B: [[int -> bool] -> bool]
  B_app: FORMULA B(b)
\end{pvsex}
is automatically transformed to \texttt{B(extend[int,nat,bool,false](b))}.

\subsection{Lambda conversions}\label{lambda-conversion}
\index{lambda conversion}\index{conversion!lambda}

Conversions are also useful (for example, in semantic encodings of dynamic
or temporal logics) in ``lifting'' operations to apply pointwise to
sequences over their argument types.  Here is an example, where
\texttt{state} is an uninterpreted (nonempty) type, and a state variable
\texttt{v} of type real is represented as a constant of type
\texttt{[state -> real]}.
\begin{session}
  th: THEORY
   BEGIN
    CONVERSION+ K_conversion
    state: TYPE+
    l: [state -> list[int]]
    x: [state -> real]
    b: [state -> bool]
    bv: VAR [state -> bool]
    s: VAR state
    box(bv): bool = FORALL s: bv(s)
    F1: FORMULA box(x > 1)
    F2: FORMULA box(b IMPLIES length(l) + 3 > x)
   END th
\end{session}
In this example, the formulas \texttt{F1} and \texttt{F2} are not type
correct as they stand, but with a \emph{lambda conversion},\index{lambda
conversion}\index{conversion!lambda} triggered by the
\texttt{K\_conversion} in the PVS prelude, these formulas are converted to
the forms
\begin{session}
  F1: FORMULA box(LAMBDA (x1: state): x(x1) > 1)
  
  F2: FORMULA
    box(LAMBDA (x3: state):
          b(x3) IMPLIES
           (LAMBDA (x2: state):
              (LAMBDA (x1: state):
                 (LAMBDA (x: state): length(l(x)))(x1) + 3)
                (x2)
             > x(x2))(x3))
\end{session}


\subsection{Conversion Processing}

In general, conversions are applied by the typechecker whenever it would
otherwise emit a type error.  In the simplest case, if an expression
\texttt{e} of type \texttt{T$_1$} occurs where an incompatible type
\texttt{T$_2$} is expected, the most recent compatible conversion
\texttt{C} is found in the context and the occurrence of \texttt{e} is
replaced by \texttt{C(e)}.  \texttt{C} is compatible if its type is
\texttt{[D -> R]}, where \texttt{D} is compatible with \texttt{T$_1$} and
\texttt{R} is compatible with \texttt{T$_2$}.

Conversions are ordered in the context; if multiple compatible conversions
are available,  the most recently declared conversion is used.  Hence, in

\begin{pvsex}
  CONVERSION c1
  \(\cdots\)
  IMPORTING th1, th2
  \(\cdots\)
  CONVERSION c2
  \(\cdots\)
  F: FORMULA 2
\end{pvsex}

For formula \texttt{F}, \texttt{c2} is the most recent conversion,
followed by the conversions in theory \texttt{th2}, those in \texttt{th1},
and finally \texttt{c1}.  Note that the relative order of the constant
declarations (e.g., \texttt{c1} and \texttt{c2} above) doesn't matter,
only the \texttt{CONVERSION} declarations.

When conversions are available on either the argument(s) or the operator
of an application, the arguments get precedence.

For an application \texttt{e(x$_1$, \ldots, x$_n$)} the possible types of
the operator \texttt{e}, and the arguments \texttt{x$_i$} are determined,
and for each operator type \texttt{[D$_1$, \ldots, D$_n$ -> R]} and
argument type \texttt{T$_i$}, if \texttt{D$_i$} is not compatible with
\texttt{T$_i$}, conversions of type \texttt{[T$_i$ -> D$_i$]} are
collected.  If such conversions are found for every argument that doesn't
have a compatible type, then those conversions are applied.  Otherwise an
operator conversion is looked for.

Note that compositions of conversion are never searched for, as this would
slow down processing too much.  If you want to use a composition, include
a conversion declaration for it.  Here is an example:
\begin{session}
  T1, T2, T3: TYPE+
  f1: [T1 -> T2]
  f2: [T2 -> T3]
  x: T1
  g: [T3 -> bool]
  CONVERSION f1, f2
  F1: FORMULA g(x)
  CONVERSION f2 o f1
  F2: FORMULA g(x)
\end{session}
In this example, \texttt{F1} leads to a type error, but when we make the
composition a conversion, the same expression in \texttt{F2} applies the
conversion rather than give a type error.

\subsection{Conversion Control}

As stated above, conversions are only applied when typechecking otherwise
fails.  In some cases, a conversion can allow a specification to
typecheck, but the meaning is different than what was intended.  This is
most likely for the \texttt{K\_conversion}, which was introduced when the
\texttt{mucalculus} theory was added to the prelude in support of the
model checker.  When a conversion is applied that fact is noted as a
message, and may be viewed using the \texttt{show-theory-messages}
command.  However, these messages are easily overlooked, so instead PVS
allows finer control over conversions.

Thus in addition to the \texttt{CONVERSION} form, the \texttt{CONVERSION-}
form is available allowing conversions to be turned off.  For uniformity,
the \texttt{CONVERSION+} form is also available as an alias for
\texttt{CONVERSION}.  \texttt{CONVERSION-} disables conversions.

The following theory illustrates the idea:
\begin{session}
  t1: THEORY
  BEGIN
   c: [int -> bool]
   CONVERSION+ c
   f1: FORMULA 3
   CONVERSION- c
   f2: FORMULA 3
  END t1
\end{session}
Here \texttt{f2} leads to a type error.

Another example is provided by the definition of the CTL temporal
operators in the prelude theory \texttt{ctlops}, which are surrounded by
\texttt{CONVERSION+} and \texttt{CONVERSION-} declarations that first
enable the \texttt{K\_conversion} then disable it at the end of the
theory.  All other conversions declared in the prelude remain enabled.
They may be disabled within any theory by using the \texttt{CONVERSION-}
form.

When theories containing conversion declarations are imported, the
conversions are imported as well.  Thus if \texttt{t2} has the
\texttt{CONVERSION+ c} declaration but no \texttt{CONVERSION-}
declaration, then \texttt{IMPORTING t1, t2} would enable the conversion,
but \texttt{IMPORTING t2, t1} would leave it disabled.

Conversion declarations may be generic or instantiated.  This
allows, for example, enabling the generic form of a conversion while
disabling particular instances.

\index{conversions|)}


\chapter{PVS Libraries}

The PVS library mechanism provides support for creating library
\emph{packages}, including, in addition to theories, lisp and Emacs lisp
functions that add new features.  We describe here the considerations for
creating and using such packages, in particular, how to structure, 
distribute, and install a package in portable manner.

Here is a typical scenario.  A set of theories are developed and
typechecked, some supporting strategies are defined in the
\texttt{pvs-strategies}\index{pvs-strategies@\texttt{pvs-strategies}}
file, and some Emacs extensions are created in
\texttt{pvs-lib.el}\index{pvs-lib.el@\texttt{pvs-lib.el}} to add new key
bindings to the Emacs prover interface.  These are all in directory
\texttt{<DIR>/lib/mylib/}.  To use the library package, create the
environment variable
\texttt{PVS\_LIBRARY\_PATH}\index{PVS_LIBRARY_PATH@\texttt{PVS\_LIBRARY\_PATH}},
and set it to \texttt{<DIR>/lib}.  Then \texttt{cd} to a new directory,
and start up PVS.  After it comes up, run
\begin{alltt}
  M-x \icmd{load-prelude-library} mylib <RET>
\end{alltt}
and the theories will be loaded as prelude extensions, the pvs-lib.el file
will be loaded into Emacs, and when the prover is invoked next, the
\texttt{pvs-strategies} file will be loaded.  When the session is ended,
the prelude library dependency is stored in the \texttt{.pvscontext}, and
the next time PVS goes to that context (either by starting it up there, or
with the \icmd{change-context} command), the files will all be loaded once
again.

When PVS creates the \texttt{.pvscontext} file, it tries to stay away from
absolute pathnames as much as it can.  In this case, the dependency that
it keeps is simply \texttt{mylib/}.  What this means is that the directory
can be copied to a different site, where the \texttt{mylib} directory is
found in a different directory, but as long as \texttt{mylib} can be found
in the \texttt{PVS\_LIBRARY\_PATH} it will be used properly.

Note that for this to work the supporting files must also be free of any
absolute pathnames.  This includes any library declarations within the
theory files.  This is possible, as the
\texttt{PVS\_LIBRARY\_PATH}\index{PVS_LIBRARY_PATH@\texttt{PVS\_LIBRARY\_PATH}}
may have many paths on it, separated by colons.  Note that a given PVS
library is a subdirectory of an element of the
\texttt{PVS\_LIBRARY\_PATH}, not directly on the path.

PVS uses the \texttt{PVS\_LIBRARY\_PATH} variable in many ways:
\begin{itemize}
\item On startup, the Emacs variable
\texttt{pvs-library-path}\index{pvs-library-path@\texttt{pvs-library-path}}
is set to the corresponding list of pathnames.  These are added to the
Emacs variable \texttt{load-path}\index{load-path@\texttt{load-path}}, so
that the Emacs \texttt{load} command can find them.  Thus if library
\texttt{lib1} contains an Emacs file \texttt{foo.el} that needs the Emacs
file \texttt{bar.el} of \texttt{lib2}, it simply loads \texttt{lib2/bar}.

\item In the lisp image, the variable \texttt{*pvs-library-path*} is
similarly set to the list of pathname components of
\texttt{PVS\_LIBRARY\_PATH}.  

\end{itemize}





\section{Auto-rewrite Declarations}
\label{auto-rewrite-decls}\index{auto-rewrites|(}

One of the problems with writing useful theories or libraries is that
there is no easy way to convey how the theory is to be used, other than in
comments or documentation.  In particular, the specifier of a theory
usually knows which lemmas should always be used as rewrites, and which
should never appear as rewrites.  Auto-rewrite declarations allow for both
forms of control.  Those that should always be used as rewrites are
declared with the \texttt{AUTO\_REWRITE+} or \texttt{AUTO\_REWRITE}
keyword, and those that should not are declared with
\texttt{AUTO\_REWRITE-}.  These will be referred to as
\emph{auto-rewrites} and \emph{stop-auto-rewrites} below.

When a proof is initiated for a given formula, all of the auto-rewrite
names in the current context that haven't subsequently been removed by
stop-auto-rewrite declarations are collected and added to the initial proof
state.  The stop-auto-rewrite declaration, in addition to removing
auto-rewrite names, also affects the following commands described in the
Prover manual.
\begin{itemize}
\setlength{\itemsep}{-5pt}
\item \texttt{auto-rewrite-theory},
\item \texttt{auto-rewrite-theories},
\item \texttt{auto-rewrite-theory-with-importings},
\item \texttt{simplify-with-rewrites},
\item \texttt{auto\-rewrite-defs},
\item \texttt{install-rewrites},
\item \texttt{auto-rewrite-explicit},
\item \texttt{grind},
\item \texttt{induct\-and-simplify},
\item \texttt{measure-induct-and-simplify}, and
\item \texttt{model-check}
\end{itemize}
These commands collect all definitions and formulas except those that
appear in \texttt{AUTO\_\-REWRITE-} declarations.  Thus suppose a theory
\texttt{T} contains the lemmas \texttt{lem1}, \texttt{lem2}, and
\texttt{lem3} and the declarations
\begin{alltt}
  AUTO_REWRITE+ lem1
  AUTO_REWRITE- lem3
\end{alltt}
Then in proving a formula of a theory that imports \texttt{T},
\texttt{lem1} is initially an auto-rewrite, and the command
\texttt{(auto-rewrite-theory "T")} will additionally install
\texttt{lem2}.  To auto-rewrite with \texttt{lem3}, simply use
\texttt{(auto-rewrite "lem3")}.  To exclude \texttt{lem1}, use
\texttt{(stop-auto-rewrite "lem1")} or \texttt{(auto-rewrite-theory "T"
:exclude "lem1")}.

The \texttt{autorewrites} theory shows a simple example.
\begin{session}
autorewrites: THEORY
BEGIN
 AUTO_REWRITE+ zero_times3
 a, b: real
 f1: FORMULA a * b = 0 AND a /= 0 IMPLIES b = 0
 AUTO_REWRITE- zero_times3
 f2: FORMULA a * b = 0 AND a /= 0 IMPLIES b = 0
END autorewrites
\end{session}
Here \texttt{f1} may be proved using only \texttt{assert}, but \texttt{f2}
requires more.

Rewrite names may have suffixes, for example, \texttt{foo!} or
\texttt{foo!!}.  Without the suffix, the rewrite is \emph{lazy}, meaning
that the rewrite will only take place if conditions and TCCs simplify to
true.  A condition in this case is a top-level \texttt{IF} or
\texttt{CASES} expression.  With a single exclamation point the
auto-rewrite is \emph{eager}, in which case the conditions are irrelevant,
though if it is a function definition it must have all arguments supplied.
With two exclamation points it is a \emph{macro} rewrite, and terms will
be rewritten even if not all arguments are provided.  See the prover guide
for more details; the notation is derived from the prover commands
\texttt{auto-rewrite}, \texttt{auto-rewrite!}, and
\texttt{auto-rewrite!!}.

In addition, a rewrite name may be disambiguated by stating that it is a
formula, or giving its type if it is a constant.  Without this any
definition or lemma in the context with the same name will be installed as
an auto-rewrite.

In order to be more uniform, these new forms of name are also available
for the \texttt{auto-rewrite} prover commands.  Thus the command
\begin{alltt}
  (auto-rewrite "A" ("B" "-2") "C" (("1" "D")))
\end{alltt}
may now be given instead as
\begin{alltt}
  (auto-rewrite "A" "B!" "-2!" "C" "1!!" "D!!")
\end{alltt}
The older form is still allowed, but is deprecated, and may not be mixed
with the new form.  Notice that in the auto-rewrite commands formula
numbers may also be used, and these may be followed by exclamation points,
but not by a formula keyword or type.

\index{auto-rewrites|)}


\index{declaration|)}

% Document Type: LaTeX
% Master File: language.tex

\chapter{Types}\label{types}
\index{type|(}

PVS specifications are strongly typed, meaning that every expression has
an associated type (although it need not be unique, more on this later).
The PVS type system is based on \emph{structural
equivalence}\index{structural equivalence} instead of \emph{name
equivalence}\index{name equivalence}, so types are closely related to
sets, in that two types are equal iff they have the same elements.
Section~\ref{type-declarations} describes the introduction of type names,
which are the simplest type expressions.  More complex type
expressions\index{type expressions} are built from these using \emph{type
constructors}\index{type constructors}.  There are type constructors for
\emph{subtypes}\index{subtypes}\index{type!subtype}, \emph{function
types}\index{function types}\index{type!function}, \emph{tuple
types}\index{tuple types}\index{type!tuple}, and \emph{record
types}\index{record types}\index{type!record}.  Function, record,
and tuple types may also be \emph{dependent}\index{dependent
types}\index{type!dependent}.  A form of \emph{type
application}\index{type application}\index{type!application} is
provided that makes it more convenient to specify parameterized subtypes.
There are also provisions for creating \emph{abstract datatypes},
described in Chapter~\ref{datatypes}.

Type expressions occur throughout a specification; in particular, they may
appear in theory parameters, type declarations, variable declarations,
constant declarations, recursive and inductive definitions, conversions,
and judgements.  In
addition, they may appear in certain expressions (coercions and local
bindings, see pages~\pageref{coercions} and~\pageref{binding-expressions},
respectively), and as actual parameters in names (page~\pageref{names}).
In the many examples which follow, type expressions will be presented in
the context of type declarations; but it must be remembered that they can
appear in any of the above places.

\pvsbnf{bnf-type-expr}{Type Expression Syntax}

\section{Subtypes}\label{subtypes}
\index{subtypes|(}\index{type!subtype|(}

Any collection of elements of a given type itself forms a type, called a
\emph{subtype}.  The type from which the elements are taken is called the
\emph{supertype}\index{supertype}.  The elements which form the subtype
are determined by a \emph{subtype predicate}\index{subtype predicate} on
the supertype.

Subtypes in PVS provide much of the expressive power of the language,
at the cost of making typechecking undecidable.  There are two forms of
subtypes.  The first is similar to the notation used to define a set:
\begin{pvsex}
  t: TYPE = \{x: s | p(x)\}
\end{pvsex}
%
where {\tt p} is a predicate on the type {\tt s}.\footnote{If {\tt x}
has been previously declared as a variable of type {\tt s}, then the
``{\tt :~s}'' may be omitted.} This has the usual set-theoretical
meaning, since types in PVS are modeled as sets.  Subtypes may also
be presented in an abbreviated form, by giving a predicate surrounded by
parentheses:
\begin{pvsex}
  t: TYPE = (p)
\end{pvsex}
%
This is equivalent to the form above.

Note that if the predicate \texttt{p} is everywhere false, then the type
is empty.  PVS supports empty types\index{empty type}\index{type!empty},
and the term \emph{type} is used to refer to any type, including the empty
type.  This is discussed in Section~\ref{type-declarations} (page~\pageref{type-declarations}).

Subtypes tend to make specifications more succinct and easier to read.
For example, in a specification such as
\begin{pvsex}
  FORALL (i:int):
    (i >= 0 IMPLIES (EXISTS (j:int): j >= 0 AND j > i))
\end{pvsex}
it is much more difficult to see what is being stated than in the
equivalent
\begin{pvsex}
  FORALL (i:nat): (EXISTS (j:nat): j > i))
\end{pvsex}
%
where {\tt nat} is defined in the prelude as
\begin{pvsex}
  naturalnumber: NONEMPTY\_TYPE = \{i:integer | i >= 0\} CONTAINING 0
  nat: NONEMPTY\_TYPE = naturalnumber
\end{pvsex}

Subtype constructors consist of a \emph{supertype}\index{supertype} and
an optional predicate on the supertype.  The primary property of a
subtype is that any element which belongs to the subtype automatically
belongs to the supertype.  In addition, functions defined on a type
automatically carry over to the subtype.

\index{TCC|(}

There are two \emph{type-correctness conditions} (\tccs) associated with
subtypes.  The first concerns \emph{empty types}\index{empty
type}\index{type!empty} as described in section~\ref{emptytypes}.  PVS
allows empty types as long as only variables range over them.  However,
allowing declarations of constants involving empty types leads to
inconsistencies.  Whenever a constant is declared, the typechecker checks
the types involved, and generates \emph{ existence} \tccs\ \index{existence
TCC}\index{TCC!existence} for those types which must be nonempty.  For
example,
%
\begin{pvsex}
  f: [int -> \{x:int | p(x)\}]
\end{pvsex}
leads to the \tcc\
\begin{pvsex}
  f_TCC1: OBLIGATION (EXISTS (x: int): p(x))
\end{pvsex}
These \tccs\ are recorded, so that the nonemptiness of a subtype need
be established only once in a theory.  However, the same \tcc\ may be
generated in different theories.  In particular, if a theory declares a
type but no constant of that type, then any theory which imports that
theory and declares a constant of that type will generate the nonempty
\tcc.  

A subtype may be guaranteed nonempty by providing a
\emph{witness},\index{witness} in which case no existence \tcc\ is
generated, though typechecking the witness itself may generate a \tcc.
The witness is provided using the {\tt
CONTAINING}\index{CONTAINING@\texttt{CONTAINING}} clause of a subtype
expression, as illustrated in the following:
\begin{pvsex}
  t: TYPE = \{x: int | 0 < x AND x < 10\} CONTAINING 1
\end{pvsex}
In this case a \tcc\ is generated with the witness in place of the
existential variable, resulting in the trivial \tcc\footnote{This \tcc\
will be proved automatically by PVS; see the
\cmd{typecheck-prove} command in the PVS System Guide~\cite{PVS:userguide}.}
\begin{pvsex}
  t_TCC1: OBLIGATION 0 < 1 AND 1 < 10
\end{pvsex}

The second \tcc\ associated with subtypes is the \emph{subtype}
\tcc,\index{subtype TCC}\index{TCC!subtype}, which come about from the use
of operations defined on subtypes which are applied to elements of the
supertype.  By this means partial functions may be handled directly,
without recourse to a partial term logic or some form of multi-valued
logic.  For instance, division in PVS is a total function, with signature
{\tt [real, nonzero\_real -> real]}.  So given the formula
\begin{pvsex}
  div_form: FORMULA (FORALL (x,y: int):
                      x /= y IMPLIES (x - y)/(y - x) = -1)
\end{pvsex}
%
the denominator is of type integer, but the signature for {\tt /} demands
a {\tt nonzero\_real}.  The typechecker thus generates a \emph{subtype}
\tcc\ whose conclusion is {\tt (y - x) /= 0}.  The premises of the \tcc\
are obtained from the expression's \emph{context}---the conditions which
lead to the {\tt /} operator---in this case {\tt x /= y}.  The \tcc\ is
then
\begin{pvsex}
  div_form_TCC1: OBLIGATION
    (FORALL (x,y: int): x /= y IMPLIES (y - x) /= 0)
\end{pvsex}
which is easily discharged by the prover.  In general, the context of an
expression is obtained from expressions involving {\tt IF-THEN-ELSE},
{\tt AND}, {\tt OR}, and {\tt IMPLIES} by translating to the {\tt
IF-THEN-ELSE} form.  Specifically,
\begin{center}
\begin{tabular}{|lc|} \hline
Expression & Context for {\tt E} \\ \hline
{\tt IF A THEN E ELSE C ENDIF} & {\tt A} \\
{\tt IF A THEN B ELSE E ENDIF} & {\tt NOT A} \\
{\tt A AND E} & {\tt A} \\
{\tt A OR E} & {\tt NOT A} \\
{\tt A IMPLIES E} & {\tt A} \\ \hline
\end{tabular}
\end{center}
Note that only these operators are treated this way; if, for example,
\texttt{IMPLIES} is overloaded it will not include the left-hand side in
the context for typechecking the right-hand side.  The \tccs\ generated
from the context of expression involving a subtype are sufficient, but not
necessary conditions which ensure that the value of the expression does
not depend on the value of functions applied outside their domain.

\index{TCC|)}

\index{type!subtype|)}\index{subtypes|)}

\section{Function Types}\label{function-types}
\index{function types|(}\index{type!function|(}

Function types have three equivalent forms:
\begin{itemize}
\item {\tt [t\(_1\), \ldots, t\(_n\) -> t]}

\item {\tt FUNCTION[t\(_1\), \ldots, t\(_n\) -> t]}

\item {\tt ARRAY[t\(_1\), \ldots, t\(_n\) -> t]}
\end{itemize}
%
where each {\tt t$_i$} is a type expression.  An element of this type is
simply a function whose domain is the sequence of types {\tt t$_1$},
\ldots, {\tt t$_n$}, and whose range is {\tt t}.  A function type is empty
if the range is empty and the domain is not.  There is no difference in
meaning between these three forms; they are provided to support different
intensional uses of the type, and may suggest how to handle the given type
when an implementation is created for the specification.

The two forms {\tt pred[t]}\index{pred@{{\tt pred}}} and {\tt
setof[t]}\index{setof@{{\tt setof}}} are both provided in the
prelude as shorthand for {\tt [t ->
bool]}.  There is no difference in semantics, as sets in
PVS are represented as predicates.  The different keywords are
provided to support different intentions; {\tt pred} focuses on
properties while {\tt setof} tends to emphasize elements.

A function type \texttt{[t$_1$,\ldots,t$_n$ -> t]} is a subtype of
\texttt{[s$_1$,\ldots,s$_m$ -> s]} iff \texttt{s} is a subtype of
\texttt{t}, $n = m$, and {\tt s$_i$} = {\tt t$_i$} for $1 \leq i \leq n$.
This leads to subtype TCCs (called \emph{domain mismatch
TCCs})\index{domain mismatch TCC}\index{TCC!domain mismatch} that state
the equivalence of the domain types.  For example, given
\begin{pvsex}
  p,q: pred[int]
  f: [\{x: int | p(x)\} -> int]
  g: [\{x: int | q(x)\} -> int]
  h: [int -> int]
  eq1: FORMULA f = g
  eq2: FORMULA f = h
\end{pvsex}
%
The following \tccs\ are generated:
\begin{pvsex}
eq1_TCC1: OBLIGATION
  (FORALL (x1: \{x : int | q(x)\}, y1 : \{x : int | p(x)\}) :
     q(y1) AND p(x1))

eq2_TCC1: OBLIGATION
  (FORALL (x1: int, y1 : \{x : int | p(x)\}) :
     TRUE AND p(x1))
\end{pvsex}

Section~\ref{conversion-examples} on page~\pageref{conversion-examples}
explains how the \texttt{restrict} conversion may be automatically applied
in some cases to eliminate the production of these TCCs.

\index{type!function|)}\index{function types|)}


\section{Tuple Types}\label{tuple-types}
\index{tuple types|(}\index{type!tuple|(}

Tuple types (also called product types) have the form {\tt [t$_1$, \ldots,
t$_n$]}, where the {\tt t$_i$} are type expressions.  Note that the 0-ary
tuple type is not allowed.  Elements of this type are tuples whose
components are elements of the corresponding type.  For example, {\tt (1,
TRUE, (LAMBDA (x:int): x + 1))} is an expression of type {\tt [int, bool,
[int -> int]]}.  Order is important.  Associated with every $n$-tuple type
is a set of projection functions: \texttt{`1}, \texttt{`2}, \ldots, (or
{\tt proj\_1}, \texttt{proj\_2}, \ldots) where the $i$th projection is of
type {\tt [[t$_1$, \ldots, t$_n$] -> t$_i$]}.  A tuple type is empty if
any of its component types is empty.  Function type domains and tuple
types are closely related, as the types {\tt [t$_1$,\ldots, t$_n$ -> t]}
and {\tt [[t$_1$,\ldots, t$_n$] -> t]} are equivalent; see
Section~\ref{tuple-exprs} for more details.

\index{type!tuple|)}\index{tuple types|)}

\section{Record Types}\label{record-types}
\index{record types|(}\index{type!record|(}

Record types are of the form {\tt [\# a$_1$:t$_1$, \ldots, a$_n$:t$_n$
\#]}.  The {\tt a$_i$} are called \emph{record accessors}\index{record
accessors} or fields and the {\tt t$_i$} are types.  Record types are
similar to tuple types, except that the order is unimportant and accessors
are used instead of projections.  Record types are empty if any of the
component types is empty.

\index{type!record|)}\index{record types|)}

\section{Dependent types}\label{dependent-types}
\index{dependent types|(}\index{type!dependent|(}

Function, tuple, and record types may be dependent; in other words, some
of the type components may depend on earlier components.  Here are some
examples:
\begin{pvsex}
  rem: [nat, d: \{n: nat | n /= 0\} -> \{r: nat | r < d\}]
  pfn: [d:pred[dom], [(d) -> ran]]
  stack: [\# size: nat, elements: [\{n:nat | n < size\} -> t] \#]
\end{pvsex}
The declaration for {\tt rem} indicates explicitly the range of the
remainder function, which depends on the second argument.  Function types
may also have dependencies within the domain types; \eg\ the second domain
type may depend on the first.  Note that for function and tuple dependent
types, local identifiers need be associated only with those types on which
later types depend.

The tuple type {\tt pfn} encodes partial functions as pairs consisting
of a predicate on the domain type and a function from the subtype
defined by that predicate to the range {\tt ran}.  If the second
component were given instead as a function of type {\tt [dom -> ran]},
then equality no longer works as intended.  For example, the absolute
value function {\tt abs} and the identity function {\tt id} are the same
on the domain {\tt nat}, so we would like to have
\begin{pvsex}
  ((LAMBDA (x:int):x >= 0),abs) = ((LAMBDA (x:int):x >= 0),id)
\end{pvsex}
%
but without the dependency this would be equivalent to {\tt abs = id}.

{\tt stack} encodes a stack as a pair consisting of a size and an array
mapping initial segments of the natural numbers to {\tt t}.  This is
similar to the {\tt pfn} example---in fact, if we were willing to use a
tuple instead of a record encoding, {\tt stack} could be declared as an
instance of the type of {\tt pfn}.

Another example, presented in~\cite{Cheng&Jones90} as a ``challenge'' to
specification languages without partial functions, is easily handled
with dependent types as shown below.
\begin{pvsex}
  subp(i:int,(j:int | i >= j)): RECURSIVE int =
       (IF (i=j) THEN 0 ELSE (subp(i, j+1)+1) ENDIF)
    MEASURE i - j
\end{pvsex}
However, some formulas that are valid with partial functions are not even
well-formed in PVS:
\begin{pvsex}
  subp_lemma: LEMMA subp(i, 0) = i OR subp(0, i) = i
\end{pvsex}
This generates unprovable TCCs.  In practice this is rarely a problem.

\index{type!dependent|)}\index{dependent types|)}

\index{type|)}

% Document Type: LaTeX
% Master File: language.tex

\chapter{Expressions}\label{expressions}
\index{expressions|(}

The PVS language offers the usual panoply of expression constructs,
including logical and arithmetic operators, quantifiers, lambda
abstractions, function application, tuples, a polymorphic
\texttt{IF-THEN-ELSE}, and function and record overrides.  Expressions may
appear in the body of a formula or constant declaration, as the predicate
of a subtype, or as an actual parameter of a theory instance.  The syntax
for PVS expressions is shown in Figures~\ref{bnf-expr} and~\ref{bnf-expr-aux}.

\pvsbnf{bnf-expr}{Expression syntax}

\pvsbnf{bnf-expr-aux}{Expression syntax (continued)}

\index{precedence|(} The language has a number of predefined operators
(although not all of these have a predefined meaning).  These are given in
Figure~\ref{precedenceops} below, along with their relative precedence
from lowest to highest.  Most of these operators are described in the
following sections.  \texttt{IN} is a part of \texttt{ LET} expressions,
\texttt{WITH} goes with override expressions, and the double colon
(\texttt{::}) is for coercion expressions.  The \texttt{o} operator is
defined in the prelude as the function composition operator.  Note that
most operators may be overloaded, see Chapter~\ref{lexical}
(page~\pageref{lexical}) for details.

\begin{figure}[htb]
\begin{center}{\small\tt
\begin{tabular}{|l|l|} \hline
{\rm Operators} & {\rm Associativity} \\ \hline
FORALL, EXISTS, LAMBDA, IN & None \\
\verb/|/ & Left \\
\verb/|-/, \verb/|=/ & Right \\
IFF, <=> & Right \\
IMPLIES, =>, WHEN & Right \\
OR, \verb|\/|, XOR, ORELSE & Right \\
AND, \&, \&\&, \verb|/\|, ANDTHEN & Right \\
NOT, \verb|~| & None \\
=, /=, ==, <, <=, >, >=, <<, >>, <<=, >>=, <|, |> & Left \\
WITH & Left \\
WHERE & Left \\
@, \# & Left \\
@@, \#\#, || & Left \\
+, -, ++, ~ & Left \\
*, /, **, // & Left \\
- & None \\
o & Left \\
:, ::, HAS\_TYPE & Left \\
\verb|[]|, <> & None \\
\verb|^|, \verb|^^| & Left \\
` & Left \\ \hline
\end{tabular}}
\end{center}\caption{Precedence Table}\label{precedenceops}
\end{figure}
\index{precedence|)}

\index{operator symbols|(}

Many of the operators may be overloaded by the user and retain their
precedence and form (\eg\ infix).  All of the infix operators may also be
given in prefix form; \texttt{x + 1} and \texttt{+(x,1)} are semantically equivalent.  Care must be taken in redefining these operators---if the
preceding declaration ends in an expression there could be an ambiguity.
To handle this situation the language allows declarations to be terminated
with a '\texttt{;}'.  For example,
\begin{pvsex}
  AND: [state, state -> state] = (LAMBDA a,b: (LAMBDA t: a(t) AND b(t)));
  OR: [state, state -> state] = (LAMBDA a,b: (LAMBDA t: a(t) OR b(t)));
\end{pvsex}
%
without the semicolon the second declaration would be seen as an infix
\texttt{OR} and the result would be a parse error.

Another common mistake when overloading operators with predefined meanings
is the assumption that overloading, for example, {\tt IMPLIES} automatically
provides an overloading for {\tt =>}.  This is not the case---they are distinct
operators (which happen to have the same meaning by default) and not syntactic
sugar.

\index{operator symbols|)}

\section{Boolean Expressions}\label{bool-exprs}
\index{boolean expressions}

The Boolean expressions include the constants \texttt{TRUE}\index{true@{\texttt{TRUE}}} and
\texttt{FALSE}\index{false@{\texttt{FALSE}}},
the unary operator \texttt{NOT}\index{not@{\texttt{NOT}}}, and
the binary operators \texttt{AND}\index{and@{\texttt{AND}}} (also written
\texttt{ \&}\index{\&}), \texttt{OR}\index{or@{\texttt{OR}}}, \texttt{
IMPLIES}\index{implies@{\texttt{IMPLIES}}}
(\texttt{=>}\index{=>@\texttt{=>}}),
\texttt{WHEN}\index{when@{\texttt{WHEN}}}, and
\texttt{IFF}\index{iff@{\texttt{IFF}}}
(\texttt{<=>}\index{<=>@\texttt{<=>}}).  The declarations for these are in
the \texttt{booleans} prelude theory.  All of these have their standard
meaning, except for \texttt{WHEN}, which is the converse of
\texttt{IMPLIES} (\ie\ $A$ \texttt{WHEN} $B$ $\equiv$ $B$ \texttt{IMPLIES}
$A$).

Equality\index{equality} (\texttt{=}\index{=}) and
disequality\index{disequality} (\texttt{/=}\index{/=}) are declared in the
prelude theories \texttt{equalities} and \texttt{notequal}.  They are both
polymorphic, the type depending on the types of the left- and right-hand
sides.  If the types are compatible, meaning that there is a common
supertype, then the (dis)equality is of the greatest common supertype.  Otherwise it is a type
error.  For example,
\begin{pvsex}
  S,T: TYPE
  s: VAR S
  t: VAR T
  eq1: FORMULA s = t
  i: VAR \setb{}x: int | x < 10\sete
  j: VAR \setb{}x: int | x > 100\sete
  eq2: FORMULA i = j
\end{pvsex}
%
\texttt{eq1} will cause a type error---remember that \texttt{S} and \texttt{T}
are assumed to be disjoint.  \texttt{eq2} is perfectly typesafe because
they have a common supertype \texttt{int} even though the subtypes have no
elements in common; the equality simply has the value \texttt{FALSE}.

When the equality is between terms of type \texttt{bool}, the semantics
are the same as for \texttt{IFF}.  There is a pragmatic difference in the
way the PVS prover processes these operators.  Equalities may be
used for rewriting, which makes for efficient proofs but is incomplete,
\ie\ the prover may fail to find the proof of a true formula.  On the other
hand the \texttt{IFF} form is complete, but may lead to a large number of
cases.  When in doubt, use equality as the prover provides commands
that turn an equality into an \texttt{IFF}.

%The decision to disallow \texttt{eq1} is a pragmatic one; the
%utility of such a declaration is questionable, and most likely the user
%has made an error in the specification.


\section{\texttt{IF-THEN-ELSE} Expressions}
\index{if-then-else@{\texttt{IF-THEN-ELSE}}}

The \texttt{IF-THEN-ELSE} expression \texttt{IF} {\em cond\/} \texttt{THEN} {\em
expr1\/} \texttt{ELSE} {\em expr2\/} \texttt{ENDIF} is polymorphic; its type is the
common type of {\em expr1\/} and {\em expr2\/}.  The {\em cond\/} must
be of type \texttt{boolean}.  Note that the \texttt{ELSE} part is not
optional as this is an expression, not an operational statement.  The
declaration for \texttt{IF} is in the \texttt{if\_def} prelude theory.  \texttt{
IF-THEN-ELSE} may be redeclared by the user in the same way as \texttt{
AND}, \texttt{OR}, etc.  Note that only \texttt{IF} is explicitly redeclared,
the \texttt{THEN} and \texttt{ELSE} are implicit.

Any number of \texttt{ELSIF} clauses may be present; they are translated into nested
\texttt{IF-THEN-ELSE} expressions.  Thus the expression
\begin{pvsex}
  IF A THEN B
  ELSIF C THEN D
  ELSE E
  ENDIF
\end{pvsex}
%
translates to
\begin{pvsex}
  IF A THEN B
  ELSE (IF C THEN D
        ELSE E
        ENDIF)
  ENDIF
\end{pvsex}

\section{Numeric Expressions}
\index{numeric expressions}

The numeric expressions include the \emph{numerals}\index{numerals} (0, 1,
2, \ldots), the unary operator \texttt{-}\index{-}, and the binary infix
operators \texttt{\char94}\index{\^}, \texttt{+}\index{+},
\texttt{-}\index{-}, \texttt{*}\index{*}, and \texttt{/}\index{/}.  The
numerals are all of type \texttt{real}\index{real@\texttt{real}}.
The typechecker has implicit judgements on numbers; \texttt{0} is known to
be \texttt{real}, \texttt{rat}, \texttt{int} and \texttt{nat}; all others
are known to be non zero and greater than zero.  The relational operators
on numeric types are \texttt{<}\index{<@\texttt{<}}, \texttt{
<=}\index{<=@\texttt{<=}}, \texttt{>}\index{>@\texttt{>}}, and
\texttt{>=}\index{>=@\texttt{>=}}.  The numeric operators and axioms are
all defined in the prelude.  As with the boolean operators, all of these
operators may be defined on new types and retain their original
precedences.

The numerals may also be treated as names, and
overloaded.\index{overloading numberals}\index{numerals!overloading} This
is particularly useful for defining algebraic structures such as groups
and rings, where it is natural to overload `\texttt{0}' and `\texttt{1}'.
Note that such use may include actual parameters, just as for names.  Thus
\texttt{groups[int].0} or \texttt{0[int]} might refer to the group zero
instantiated with the integer carrier set.

\section{Characters and String Expressions}
\index{string expressions}

String expressions are expressions enclosed in double quotes `\texttt{"}',
for example,
\begin{pvsex}
  "This is a string"
\end{pvsex}
Strings consist of eight bit ASCII characters.  To include control
characters or characters above the usual seven bits, use a back slash
`\verb|\|', as described in the following table.

\begin{tabular}{ll}
\verb|\a| & \verb|^G| (BEL) \\
\verb|\b| & \verb|^H| (backspace) \\
\verb|\f| & \verb|^L| (form feed) \\
\verb|\n| & \verb|^J| (new line) \\
\verb|\r| & \verb|^M| (carriage return) \\
\verb|\t| & \verb|^I| (horizontal tab) \\
\verb|\v| & \verb|^K| (vertical tab) \\
\verb|\"| & double quote \\
\verb|\\| & backslash \\
\verb|\x|NN & byte with hexadecimal value NN (2 digits) \\
\verb|\|NNN & byte with decimal value NNN (3 digits) \\
\verb|\0|NNN & byte with octal value NNN (3 digits) \\
\end{tabular}

Strings are finite sequences of characters, which in turn are represented
by a datatype.
\begin{pvsex}
  character: DATATYPE
   BEGIN
    char(code:below[256]):char?
   END character
\end{pvsex}
When a string is parsed, it is internally converted to a conversion of a
list of characters to a finite sequence.  The following lemm is thus
trivially true, because both sides are actually the same term.
\begin{pvsex}
  string_rep: LEMMA
    "foo" = list2finseq(cons(char(102),
                             cons(char(111),
                                  cons(char(111), null))))
\end{pvsex}
Note that there is no special notation for characters; this is because the
\texttt{extract1} conversion will automatically convert a string of length
one to a character.  Note also that because of the \texttt{finseq\_appl}
conversion, a specific character may be extracted from a string simply by
applying it.  For example the following will typecheck
\begin{pvsex}
  f: character = "f"
  char_test: LEMMA "foo"(0) = f
\end{pvsex}  

\section{Applications}
\index{application expressions}

Function application is specified as in ordinary mathematics; thus the
application of function \texttt{f} to expression \texttt{x} is denoted \texttt{
f(x)}.  Those operator symbols that are binary functions, and their
applications, may be written in prefix or the usual infix notation.  For
example, \texttt{(3 + 5) = (2 * 4)} may be written as \texttt{=(+(3,5),
*(2,4))}.

PVS supports higher-order types, so that functions may yield functions
as values or be curried\index{curried applications}.  For example, given
\texttt{f} of type \texttt{[int -> [int, int -> int]]}, \texttt{f(0)(2,3)}
yields an \texttt{int}.

If the application involves a dependent function type then the result
type of the application is substituted for accordingly.  For example,
\begin{pvsex}
  f: [a:int, b:\setb{}x:int | a < x\sete -> \setb{}y:int | a < y & y <= b\sete]
\end{pvsex}
the application \texttt{f(2,3)} is of type \texttt{\setb{}y:int | 2 < y \& y <=
3\sete}.  This application will also lead to the subtype \tcc\ \texttt{2 < 3}.

Application and tuple expressions have a special relation, due to the
type equivalence of \texttt{[t$_1$,\ldots,t$_n$ -> t]} and \texttt{
[[t$_1$,\ldots,t$_n$] -> t]}, see Section~\ref{tuple-exprs} for details.

\section{Binding Expressions}\label{binding-expressions}
\index{binding expressions}

The binding expressions are those which create a local scope for
variables, including the quantified expressions and
$\lambda$-expressions.  Binding expressions consist of an operator, a
list of bindings, and an expression.  The operator is one of the
keywords \texttt{FORALL}\index{forall@\texttt{FORALL}}, \texttt{
EXISTS}\index{exists@\texttt{EXISTS}}, or \texttt{LAMBDA}\index{lambda@{\texttt{LAMBDA}}}.\footnote{Set
expressions are also binding expressions; see Section~\ref{set-exprs} (page~\pageref{set-exprs}).}
The bindings specify the variables bound by the operator; each variable
has an id and may also include a type or a constraint.  Here is a
contrived example:
\begin{pvsex}
  x,y,z,d,e: VAR real
  ex1: AXIOM FORALL x,y,z: (x + y) + z = x + (y + z)
  ex2: AXIOM FORALL (x,y,z: nat): x * (y + z) = (x * y) + (x * z)
  ex3: AXIOM FORALL (n: num | n /= 0): EXISTS (x | x /= 0): x = 1/n
\end{pvsex}
%
In \texttt{ex1}, variables \texttt{x}, \texttt{y}, and \texttt{z} are all of type
\texttt{real}.  In \texttt{ex2} these same variables are of type \texttt{nat},
shadowing the global declarations.  \texttt{ex3} illustrates
the use of constraints; this is equivalent to the declaration
\begin{pvsex}
  ex3: AXIOM FORALL (n: \setb{}n: num | n /= 0\sete):
               EXISTS (x: \setb{}x | x /= 0\sete): x = 1/n
\end{pvsex}

Quantified expressions\index{quantified expressions} are introduced with
the keywords \texttt{FORALL} and \texttt{EXISTS}.  These expressions are
of type \texttt{boolean}.

Lambda expressions\index{lambda expressions} denote unnamed functions.
For example, the function which adds \texttt{3} to an integer may be
written
\begin{pvsex}
  (LAMBDA (x: int): x + 3)
\end{pvsex}
%
The type of this expression is the function type \texttt{[int ->
numfield]}.\footnote{\texttt{numfield} sits between \texttt{number} and
\texttt{real}, and is where the field operators are introduced.  See
Section~{prelude-numbers}.}  In addition, when the range is \texttt{bool},
a lambda expression may be represented as a set expression; see
Section~\ref{set-exprs}.

All of the binding expressions may involve dependent
types\index{dependent types} in the bindings, \eg
\begin{pvsex}
  FORALL (x: int), (y: \setb{}z: int | x < z\sete): p(x,y)
\end{pvsex}
%
Note that in the instantiation of such an expression during a proof will
generally lead to a subtype \tcc.  For example, substituting \texttt{e$_1$} for
\texttt{x} and \texttt{e$_2$} for \texttt{y} will lead to the \tcc\ \texttt{e$_1$ <
e$_2$}.\footnote{Such \tccs\ may never be seen, as they tend to be
proved automatically during a proof; more complicated examples may be
given, for which the prover would need help from the user.  In addition,
a false \tcc\ can show up, \eg\ substituting \texttt{2} for \texttt{x} and
\texttt{1} for \texttt{y}.  This means that the corresponding expression is
not type correct.}

Constant names may be treated as binding expressions by using a
\texttt{!}  suffix.  For example,
\begin{pvsex}
foo! (x : int) : e
\end{pvsex}
is equivalent to
\begin{pvsex}
foo( LAMBDA (x : int) : e)
\end{pvsex}

\section{\texttt{LET} and \texttt{WHERE} Expressions}
\index{let expressions@{\texttt{LET} expressions}}
\index{where expressions@{\texttt{WHERE} expressions}}

\texttt{LET} and \texttt{WHERE} expressions are provided for convenience,
making some forms easier to read.  Both of these forms provide local
bindings for variables that may then be referenced in the body of the
expression, thus reducing redundancy and allowing names to be provided for common subterms.
Here are two examples:
\begin{pvsex}
  LET x:int = 2, y:int = x * x IN x + y
  x + y WHERE x:int = 2, y:int = x * x
\end{pvsex}
%
The value of each of these expressions is 6.

\texttt{LET} and \texttt{WHERE} expressions are internally translated to
applications of lambda expressions; in this case both expressions
translate to
\begin{pvsex}
  (LAMBDA (x:int) : (LAMBDA (y:int) : x + y)(x * x))(2)
\end{pvsex}
%
These translations should be kept in mind when the semantics of these
expressions is in question.

The type declaration is optional, so the above could be written as
\begin{pvsex}
  LET x = 2, y = x * x IN x + y
  x + y WHERE x = 2, y = x * x
\end{pvsex}
In this case the typechecking of these expressions depends on whether
\texttt{x} and/or \texttt{y} have been previously declared as variables.
If they have, then those delarations are used to determine the type.
Otherwise, the right-hand side of the \texttt{=} is typechecked, and if it
is unambiguous is used to determine the type of the variable.  This is 
one way in which these expressions differ from their translation.
It is usually better to either reference a variable or give the type, as
the typechecker uses the ``natural'' type of the expression as the type of
the variable, which can lead to extra \tccs.

The \texttt{LET} expression has a limited form of pattern matching over
tuples.  An example is
\begin{pvsex}
  p: VAR [int, int]
  +(p): int = LET (m, n) = p IN m + n
\end{pvsex}
which is shorter than the equivalent
\begin{pvsex}
  p: VAR [int, int]
  +(p): int = LET m = p`1, n = p`2 IN m + n
\end{pvsex}


\section{Set Expressions}\label{set-exprs}

In PVS, sets of elements of a type \texttt{t} are represented as
predicates, \ie\ functions from \texttt{t} to \texttt{bool}.  The type of a
set may be given as \texttt{[t -> bool]}, \texttt{pred[t]}, or \texttt{
setof[t]}, which are all type equivalent.\footnote{The prelude theory
\texttt{defined\_types} also defines \texttt{PRED}, \texttt{predicate}, \texttt{
PREDICATE}, and \texttt{SETOF} as alternate equivalents.}
The choice depends wholly on the intended use of the type.
Similarly, a set may be given in the form \texttt{(LAMBDA (x:\ t):\
p(x))} or \texttt{\setb{}x:\ t | p(x)\sete}; these are equivalent
expressions.\footnote{In fact, internally they are represented by the
same abstract syntax, they simply print differently.} Note that the
latter form may also represent a type---this usually causes no
confusion as the context generally makes it clear which is expected.
The usual functions and properties of sets are provided in the prelude
theory \texttt{sets}.


\section{Tuple Expressions}\label{tuple-exprs}
\index{tuple expressions}

A tuple expression of the type \texttt{[t$_1$,\ldots,t$_n$]} has the form
\texttt{(e$_1$,\ldots,e$_n$)}.  For example, \texttt{(2, TRUE, (LAMBDA x:\ x +
1))} is of type \texttt{[nat, bool, [nat -> nat]]}.  0-tuples are not
allowed, and 1-tuples are treated simply as parenthesized expressions.
The following relation holds between function types and tuple types:
\begin{pvsex}
  [[t\(\sb{1}\),\ldots,t\(\sb{n}\)] -> t] \(\equiv\) [t\(\sb{1}\),\ldots,t\(\sb{n}\) -> t]
\end{pvsex}
%
This equivalence is most important in theory parameters; it allows one
theory to take the place of many.  For example the \texttt{functions}
theory from the prelude may be instantiated by the reference
\texttt{injective?[[int,int,int],int]}.  Applications of an element \texttt{f} of
this type include \texttt{f(1,2,3)}, \texttt{f((1,2,3))}, and \texttt{f(e)},
where \texttt{e} is of type \texttt{[int,int,int]}.

\section{Projection Expressions}\label{projection-exprs}
\index{projection expressions}

The components of an expression whose type is a tuple can be accessed
using the projection operators \texttt{`1}, \texttt{`2}, \ldots or
\texttt{PROJ\_1}, \texttt{PROJ\_2}, \ldots.  The former are preferred.
Like reserved words, projection expressions are case insensitive and may
not be redeclared.  For the most part, projection expressions are
analogous to field accessors for record types.  For example,
\begin{pvsex}
  t: [int, bool, [int -> int]]
  ft: FORMULA t`2 AND t`1 > t`3(0)
  ft_deprecated: FORMULA PROJ_2(t) AND PROJ_1(t) > (PROJ_3(t))(0)
\end{pvsex}

Projection expressions may be used without an argument as long as the
context determines the tuple type involved.  For example, in the following
it is obvious what tuple type is involved.
\begin{pvsex}
  F: [[[int, bool, [int -> int]] -> bool] -> bool]
  FP: FORMULA F(PROJ_2)
\end{pvsex}
Note that the \texttt{PROJ} keyword must be used in such cases, as, e.g.,
\texttt{`2} is not an expression.  In the following example we see that
the context does not provide enough information.
\begin{pvsex}
  PP: FORMULA PROJ_2 = PROJ_2
\end{pvsex}
To deal with such situations, the syntax for projections has been extended
to allow the tuple type to be provided.
\begin{pvsex}
  PP: FORMULA PROJ_2[[int, bool, [int -> int]]] = PROJ_2
\end{pvsex}
In this case only one of the operators needs to be annotated.  This looks
like a use of actual parameters, but it is not, as the \texttt{PROJ} is
not a name, and does not belong to a theory.


\section{Record Expressions}\label{record-expressions}
\index{record expressions}

Record expressions are of the form \texttt{(\# a$_1$ := e$_1$, \ldots,
a$_n$ := e$_n$ \#)}, which has type \texttt{[\# a$_1$:\ t$_1$, \ldots,
a$_n$:\ t$_n$ \#]}, where each \texttt{e$_i$} is of type \texttt{t$_i$}.
Partial record expressions are not allowed; all fields must be given.  If
it is desired to give a partial record, declare an uninterpreted constant
or variable of the record type, and use override expressions to specify
the given record at the fields of interest.  For example,
\begin{pvsex}
  rc: [# a, b : int #]
  re: [# a, b : int #] = rc WITH [`a := 0]
\end{pvsex}

The type of a record expression is determined by the type of its
components.  Thus \texttt{(\# a := 3, b := 2 \#)} is of type \texttt{[\# a,
b: real \#]}.  This means that a record expression is never of a dependent
record type directly, though it may be used where a dependent record is
expected, and \tccs\ may be generated as a result.  For example,
\begin{pvsex}
  R: TYPE = [# a: int, b: \setb{}x: int | x < a\sete #]
  r: R = (# a := 3, b := 4 #)
\end{pvsex}
%
leads to the (unprovable) \tcc\ \texttt{4 < 3}.

Record expressions may be introduced without introducing the record type
first, and the type of a record expression is determined by its
components, independently of any previously declared record type.  For
this reason record types do not automatically generate associated accessor
functions.

\section{Record Accessors}

The components of an expression of a record type are accessed using the
corresponding field name.  There are two forms of access.  For example if
\texttt{r} is of type \texttt{[\# x, y: real \#]}, the x-component may be
accessed using either \texttt{r`x} or \texttt{x(r)}.  The first form is
preferred as there is less chance for ambiguity.

As noted above, accessors are not stand-alone functions.  However, you can
define your own functions to provide this capability, and even use the
same name.  For example:
\begin{pvsex}
  point: TYPE = [# x, y: real #]
  x(p:point): real = p`x
  y(p:point): real = p`y
\end{pvsex}
Now \texttt{x} and \texttt{y} may be provided wherever a function is
expected.  Note that this means that a subsequent expression of the form
\texttt{x(p)} could be ambiguous, but the record field accessor is always
preferred, so in practice such ambiguities don't arise.

\section{Cotuple Expressions}\label{cotuple-expressions}
\index{cotuple expression}

Elements of cotuple types \texttt{[t$_1$ + \ldots + t$_n$]} are constructed
with the \emph{injection} operators \texttt{IN\_$i$} of type
\texttt{[t$_i$ -> [t$_1$ + \ldots + t$_n$]]}.  Thus if $e$ is of type
\texttt{t$_i$}, \texttt{IN\_$i$($e$)} is of the cotuple type.  If $x$ is
an element of a cotuple type, \texttt{IN?\_$i$($x$)} is a boolean that
tests if $x$ belongs to the $i^{th}$ component, and if it does,
\texttt{OUT\_$i$($x$)} returns the associated value of type
\texttt{t$_i$}.  Note that this is similar to a datatype of the form
\begin{pvsex}
  cotup: DATATYPE
   BEGIN
    IN_1(OUT_1: t\(\sb{1}\)): IN?_1
    \(\cdots\)
    IN_\(n\)(OUT_\(n\): t\(\sb{n}\)): IN?_\(n\)
   END cotup
\end{pvsex}
The differences are that cotuples are not recursive, do not generate all
the functions and axioms associated with datatypes, and allow for any
number of component types---using datatypes a new one would have to be
given for each arity.

The analogy works also for the \texttt{CASES} expression described in
Section~\ref{cases-expressions}.  This allows access to the values of a
cotuple element.  It has the form
\begin{pvsex}
  CASES \(e\) OF
    IN_1(x1): f\(\sb{1}\)(x1),
    \vdots
    IN_\(n\)(x\(n\)): f\(\sb{n}\)(x\(n\))
  ENDCASES
\end{pvsex}
where each \texttt{f$_i$} is an expression of type \texttt{[t$_i$ ->
$T$]}, and the common return type $T$ is the type of the \texttt{CASES}
expression.  For example, if \texttt{x} is of type \texttt{[int + bool +
[int -> int]}, the following expression will return a boolean value.
\begin{pvsex}
  CASES x OF
    IN_1(i): i > 0,
    IN_2(b): NOT b,
    IN_3(f): FORALL (n: int): f(f(n)) = f(n)
  ENDCASES
\end{pvsex}
If there are any missing components in the \texttt{CASES} expression, a
\emph{cases \tcc}\index{cases TCC}\index{TCC!cases} will be generated
stating that the cotuple expression must be one of the given selections,
unless there is an \texttt{ELSE} selection.

Like the projection operators \texttt{PROJ\_$i$}, the \texttt{IN\_$i$},
\texttt{OUT\_$i$} and \texttt{IN?\_$i$} operators make be disambiguated by
adding the cotuple type reference to the operator, for example,
\texttt{IN\_2[int + int](3)} or \texttt{IN?\_1[coT]}.  Note that although
they have the form of actual parameters, they are not, as these operators
are built in and not associated with any theory.  Also, for brevity, only
the cotuple type is given, not the full type of the operator.  There are a
number of axioms associated with cotuples that are built in to the PVS
typechecker and prover.


\section{Override Expressions}
\index{override expression}
\index{update expression}
\index{with expression}

Functions, tuples, records, and datatype elements may be ``modified'' by
means of the override expression.  The result of an override expression is
a function, tuple, record, or datatype element that is exactly the same as
the original, except that at the specified arguments it takes the new
values.  For example,
\begin{pvsex}
  identity WITH [(0) := 1, (1) := 2]
\end{pvsex}
%
is the same function as the \texttt{identity} function (defined in the
prelude) except at argument values \texttt{0} and \texttt{1}.  This is exactly
the same expression as either of
\begin{pvsex}
  (identity WITH [(0) := 1]) WITH [(1) := 2] {\rm or}
  (LAMBDA x: IF x = 1 THEN 2 ELSIF x = 0 THEN 1 ELSE identity(x))
\end{pvsex}

This order of evaluation ensures that functions remain total, and allows
for the possibility of expressions such as
\begin{pvsex}
  identity WITH [(c) := 1, (d) := 2]
\end{pvsex}
where \texttt{c} and \texttt{d} may or may not be equal.  If they are
equal, then the value of the override expression at the common argument is
\texttt{2}.

More complex overrides can be made using nested arguments; for example,
\begin{pvsex}
  R: TYPE = [# a: int, b: [int -> [int, int]] #]
  r1: R
  r2: R = r1 WITH [`a := 0, `b(1)`2 := 4]
\end{pvsex}
{\tt r2} is equivalent to
\begin{pvsex}
  (# a := 0,
     b := LAMBDA (x: int):
           IF x = 1
           THEN (r1`b(x)`1, 4)
           ELSE r2`b(x)
           ENDIF #)
\end{pvsex}

Updating a datatype element amounts to updating the accessor(s) associated
with a constructor.  For example, if \texttt{lst} is of type
\texttt{(cons?[nat])}, then \texttt{lst WITH [`car := 3]} returns a list
that is the same as \texttt{lst}, but whose first element is \texttt{3}.
If \texttt{lst} is given type \texttt{list[nat]}, then the same override
expression generates a \tcc\ obligation to prove that \texttt{lst} is a
\texttt{cons?}.  Because accessors may be both dependent and overloaded,
\tccs\ may get complicated.  For example,
\begin{pvsex}
  dt: DATATYPE
  BEGIN
   c0: c0?
   c1(x: int, a: \setb{}z: (even?) | z > x\sete, b: int): c1?
   c2(x: int, a: \setb{}n: nat | n > x\sete, c: int): c2?
  END dt
\end{pvsex}
If \texttt{d} is of type \texttt{dt}, the update expression \texttt{d WITH
[a := y]} leads to the \tcc
\begin{pvsex}
  f1_TCC1: OBLIGATION
    (c1?(d) AND even?(y) AND y > x(d)) OR
     (c2?(d) AND y >= 0 AND y > x(d));
\end{pvsex}

Another form of override expression is the maplet, indicated using
\texttt{|->} in place of \texttt{:=}.  This is used to extend the domain
of the corresponding element; for example, if \texttt{f:[nat -> int]} is
given, then \texttt{f WITH [(-1) |-> 0]} is a function of type
\texttt{[\setb{}i:int | i >= 0 OR i = -1\sete -> int]}.  This is especially useful
with dependent types, see Section~\ref{dependent-types}.  Domain extension
is also possible for record and tuple types; for example, \texttt{r1 WITH
[`c |-> 3]} is of type \texttt{[\# a:\ int, b:\ [int -> [int,int]], c:\ int
\#]}, and if \texttt{t1} is of type \texttt{[int, bool]}, then \texttt{t1
WITH [`3 |-> 1]} is of type \texttt{[int, bool, int]}.  It is an error to
extend a tuple type such that gaps are left, so \texttt{t1 WITH [`4 |->
1]} is illegal, though \texttt{t1 WITH [`3 |-> 1, `4 |-> 1]} is allowed.
Gaps would also be left if nested arguments were given, so \texttt{r1 WITH
[`c(0) |-> 0]} is also illegal.  It would have to be given as \texttt{r1
WITH [`c := LAMBDA (x:\ int):\ IF x = 0 THEN 0 ELSE $\cdots$ ENDIF]}, where
the gap $\cdots$ now has to be filled in.  Domain extension is not
possible for datatype elements, as a new datatype theory would need to be
generated for each such extension.

In the past, the two forms of assignment (using \texttt{:=} and
\texttt{|->}) were merely alternative notation, and domains would be
extended automatically whenever the typechecker could not determine that
the argument belonged to the domain.  In most cases, extending the domain
unnecessarily is harmless.  However, when terms get large, the types can
get cumbersome, slowing down the system dramatically.  Even worse, when
domains are extended and matched against a rewrite rule with the original
type, the match can fail, and the automatic rewrite will not be triggered.
For this reason, it is always best to use the maplet on function types
only when actually extending the domain.

\section{Coercion Expressions}\label{coercions}

Coercion expressions are of the form \texttt{expr ::\ type-expr}, indicating
that the expression \texttt{expr} is expected to be of type \texttt{
type-expr}.  This serves two purposes.  First, although PVS allows a
liberal amount of overloading, it cannot always disambiguate things for
itself, and coercion may be needed.  For example, in
\begin{pvsex}
  foo: int
  foo: [int -> int]
  foo: LEMMA foo = foo::int
\end{pvsex}
%
the coercion of \texttt{foo} to \texttt{int} is needed, because otherwise the
typechecker cannot determine the type.  Note that only one of the sides
of the equation needs to be disambiguated.

The second purpose of coercion is as an aid to typechecking; by
providing the expected type in key places within complex expressions,
the resulting \tccs\ may be considerably simplified.

% Master File: language.tex

\section{Tables}
\index{tables|(}

Many expressions are easier to express and to read when presented in
tabular form, as described in~\cite{Heitmeyer94:SCR-checks,Parnas92}.
There are many types of tables, ten different interpretations are
described in~\cite{Parnas92} alone.  Rather than provide support for all
these tables, we chose to support a simple form of table initially,
providing extensions in later versions of PVS as the need arises.

PVS provides a form of table expressions that allows simple
tables\footnote{In Parnas' terms~\cite{Parnas92}, these tables are
\emph{normal function tables} of one or two dimensions.} to be presented,
and supports \emph{table consistency conditions}\index{table consistency}.
One of the consistency conditions (the \emph{Mutual Exclusion
Property}\index{mutual exclusion property} or
\emph{disjointness}\index{disjointness property}) requires the pairwise
conjunction of a set of formulas to be false; another (the \emph{Coverage
Property}\index{coverage property}) requires the disjunction of a set of
formulas to be true.

Tables are supported by means of the more generic \texttt{COND}
expression, which provides the semantic foundation.  In the following
sections, we first describe the \texttt{COND} expression, and then
\texttt{TABLE} expressions.

\subsection{\texttt{COND} Expressions}
\index{cond expressions@\texttt{COND} expressions|(}

The \texttt{COND} construct is a multi-way extension to the polymorphic
\texttt{IF-THEN-ELSE} construct of PVS\@.  Its form is
\begin{pvsex}
  COND
      be\_1 -> e\_1,
      be\_2 -> e\_2,
        \ldots
      be\_n -> e\_n
  ENDCOND
\end{pvsex}
where the \texttt{be\_}i's are boolean expressions, and the \texttt{e\_}i's are
expressions of some common supertype.  It is required that the \texttt{be\_}i's are pairwise disjoint and that their disjunction is a tautology:
these constraints are generated as \emph{disjointness}\index{disjointness
TCC}\index{TCC!disjointness} and \emph{coverage}\index{coverage
TCC}\index{TCC!coverage} TCCs that must be discharged before PVS will
consider a \texttt{COND} expression fully type-correct.
\begin{pvsex}
  foo_TCC1: OBLIGATION NOT (be\_1 AND be\_2) AND\ldots{}AND NOT (be\_n-1 AND be\_n)
  foo_TCC2: OBLIGATION be\_1 OR be\_2 OR\ldots{}OR be\_n
\end{pvsex}

Notice that a \texttt{COND} expression with $n$ clauses generates $O(n^2)$
clauses in its disjointness TCC\@.  

Assuming its associated TCCs are discharged, the schematic \texttt{COND}
shown above is equivalent to the following \texttt{IF-THEN-ELSE} form,
which is its semantic definition.

\begin{pvsex}
  IF be\_1 THEN e\_1
  ELSIF be\_2 THEN e\_2
          \ldots
  ELSIF be\_n-1 THEN e\_n-1
  ELSE e\_n
\end{pvsex}

The \texttt{COND} may include an \texttt{ELSE} clause:
\begin{pvsex}
  COND
      be\_1 -> e\_1,
      be\_2 -> e\_2,
        \ldots
      ELSE -> e\_n
  ENDCOND
\end{pvsex}
This form does not require the coverage TCC and is equivalent to the
\texttt{IF-THEN-ELSE} form shown above.

Using \texttt{COND}, we can translate the following tabular
specification of the \emph{sign} function
\[
\begin{array}{|c||c|c|c|}
\hline
 & x<0 & x=0 & x>0\\
\hline
\emph{sign}(x) & -1 & 0 & 1\\
\hline
\end{array}
\]
into
\begin{pvsex}
  sign(x): int = COND
                    x<0 -> -1, 
                    x=0 -> 0,
                    x>0 -> 1
                 ENDCOND
\end{pvsex}

Two dimensional tables can be generated by nested \texttt{COND}s.  For
example, the following table defining the value for \texttt{safety\_injection}
\begin{center}
\begin{tabular}{|c||c|c|}
\hline
modes & \multicolumn{2}{c|}{conditions} \\
\hline
\hline
normal & false & true\\
\hline
low & not overridden & overridden \\
\hline
voter\_failure & true & false\\
\hline
\hline
safety\_injection & on & off\\
\hline
\end{tabular}
\end{center}
can be represented as
\begin{pvsex}
  safety\_injection(mode, overridden): on\_off = 
    COND
      mode=normal -> off,
      mode=low -> (COND NOT overridden -> on, overridden -> off ENDCOND),
      mode=voter\_failure -> on
    ENDCOND
\end{pvsex}

Notice that \texttt{mode=low} provides the ``left context'' used in
generating the TCCs for the nested \texttt{COND}.  This causes some
redundancy in highly structured two dimensional tables as the
following example shows.
\begin{center}
\begin{tabular}{|c||c|c|}
\hline
 & \multicolumn{2}{c|}{input}\\
\cline{2-3}
state & x & y\\
\hline
\hline
a & a & b\\
\hline
b & b & b\\
\hline
\end{tabular}
\end{center}
This translates to
\begin{pvsex}
  COND
    state=a -> COND input=x -> a,input=y -> b ENDCOND,
    state=b -> COND input=x -> b,input=y -> b ENDCOND
  ENDCOND
\end{pvsex}
The coverage TCCs generated for the two inner \texttt{COND}s will have the form
\begin{pvsex}
   foo\_TCC2 : OBLIGATION state=a IMPLIES input=x OR input=y
   foo\_TCC3 : OBLIGATION state=b IMPLIES input=x OR input=y
\end{pvsex}
whereas, because of the disjointness and coverage of $\{$\texttt{a}, \texttt{b}$\}$, the correct TCC is the simpler form
\begin{pvsex}
   foo\_TCC: OBLIGATION input=x OR input=y
\end{pvsex}
The source of the error here is that our translation of the original
table is too simple-minded.  A better translation is the following.
\begin{pvsex}
  LET
    x1 = COND input=x -> a, input=y -> b ENDCOND,
    x2 = COND input=x -> b, input=y -> b ENDCOND
  IN
    COND state=a -> x1, state=b -> x2 ENDCOND
\end{pvsex}
And this generates the correct TCCs.

Note that if the \texttt{be\_i}'s are members of an enumerated type, then
the standard PVS \texttt{CASES} construct should be used instead of \texttt{COND}, since there is no need to generate TCCs in these cases.  For
example, if in the previous example $\{$ \texttt{a}, \texttt{b} $\}$ and
$\{$ \texttt{x}, \texttt{y} $\}$ had been enumerated types, then the table
could have been expressed as
\begin{pvsex}
  CASES state OF
    a: CASES input OF x: a, y: b ENDCASES,
    b: CASES input OF x: b, y: b ENDCASES
  ENDCASES
\end{pvsex}
and no TCCs would be generated.

If the \texttt{be\_i}'s are all equalities with the same left hand side,
whose right hand sides are ground arithmetic terms (involving only
numbers, \texttt{+}, \texttt{-}, \texttt{*}, \texttt{/}) then the typechecker
directly checks for coverage and disjointness so no \tccs\ are generated
in this case.

\index{cond expressions@\texttt{COND} expressions|)}


\subsection{Table Expressions}

The \texttt{COND} and \texttt{CASES} constructs (see datatypes on page~\pageref{datatypes}) provide the semantic
foundation for our treatment of tables in PVS; for convenience, we
also provide a \texttt{TABLE} construct that provides more attractive
syntax for the important special cases of regular one and
two-dimensional tables.  The example above can be written in the
alternative form.
\begin{pvsex}
   TABLE
%               ---------------------
               |[ input=x | input=y ]|
%     -------------------------------
      | state=a |    a    |    b    ||
%     -------------------------------
      | state=b |    b    |    b    ||
%     -------------------------------
   ENDTABLE
\end{pvsex}
This will translate internally into the \texttt{LET} and \texttt{COND} form
shown earlier.  Note that the horizontal lines are simply PVS
comments.\footnote{The \LaTeX\ generation translates these constructs into attractively
typeset tables.  See the PVS System Guide~\cite{PVS:userguide} for details.}

The row and column headers to a \texttt{TABLE} construct are arbitrary
boolean expressions.   In cases where the expressions are all of the
form \texttt{id=x}, the \texttt{id} can be factored out to produce simpler
tables of the following form.
\begin{pvsex}
   TABLE state,    input
%               ----------
               |[ x | y ]|
%        -----------------
         |  a   | a | b ||
%        -----------------
         |  b   | b | b ||
%        -----------------
   ENDTABLE
\end{pvsex}
In this form, as the headings are enumeration constructs this is
internally represented as a \texttt{CASES} construct, and so generates
no \tccs\ (the previous version generates 5 \tccs).

One-dimensional tables can be presented in both ``horizontal'' and
``vertical'' forms.  The \emph{sign} function example can be
presented as a ``vertical'' table as follows.
\begin{pvsex}
  sign(x): int = TABLE 
%                ------------
                |[ x<0 | -1 ]|
%                ------------
                 | x=0 |  0 ||
%                ------------
                 | x>0 |  1 ||
%                ------------
   ENDTABLE
\end{pvsex}

And as a horizontal one as follows.
\begin{pvsex}
  sign(x): int = TABLE 
%                --------------------
                 |[ x<0 | x=0 | x>0 ]|
%                -------------------
                 |   -1 |  0  |  1  ||
%                --------------------
   ENDTABLE
\end{pvsex}


A more complex two-dimensional example is provided by the mode
transition tables used in SCR\@.  These have the following form.
\[
\begin{array}{|c|c|c|}
\hline
\mbox{current mode} & \mbox{Event} & \mbox{New Mode} \\
\hline
m_{1} & e_{1,1} & m_{1,1} \\
    & e_{1,2} & m_{1,2} \\
    & \ldots & \ldots\\
    & e_{1,k_1} & m_{1,k_1} \\
\hline
m_{2} & e_{2,1} & m_{2,1} \\
    & e_{2,2} & m_{2,2} \\
    & \ldots & \ldots\\
    & e_{2,k_2} & m_{2,k_2} \\
\hline
\ldots  & \ldots & \ldots\\  
\hline
m_{p} & e_{p,1} & m_{p,1} \\
    & e_{p,2} & m_{p,2} \\
    & \ldots & \ldots\\
    & e_{p,k_p} & m_{p,k_p} \\
\hline
\end{array}
\]
And translate to the following form.
\smaller
\begin{center}
\begin{pvsex}
TABLE mode
%--------------------------------
     |  m\_1 | TABLE event
                  | e\_1,1 | m\_1,1 ||
                  | e\_1,2 | m\_1,2 ||
           \ldots
                  | e\_1,k1| m\_1,k1||
             ENDTABLE ||
%--------------------------------
     |  m\_2 | TABLE event
                  | e\_2,1 | m\_2,1 ||
                  | e\_2,2 | m\_2,2 ||
           \ldots
                  | e\_2,k2| m\_2,k2||
             ENDTABLE ||
%--------------------------------
        \ldots
%--------------------------------
     |  m\_p | TABLE event
                  | e\_p,1 | m\_p,1 ||
                  | e\_p,2 | m\_p,2 ||
           \ldots
                  | e\_p,kp| m\_p,kp||
             ENDTABLE ||
%--------------------------------
ENDTABLE
\end{pvsex}
\end{center}

The last row or column heading in a table may contain the \texttt{ELSE}
keyword, which has the same meaning as for the corresponding \texttt{COND}
or \texttt{CASES} expression.

The table may also have blank entries (except in the headings).  These
represent illegal values; in other words the entry may never be reached.
This is represented by generation of a TCC indicating that the
formulas corresponding to the row and column headings for that entry
cannot both be true.

Note that this is different than having ``don't care'' values.  If you
want to add don't care entries, make sure that you use an array; the table
\begin{pvsex}
DC: int
TABLE
         |[ x < 0 | x = 0 | x > 0 ]|
  | y < 0 |   1   |   0   |   DC  ||
  | y = 0 |  DC   |   2   |    3  ||
  | y > 0 |   -2  |  DC   |    0  ||
  ENDTABLE
\end{pvsex}
may seem like any integer may appear in place of \texttt{DC}, but it must
always be the same integer, which is probably not intended.  The right way
to do this is
\begin{pvsex}
DC(n:nat): int
TABLE
         |[ x < 0 | x = 0 | x > 0 ]|
  | y < 0 |   1   |   0   | DC(2) ||
  | y = 0 | DC(0) |   2   |    3  ||
  | y > 0 |   -2  | DC(1) |    0  ||
  ENDTABLE
\end{pvsex}

\index{tables|)}


\index{expression|)}

% Document Type: LaTeX
% Master File: language.tex

\chapter{Theories}\label{theories}
\index{theories}

Specifications in \pvs\ are built from \emph{theories}, which provide
genericity, reusability, and structuring.  \pvs\ theories may be
parameterized.  A theory consists of a \emph{theory
identifier}, a list of formal \emph{parameters}, an \texttt{EXPORTING}
clause, an \emph{assuming part}, a \emph{theory body}, and an ending
id.  The syntax for theories is shown in Figure~\ref{bnf-theory}.

\pvsbnf{bnf-theory}{Theory Syntax}

Everything is optional except the identifiers and the keywords.  Thus
the simplest theory has the form
\begin{pvsex}
  triv : THEORY
    BEGIN
    END triv
\end{pvsex}

The formal parameters, assuming, and theory body consist of declarations
and \emph{importings}.  The various declarations are described in
Section~\ref{declarations}.  In this section we discuss the restrictions
on the allowable declarations within each section, the formal parameters,
the assuming part, and the exportings and importings.

The \texttt{groups} theory below illustrates these concepts.  It views a
group as a 4-tuple consisting of a type \texttt{G}, an identity element
\texttt{e} of \texttt{G}, and operations \texttt{o}\footnote{Recall that
\texttt{o} is an infix operator.} and \texttt{inv}.  Note the use of the
type parameter \texttt{G} in the rest of the formal parameter list.  The
assuming part provides the group axioms.  Any use of the \texttt{groups}
theory incurs the obligation to prove all of the \texttt{ASSUMPTION}s.
The body of the \texttt{groups} theory consists of two theorems, which can
be proved from the assumptions.

\pvstheory{groups-alltt}{Theory \texttt{groups}}{groups-alltt}

\section{Theory Identifiers}

The theory identifier introduces a name for a theory; as described in
Section~\ref{names}, this identifier can be used to help disambiguate
references to declarations of the theory.

In the \pvs\ system, the set of theories currently available to the
session form a \emph{context}.  Within the context theory names must be
unique.  There is an initial context available, called the prelude
(described in Appendix~\ref{prelude}), that provides, among other things,
the Boolean operators, equality, and the \texttt{real}, \texttt{rational},
\texttt{integer}, and \texttt{naturalnumber} types and their associated
properties.  The only difference between the prelude and user-defined
theories is that the prelude is automatically imported in every theory,
without requiring an explicit \rsv{IMPORTING} clause.

The end identifier must match the theory identifier, or an error is
signaled.


\section{Theory Parameters}\label{parameters}
\index{theory parameters|(}
\index{formal parameters|see{theory parameters}}

The theory parameters allow theory schemas to be specified.  This
provides support for \emph{universal polymorphism}\index{polymorphism}

Theory parameters may be types, subtypes, or constants, and importings may
be interspersed.  Theory parameters must have unique identifiers.  The
parameters are ordered, allowing later parameters to refer to earlier
ones.  This is another form of dependency, akin to dependent types (see
Section~\ref{dependent-types}).  A theory is \emph{ instantiated} from
within another theory by providing \emph{actual parameters}\index{actual
parameters} to substitute for the formals.  Actual parameters may occur in
importings, exportings, theory declarations, and names.  In each case they
are enclosed in braces (\texttt{[} and \texttt{]}) and separated with
commas.

The actuals must match the formals in number, kind, and (where
applicable) type.  In this matching process the importings, which
must be enclosed in parentheses, are ignored.  For example, given the
theory declaration
\begin{pvsex}
  T [t: TYPE,
     subt: TYPE FROM t
     (IMPORTING orders[subt]) <=: (partial_order?),
     c: subt,
     d: \setb{}x:subt | c <= x\sete]
\end{pvsex}
a valid instance has five actual parameters; an example is
\begin{pvsex}
  T[int, \setb{}x:nat | x < 10\sete, <=, 5, 6]
\end{pvsex}
%
Note that the matching process may lead to the generation of \emph{actual}
\tccs.\index{actual TCC}\index{TCC!actual}

\index{theory parameters|)}


\section{Importings and Exportings}\label{importings}

The importing and exporting clauses form a hierarchy, much like the
subroutine hierarchy of a programming language.

Names declared in a theory may be made available to other theories in the
same context by means of the \texttt{EXPORTING} clause.  Names exported by
a given theory may be imported into a second theory by means of the
\texttt{IMPORTING} clause.  Names that are exported from one theory are
said to be \emph{visible} to any theory which uses the given theory.  In
this section we describe the syntax of the \texttt{EXPORTING} and
\texttt{IMPORTING} clauses and give some simple examples.

\pvsbnf{bnf-exporting}{Importing and Exporting Syntax}


\subsection{The \texttt{EXPORTING} Clause}
\index{exporting@\texttt{EXPORTING}|(}

The \texttt{EXPORTING} clause specifies the names declared in the theory
which are to be made available to any theory \texttt{IMPORTING} it.  It may
also specify instances of the theories which it imported to be exported.
The syntax of the \texttt{EXPORTING} clause is given in
Figure~\ref{bnf-exporting}.

\noindent
The \texttt{EXPORTING} clause is optional; if omitted, it defaults to
\begin{alltt}
  EXPORTING ALL WITH ALL
\end{alltt}

Any declared name may be exported except for variable declarations and formal parameters.
When \texttt{ALL} is specified for the \emph{ExportingNames}, all
entities declared in the theory aside from the variables are exported.
If a list of names is specified, then these are exported.  Finally, when
a list of names follows \texttt{ALL BUT}, all names aside from these are exported.

Since PVS supports overloading, it is possible that the exported name will
be ambiguous.  Such names may be disambiguated by including the type, if
it is a constant, or by including one of the keywords \texttt{TYPE} or
\texttt{FORMULA}.  The keyword \texttt{TYPE} is used for any type
declaration, and \texttt{FORMULA} is used for any formula declaration
(including \texttt{AXIOM}s, \texttt{LEMMA}s, etc.)  If not disambiguated,
all declarations (except variables and formals) with the specified id will
be exported.

When names are specified they are checked for \emph{completeness}.
This means that when a name is exported all of the names on which the
corresponding declaration(s) depend must also be exported.  Thus, for
example, given the following declarations
\begin{alltt}
  sometype: TYPE
  someconst: sometype
\end{alltt}
it would be illegal to export \texttt{someconst} without also exporting
\texttt{sometype}.  Note that this check is unnecessary if exporting
\texttt{ALL} without \texttt{BUT}.

In some cases it is desirable (or necessary for completeness) to
export some of the instances of the theories which are used by the
given theory.  This is done by specifying a \texttt{WITH} subclause as a
part of the \texttt{EXPORTING} clause.  The \texttt{WITH} subclause may be
\texttt{ALL}, indicating that all instances of theories used by the given
theory are exported.  If \texttt{CLOSURE} is specified, then the
typechecker determines the instances to be exported by a \emph{
completion analysis}\index{completion analysis} on the exported
names.  Completion analysis determines those entities that are
directly or indirectly referenced by one of the exported
names.\footnote{Proofs are not used in completion analysis.} Finally,
a list of theory names may be given; in this case the theory names
must be complete in the sense that if an exported name refers to an
entity in another theory instance, then that theory instance must be
exported also.  Other theory instances may also be exported even if
not actually needed for completeness in this sense.  The \texttt{WITH}
subclause may only reference theory instances, \ie\ theory names with
actuals provided for all of the corresponding formal parameters.

As a practical matter, it is probably best not to include an
\texttt{EXPORTING} clause unless there is a good reason.  That way
everything that is declared will be visible at higher levels of the
\texttt{IMPORTING} chain.

\index{exporting@\texttt{EXPORTING}|)}

\subsection{\texttt{IMPORTING} Clauses}
\index{importings|(}

\texttt{IMPORTING} clauses import the visible names of another theory.
\texttt{IMPORTING} clauses may appear in the formal parameters list, the
assuming part, or the theory part of a theory.  In addition, theory
abbreviations implicitly import the theory name that they abbreviate (see
Section~\ref{theory-abbreviations}).

The names appearing in an \texttt{IMPORTING} or theory abbreviation specifies a
theory and optionally gives an instance of that theory, by providing
actual parameters corresponding to the formal parameters of the theory
used.  \texttt{IMPORTING}s are cumulative; entities made visible at some point
in a theory are visible to every declaration following.

An \texttt{IMPORTING} with actual parameters provided is said to be a \emph{
theory instance}.\index{theory instance} We use the same terminology for
an \texttt{IMPORTING} that refers to theory that has no formal parameters.
Otherwise it is referred to as a \emph{generic}\index{generic reference}
reference.

A single theory may appear in any number of \texttt{IMPORTING}s of another
theory, both instantiated and generic.  Obviously, any time there is
more than one \texttt{IMPORTING} of a given theory there is a chance for
ambiguity.  Section~\ref{names} discusses such ambiguities, explaining
how the system attempts to resolve them and how the user can
disambiguate in situations where the system cannot.

An \texttt{IMPORTING} forms a relation between the theory containing the
\texttt{IMPORTING} and the theory referenced.  The transitive closure of
the \texttt{IMPORTING} relation is called the \emph{importing chain} of a
theory.  The importing chain must form a directed acyclic graph; hence a
theory may not end up importing itself, directly or indirectly.
\index{importings|)}


\section{Theory Abbreviations}\label{theory-abbreviations}
\index{theory abbreviations}

A theory abbreviation introduces a new name for a theory instance,
providing an alternate means for referring to the instance.  For
example, given the declaration
\begin{pvsex}
  fsets: THEORY = sets[[integer -> integer]]
\end{pvsex}
where \texttt{sets} is a theory in which the function \texttt{member} is
declared, the name \texttt{sets[[integer -> integer]].member} may instead
be written as \texttt{fsets.member}.  The body of a theory abbreviation
must refer to a theory of the current context or the prelude, and the
actual parameters must be compatible (in number, kind, and type) with
the corresponding theory parameters\index{theory parameters}.

In addition to providing an abbreviation, such declarations do an
implicit \texttt{IMPORTING}\index{IMPORTING@\texttt{IMPORTING}} of the theory instance.
Theory abbreviations may not be exported.


\section{Assuming Part}\label{assuming}

The assuming part consists of top-level declarations and \texttt{IMPORTING}s.
The assuming part precedes the theory part, so the
theory part may refer to entities declared in the assuming part.  The
grammar for the assuming part is given in Figure~\ref{bnf-assuming}.

\pvsbnf{bnf-assuming}{Assuming Syntax}

The primary purpose of the assuming part is to provide constraints on the
use of the theory, by means of \texttt{ASSUMPTION}s.  These are formulas
expressing properties that are expected to hold of any instance of the
theory.  They are generally stated in terms of the formal parameters, and
when instantiated they become \emph{assuming} \tccs\ \index{assuming
TCC}\index{TCC!assuming} \texttt{OBLIGATION}s which must be discharged.  For
example, given the theory \texttt{groups} above, the importing
\begin{pvsex}
  IMPORTING groups[int, 0, +, -]
\end{pvsex}
generates the following obligations
\begin{pvsex}
  % Assuming TCC generated for  groups[int, 0, +, -]
  IMPORTING1_TCC1: OBLIGATION
    (FORALL (a: int), (b: int), (c: int):
         +(a, (+(b, c))) = +((+(a, b)), c))
  
  % Assuming TCC generated for  groups[int, 0, +, -]
  IMPORTING1_TCC2: OBLIGATION (FORALL (a: int):
         +(0, a) = a AND +(a, 0) = a)
  
  % Assuming TCC generated for  groups[int, 0, +, -]
  IMPORTING1_TCC3: OBLIGATION (FORALL (a: int):
         +(-(a), a) = 0 AND +(a, -(a)) = 0)
\end{pvsex}

Except for the variable declarations, the declarations of the assumings
are all externally visible.  
  
The dynamic semantics of an \emph{assuming} part of a theory is as
follows.  Internal to the theory, assumptions are used exactly as axioms
would be used.  Externally, for each import of a theory, the assumptions
have to be discharged (i.e., proved) with the actual parameters replacing
the formal parameters.  Note that in terms of the proof chain, every proof
in a theory depends on the proofs of the assumptions.

Assuming \tccs\ are generated when a theory is instantiated, which may or
may not occur when it is imported.  Thus if a theory with assumptions is
imported generically, the assuming \tccs\ are not generated until some
reference is instantiated.  If a theory instance is imported, then the
assuming \tccs\ precede the importing in the dynamic semantics.  Note that
this may not make sense, as the assumings may refer to entities that are
not visible until after the theory is imported.  Thus the following is
illegal.
\begin{pvsex}
  assuming_test[n: nat, m: {x:int | x < n}]: THEORY
  BEGIN
   ASSUMING
    rel_prime?(x, y: int): bool = EXISTS (a, b: int): x*a + y*b = 1
    rel_prime: ASSUMPTION rel_prime?(n,m)
   ENDASSUMING
  END assuming_test

  assimp: THEORY
  BEGIN
   IMPORTING assuming_test[4, 2]
  END assimp
\end{pvsex}
And leads to the following error message.
\begin{pvsex}
  Error: assumption refers to 
    assuming_test[4, 2].rel_prime?,
  which is not visible in the current theory
\end{pvsex}
There are a number of ways to solve this problem.  Perhaps the simplest is
to first import the theory generically, then import the instance.
\begin{pvsex}
   IMPORTING assuming_test
   IMPORTING assuming_test[4, 2]
\end{pvsex}
Now the reference to \texttt{rel\_prime?} makes sense in the assuming
\tcc\ generated for the second importing.

In this case, another solution is to simply define \texttt{rel\_prime?} as
a \emph{macro} (see Section~\ref{macro-declarations}).
\begin{pvsex}
  rel_prime?(x, y: int): MACRO bool = EXISTS (a, b: int): x*a + y*b = 1
\end{pvsex}
Of course, this will not work if the declaration in question is a
recursive or inductive definition.

Another solution is to provide the declaration in a theory that is
imported in both the theory with the assuming and the theory importing
that theory.
\begin{pvsex}
  rel_prime[y:int]: THEORY
  BEGIN
   rel_prime?(x: int): bool = EXISTS (a, b: int): x*a + y*b = 1
  END assth2

  assuming_test[n: nat, m: {x:int | x < n}]: THEORY
  BEGIN
   ASSUMING
    IMPORTING rel_prime[m]
    rel_prime: ASSUMPTION rel_prime?(n)
   ENDASSUMING
  END assuming_test2

  assuming_imp: THEORY
  BEGIN
   IMPORTING rel_prime[2], assuming_test[4, 2]
  END assuming_imp
\end{pvsex}
Now the reference to \texttt{rel\_prime?} in the assuming \tcc\ associated
with \texttt{assuming\_test[4, 2]} is the same as the previously imported
instance, so there is no problem.  In the theory \texttt{assuming\_imp},
\texttt{rel\_prime} may also be imported generically.  However, if
\texttt{rel\_prime} is not imported, or is imported with a different
parameter (e.g., \texttt{rel\_prime[3]}) then the above error is produced.


\section{Theory Part}

The theory part consists of top-level declarations and \texttt{IMPORTING}s.
Declarations are ordered; references may not be made to declarations
which occur later in the theory.  The theory part usually contains the
main body of the theory.  Assuming declarations are not allowed in the
theory part.  The grammar for the theory part is given in
Figure~\ref{bnf-theory-part}.

\pvsbnf{bnf-theory-part}{Theory Part Syntax}

% Document Type: LaTeX
% Master File: interpretations-final.tex
\documentclass[11pt,twoside,openright,titlepage]{cslreport}
\usepackage{relsize}
\usepackage{makebnf}
\usepackage{alltt}
%\usepackage{doublespace}

\usepackage{cite}
\usepackage{/homes/rushby/tex/oz}
%\usepackage{/homes/rushby/tex/cslrep}
\usepackage{url}
\usepackage{psfig}
\usepackage{times}
\usepackage{/homes/owre/tex/session}
\usepackage{boxedminipage}
\def\mapb{\char"7B\char"7B}
\def\mape{\char"7D\char"7D}
\def\setb{\char"7B}
\def\sete{\char"7D}
\newcommand{\specware}{{\sc Specware}}
\textwidth 5.5in
\oddsidemargin .65in
\evensidemargin .41in
\raggedbottom
\sloppy


\begin{document}
\begin{titlepage}
\title{\textbf{\larger Theory Interpretations in PVS}}
\author{Sam Owre \and N. Shankar
\date{April 2001}
\cslreportnumber{SRI-CSL-01-01}
\maketitle
\noindent
%\hspace*{-1in}
\raisebox{-0.8cm}[1cm][1cm]{\srilogo}
\acknowledge{Funded by NASA Langley Research Center contract numbers
NAS1-20334 and NAS1-0079 and DARPA/AFRL contract number F33615-00-C-3043.}
\end{titlepage}

\cleardoublepageblank
\pagenumbering{roman}

\begin{abstract}
\thispagestyle{plain}

We describe a mechanism for theory interpretations in PVS.  The
mechanization makes it possible to show that one collection of theories is
correctly interpreted by another collection of theories under a
user-specified interpretation for the uninterpreted types and constants.
A theory instance is generated and imported, while the axiom instances are
generated as proof obligations to ensure that the interpretation is valid.
Interpretations can be used to show that an implementation is a correct
refinement of a specification, that an axiomatically defined specification
is consistent, or that a axiomatically defined specification captures its
intended models.

In addition, the theory parameter mechanism has been extended with a
notion of \emph{theory as parameter} so that a theory instance can be
given as an actual parameter to an imported theory.  Theory
interpretations can thus be used to refine an abstract specification or to
demonstrate the consistency of an axiomatic theory.  In this report we
describe the mechanism in detail.  This extension is a part of PVS version
3.0, which will be publicly released in mid-2001.

\end{abstract}

\tableofcontents
\cleardoublepage
\setcounter{page}{0} 
\pagenumbering{arabic}

\chapter{Introduction}

Theory interpretations have a long history in first-order
logic~\cite{Shoenfield,Enderton,Monk76}\@.  They are used to show that the
language of a given source theory $S$ can be interpreted within a target
theory $T$ such that the corresponding interpretation of axioms of $S$
become theorems of $T$\@.  This demonstrates the consistency of $S$
relative to $T$, and also the decidability of $S$ modulo the decidability
of $T$\@.  Theories and theory interpretations have also become important
in higher-order logic and type theory with languages such as {\sc
Ehdm}~\cite{EHDM:manuals}, IMPS~\cite{Farmer:interpretations},
HOL~\cite{Windley92}, Maude~\cite{Maude}, Extended
ML~\cite{SannellaDT:essential-concepts97}, and
\specware~\cite{SrinivasJullig95}\@.  In these languages, theories are
used as structuring mechanisms for large specifications so that abstract
theories can be refined into more concrete ones through interpretation.
In this report, we describe a theory interpretation mechanism for the PVS
specification language.

Specification languages and programming languages usually have some
mechanism for packaging groups of definitions into modules.  Lisp and Ada
have \emph{packages}\@.  Standard ML has a module system consisting of
signatures, structures corresponding to a signature, and functors that map
between structures.  Packages can be made generic by allowing certain
declarations to serve as parameters that can be instantiated when the
package is imported.  Ada has \emph{generic} packages that allow
parameters.  SML \emph{functors} can be used to construct parametric
modules.  C++ allows \emph{templates}.

In specification languages, a \emph{theory} groups together related
declarations of constants, types, axioms, definitions, and theorems.  One
way of demonstrating the consistency of such a theory is by providing an
interpretation for the uninterpreted types and constants under which the
axioms are valid.  The definitions and theorems corresponding to a valid
interpretation can then be taken as valid without further proof as long as
they have been verified in the source theory.  The technique of
interpreting one axiomatic theory in another has many uses.  It can be
used to demonstrate the consistency or decidability of the former theory
with respect to the latter theory.  It can also be used to refine an
abstract theory down to an executable implementation.

Interpretations are also useful in showing that the axioms capture the
intended models.  For example, a clock synchronization algorithm was
developed in \textsc{Ehdm} and was later shown to be consistent using the
mappings, but it turned out that in one place $<$ was used instead of
$\leq$, and because of this a set of perfectly synchronized clocks was
actually disallowed by the model.  Using interpretations in this way is
similar to testing in allowing for the exploration of the space of models
for the theory.

Parametric theories in PVS share some of the features of theory
interpretations.  Such theories can be defined with formal parameters
ranging over types and individuals, for example,\footnote{Note that the
number 0 here is overloaded, and treated as an identifier.}
{\smaller\begin{alltt}
group[G: TYPE, + : [G, G -> G], 0: G, -: [G -> G]]: THEORY
  BEGIN
    \vdots
  END group
\end{alltt}}
An instance of the theory \texttt{group} can be imported by supplying
actual parameters, the type \texttt{int} of integers, integer addition
{\tt +}, zero \texttt{0}, and integer negation {\tt -}, corresponding to
the formal parameters, as in {\tt group[int, +, 0, -]}\@.  A theory can
include assumptions about the parameters that have to be discharged when
the actual parameters are supplied.  For example, the group axioms can
be given as assumptions in the \texttt{group} theory above.  However,
there are some crucial differences between parametric theories and theory
interpretations.  In particular, if axioms are always specified as
assumptions, then the theory can be imported only by discharging these
assumptions.  It is necessary to have separate mechanisms for importing a
theory with the axioms, and for interpreting a theory by supplying a valid
interpretation, that is, one that satisfies its axioms.

The PVS theory interpretation mechanism is quite similar to that for
theory parameterization.  The axiomatic specification of groups could
alternately be given in a theory
{\smaller\begin{alltt}
group: THEORY
 BEGIN
  G: TYPE+
  +: [G, G -> G]
  0: G
  -: [G -> G]
   \vdots
 END group
\end{alltt}}
The group axioms are declared in the body of the theory.  Such a theory
can be interpreted by writing \texttt{\smaller group\mapb{}G := int, + :=
+, 0 := 0, - := -\mape{}}\@.  Here the left-hand sides refer to the
uninterpreted types and constants of theory \texttt{group}, and the
right-hand sides are the interpretations.  This notation resembles that of
theory parameterization and is used in contexts where a theory is
imported.  The corresponding instances of the group axioms are generated
as proof obligations at the point where the theory is imported.  The
result is a theory that consists of the corresponding mapping of the
remaining declarations in the theory \texttt{group}\@.  This allows the
theory \texttt{group} to be used in other theories, such as rings and
fields, and also allows the theory \texttt{group} to be suitably
instantiated by group structures.

Theory interpretations largely subsume parametric theories in the sense
that the theory parameters and the corresponding assumings can instead be
presented as uninterpreted types and constants and axioms so that the
actual parameters are given by means of an interpretation.  However, a
parametric theory with both assumings and axioms involving the parameters
is not equivalent to any interpreted theory, as the parameters may be
instantiated without the need to prove the axioms.  It is also useful to
have parametric theories as a convenient way of grouping together all the
parameters that must be provided whenever the theory is used.  For
example, typical theory parameters such as the size of an array, or the
element type of an aggregate datatype such as an array, list, or tree, are
required as inputs whenever the corresponding theories are used.  While
this kind of parameterization can be captured by theory interpretations,
it would not capture the intent that these parameters are \emph{required}
inputs wherever the theory is used.  Furthermore, when an operation from a
parametric theory is used, PVS attempts to figure out the actual
parameters based on the context of its use.  It can do this because the
formal parameters are precisely delimited.  The corresponding inference is
harder for theory interpretations since there might be many possible
interpretations that are compatible with the context of the operations
use.

In addition to the uninterpreted types and constants in a source theory
$S$, the PVS theory interpretation mechanism can also be used to interpret
any theories that are imported into $S$ by means of the \texttt{THEORY}
declaration.  The interpretation of a theory declaration for $S'$ imported
within $S$ must itself be a theory interpretation of $S'$\@.  Two distinct
importations of a theory $S'$ within $S$ can be given distinct
interpretations.  A typical situation is when two theories $R_1$ and $R_2$
both import a theory $S$ as $S_1$ and $S_2$, respectively.  A theory $T$
importing both $R_1$ and $R_2$ might wish to identify $S_1$ and $S_2$
since, otherwise, these would be regarded as distinct within $T$\@.  This
can be done by importing an instance $S'$ of $S$ into $T$ and importing
$R_1$ with $S_1$ interpreted by $S'$ and $R_2$ with $S_2$ interpreted as
$S'$\@.  With theory interpretations, we have also extended parametric
theories in PVS to take theories as parameters.  For example, we might
have a theory \texttt{group\_homomorphism} of group homomorphisms that
takes two groups \texttt{G1} and \texttt{G2} as parameters as in the
declaration
\begin{alltt}
 group_homomorphism[G1, G2: THEORY group]: THEORY \ldots
\end{alltt}
The actual parameters for these theory formals must be
interpretations \texttt{G1'} and \texttt{G2'}
of the theory \texttt{group}\@.

Another typical requirement in a theory interpretation mechanism is the
ability to map a source type to some quotient with respect to an
equivalence relation over a target type.  For example, rational numbers
can be interpreted by means of a pair of integers corresponding to the
numerator and denominator, but the same rational number can have multiple
such representations.  We show how it is possible to define quotient types
in PVS and use these types to capture interpretations where the equality
over a source type is mapped to an equivalence relation over a target
type.

The implementation of theory interpretation in PVS is described in the
following chapters.  This report assumes the reader is already familiar
with the PVS language; for details see the PVS Language
Manual~\cite{PVS:language}.  Chapter 2 deals with mappings, explaining the
basic concepts and introduces the grammar.  Chapter 3 introduces theory
declarations and theories as parameters which allow any valid
interpretation of the formal parameter theory as an actual parameter.
Chapter 4 describes a new command for viewing theory instances.  Chapter 5
compares PVS interpretations with other systems, Chapter 6 describes
future work, and we conclude with Chapter 7.


\chapter{Mappings}\label{mappings}

Theory interpretations in PVS provide mappings for uninterpreted types and
constants of the \emph{source} theory into the current
(\emph{interpreting}) theory.  Applying a mapping to a source theory
yields an \emph{interpretation} (or \emph{target}) theory.  A mapping is
specified by means of the \emph{mapping} construct, which associates
uninterpreted entities of the source theory with expressions of the target
theory.  The mapping construct is an extension to the PVS notion of
``name''.  The changes to the existing grammar are given in
Figure~\ref{mapping-bnf}.

\begin{figure}
\setlength{\sessionboxwidth}{\linewidth}
\addtolength{\sessionboxwidth}{-\arrayrulewidth}
\addtolength{\sessionboxwidth}{-\tabcolsep}
\begin{boxedminipage}[b]{\sessionboxwidth}
\begin{bnf}

\production{TheoryName}
{\opt{Id \lit{@}} Id \opt{Actuals} \opt{Mappings}}

\production{Name}
{\opt{Id \lit{@}} IdOp \opt{Actuals} \opt{Mappings} \opt{\lit{.} IdOp}}

\production{Mappings}
{\lit{\mapb{}} \ites{Mapping}{,} \lit{\mape{}}}

\production{Mapping}
{MappingLhs MappingRhs}

\production{MappingLhs}
{IdOp \rep{Bindings} \opt{\lit{:} \brc{\lit{TYPE} \choice \lit{THEORY}
\choice TypeExpr}}}

\production{MappingRhs}
{\lit{:=} \brc{Expr \choice TypeExpr}}

\end{bnf}
\end{boxedminipage}
\caption{Grammar for Names with Mappings}\label{mapping-bnf}
\end{figure}

The mapping construct defines the basic translation, but to be a theory
interpretation the mapping must be consistent: if type \texttt{T} is
mapped to the type expression \emph{E}, then a constant \texttt{t} of type
\texttt{T} must be mapped to an expression \emph{e} of type \emph{E}.  In
addition, all axioms and theorems of the source theory must be shown to
hold in the target theory under the mapping.  Since the theorems are
provable from the axioms, it is enough to show that the translation of the
axioms hold.  Axioms whose translations do not involve any
uninterpreted types or constants of the source theory are converted to
proof obligations.  Otherwise they remain axioms.

Theory interpretation may be viewed as an extension of theory
parameterization.  Given a theory named \texttt{T}, the instance
\texttt{T[a$_1$,\ldots,a$_n$]\mapb{}c$_1$:= e$_1$,\ldots,c$_m$:=
e$_m$\mape{}} is the same as the original theory, with the \emph{actuals}
\texttt{a$_i$} substituted for the corresponding formal parameters, and
e$_i$ substituted for \texttt{c$_i$}, which must be an uninterpreted type
or constant declaration.  Declarations that appear in the target of a
substitution in the mapping are not visible in the importing theory.  Some
axioms are translated to proof obligations.  The substituted forms of any
remaining axioms, definitions, and lemmas are available for use, and are
considered proved if they are proved in the uninterpreted theory.

The following simple example illustrates the
basic concepts.
\begin{session}
th1[T: TYPE, e: T]: THEORY
 BEGIN
  t: TYPE+
  c: t
  f: [t -> T]
  ax: AXIOM EXISTS (x, y: t): f(x) /= f(y)
  lem1: LEMMA EXISTS (x:T): x /= e
 END th1
\end{session}
\begin{session}
th2: THEORY
 BEGIN
  IMPORTING th1[int, 0]
               \mapb{} t := bool,
                  c := true,
                  f(x: bool) := IF x THEN 1 ELSE 0 ENDIF \mape{}
  lem2: LEMMA EXISTS (x:int): x /= 0
 END th2
\end{session}
\noindent Here theory \texttt{th1} has both actual parameters and
uninterpreted types and constants, as well as an axiom and
a lemma.  Theory \texttt{th2} imports \texttt{th1}, making the
following substitutions:
\setlength{\jot}{-2pt}
\setlength{\abovedisplayskip}{0pt}
\setlength{\belowdisplayskip}{0pt}
{\smaller\begin{eqnarray*}
\texttt{T} & \leftarrow & \texttt{int} \\
\texttt{e} & \leftarrow & \texttt{0} \\
\texttt{t} & \leftarrow & \texttt{bool} \\
\texttt{c} & \leftarrow & \texttt{true} \\
\texttt{f} & \leftarrow & \texttt{LAMBDA (x:\ bool):\ IF x THEN 1 ELSE 0 ENDIF} \\
\end{eqnarray*}}
Note that the mapping for \texttt{f} uses an abbreviated form of
substitution.  Typechecking this leads to the following proof obligation.
\begin{session}
IMP_th1_ax_TCC1: OBLIGATION
  EXISTS (x, y: bool):
    IF x THEN 1 ELSE 0 ENDIF /= IF y THEN 1 ELSE 0 ENDIF;
\end{session}
This is simply the interpretation of the \texttt{ax} axiom and is easily
proved.  The lemma \texttt{lem1} can be proved from the axiom, and may
be used directly in proving \texttt{lem2} using the proof command
\texttt{(LEMMA "lem1")}.

Note that once the TCC has been proved, we know that \texttt{th1} is
consistent.  If we had left out the mapping for \texttt{f}, then the TCC
would not be generated, and the translation of theory \texttt{th1} would
still contain an axiom and not necessarily be consistent.

% Note that we used a lambda form in the axiom,
% rather than \texttt{f}.  This is because logically the generated proof
% obligation precedes the importing, which is only meaningful if the
% obligation is provable.  Hence \texttt{f} is not visible in the proof
% obligation and should not appear in any axiom of the theory.\footnote{We
% may allow this in future versions of PVS by automatically expanding
% non-recursive definitions as a part of substitution, treating them as a
% kind of macro.}  After the importing, of course, \texttt{f} is visible as
% seen in \texttt{lem2}.

% Note that mappings make theory parameters optional---they may be
% eliminated by moving the formal parameters to the theory body and turning
% assumptions into axioms.  The theory could then be instantiated using
% mappings instead of actual parameters.  Theory parameters are still
% useful, however.  First, they may be used to distinguish between the
% parameters to the system being specified and the entities defined by the
% system.  For example, in describing a protocol that works for any number
% of processes, it is more natural to make the number of processes a formal
% parameter rather than an uninterpreted constant.  Second, the
% typechecker can frequently infer the values of the actual parameters when
% a theory is imported generically, but mappings must be explicitly given.
% Although in principle the typechecker might be extended to infer mappings,
% it is hard to see how to do this efficiently.

One advantage to using mappings instead of parameters is that not all
uninterpreted entities need be mapped, whereas for parameters either all
or none must be given.  For example, consider the following theory.
\begin{session}
example1[T: TYPE, c: T]: THEORY
 BEGIN
  f(x: T): int = IF x = c THEN 0 ELSE 1 ENDIF
 END example1
\end{session}
\noindent It may be desirable to import this where \texttt{T} is always
\texttt{real}, and \texttt{c} is left as a parameter, but there is
currently no mechanism for this.  One could envision partial importings
such as \texttt{IMPORTING example1[real, \_]}, but it is not clear that
this is actually practical---in particular, the syntax for providing the
missing parameters is not obvious.  With mappings, on the other hand, we
can define \texttt{example1} as follows.
\begin{session}
example1: THEORY
 BEGIN
  T: TYPE
  c: T
  f(x: T): int = IF x = c THEN 0 ELSE 1 ENDIF
 END example1
\end{session}
\noindent Then we can refer to this theory from another theory as in the
following.
\begin{session}
example2: THEORY
 BEGIN
  th: THEORY = example1\mapb{}T := real\mape{}
  frm: FORMULA f\mapb{}c := 3\mape{} = f
 END example2
\end{session}
\noindent The \texttt{th} theory declaration just instantiates \texttt{T},
leaving \texttt{c} uninterpreted.  The first reference to \texttt{f} maps
\texttt{c} to \texttt{3}, whereas the second reference leaves it
uninterpreted though it is still a \texttt{real}.  Note that formula
\texttt{frm} is unprovable, since the uninterpreted \texttt{c} from the
second reference may or may not be equal to \texttt{3}.

As described in the introduction, an important aspect of mappings is the
support for quotient types.  In \textsc{Ehdm} this was done by
interpreting equality, but in PVS we instead define a theory of
equivalence classes, and allow the user to map constants to equivalence
classes under congruences.  For example, the \texttt{stacks} datatype
might be implemented using an array as follows.
\begin{session}
stack[t:TYPE]: DATATYPE
 BEGIN
  empty: empty?
  push(top:t, pop: stack): nonempty?
 END stack
\end{session}
\begin{session}\label{cstack}
cstack[t: TYPE+]: THEORY
 BEGIN
  cstack: TYPE = [# size: nat, elems: [nat -> t] #]
  cempty?(s: cstack): bool = (s`size = 0)
  some_t: t = epsilon(LAMBDA (x:t): true)
  cempty: (cempty?) =
    (# size := 0,
       elems := LAMBDA (n: nat): some_t #)
  cnonempty?(s: cstack): bool = (s`size /= 0)
  ctop(s: (cnonempty?)): t = s`elems(s`size - 1)
  cpop(s: (cnonempty?)): cstack = s WITH [`size := s`size - 1]
  cpush(x: t)(s: cstack): (cnonempty?) =
    (# size := s`size + 1,
       elems := s`elems WITH [(s`size) := x] #)
  ce: equivalence[cstack] =
    LAMBDA (s1, s2: cstack):
     s1`size = s2`size AND
     FORALL (n: below(s1`size)): s1`elems(n) = s2`elems(n)

  estack: TYPE = Quotient(ce)
  CONVERSION+ EquivClass(ce), rep(ce), lift(ce)
  \ldots
\end{session}
\texttt{Quotient}, \texttt{EquivClass}, and \texttt{rep} are defined in
the prelude theory \texttt{QuotientDefinition}, shown here in part.
\begin{session}
QuotientDefinition[T : TYPE] : THEORY
BEGIN
  R : VAR set[[T, T]]
  S : VAR equivalence[T]
  x, y, z : VAR T

  EquivClass(R)(x) : set[T] = { z | R(x, z) }
  \ldots
  Quotient(S) : TYPE =
    { P : set[T] | EXISTS x : P = EquivClass(S)(x) }
  \ldots
  rep(S)(P: Quotient(S)): T = choose(P)
  \ldots
END QuotientDefinition
\end{session}
The \texttt{lift} function is defined in the prelude theory
\texttt{QuotientExtensionProperties} as follows.
\begin{session}
QuotientExtensionProperties[X, Y : TYPE] : THEORY
BEGIN
  S : VAR equivalence[X]

  lift(S)(g : (PreservesEq[X, Y](S)))(P : Quotient(S)) : Y
    = g(rep(S)(P))
  \ldots
END QuotientExtensionProperties
\end{session}
This allows functions on concrete stacks to be lifted to functions on
equivalence classes, so long they satisfy the \texttt{PreservesEq}
relation, i.e., they produce the same values on \texttt{S}-equivalent
elements.

With these conversions in place, we can finish the specification of
\texttt{cstack} as follows.
\begin{session}
  \ldots
  IMPORTING stack[t]\mapb{} stack := estack,
                       empty? := cempty?,
                       nonempty? := cnonempty?,
                       empty := cempty,
                       top(s: (cnonempty?)) := ctop(s),
                       pop(s: (cnonempty?)) := cpop(s),
                       push(x: t, s: cstack) := cpush(x)(s) \mape{}
 END cstack
\end{session}
\noindent Here the source type \texttt{stack} is mapped to the quotient
type \texttt{estack} defined by the concrete equality \texttt{ce}.  The
\texttt{empty?} and \texttt{nonempty?} predicates are mapped to predicates
on \texttt{estack}s, using the \texttt{rep(ce)} conversion.  The
\texttt{empty} constructor is then mapped to its equivalence class.

\texttt{top}, \texttt{pop},

The mapping for \texttt{push} is more involved; \texttt{cpush} must first
be lifted in order to apply it to the abstract stack \texttt{s}.  This is
applied automatically by the conversion mechanism of PVS.  The application
of \texttt{lift} generates the proof obligation that \texttt{cpush}
preserves the equivalences, that is, it is a congruence.  This mapping
generates a large number of proof obligations, because the \texttt{stack}
datatype generates a \texttt{stacks\_adt} theory with a large number of
axioms, for example, extensionality, well-foundedness, and induction.

The PVS interpretations mechanism is much simpler to implement than the
one in \textsc{Ehdm}---equality is not a special case, but simply an
aspect of mapping a type to an equivalence class.  The technique of
mapping types to equivalence classes is quite useful, and captures the
notion of behavioral equivalence outlined
in~\cite{SannellaDT:essential-concepts97}.  In fact it is more general, in
that it works for any equivalence relation, not just those based on
observable sorts.


\chapter{Theory Declarations}

With the mapping mechanism, it is easy to specify a general theory and
have it stand for any number of instances.  For example, groups, rings,
and fields are all structures that can be given axiomatically in terms of
uninterpreted types and constants.  This works well when considering one
such structure at a time, but it is difficult to specify theories that
involve more than one structure, for example, group homomorphisms.
Importing the original theory twice is the same as importing it once, and
an attempted definition of a homomorphism would turn into an automorphism.
In this case what is needed is a way to specify multiple different
``copies'' of the original theory.  This is accomplished with \emph{theory
declarations}, which may appear in either the theory parameters or the
body of a theory.  A theory declaration in the formal parameters is
referred to as a \emph{theory as parameter}.\footnote{The term
\emph{theory parameter} refers to a parameter of a theory, so we use the
term \emph{theory as parameter} instead.}  Theory declarations allow
theories to be encapsulated, and instantiated copies of the implicitly
imported theory are generated.
\begin{figure}[!b]
\setlength{\sessionboxwidth}{\linewidth}
\addtolength{\sessionboxwidth}{-\arrayrulewidth}
\addtolength{\sessionboxwidth}{-\tabcolsep}
\begin{boxedminipage}[b]{\sessionboxwidth}
\begin{bnf}\smaller

\production{TheoryFormalDecl}
{TheoryFormalType \choice TheoryFormalConst \choice TheoryDecl}

\production{TheoryDecl}
{Id \lit{:} \lit{THEORY} TheoryDeclName}

\production{TheoryDeclName}
{\opt{Id \lit{@}} Id \opt{Actuals} \opt{TheoryDeclMappings}}

\production{TheoryDeclMappings}
{\lit{\mapb{}} \ites{TheoryDeclMapping}{,} \lit{\mape{}}}

\production{TheoryDeclMapping}
{MappingLhs TheoryDeclMappingRhs}

\production{TheoryDeclMappingRhs}
{MappingSubst \choice MappingDef \choice MappingRename}

\production{MappingSubst}
{\lit{:=} \brc{Expr \choice TypeExpr}}

\production{MappingDef}
{\lit{=} \brc{Expr \choice TypeExpr}}

\production{MappingRename}
{\lit{::=} \brc{IdOp \choice Number}}

\end{bnf}
\end{boxedminipage}
\caption{Grammar for Theory Declarations}\label{theory-parameter-bnf}
\end{figure}

For example, an (additive) group is normally thought of as a 4-tuple
consisting of a set $G$, a binary operator $+$, an identity element $0$,
and an inverse operator $-$ that satisfies the usual group axioms.  Using
theory interpretations, we simply define this as follows:
\begin{session}
group: THEORY
 BEGIN
  G: TYPE+
  +: [G, G -> G]
  0: G
  -: [G -> G]
  x, y, z: VAR G
  associative_ax: AXIOM FORALL x, y, z: x + (y + z) = (x + y) + z
  identity_ax: AXIOM FORALL x: x + 0 = x
  inverse_ax: AXIOM FORALL x: x + -x = 0 AND -x + x = 0
  idempotent_is_identity: LEMMA x + x = x => x = 0
 END group
\end{session}

As described in Chapter~\ref{mappings}, we can use mappings to create
specific instances of groups.  For example, {\smaller\begin{alltt}
group\mapb{}G := int, + := +, 0 := 0, - := -\mape{}
\end{alltt}}
\noindent is the additive group of integers, whereas
{\smaller\begin{alltt}
group\mapb{}G := nzreal, + := *, 0 := 1, - := LAMBDA (r:nzreal):\ 1/r\mape{}
\end{alltt}}
\noindent is the multiplicative group of nonzero reals.

This works nicely, until we try to define the notion of a group
homomorphism.  At this point we need two groups, both individually
instantiable.  We could simply duplicate the group specification, but
this is obviously inelegant and error prone.  Using theories as
parameters, we may define group homomorphisms as follows.
\begin{session}
group_homomorphism[G1, G2: THEORY group]: THEORY
 BEGIN
  x, y: VAR G1.G
  f: VAR [G1.G -> G2.G]
  homomorphism?(f): bool = FORALL x, y: f(x + y) = f(x) + f(y)
  hom_exists: LEMMA EXISTS f: homomorphism?(f)
 END group_homomorphism
\end{session}
\noindent Here \texttt{G1} and \texttt{G2} are theories as parameters to a
generic homomorphism theory that may be instantiated with two different
groups.  Hence we may import \texttt{group\_homomorphism}, for example, as
\begin{session}
IMPORTING group_homomorphism[group\mapb{}G := int, + := +, 0 := 0, - := -\mape{}
                             group\mapb{}G := nzreal, + := *, 0 := 1,
                                 - := LAMBDA (x: nzreal): 1/x\mape{}]
\end{session}

There is a subtlety here that needs emphasizing; \texttt{G1} and
\texttt{G2} are two \emph{distinct} copies of theory \texttt{group}.
For example, consider the addition of the following lemma to
\texttt{group\_homomorphism}.
\begin{session}
oops: LEMMA G1.0 = G2.0
\end{session}
\noindent If \texttt{G1} and \texttt{G2} are treated as the same
\texttt{group} theory, this is a provable lemma.  But then after the
importing given above we would be able to show that \texttt{0 = 1}.  Even
worse, the two different instances of groups may not even be type
compatible, so the \texttt{oops} lemma should not even typecheck.

We have solved this in PVS by expanding theories \texttt{G1} and
\texttt{G2} in place, within \texttt{group\_homomorphism}, as shown in
Figure~\ref{group_homo_ppe}.  Declarations within these expansions have
identifiers that guarantee they are distinct from each other and from the
original group theory.  Thus the \texttt{oops} lemma generates a type
error, as \texttt{G1.G} and \texttt{G2.G} are incompatible types, though
as they are uninterpreted they may later be mapped to compatible types.

The identifiers for a theory declaration are generated by prepending the
theory declaration identifier to each of the mapped declarations of the
source theory.  Hence for \texttt{G1}, all of the declarations of
\texttt{group} are mapped, but with \texttt{G1} prepended.  This can be
continued: if a declaration
\begin{alltt}
 gh: THEORY group_homomorphism
\end{alltt}
appears in another theory, then the type \texttt{gh.G1.G} will be created, etc.

This introduces new possibilities.  When expanding a theory the
mappings are substituted and the original declarations disappear.
However, it may be preferable to create definitions rather than
substitutions.  In addition, it is sometimes useful to simply rename the
types or constants of a theory.  For example, consider the following group
instance
\begin{session}
G1: THEORY = group\mapb{}G := int, + := +, 0 := 0, - := -\mape{}
\end{session}
\noindent which generates the following theory.
\label{group-instances-start}
\begin{session}
G1: THEORY
 BEGIN
  x, y, z: VAR int
  idempotent_is_identity: LEMMA x + x = x => x = 0
 END G1
\end{session}
To create definitions, use \texttt{=} instead of \texttt{:=}, as
in the following.
\begin{session}
G2: THEORY = group\mapb{}G = int, + = +, 0 = 0, - = -\mape{}
\end{session}
\noindent Now we get the following theory.
\begin{session}
G2: THEORY
 BEGIN
  G: TYPE+ = int
  +: [G, G -> G] = +
  0: G = 0
  -: [G -> G] = -
  x, y, z: VAR G
  idempotent_is_identity: LEMMA x + x = x => x = 0
 END G2
\end{session}
Finally, to simply rename the uninterpreted types and constants, use
\texttt{::=} as in the following.
\begin{session}
G3: THEORY = group\mapb{}G ::= MG, + ::= *, 0 ::= 1, - ::= inv\mape{}
\end{session}
\noindent The generated theory instance specifies multiplicative groups as
follows.
\begin{session}
G3: THEORY
 BEGIN
  MG: TYPE+
  *: [MG, MG -> MG]
  1: MG
  inv: [MG -> MG]
  x, y, z: VAR MG
  associative_ax: AXIOM FORALL x, y, z: x * (y * z) = (x * y) * z
  identity_ax: AXIOM FORALL x: x * 1 = x
  inverse_ax: AXIOM FORALL x: x * inv(x) = 1 AND inv(x) * x = 1
  idempotent_is_identity: LEMMA x * x = x => x = 1
 END G3
\end{session}
The right-hand side of a renaming mapping must be an identifier, operator,
or number, and must not create ambiguities within the generated theory.
Note that renamed declarations are still uninterpreted, and may themselves
be given interpretations, as in
\begin{session}
G3i: THEORY = G3\mapb{}MG := nzreal, * := *, 1 := 1,
                  inv := LAMBDA (r: nzreal): 1/r\mape{}
\end{session}

Finally, we can mix the different forms of mapping, to give a partial
mapping.
\begin{session}
G4: THEORY = group\mapb{}G = nzreal, + := *, 0 ::= one\mape{}
\end{session}
\noindent This generates the following theory instance.
\begin{session}
G4: THEORY
 BEGIN
  G: TYPE+ = nzreal;
  one: nzreal;
  -: [nzreal -> nzreal]
  x, y, z: VAR nzreal
  identity_ax: AXIOM FORALL (x: nzreal): x * one = x
  inverse_ax: AXIOM FORALL (x: nzreal):
                      x * -x = one AND -x * x = one
  idempotent_is_identity: LEMMA x * x = x => x = one
 END G4
\end{session}\label{group-instances-end}
Note that \texttt{associative\_ax} has disappeared---it has become a TCC
of the importing theory---whereas the other axioms are not so transformed
because they still reference uninterpreted types or constants.

With theories as parameters we have another situation in which mappings
are more convenient than theory parameters.  Many times the same set of
parameters is passed through an entire theory hierarchy.  If there are
assumings, then these must be copied.  For example, consider the
following theory.
\begin{session}
th[T: TYPE, a, b: T]: THEORY
 BEGIN
  ASSUMING
   A: ASSUMPTION a /= b
  ENDASSUMING
  ...
 END th
\end{session}
\noindent To import this theory, you simply provide a type and two
different elements of that type.  But suppose you wish to import this
theory from a theory that has the same parameters.  In this case the
assumption must also be copied, as there is otherwise no way to prove the
resulting obligation.  This can (and frequently does) lead to a tower of
theories, all with the same parameters and copies of the same assumptions,
as well as proofs of the same obligations.

There are ways around this, of course.  Most assumptions may be turned
into type constraints, as in the following.
\begin{session}
th[T: TYPE, a: T, b: \setb{}x: T | a /= x\sete{}]: THEORY
 ...
\end{session}
\noindent But this introduces an asymmetry in that \texttt{a} and
\texttt{b} now belong to different types, and the type predicate still
must be provided up the entire hierarchy.

Using a theory as a parameter, we may instead define \texttt{th} as
follows.
\begin{session}
th: THEORY
 BEGIN
  T: TYPE,
  a, b: T
  A: AXIOM a /= b
  ...
 END th
\end{session}
\noindent We then parameterize using this theory (which is implicitly
imported):
\begin{session}
th_1[t: THEORY th]: THEORY ...
\end{session}
\noindent We have encapsulated the uninterpreted types and constants into
a theory, and this is now represented as a single parameter.  Axiom
\texttt{A} is visible within theory \texttt{th\_1}, and no proof
obligations are generated since no mapping was given for \texttt{th}.  Now
we can continue defining new theories as follows.
\begin{session}
th_2[t: THEORY th]: THEORY IMPORTING th_1[t] ...
th_3[t: THEORY th]: THEORY IMPORTING th_2[t] ...
  \vdots
\end{session}
\noindent None of these generate proof obligations, as no mappings are
provided.

We may now instantiate \texttt{th\_n}, for example, with the following.
\begin{session}
IMPORTING th_n[th\mapb{}T := int, a := 0, b := 1\mape{}]
\end{session}
\noindent Now the substituted form of the axiom becomes a proof obligation
which, when proved, provides evidence that the theory \texttt{th} is
consistent.

% \chapter{Theory Declarations and Theory Abbreviations}

With the introduction of theories as parameters, it is natural to allow
theory declarations that may be mapped, in the same way that instances may
be provided for theories as parameters.  Thus the
\texttt{group\_homomorphism} may be rewritten as follows:
\begin{session}
group_homomorphism: THEORY
 BEGIN
  G1, G2: THEORY group
  x, y: VAR G1.G
  f: VAR [G1.G -> G2.G]
  homomorphism?(f): bool = FORALL x, y: f(x + y) = f(x) + f(y)
  hom_exists: LEMMA EXISTS f: homomorphism?(f)
 END group_homomorphism
\end{session}
\noindent Again, the choice between using theories as parameters or theory
declarations is really a question of taste, as they are largely
interchangeable.

As with theories as parameters, copies must be made for \texttt{G1} and
\texttt{G2}.  Note that this means that there is a difference between
theory abbreviations and theory declarations, as the former do not involve
any copying.  We decided to use the old form of theory abbreviation to
define theory declarations, and to extend the \texttt{IMPORTING} expressions to
allow abbreviations, as shown in Figure~\ref{importing-bnf}.  Thus instead of
\begin{session}
funset: THEORY = sets[[int -> int]]
\end{session}
\noindent which creates a copy of sets, use
\begin{session}
IMPORTING sets[[int -> int]] AS funset
\end{session}
\noindent which imports \texttt{sets[[int -> int]]} and abbreviates it as
\texttt{funset}.

\begin{figure}
\setlength{\sessionboxwidth}{\linewidth}
\addtolength{\sessionboxwidth}{-\arrayrulewidth}
\addtolength{\sessionboxwidth}{-\tabcolsep}
\begin{boxedminipage}[b]{\sessionboxwidth}
\begin{bnf}

\production{Importing}
{\lit{IMPORTING} \ites{ImportingItem}{,}}

\production{ImportingItem}
{TheoryName \opt{\lit{AS} Id}}

\end{bnf}
\end{boxedminipage}
\caption{Grammar for Importings}\label{importing-bnf}
\end{figure}

\chapter{Prettyprinting Theory Instances}

Mappings can get fairly complex, especially if actual parameters are
involved, and it may be desirable to see the specified theory instance
displayed with all the substitutions performed.  To support this, we have
provided a new PVS command: \texttt{prettyprint-theory-instance}
(\texttt{M-x ppti}).  This takes two arguments: a theory instance, which
in general is a theory name with actual parameters and/or mappings, and a
context theory, in which the theory instance may be typechecked.  The
simplest way to use this command is to put the cursor on the theory name
as it appears in a theory as parameter, theory declaration, or
importing---when the command is issued it then defaults to the theory
instance under the cursor and the current theory is the default
context theory.  For example, putting the cursor on
\texttt{group\_homomorphism} in the following and typing \texttt{M-x ppti}
followed by two carriage returns\footnote{The first uses the theory name
instance at the cursor, and the second uses the current theory as the
context.} generates a buffer named \texttt{group\_homomorphism.ppi}.
All instances of a given theory generate the same buffer name.
\begin{session}
IMPORTING group_homomorphism[\mapb{}G := int, + := +, 0 := 0, - := -\mape{}
                             \mapb{}G := nzreal, + := *, 0 := 1,
                               - := LAMBDA (x: nzreal): 1/x\mape{}]
\end{session}
\noindent This buffer has the following contents.
\begin{session}
% Theory instance for
  % group_homomorphism[groups\mapb{} G := int, + := +,
  %                             - := -, 0 := 0 \mape{},
  %                    groups\mapb{} G := nzreal, + := *,
  %                             - := (LAMBDA (x: nzreal): 1 / x),
  %                             0 := 1 \mape{}]
group_homomorphism_instance: THEORY
 BEGIN

  IMPORTING groups\mapb{} G := int, + := +, - := -, 0 := 0 \mape{}

  IMPORTING groups\mapb{} G := nzreal, + := *,
                     - := (LAMBDA (x: nzreal): 1 / x), 0 := 1 \mape{}

  x, y: VAR int

  f: VAR [int -> nzreal]

  homomorphism?(f): bool =
    FORALL (x: int), (y: int): f(x + y) = f(x) * f(y)

  hom_exists: LEMMA EXISTS (f: [int -> nzreal]): homomorphism?(f)
 END group_homomorphism_instance
\end{session}
The group instances shown on
pages~\pageref{group-instances-start}--\pageref{group-instances-end}
provide more examples of the output produced by
\texttt{prettyprint-theory-instance}.

\chapter{Comparison with Other Systems}

In this chapter we compare PVS theory interpretations to existing
programming and specification mechanisms of other systems.
The \textsc{Ehdm} system~\cite{EHDM:Language} has a notion of a mapping
module that maps a source module to a target module.  When a mapping
module is typechecked, a new module is automatically created that
represents the substitution of the interpretations for the body of the
source theory.  Equality is allowed to be mapped in \textsc{Ehdm}, in
which case it must be mapped to an equivalence relation.  In PVS, mappings
are provided as a syntactic component of names, and are essentially an
extension of theory parameters.  Equality is not treated specially, but is
handled by mapping a given type to a quotient type.

IMPS~\cite{Farmer:imps-cade,Farmer94} also supports theory
interpretations.  It is similar to \textsc{Ehdm} in that it has a special
\texttt{def-translation} form that takes a source theory, target
theory, sort association list, and constant association list, and generates a
theory translation.  Obligations may be generated that ensure that every
axiom of the source theory is a theorem of the target theory.  If these
are proved the translation is treated as an interpretation.  There is no
mechanism for mapping equality.  As with both PVS and \textsc{Ehdm},
defined sorts and constants of the source theory are automatically
translated.  A more detailed comparison between IMPS and an earlier
version of PVS appears in an unpublished report by
Kamm\"{u}ller~\cite{Kammuller:comparison}.

In Maude~\cite{Maude} and its precursor OBJ~\cite{OBJ:intro} it is
possible to
define \texttt{modules} that represent transition systems of a rewrite
theory whose states are equivalence classes of ground terms and whose
transitions are inference rules in \emph{rewriting logic}.  A given module
may import another module, either \texttt{protecting} it, which means that
the importing module adds no \emph{junk} or \emph{confusion}, or
\texttt{including} it, which imposes no such restrictions.  In addition to
modules, Maude has \emph{theories}, which are used to declare module
interfaces.  These may appear as module parameters, as in
$M[X_{1}::T_{1},\ldots,X_{n}::T_{n}]$, where the $X_{i}$ are \emph{labels}
and the $T_{i}$ are names of theories.  These theory parameters (source
theories) may be instantiated by target theories or modules using
\emph{views}, which indicate how each sort, function, class, and message
of the source theory is mapped to the target theory.  However, Maude
currently does not support the generation of proof obligations from source
theory axioms, so views are simply theory translations, not
interpretations.

The programming language Standard ML~\cite{ML-report} has a module
system where modules are given by \emph{structures} with a given
\emph{signature}, and parametric modules are \emph{functors} mapping
structures of a given signature to structures.  The PVS mechanism
of using theories as parameters resembles SML functors but for a
specification language rather than a programming language. 
Sannella and Tarlecki~\cite{SannellaDT:essential-concepts97} describe a
version of the ML module system in which there are \emph{specifications}
containing \emph{sorts}, \emph{operations}, and \emph{axioms}.  For
example, the signature of stacks is the following.
\begin{eqnarray*}
\emph{STACK} = & \textbf{sorts} & \emph{stack} \\
               & \textbf{opns} & \emph{empty} : \emph{stack} \\
               &               & \emph{push} : \texttt{int} \times \emph{stack} \rightarrow \emph{stack} \\
               &               & \emph{pop} :
                                 \emph{stack} \rightarrow \emph{stack} \\
               &               & \emph{top} :
                                 \emph{stack} \rightarrow \texttt{int} \\
               &               & \emph{is\_empty} :
                                 \emph{stack} \rightarrow \texttt{bool} \\
               & \textbf{axioms} & \emph{is\_empty}(\emph{empty}) =
                                   \texttt{true} \\
               &               &
               \forall\emph{s}:\emph{stack}.\forall\emph{n}:\texttt{int}.
                 \emph{is\_empty}(\emph{push}(\emph{n},\emph{s}))
                     = \texttt{false} \\
               &               &
               \forall\emph{s}:\emph{stack}.\forall\emph{n}:\texttt{int}.
                 \emph{top}(\emph{push}(\emph{n},\emph{s})) = \emph{n} \\
               &               &
               \forall\emph{s}:\emph{stack}.\forall\emph{n}:\texttt{int}.
                 \emph{pop}(\emph{push}(\emph{n},\emph{s})) = \emph{s} \\
\end{eqnarray*}
The following algebra is a \emph{realization} of the above specification
that corresponds to that of \texttt{cstack} on page~\pageref{cstack}.
{\smaller\begin{alltt}
  structure S2 : STACK =
      struct
          type stack = (int -> int) * int
          val empty = ((fn k => 0), 0)
          fun push (n, (f, i))
                = ((fn k => if k = i then n else f k), i+1)
          fun pop (f, i) = if i = 0 then (f, 0) else (f, i-1)
          fun top (f, i) = if i = 0 then 0 else f(i-1)
          fun is_empty (f, i) = (i=0)
\end{alltt}}
Note however, that the stacks \emph{empty} and
\emph{pop}(\emph{push}(\texttt{6},\emph{empty})) are not equal.  Thus they
distinguish the \emph{observable} sorts, in this case \texttt{int} and
\texttt{bool}, which are the only data directly visible to the user.  The
above two terms are not \emph{observable computations}, so it does not
matter that they are different.  In general, two different algebras are
\emph{behaviorally equivalent} if all observable computations yield the
same results. Note that choosing observable values based on sorts is a bit
coarse: for example, there may be two \texttt{int}-valued variables, one of
which is observable and one that represents an internal pointer.  Mapping
to equivalence classes is more general, as it is easy to capture
behavioral equivalence.

The induction theorem prover Nqthm~\cite{boyer-moore88,BoyerGoldschlag91}
has a feature called \texttt{FUNCTIONALLY-INSTANTIATE} that can be used to
derive an instance of a theorem by supplying an interpretation for some of
the function symbols used in defining the theorem.  The corresponding
instances of any axioms concerning these function symbols must be
discharged.  Such axioms can be introduced as conservative extensions as
definitions with the \texttt{DEFUN} declaration or through witnessed
constraints using the \texttt{CONSTRAIN} declaration, or they can be
introduced nonconservatively through an \texttt{ADD-AXIOM}
declaration.  While the functional instantiation mechanism is similar in
flavor to PVS theory interpretations, the underlying logic of Nqthm is a
fragment of first-order logic whose expressive power is more limited
than the higher-order logic of PVS.  In addition, Nqthm lacks types and
structuring mechanisms such as parametric theories.

The \specware{} language~\cite{SrinivasJullig95} employs theory
interpretations as a mechanism for the stepwise refinement of
specifications into executable code.  \specware{} has constructs for
composing specifications while identifying the common components, and for
compositionally refining specifications so that the refinement of a
specification can be composed from the refinement of its components.
Unlike PVS, \specware{} has the ability to incorporate multiple logics
and translate specifications between these logics.  A theory is an
independent unit of specification in PVS and hence there is no support for
composing theories from other theories.  However, the operations in
\specware{} can largely be simulated by means of theories and theory
interpretations in PVS.

In summary, theory interpretation has been a standard tool in
specification languages since the early work on HDM~\cite{HDM:Handbook}
and Clear~\cite{BURSTALL&GOGUEN}.  PVS implements theory interpretations
as a simple extension of the mechanism for importing parametric theories.
PVS theory interpretations subsume the corresponding capabilities
available in other specification frameworks.


\chapter{Future Work}

A number of interesting extensions may be contemplated for
the future.

\paragraph{Mapping of interpreted types and constants---}

There are two aspects: one is simply a convenience where, for
example, we might have a tuple type declaration \texttt{T: TYPE = [T1, T2,
T3]} and want to map it to \texttt{position: TYPE = [real, real, real]} by
simply giving the map \texttt{\mapb{}T := position\mape{}}.

The second aspect is where the mapping is between two different kinds, for
example mapping a record type to a function type.  This requires
determining the corresponding components as well as making explicit the
underlying axioms.  For example, record types satisfy extensionality, and
if they are mapped to a different type the implicit extensionality axiom
must be translated to a proof obligation.

\paragraph{Rewriting with congruences---}

In theory substitution, if a type is mapped to a quotient type then
equality over this type is mapped to equality over the quotient type.
If $T$ is an uninterpreted type, $\equiv$ an equivalence relation over
$T'$, and $T'/\equiv$ the quotient type, then \texttt{=[$T$]} is mapped to
\texttt{=[$T'/\equiv$]}, which is equivalent to $\equiv$.  An equational
formula thus still has the form of a rewrite.  However, to apply such a
rewrite one generally needs to do some lifting.  The following is a simple
example.
\begin{session}
th: THEORY
 BEGIN
  T: TYPE
  a, b: T
  f, g: [T -> T]
  \ldots \emph{Some axioms involving f, g, a, and b}
  lem: LEMMA f(a) = g(b)
 END th
th2: THEORY
 BEGIN
  ==(x, y: int): bool = divides(3, x - y)
  IMPORTING th\mapb{}T := E(==),
                a := equiv_class(==)(2),
                b := equiv_class(==)(1),
                f := LAMBDA (x: E(==)): equiv_class(rep(x) - 1),
                g := LAMBDA (x: E(==)): equiv_class(rep(x) - 2)\mape{}
  \ldots
 END th2
\end{session}
\noindent To rewrite with \texttt{lem}, \texttt{a} must first be lifted to
its equivalence class, then the rewrite is applied and the result is then
projected back using \texttt{rep}.  To do this requires some modification
to the rewriting mechanism of the prover.

\paragraph{Consistency Analysis---}

With a single independent theory such as groups, it is easy to generate a
mapping in which all axioms become proof obligations, and see directly
that the theory is consistent.  On the other hand, if many theories are
involved in which compositions of mappings are involved, this may become
quite difficult.  What is needed is a tool that analyzes a mapped theory
to see if it is consistent, and reports on any remaining axioms and
uninterpreted declarations.  This is similar in spirit to proof chain
analysis, but works at the theory level rather than for individual
formulas.

\paragraph{Semantics of Mappings---}

The semantics of theory interpretations needs to be formalized and added
to the PVS semantics report~\cite{PVS-Semantics:TR}.

\chapter{Conclusion}

Theory interpretations are used to embed an interpretation of an abstract
theory in a more concrete one.  In this way, they allow an abstract
development to be reused at the more concrete level.  Theory
interpretations can be used to refine a specification down to code.
Theory interpretations can also be used to demonstrate the consistency of
an axiomatic theory relative to another theory.

Parametric theories in PVS provide some but not all of the functionality
of theory interpretations.  In particular, they do not allow an abstract
theory to be imported with only a partial parameterization.  Theory
interpretations have been implemented in PVS version 3.0, which will be
released in mid-2001.  The current implementation allows the
interpretation of uninterpreted types and constants in a theory, as well
as theory declarations.  PVS has also been extended so that a theory may
appear as a formal parameter of another theory.  This allows related sets
of parameters to be packaged as a theory.  Quotient types have been
defined within PVS and used to admit interpretations of types where the
equality on a source type is treated as an equivalence relation on a
target type.

Theory interpretations have been implemented in PVS as an extension of the
theory parameter mechanism.  This way, theory interpretations are
an extension of an already familiar concept in PVS and can be used in
place of theory parameters where there is a need for greater
flexibility in the instantiation.  The proof obligations generated by
theory interpretations are similar to those for parametric theories
with assumptions.  

A number of extensions related to theory interpretations remain to be
implemented.  First, we plan to extend theory interpretations to the case
of interpreted types and constants.  This poses some challenges since
there are implicit operations and axioms associated with certain type
constructors.  Second, the rewriting mechanisms of the PVS prover need to
be extended to rewrite relative to a congruence.  This means that if we
are only interested in $f(a)$ up to some equivalence that is preserved by
$f$, then we could rewrite $a$ up to equivalence rather than equality.
Third, the PVS semantics have to be extended to incorporate theory
interpretations.  Finally, the PVS ground evaluator has to be extended to
handle theory interpretations.  Currently, the ground evaluator generates
code corresponding to a parametric theory and this code is reused with the
actual parameters used as arguments to the operations.  Theory
interpretations cannot be treated as arguments in this manner since there
is no fixed set of parameters; parameters can vary according to the
interpretation.  Also, non-executable operations can become executable as
a result of the interpretation.

In summary, we believe that theory interpretations are a significant
extension to the PVS specification language.  Our implementation of this
in PVS3.0 is simple yet powerful.  We expect theory interpretations to be
a widely used feature of PVS.

\newpage
\bibliographystyle{alpha}
\addcontentsline{toc}{chapter}{Bibliography}
\bibliography{../pvs}
\end{document}

% Master File: language.tex
% Document Type: LaTeX

\chapter{Name Resolution}\label{names}\label{resolution}

Names in PVS are used to denote theories, variables, constants, and
formulas.  New names are introduced by declarations.  The syntax of names
is given in Figure~\ref{bnf-names}.

\pvsbnf{bnf-names}{Name Syntax}

The simplest form of a name is an \emph{idop}, \ie\ an identifier or
operator symbol.  This is generally all that is needed, unless names are
overloaded.

The overloading of names, both from different theories and within a single
theory, is allowed as long as there is some way for the system to
distinguish references to them.  Names from different theories may be
distinguished by prefixing them with the theory name.  Within a theory,
all names of the same kind must be unique, except for expression kinds;
which need only be unique up to the signature.  This is because the
signature is enough to distinguish these declarations.  For example, if
\texttt{<} is declared to have signature \texttt{[bool,int -> bool]}, the
system will recognize from the context that \texttt{TRUE < 3} contains a
reference to this declaration, whereas \texttt{2 < 3} does
not.\footnote{Of course, this assumes that \texttt{TRUE} has not itself
been overloaded.}  If the use of the name is not enough to distinguish,
coercion may be used to specify the signature directly (see
page~\pageref{coercions}).  Theory parameters must be unique across all
kinds.

There are three possible forms for names (two for theory names, which
appear in \texttt{IMPORTING}s, \texttt{EXPORTING} \texttt{WITH}s, and
theory declarations).  Given a theory named \emph{theoryid}, with formal
parameters $f_1,\ldots,f_n$, that contains a declaration named \emph{id},
the following three forms may be used to reference the declaration in a
theory that imports \emph{theoryid}:
\begin{itemize}
\item \emph{theoryid}\texttt{[$a_1,\ldots,a_n$]}.\emph{id}

\item \emph{id}\texttt{[$a_1,\ldots,a_n$]}

\item \emph{id}
\end{itemize}
where the $a_i$ are expressions or type expressions that are compatible
with the formal parameters as described in Section~\ref{parameters}.  Note
that any of these forms may have \emph{mappings} immediately after the
actual parameters.  As described in Section~\ref{mappings}, these can be
viewed as an extension of the actuals.  Note also that theory names allow
different kinds of mappings.  The forms above are listed in order of
increasing likelihood of ambiguity---that is, names that are given with
just an \emph{id} are far more likely to produce an ambiguity than those
further up.  Note that even the top form may be ambiguous, as \emph{id}
may be declared more than once in \emph{theoryid}.  If this is the case,
then either the context will disambiguate the name or a type will have to
be supplied in the form of a coercion expression, \eg\
\texttt{\emph{id}::~nat}.  This kind of ambiguity is allowed only for
constants (including functions and recursive functions) and variables.

Names are resolved based on the expected type and the number and types of
arguments to which the name is applied.  The expected type is generally
determined from the context of the name, for example in
\begin{pvsex}
  c1: int = c2
\end{pvsex}
\texttt{c2} has expected type \texttt{int}.  For most expressions, this is
straight-forward, but applications create special problems.  For example,
in
\begin{pvsex}
  f: FORMULA c1 = c2
\end{pvsex}
we know that the equality (which \emph{is} an application) has range type
\texttt{boolean} since it is a formula, but this gives no information
about the types of the arguments.  We will first describe the simpler
situation, and then explain how names used as operators of an application
are resolved.

In general, the typechecker works by first collecting possible types for
the expressions, and then chooses from among the possible types using the
expected type, which is determined from the context of the expression.
The expected type is used to resolve ambiguities, but otherwise does not
contribute to the type of an expression.  Thus if \texttt{2 + 3}
typechecks, and \texttt{+} has not been redeclared, then it has type
\texttt{number\_field} regardless of its context.  However, for the purpose
of checking for TCCs, it may be treated as having a different type
depending on the expected type and the available judgements.

% Document Type: LaTeX
% Master File: language.tex

\chapter{Abstract Datatypes}\label{datatypes}\label{adts}

PVS provides a powerful mechanism for defining abstract datatypes.  This
mechanism is akin to, but more sophisticated than, the \emph{shell}
principle of the Boyer-Moore prover~\cite{Boyer-Moore79}).  A PVS datatype
is specified by providing a set of \emph{constructors} along with
associated \emph{accessors} and \emph{recognizers}.  When a datatype is
typechecked, a new theory is created that provides the axioms and
induction principles needed to ensure that the datatype is the initial
algebra defined by the constructors.

\pvsbnf{bnf-adts}{Datatype Syntax}

The syntax for PVS datatypes is given in Figure~\ref{bnf-adts}.  Datatypes
may appear at the \emph{top-level} as with theory declarations, or
\emph{in-line} as a declaration within a theory.\footnote{Enumeration
types are actually in-line datatypes---see Section~\ref{enum-types}.}
Typechecking a top-level datatype named \texttt{foo} causes the generation
of a new PVS file named \texttt{foo\_adt.pvs} containing up to three
theories as described below.  Typechecking an in-line datatype has the
effect of adding new declarations to the current theory, effectively
replacing the in-line datatype.  In-line datatypes are more restricted:
they may not have formal parameters or assuming parts, and they will not
generate the recursive combinators described below.  The declarations
generated for an in-line datatype may be viewed using the
\texttt{M-x~prettyprint-expanded} command (see the \emph{PVS System
Guide}~\cite{PVS:userguide}).

\section{A Datatype Example: \texttt{stack}}\label{stacks-adt}
An example of a datatype is \texttt{stack}:
\begin{session}
  stack[T: TYPE]: DATATYPE
   BEGIN
    empty: empty?
    push(top:T, pop:stack): nonempty?
   END stack
\end{session}
The \texttt{stack} datatype has two \emph{constructors}, \texttt{empty} and
\texttt{push}, that allow stack elements to be constructed.  For example,
the term \texttt{push(1, empty)} is an element of type \texttt{stack[int]}.
The \emph{recognizers} \texttt{empty?}\ and \texttt{nonempty?}\ are predicates
over the \texttt{stack} datatype that are true when their argument is
constructed using the corresponding constructor.  Given a \texttt{stack}
element that is known to be \texttt{nonempty?}, the \emph{accessors}
\texttt{top} and \texttt{pop} may be used to extract the first and second
arguments.

Typechecking the \texttt{stack} specification automatically creates a new
file \texttt{stack\_adt.pvs}, that contains the material found in
the next five figures.  This new file contains three theories:
\texttt{stack\_adt}, \texttt{stack\_adt\_map}, and
\texttt{stack\_adt\_reduce}.

\pvstheory{stack_adtA-alltt}{Theory \texttt{stack\_adt} (continues)}{stack_adtA-alltt}
\pvstheory{stack_adtB-alltt}{Theory \texttt{stack\_adt} (continues)}{stack_adtB-alltt}
\pvstheory{stack_adtC-alltt}{Theory \texttt{stack\_adt} (continues)}{stack_adtC-alltt}
\pvstheory{stack_adtD-alltt}{Theory \texttt{stack\_adt\_map}}{stack_adtD-alltt}
\pvstheory{stack_adtE-alltt}{Theory \texttt{stack\_adt\_reduce}}{stack_adtE-alltt}

The first theory \texttt{stack\_adt} is parametric in type \texttt{T}.
This is a specification of ``stacks of \texttt{T}'', where \texttt{T} may
be instantiated by any defined type when the stacks datatype is imported.
Thus ``stacks of integers'' as well as ``stacks of stacks of integers''
may be defined using this theory.  The first few lines of the theory
define the main type of stacks \texttt{stack}, the recognizers
\texttt{emptystack?} and \texttt{nonemptystack?}, the constructors
\texttt{empty} and \texttt{push}, and the accessors \texttt{top} and
\texttt{pop} are declared.

The \texttt{stack\_ord} function is defined, and an axiom provided for
it's definition.  This is provided instead of a disjointness axiom,
because the disjointness axiom becomes difficult to generate and use if
the number of constructors is large.  The disjointness comes from the fact
that the natural numbers are distinct.  The \texttt{ord} function is then
defined to return \texttt{0} on an empty stack and \texttt{1} on a
nonempty stack.  This is the same function as \texttt{stack\_ord}, but is
easier to use.

Then a series of axioms are given.  The
\texttt{stack\_empty\_extensionality} axiom states that there is only one
bottom element of the datatype: \texttt{empty}.
\texttt{stack\_push\_extensionality} states that any two stacks that have
the same \texttt{top} and \texttt{pop} (have the same components) are the
same.  The \texttt{stack\_push\_eta} axiom states that \texttt{pop}ping
and \texttt{push}ing the same element off and onto a stack results in a
stack identical to the original.  \texttt{stack\_top\_push} says that if
you \texttt{push} and element on a stack, you get that same element when
you \texttt{pop} it back off.  \texttt{stack\_pop\_push} says that pushing
something on a stack and then popping it back off results in the original
stack.

The \texttt{stack\_inclusive} axiom states that all stacks are either
\texttt{empty?} or \texttt{nonempty?}.  The PVS prover builds this axiom
in, so that it rarely needs be cited by a user.

\newpage
The next axiom, \texttt{stack\_induction}, introduces an induction formula
for stacks stating that any predicate $p$ of stacks that
\begin{enumerate}
\item holds for the empty stack (the base case), and
\item if $p$ holds for some stack then $p$ holds for the result of
\texttt{push}ing anything of the right type onto that stack (the induction
step),
\end{enumerate}
then $p$ holds for all stacks.

Then some useful functions are defined over stacks.  The stack predicate
\texttt{every} takes as arguments a predicate over \texttt{T} and a stack
and returns \texttt{TRUE} iff all elements on the stack satisfy the given
predicate.  \texttt{every} is introduced in both curried and uncurried
forms.  The stack predicate \texttt{some} is dual to \texttt{every},
returning \texttt{TRUE} iff there is some element on the stack that
satisfies the predicate.  The \texttt{subterm} predicate takes two stacks
and returns \texttt{TRUE} if and only if the first argument stack is a
subterm of the second.  That is, if the second stack consists of the first
stack with some (perhaps zero) elements pushed onto it.  The \texttt{<<}
predicate is the strict (irreflexive) \texttt{subterm} predicate.  Thus
for all stacks $s$, \texttt{subterm}$(s,s)$ holds, but for no stack $s$
does \texttt{<<}$(s,s)$ hold.  An alternative equivalent definition of
\texttt{<<} is as follows:
\begin{pvsex}
  <<(x: stack, y: stack): boolean = subterm(x,y) AND NOT x = y
\end{pvsex}
However, this definition is more awkward to use in a proof, as the
recursion is hidden in the definition of \texttt{subterm}.  For this
reason the definitions for \texttt{every}, \texttt{some},
\texttt{subterm}, and \texttt{<<}, are each defined as standalone
functions, though some of them could be defined in terms of the others.

The last four declarations of the theory \texttt{stack\_adt} are functions
which reduce a stack to a natural number or to an ordinal.  These
functions are useful for simplifying the proof of termination of
user-defined functions over stacks.  Recall that PVS requires recursive
functions to include a \emph{measure}, which is used to generate
termination conditions.  The primary use of the recursive combinator is to
allow measure functions to be specified.  The function
\texttt{reduce\_nat} takes a natural number and a function.  The natural
number is used for the empty stack, and then for each element on the
stack, the input function is applied to the element from the stack and the
current reduced natural number, returning a natural number.  The function
\texttt{reduce\_nat} returns the final natural number.  The function
\texttt{REDUCE\_nat} is analogous to \texttt{reduce\_nat}, except that the
reducing function is also given the entire contents of the stack.  This
version of reduction can be useful for complicated measures that involve,
for example, the number of repeated elements appearing on the stack.  The
simpler form of reduce is difficult to apply to such situations.  The
functions \texttt{reduce\_ordinal} and \texttt{REDUCE\_ordinal} are
analogous to \texttt{reduce\_nat} and \texttt{REDUCE\_nat} except that
they return ordinal numbers instead of natural numbers.  It is rare that a
termination argument requires the use of ordinals, so the simpler
\texttt{reduce\_nat} form is more often used.  This completes the
description of the \texttt{stack\_adt} theory.

The second theory in the file \texttt{stack\_adt.pvs} is
\texttt{stack\_adt\_map}.  This theory takes two types \texttt{T} and
\texttt{T1} as parameters, imports the \texttt{stack\_adt} theory, and
defines a mapping from \texttt{stacks[T]} to \texttt{stacks[T1]}.  The
higher-order \texttt{map} function takes a function \texttt{f} of type
\texttt{[T -> T1]}, and a stack of \texttt{T}, and returns a stack of
\texttt{T1} obtained by applying \texttt{f} to each element on the input
stack.  \texttt{map} is defined in both curried and uncurried forms.
\texttt{map} couldn't reside in the \texttt{stack\_adt} theory because
that theory has only one type parameter, while the \texttt{map} functions
require two: In order to construct and access stacks in two theories,
\texttt{map} must be parameterized in the two types.

Also in the \texttt{stack\_adt\_map} is a relational \texttt{every}
function.  It lifts a relation \texttt{R} between \texttt{T} and \texttt{T1},
to stacks of \texttt{T} and \texttt{T1}.  It is true if the stacks are the
same size, and corresponding elements satisfy \texttt{R}.

The third and final theory generated from \texttt{stack\_pvs} is
\texttt{stack\_adt\_reduce}.  This theory provides a generalized version
of \texttt{reduce\_nat} and \texttt{REDUCE\_nat}.  It takes as parameters
a type \texttt{T} and a range type \texttt{range}.  It defines a
generalized \texttt{reduce} which reduces stacks of \texttt{T} to elements
of \texttt{range}.  The functions \texttt{reduce\_nat},
\texttt{REDUCE\_nat}, \texttt{reduce\_ordinal}, and
\texttt{REDUCE\_ordinal} could have been defined using
\texttt{stack\_adt\_reduce}, but the direct definitions are provided for
additional user convenience.  The generalized \texttt{reduce} can be used
to provide evidence of termination of user-defined functions, but the
predefined versions such as \texttt{reduce\_nat} are easier to use in most
cases.

\section{Datatype Details}

In general, a datatype declaration has the form
\begin{pvsex}
  adt: DATATYPE WITH SUBTYPES S\(\sb{1}\), \ldots, S\(\sb{n}\)
    BEGIN
     cons\(\sb1\)(acc\(\sb{11}\): T\(\sb{11}\), \ldots, acc\(\sb{1{n\sb1}}\): T\(\sb{1{n\sb1}}\)): rec\(\sb1\) : S\(\sb{i\sb{1}}\)
     \vdots
     cons\(\sb{m}\)(acc\(\sb{m1}\): T\(\sb{m1}\), \ldots, acc\(\sb{1n\sb{m}}\): T\(\sb{1n\sb{m}}\)): rec\(\sb{m}\) : S\(\sb{i\sb{m}}\)
    END adt
\end{pvsex}
%
where the \texttt{cons$_i$} are the
\emph{constructors}\index{constructor}\index{datatype!constructor}, the
\texttt{acc$_{ij}$} are the
\emph{accessors}\index{accessor}\index{datatype!accessor}, the
\texttt{T$_{ij}$} are type expressions, and the \texttt{rec$_i$} are
\emph{recognizers}\index{recognizer}\index{datatype!recognizer}.  Each
line is referred to as a \emph{constructor
specification}\index{constructor specification}\index{datatype!constructor
specification}.  There are a number of restrictions enforced on
constructor specifications:
\begin{itemize}

\item The datatype identifier may not be used for a recognizer,
accessor, or subtype:\newline
($\texttt{adt} \not\equiv \texttt{rec}_i$ for all $i$, $\texttt{adt}
\not\equiv \texttt{acc}_{ij}$ for all $i$ and $j$, and $\texttt{adt}
\not\equiv \texttt{S}_i$ for all $i$).

\item The subtype names must be unique:
($i \neq j \Rightarrow \texttt{S}_i \not\equiv \texttt{S}_j$)

\item Each subtype name must be used at least once.

\item The constructor names must be unique:
($i \neq j \Rightarrow \texttt{cons}_i \not\equiv \texttt{cons}_j$).

\item The recognizer names must be unique:
($i \neq j \Rightarrow \texttt{rec}_i \not\equiv \texttt{rec}_j$).

\item No identifier may be used as both a constructor and a recognizer:\newline
($\texttt{cons}_i \not\equiv \texttt{rec}_j$ forall $i$ and $j$).

\item Duplicate accessor identifiers are not allowed within a single
constructor specification:
($j \neq k \Rightarrow \texttt{acc}_{ij} \not\equiv \texttt{acc}_{ik}$).

\end{itemize}

As seen in the \texttt{stack} example, datatypes may be recursive; this is
the case when the type of one or more of the accessors reference the
datatype.  In PVS, all such occurrences must be positive, where a type
occurrence \texttt{T} is positive in a type expression $\tau$ iff either
\begin{itemize}
\item $\tau\equiv \texttt{T}$.

\item $\tau\equiv \{x:\tau'|p(x)\}$ and the occurrence \texttt{T} is
positive in $\tau'$.

\item $\tau\equiv [{\tau_1} \rightarrow {\tau_2}]$ and the occurrence
\texttt{T} is positive in $\tau_2$\@.  For example, \texttt{T} occurs
positively in \texttt{sequence[T]} where \texttt{sequence[T]} is defined
in the PVS prelude as the function type \texttt{[nat -> T]}\@.

\item $\tau \equiv [\tau_1,\ldots, \tau_n]$ and the occurrence \texttt{T}
is positive in some $\tau_i$.

\item $\tau\equiv [\#\ l_1 : \tau_1, \ldots, l_n : \tau_n\ \#]$ and the occurrence \texttt{T} is positive in some $\tau_i$\@. 

\item $\tau\equiv \mbox{\emph{datatype}}[\tau_1,\ldots, \tau_n]$, where
\emph{datatype} is a previously defined datatype and the occurrence
\texttt{T} is positive in $\tau_i$, where $\tau_i$ is a \emph{positive
parameter} of \emph{datatype}\@.
\end{itemize}

When a top-level datatype is given with formal type parameters, they are
checked for whether their occurrences are all positive; this is used as
described above for any datatype that imports this one, as well as
determining some of the declarations described below.

When a datatype is typechecked, a number of new declarations are
generated:
\begin{itemize}

\item The datatype identifier is used to create an uninterpreted type
declaration.  In general, the term \emph{datatype} refers to this type.

\item Each recognizer is used to declare an uninterpreted subtype of the
datatype.

\item Each subtype identifier is used to declare an interpreted type that
is the disjunction of the types given by the recognizers that reference
the subtype identifier in the constructor specification.

\item Each constructor and accessor is used to generate a constant
declaration.

\item An \texttt{\emph{id}\_ord} uninterpreted function is created, and an
axiom \texttt{\emph{id}\_ord\_defaxiom} defines its values.  This is
provided instead of a disjointness axiom, because the disjointness axiom
becomes difficult to generate and use when the number of constructors is
large.

\item An \texttt{ord} function is generated that gives a zero-based number
to each constructor (e.g., \texttt{ord(null) = 0} and \texttt{cons(1,null)
= 1}).  This is mostly useful for enumeration types.

\item An extensionality axiom is generated for each constructor
specification.

\item An eta axiom is generated for each constructor specification
that has accessors.

\item For each accessor an axiom is created that says that the accessor
composed with the corresponding constructor returns the correct value; \eg\
\begin{pvsex}
  acc\(\sb{ij}\)(cons\(\sb{i}\)(e\(\sb{i1}\),\ldots, e\(\sb{i{m\sb{i}}})\) = e\(\sb{ij}\)
\end{pvsex}

\item An inclusive axiom is generated that says that every element of
the datatype belongs to at least one recognizer subtype.  This axiom is
not actually needed in practice as the prover checks for this directly.

\item Two induction schemes are provided for proving properties of the
datatype.

\item If there is at least one constructor with accessors,\footnote{Note
that enumeration types have no accessors.}  and there are positive type
parameters to the datatype, then \texttt{every} and \texttt{some}
functions are defined that provide a predicate on the datatype in terms of
the positive types.

\item The \texttt{subterm} and \texttt{<<} (irreflexive subterm) functions
are defined, and an axiom is generated that states that \texttt{<<} is
well-founded.  This allows it to be used as an ordering relation in
recursive function definitions.

\item If there is at least one constructor with
accessors,\addtocounter{footnote}{-1}\footnotemark{} the
\texttt{reduce\_nat}, \texttt{REDUCE\_nat}, \texttt{reduce\_ordinal}, and
\texttt{REDUCE\_ordinal} recursion combinators are defined.  These provide
a means for defining notions like the size or depth of a datatype term.

Note that accessor subtypes involving the datatype are
``lifted''.  The following example shows why.
\begin{pvsex}
  dt: DATATYPE
   BEGIN
    c0: c0?
    c1(a1: \setb{}x: list[dt] | length(x) > 0\sete): c1?
    c2(a2: \setb{}x: list[dt] | every(c0?)(x)\sete): c2?
   END dt
\end{pvsex}
Consider the \texttt{reduc\_nat} function.  The signature for the lifted
mapping function for \texttt{c1} and \texttt{c2} are the same:
\texttt{[list[nat] -> nat]}.  It's obvious the mapping function for
\texttt{c2} function could have the signature \texttt{[\setb{}x: list[nat]
| length(x) > 0\sete{} -> nat]}, but there is no obvious way to map
\texttt{c2} without lifting it.  Since it is not trivial to determine
which predicates map nicely, we lift them all.  In the future we may
provide heuristics that refine this.

\item If some type parameter is positive a \texttt{map} function is
generated in a separate theory.  Every positive type parameter in the
datatype is associated with a pair of \texttt{map} parameters, which form
the domain and range of a corresponding function argument.  Given a set of
such functions and a term of the datatype, \texttt{map} returns a term
that has the same structure, but with the ``leaf'' elements replaced by
the function values.

\item A separate theory is generated for the \texttt{reduce} and
\texttt{REDUCE} functions.  These generalize the \texttt{reduce} functions
above to an arbitrary range type.

\end{itemize}

Note that in the stack example, the \texttt{stack} type is nonempty, since
\texttt{empty} is an element of \texttt{stack} even if the parameter type
\texttt{T} is instantiated with an empty type.  However, there is no
requirement that a datatype be nonempty, though if it is imported and a
constant is declared to be of that type, a TCC will be generated as
described on page~\pageref{emptytypes} in section~\ref{emptytypes}.

The \texttt{stack\_adt} theory is parameterized in the type \texttt{T},
and introduces the uninterpreted type \texttt{stack}.  Under normal
circumstances, this would imply no relation between, for example,
\texttt{stack[nat]} and \texttt{stack[int]}.  However, since every
occurrence of \texttt{T} in the accessor types is positive, we can infer
that \texttt{stack[nat]} is a subtype of \texttt{stack[int]}.  In general,
given a type $T$ and a subtype $S \equiv \setb{}x:T | p(x)\sete$, then
\texttt{stack[$S$]} is treated the same as $\setb{}s:
\texttt{stack[}T\texttt{]} | \texttt{every}(p)(s)\sete$.  When a datatype
has a mix of positive and nonpositive type parameters, the subtype
relation only holds for the positive ones.  For example, in the datatype
\begin{session}
  dt[T1, T2: TYPE, c: T1]: DATATYPE
   BEGIN
    c(a1: T1, a2: [T2 -> T1]): c?
   END dt
\end{session}
\texttt{T1} is positive and \texttt{T2} is not, so \texttt{dt[nat, nat,
0]} is a subtype of \texttt{dt[int, nat, 0]}, but is not a subtype of
\texttt{dt[nat, int, 0]}, nor is it a subtype of \texttt{dt[nat, nat, 1]}.

More complex datatypes lead to correspondingly more complex declarations;
for example, in the following contrived datatype
\begin{session}
  adt1[t1,t2: TYPE, c:t1]: DATATYPE
   BEGIN
    bottom: bottom?
    c1(a11:t1, a12: [t2 -> int]): c1?
    c2(a21:adt1, a22:[nat -> adt1], a23: list[adt1]): c2?
    c3(a31:[list[int] -> adt1],
       a32:[# a: adt1, b: [int -> adt1] #],
       a33:[adt1, [set[int] -> adt1]]) : c3?
   END adt1
\end{session}
the curried \texttt{every} is generated as follows:
\begin{session}
  every(p: PRED[t1])(a1: adt1):  boolean =
      CASES a1
        OF bottom: TRUE,
           c1(c11_var, c12_var): p(c11_var),
           c2(c21_var, c22_var, c23_var):
             every(p)(c21_var) AND
              every(every(p))(c22_var) AND every[adt1](every(p))(c23_var),
           c3(c31_var, c32_var, c33_var):
                  (FORALL (x1: list[int]): every(p)(c31_var(x1)))
              AND every(p)(a(c32_var))
              AND FORALL (x: int): every(p)(b(c32_var)(x))
              AND every(p)(c33_var`1)
              AND FORALL (x: set[int]): every(p)(c33_var`2(x))
        ENDCASES;
\end{session}
Note that this is only defined for predicates over \texttt{t1}, since
the occurrence of \texttt{t2} in the constructor specification for
\texttt{c2} is not positive.

As with record types, constructor selectors may be dependent.  Here is a
simple example.
\begin{session}
  depdt: DATATYPE
   BEGIN
    b: b?
    c(x: int, y: \setb{}z: int | z < x\sete): c?
   END depdt
\end{session}

\section{Datatype Subtypes}

The \texttt{WITH SUBTYPES} keyword introduces a set of subtype names.
These are useful, for example, in defining the nonterminals of a language.
For example, we might try to describe a simple typed lambda calculus:
\begin{eqnarray*}
T & ::= & B \;|\; T \rightarrow T \\
E & ::= & x \;|\; \lambda x:T.E \;|\; E(E)
\end{eqnarray*}
This is difficult to express using datatypes without subtypes, but is
reasonably straightforward with them:\footnote{\texttt{TYPE},
\texttt{LAMBDA}, and \texttt{VAR} are PVS keywords, so variants are used
here.}
\begin{session}
tlc: DATATYPE WITH SUBTYPES typ, expr
 BEGIN
 base_type(n:nat): base_type? : typ
 fun_type(dom, ran: typ): fun_type? : typ
 expr_var(n:nat): expr_var? : expr
 lambda_expr(lvar:(expr_var?), ltype: typ, lexpr: expr)
                            : lambda_expr? : expr
 application(fun, arg: expr): application? : expr
 END tlc
\end{session}
In addition to the usual generated declarations, this generates
\begin{session}
  typ((x: tlc)): boolean = base_type?(x) OR fun_type?(x);
  typ: TYPE = \setb{}x: tlc | base_type?(x) OR fun_type?(x)\sete
  expr((x: tlc)): boolean =
     expr_var?(x) OR lambda_expr?(x) OR application?(x);
  expr: TYPE =
     \setb{}x: tlc | expr_var?(x) OR lambda_expr?(x) OR application?(x)\sete
\end{session}
immediately after the declarations generated for the recognizers, so they
may be referenced in the accessor types.  Note that only a single
induction scheme is generated.  To induct over a particular subtype,
extend the property of interest to the entire datatype so that it returns
true for everything else.


\section{\texttt{CASES} Expressions}\label{cases-expressions}
\index{cases expressions}

The \texttt{CASES} expression uses a simple form of pattern-matching on
abstract datatypes.  Patterns are of the form $c(x_1,\ldots, x_n)$ where
$c$ is an $n$-ary constructor and $x_1,\ldots, x_n$ is a list of distinct
variables.  Patterns here are simple so that certain logical properties of
the expression are easy to check.  Patterns are not defined in the grammar
but in the type rules, since the notion of a variable or a constructor is
only defined in the type rules.

For example, if \texttt{x} is of type \texttt{stack}, the cases expression
\begin{pvsex}
  CASES x OF
    empty : FALSE,
    push(y, z) : even?(y) AND empty?(z)
  ENDCASES
\end{pvsex}
is \texttt{TRUE} if \texttt{x} is a singleton even integer, and otherwise is
false.  This expression can be translated into
\begin{pvsex}
  IF empty?(x)
     THEN FALSE
     ELSE LET (y, z) = (car(x), cdr(x))
           IN even?(y) AND empty?(z)
  ENDIF
\end{pvsex}

The \texttt{CASES} expression also allows an \texttt{ELSE} clause, which
comes last and covers all constructors not previously mentioned in a
pattern.  If the \texttt{ELSE} clause is missing, and not all constructors
have been mentioned, then a \emph{cases TCC}\index{cases
TCC}\index{TCC!cases} is generated which states that the expression is not
any of the missing elements.  For example, if the \texttt{x} above is
declared to be a subtype of \texttt{stack} in which \texttt{empty} is
excluded, then the \texttt{empty} case can safely be left out, and a \tcc\
will be generated that obligates the user to prove that \texttt{x} is not
\texttt{empty}.  There is a trade-off here between simpler specifications
and simpler verifications; if the \texttt{empty} case is left in, then
there is no obligation to prove, but the extra case clutters up the
specification, and can mislead the reader into thinking that the
\texttt{empty} case is possible.  In general, we feel that the
specification should be as perspicuous as possible, even if it means a
little more work behind the scenes.


\appendix
\chapter{The Grammar}\label{grammar}

The complete \pvs\ grammar is presented in this Appendix, along with a
discussion of the notation used in presenting the grammar.

The conventions used in the presentation of the syntax are as follows.
\index{syntax!conventions}

\begin{itemize}

\item Names in {\it italics\/} indicate syntactic classes and
metavariables ranging over syntactic classes.

\item The reserved words of the language are
      printed in \lit{tt font, UPPERCASE}.

\item An optional part {\it A\/} of a clause is enclosed in square brackets:
\opt{{\it A\/}}.

\item Alternatives in a syntax production are separated by a bar
(``\choice''); a list of alternatives that is embedded in the right-hand
side of a syntax production is enclosed in brackets, as in

\begin{bnf}
\production{ExportingName}
{IdOp \opt{\lit{:} \brc{TypeExpr \choice \lit{TYPE} \choice \lit{FORMULA}}}}
\end{bnf}


\item Iteration of a clause {\it B\/} one or more times is indicated by
enclosing it in brackets followed by a plus sign: \ite{{\it B\/}};
repetition zero or more times is indicated by an asterisk instead of the
plus sign: \rep{{\it B\/}}.

\item A double plus or double asterisk indicates a clause separator; for
example, \reps{{\it B\/}}{,} indicates zero or more repetitions of the
clause {\it B} separated by commas.

\item Other items printed in tt font on the right hand side of
      productions are literals.  Be careful to distinguish where BNF
symbols occur as literals, \eg\ the BNF brackets \brc{} versus the
literal brackets \lit{\{\}}.

\end{itemize}

\subsubsection*{Specification}
\par\noindent
\spvsbnf{bnf-theory}

\subsubsection*{Importings and Exportings}
\par\noindent
\spvsbnf{bnf-exporting}

\subsubsection*{Assumings}
\par\noindent
\spvsbnf{bnf-assuming}

\subsubsection*{Theory Part}
\par\noindent
\spvsbnf{bnf-theory-part}

\subsubsection*{Declarations}
\par\noindent
\spvsbnf{bnf-decls}

\subsubsection*{Type Expressions}
\par\noindent
\spvsbnf{bnf-type-expr}

\subsubsection*{Expressions}
\par\noindent
\spvsbnf{bnf-expr}

\subsubsection*{Expressions (continued)}
\par\noindent
\spvsbnf{bnf-expr-aux}

\subsubsection*{Names}
\par\noindent
\spvsbnf{bnf-names}

\subsubsection*{Identifiers}
\par\noindent
\spvsbnf{bnf-lexical}

\subsubsection*{Datatypes}
\par\noindent
\spvsbnf{bnf-adts}

%% Derived from John Rushby's prelude.tex, modified for NFSS2
%
% define variants of the \LaTeX macro that avoid using \sc
% for use in headings
%

% Define fonts that work in math or text mode
\def\dwimrm#1{\ifmmode\mathrm{#1}\else\textrm{#1}\fi}
\def\dwimsf#1{\ifmmode\mathsf{#1}\else\textsf{#1}\fi}
\def\dwimtt#1{\ifmmode\mathtt{#1}\else\texttt{#1}\fi}
\def\dwimbf#1{\ifmmode\mathbf{#1}\else\textbf{#1}\fi}
\def\dwimit#1{\ifmmode\mathit{#1}\else\textit{#1}\fi}
\def\dwimnormal#1{\ifmmode\mathnormal{#1}\else\textnormal{#1}\fi}

\def\BigLaTeX{{\rm L\kern-.36em\raise.3ex\hbox{\small\small A}\kern-.15em
    T\kern-.1667em\lower.7ex\hbox{E}\kern-.125emX}}
\def\BoldLaTeX{{\bf L\kern-.36em\raise.3ex\hbox{\small\small\bf A}\kern-.15em
    T\kern-.1667em\lower.7ex\hbox{E}\kern-.125emX}}
%\def\labelitemi{$\bullet$}
\def\labelitemii{$\circ$}
\def\labelitemiii{$\star$}
\def\labelitemiv{$\diamond$}
\newcommand{\tcc}{{\small\small TCC}}
\newcommand{\tccs}{\tcc s}
\newcommand{\emacs}{{Emacs}}
\newcommand{\Emacs}{\emacs}
\newcommand{\ehdm}{{E{\small\small HDM}}}
\newcommand{\Ehdm}{\ehdm}
\newcommand{\tm}{$^{\mbox{\tiny TM}}$}
\newcommand{\hozline}{{\noindent\rule{\textwidth}{0.4mm}}}

\newcommand{\allclear}%
  {\mbox{\boldmath$\stackrel{\raisebox{-.2ex}[0pt][0pt]%
              {$\textstyle\oslash$}}{\displaystyle\bot}$}}

\newenvironment{private}{}{}

\newenvironment{smalltt}{\begin{alltt}\small}{\end{alltt}}

\newlength{\hsbw}

\newenvironment{session}%
  {\begin{flushleft}
   \setlength{\hsbw}{\linewidth}
   \addtolength{\hsbw}{-\arrayrulewidth}
   \addtolength{\hsbw}{-\tabcolsep}
   \begin{tabular}{@{}|c@{}|@{}}\hline 
   \begin{minipage}[b]{\hsbw}
   \begingroup\small\mbox{ }\\[-1.8\baselineskip]\begin{alltt}}
  {\end{alltt}\endgroup\end{minipage}\\ \hline 
   \end{tabular}
   \end{flushleft}}

\newenvironment{smallsession}%
  {\begin{flushleft}
   \setlength{\hsbw}{\linewidth}
   \addtolength{\hsbw}{-\arrayrulewidth}
   \addtolength{\hsbw}{-\tabcolsep}
   \begin{tabular}{@{}|c@{}|@{}}\hline 
   \begin{minipage}[b]{\hsbw}
   \begingroup\footnotesize\mbox{ }\\[-1.8\baselineskip]\begin{alltt}}%
  {\end{alltt}\endgroup\end{minipage}\\ \hline 
   \end{tabular}
   \end{flushleft}}

\newenvironment{spec}%
  {\begin{flushleft}
   \setlength{\hsbw}{\textwidth}
   \addtolength{\hsbw}{-\arrayrulewidth}
   \addtolength{\hsbw}{-\tabcolsep}
   \begin{tabular}{@{}|c@{}|@{}}\hline 
   \begin{minipage}[b]{\hsbw}
   \begingroup\small\mbox{ }\\[-0.2\baselineskip]}%
  {\endgroup\end{minipage}\\ \hline 
   \end{tabular}
   \end{flushleft}}

\newcommand{\memo}[1]%
  {\mbox{}\par\vspace{0.25in}%
   \setlength{\hsbw}{\linewidth}\addtolength{\hsbw}{-1.5ex}%
   \noindent\fbox{\parbox{\hsbw}{{\bf Memo: }#1}}\vspace{0.25in}}

\newcommand{\nb}[1]%
  {\mbox{}\par\vspace{0.25in}%
   \setlength{\hsbw}{\linewidth}\addtolength{\hsbw}{-1.5ex}%
   \noindent\fbox{\parbox{\hsbw}{{\bf Note: }#1}}\vspace{0.25in}}

\newcommand{\comment}[1]{}
\newcommand{\exfootnote}[1]{}
%\newcommand{\ifelse}[2]{#1}
\sloppy
\clubpenalty=100000
\widowpenalty=100000
%\displaywidowpenalty=100000
\setcounter{secnumdepth}{3} 
\setcounter{tocdepth}{3}
\setcounter{topnumber}{9}
\setcounter{bottomnumber}{9}
\setcounter{totalnumber}{9}
\renewcommand{\topfraction}{.99}
\renewcommand{\bottomfraction}{.99}
\renewcommand{\floatpagefraction}{.01}
\renewcommand{\textfraction}{.2}
\font\largett=cmtt10 scaled\magstep1
\font\Largett=cmtt10 scaled\magstep2
\font\hugett=cmtt10 scaled\magstep3


%\addcontentsline{toc}{chapter}{Bibliography}
\bibliographystyle{plain}
\bibliography{../pvs}

%\addcontentsline{toc}{chapter}{Index}   %% Put entry in T-O-C
%%\printindex  %% printindex makes extra call to "theindex"
{\smaller
\printindex
%% Document Type: LaTeX
% Master File: language.tex
\documentclass[12pt]{book}
\usepackage{alltt}
\usepackage{makeidx}
\usepackage{relsize}
\usepackage{boxedminipage}
\usepackage{url}
\usepackage{../../pvs}
\usepackage{../makebnf}
\usepackage[chapter]{tocbibind}
\usepackage{fancyvrb}
\usepackage[dvipsnames,usenames]{color}

\usepackage{amssymb}
\usepackage{mathpazo}
\usepackage{fontspec}
\setmainfont[Ligatures=TeX]{XITS}
\setmonofont{DejaVu Sans Mono}[Scale=MatchLowercase]
%\setmonofont{Free Mono}[Scale=0.8]
\usepackage[math-style=ISO]{unicode-math}
\renewcommand{\leadsto}{\rightsquigarrow}
%\setmathfont{XITS Math}

\topmargin -10pt
\textheight 8.5in
\textwidth 6.0in
\headheight 15 pt
\columnwidth \textwidth
\oddsidemargin 0.5in
\evensidemargin 0.5in   % fool system for page 0
\setcounter{topnumber}{9}
\renewcommand{\topfraction}{.99}
\setcounter{bottomnumber}{9}
\renewcommand{\bottomfraction}{.99}
\setcounter{totalnumber}{10}
\renewcommand{\textfraction}{.5}
\renewcommand{\floatpagefraction}{.1}
\usepackage{fancyhdr}
\pagestyle{fancy}
\raggedbottom

%\setcounter{secnumdepth}{1}

\index{type correctness condition|see{TCC}}
\makeindex

\usepackage[bookmarks=true,hyperindex=true,colorlinks=true,linkcolor=Brown,citecolor=blue,backref=page,pagebackref=true,plainpages=false,pdfpagelabels]{hyperref}

%\input{/project/pvs/doc/prelude}
\def\labelitemii{$\circ$}
\def\labelitemiii{$\star$}
\def\labelitemiv{$\diamond$}
\newcommand{\tcc}{{\small\small TCC}}
\newcommand{\tccs}{\tcc s}

%\renewcommand{\memo}[1]{\mbox{}\par\vspace{0.25in}\noindent\fbox{\parbox{.95\linewidth}{{\bf Memo: }#1}}\vspace{0.25in}}

\newcommand{\eg}{{\em e.g.\/},}
\newcommand{\ie}{{\em i.e.\/},}

\newcommand{\pvs}{PVS}

\newcommand{\ch}{\choice}
\newcommand{\rsv}[1]{{\rm\tt #1}}

\newcommand{\lpvstheory}[3]{\figurehead{\hozline\smaller\smaller\begin{alltt}}%
                           \figuretail{\end{alltt}\vspace{-0in}\hozline}%
                           \figurelabel{#3}\figurecap{#2}%
                           \begin{longfigure}\input{#1}\end{longfigure}}

\newcommand{\pvstheory}[3]
  {\begin{figure}[htb]\begin{boxedminipage}{\textwidth}%
   {\smaller\smaller\begin{alltt} \input{#1}\end{alltt}}\end{boxedminipage}%
   \caption{#2}\label{#3}\end{figure}}

\newcommand{\bpvstheory}[3]
  {\begin{figure}[b]\begin{boxedminipage}{\textwidth}%
   {\smaller\smaller\begin{alltt} \input{#1}\end{alltt}}\end{boxedminipage}%
   \caption{#2}\label{#3}\end{figure}}

\newcommand{\spvstheory}[1]
  {\vspace{0.1in}\par\noindent\begin{boxedminipage}{\textwidth}%
   {\smaller\smaller\begin{alltt} \input{#1}\end{alltt}}\end{boxedminipage}\vspace{0.1in}%
   }
%\newenvironment{spvstext}%
%  {\vspace{0.1in}\par\noindent\begin{boxedminipage}{\textwidth}%
%   \smaller\smaller\begin{alltt}}%
%  {\end{alltt}\end{boxedminipage}\vspace{0.1in}%
%   }


%  {\begin{boxedminipage}{\textwidth}{\smaller\smaller\begin{alltt}#1\end{alltt}}\end{boxedminipage}}
%\newenvironment{spvstheory}{\par\noindent\begin{boxedminipage}{\textwidth}\smaller\smaller\begin{alltt}}{\end{alltt}\end{boxedminipage}}

\newenvironment{pvsex}%
  {\setlength{\topsep}{0in}\smaller\begin{alltt}}%
  {\end{alltt}}

\newcommand{\pvsbnf}[2]
  {\begin{figure}[htb]\begin{boxedminipage}{\textwidth}%
   \input{#1}\end{boxedminipage}\caption{#2}\label{#1}\end{figure}}

\newcommand{\spvsbnf}[1]
  {\begin{boxedminipage}{\textwidth}\input{#1}\end{boxedminipage}}

\newcommand{\pidx}[1]{{\rm #1}} % primary index entry
\newcommand{\sidx}[1]{{\rm #1}} % secondary index entry
\newcommand{\cmdindex}[1]{\index{#1@\cmd{#1}}}
\newcommand{\icmd}[1]{\cmd{#1}\cmdindex{#1}}
\newcommand{\iecmd}[1]{\ecmd{#1}\cmdindex{#1}}
\newcommand{\buf}[1]{\texttt{#1}}
\newcommand{\ibuf}[1]{\buf{#1}\index{#1 buffer@\buf{#1} buffer}\index{buffers!\buf{#1}}}

\newenvironment{pvscmds}%
  {\par\noindent\smaller%
   \begin{tabular*}{\textwidth}{|l@{\extracolsep{\fill}}l@{\extracolsep{\fill}}l|}\hline%
     {\it Command} & {\it Aliases} & {\it Function}\\ \hline}%
  {\hline\end{tabular*}\vspace{0.1in}}

\newenvironment{pvscmdsna}%
  {\par\noindent\smaller%
   \begin{tabular*}{\textwidth}{|l@{\extracolsep{\fill}}l|}\hline%
     {\it Command} & {\it \,\,Function}\\ \hline}%
  {\hline\end{tabular*}\vspace{0.1in}}

\newcommand{\cmd}[1]{{\tt #1}}
\newcommand{\ecmd}[1]{{\tt M-x #1}}

\newcommand{\latex}{\LaTeX}                  %  LaTeX
\newcommand{\sun}{{S{\smaller\smaller UN}}}                 %  Sun
\newcommand{\sparc}{{S{\smaller\smaller PARC}}}             %  Sparc
\newcommand{\sunos}{{S{\smaller\smaller UN}OS}}             %  SunOS
\newcommand{\solaris}{{\em Solaris\/}}        %  Solaris
\newcommand{\sunview}{{S{\smaller\smaller UN}V{\smaller\smaller IEW}}} %SunView
\newcommand{\unix}{{U{\smaller\smaller NIX}}}               %  Unix
\newcommand{\lisp} {{\sc Lisp}}              %  Lisp
\newcommand{\gnu}{{Gnu Emacs}}           %  Gnu Emacs
\newcommand{\gnuemacs}{{Gnu Emacs}}      %  Gnu Emacs
\newcommand{\emacsl}{{Emacs-Lisp}}       %  Emacs Lisp
\newcommand{\shell}{{\sc Csh}}               %  C-shell

\newcommand{\update}[3]{#1\{#2\leftarrow #3\}}
\newcommand{\interp}[3]{\cal{M}\dlb {\tt #1 : #2 }\drb #3}
\newcommand{\myforall}[2]{(\forall{#1 .}\ #2)}
\newcommand{\myexists}[2]{(\exists{#1 .}\ #2)}
\newcommand{\mth}[1]{$ #1 $}
\newcommand{\labst}[2]{(\lambda{#1}.\ #2)}
\newcommand{\app}[2]{(#1\ #2)}
\newcommand{\problem}[1]{{\bf Exercise: } {\em #1}}
\newcommand{\rectype}[1]{[\# 1 \#]}
\newcommand{\recttype}[1]{{\tt [\# 1 \#]}}
\newcommand{\dlb}{\lbrack\!\lbrack}
\newcommand{\drb}{\rbrack\!\rbrack}
\newcommand{\cross}{\times}
\newcommand{\key}[1]{{\tt #1}}
\newcommand{\keyindex}[1]{\index{#1@\key{#1}}}
\newcommand{\ikey}[1]{\key{#1}\keyindex{#1}}
\newcommand{\keyword}[1]{{\smaller\texttt{#1}}}

\newenvironment{keybindings}%
  {\begin{center}\begin{tabular}{|l|l|}\hline Key & Function\\ \hline}%
  {\hline\end{tabular}\end{center}}
\def\rmif{\mbox{\bf if\ }}
\def\rmiff{\mbox{\bf \ iff \ }}
\def\rmthen{\mbox{\bf \ then }}
\def\rmelse{\mbox{\bf \ else }}
\def\rmend{\mbox{\bf end}}
\def\rmendif{\mbox{\bf \ endif}}
\def\rmotherwise{\mbox{\bf otherwise}}
\def\rmwith{\mbox{\bf \ with\ }}
\def\mapb{\char"7B\char"7B}
\def\mape{\char"7D\char"7D}
\def\setb{\char"7B}
\def\sete{\char"7D}

% ---------------------------------------------------------------------
% Macros for little PVS sessions displayed in boxes.
%
% Usage: (1) \setcounter{sessioncount}{1} resets the session counter
%
%        (2) \begin{session*}\label{thissession}
%             .
%              < lines from PVS session >
%             .
%            \end{session*}
%
%            typesets the session in a numbered box in ALLTT mode.
%
%  session instead of session* produces unnumbered boxes
%
%  Author: John Rushby
% ---------------------------------------------------------------------
\newlength{\hsbw}
\newenvironment{session}{\begin{flushleft}
 \setlength{\hsbw}{\linewidth}
 \addtolength{\hsbw}{-\arrayrulewidth}
 \addtolength{\hsbw}{-\tabcolsep}
 \begin{tabular}{@{}|c@{}|@{}}\hline 
 \begin{minipage}[b]{\hsbw}
% \begingroup\small\mbox{ }\\[-1.8\baselineskip]\begin{alltt}}{\end{alltt}\endgroup\end{minipage}\\ \hline
 \begingroup\sessionsize\vspace*{1.2ex}\begin{alltt}}{\end{alltt}\endgroup\end{minipage}\\ \hline
 \end{tabular}
 \end{flushleft}}
\newcounter{sessioncount}
\setcounter{sessioncount}{0}
\newenvironment{session*}{\begin{flushleft}
 \refstepcounter{sessioncount}
 \setlength{\hsbw}{\linewidth}
 \addtolength{\hsbw}{-\arrayrulewidth}
 \addtolength{\hsbw}{-\tabcolsep}
 \begin{tabular}{@{}|c@{}|@{}}\hline 
 \begin{minipage}[b]{\hsbw}
 \vspace*{-.5pt}
 \begin{flushright}
 \rule{0.01in}{.15in}\rule{0.3in}{0.01in}\hspace{-0.35in}
 \raisebox{0.04in}{\makebox[0.3in][c]{\footnotesize \thesessioncount}}
 \end{flushright}
 \vspace*{-.57in}
 \begingroup\small\vspace*{1.0ex}\begin{alltt}}{\end{alltt}\endgroup\end{minipage}\\ \hline 
 \end{tabular}
 \end{flushleft}}
\def\sessionsize{\footnotesize}
\def\smallsessionsize{\small}
\newenvironment{smallsession}{\begin{flushleft}
 \setlength{\hsbw}{\linewidth}
 \addtolength{\hsbw}{-\arrayrulewidth}
 \addtolength{\hsbw}{-\tabcolsep}
 \begin{tabular}{@{}|c@{}|@{}}\hline 
 \begin{minipage}[b]{\hsbw}
 \begingroup\smallsessionsize\mbox{ }\\[-1.8\baselineskip]\begin{alltt}}{\end{alltt}\endgroup\end{minipage}\\ \hline 
 \end{tabular}
 \end{flushleft}}
\newenvironment{spec}{\begin{flushleft}
 \setlength{\hsbw}{\textwidth}
 \addtolength{\hsbw}{-\arrayrulewidth}
 \addtolength{\hsbw}{-\tabcolsep}
 \begin{tabular}{@{}|c@{}|@{}}\hline 
 \begin{minipage}[b]{\hsbw}
 \begingroup\small\mbox{
}\\[-0.2\baselineskip]}{\endgroup\end{minipage}\\ \hline 
 \end{tabular}
 \end{flushleft}}
\newcommand{\memo}[1]{\mbox{}\par\vspace{0.25in}%
\setlength{\hsbw}{\linewidth}%
\addtolength{\hsbw}{-2\fboxsep}%
\addtolength{\hsbw}{-2\fboxrule}%
\noindent\fbox{\parbox{\hsbw}{{\bf Memo: }#1}}\vspace{0.25in}}
\newcommand{\nb}[1]{\mbox{}\par\vspace{0.25in}\setlength{\hsbw}{\linewidth}\addtolength{\hsbw}{-1.5ex}\noindent\fbox{\parbox{\hsbw}{{\bf Note: }#1}}\vspace{0.25in}}

%%% Local Variables: 
%%% mode: latex
%%% TeX-master: t
%%% End: 


\begin{document}

\begin{titlepage}
\renewcommand{\thepage}{title}
\vspace*{1in}
\noindent
\rule[1pt]{\textwidth}{2pt}
\begin{center}
\newfont{\pvstitle}{cmss17 scaled \magstep4}
\textbf{\pvstitle PVS Language Reference}
\end{center}
\begin{flushright}
{\Large Version 7.1 {\smaller$\bullet$} August 2020}
\end{flushright}
\rule[1in]{\textwidth}{2pt}
\vspace*{2in}
\begin{flushleft}
S.~Owre\\
N.~Shankar\\
J.~M.~Rushby\\
D.~W.~J.~Stringer-Calvert\\
{\smaller\url{{Owre,Shankar,Rushby,Dave_SC}@csl.sri.com}}\\
{\smaller\url{http://pvs.csl.sri.com/}}
\end{flushleft}
\vspace*{1in}
\vbox{\hbox to \textwidth{{\Large SRI International\hfill}}%
\hbox to \textwidth{{\small\sf%
Computer Science Laboratory $\bullet$ 333 Ravenswood Avenue $\bullet$ Menlo Park CA 94025\hfil}}}
\end{titlepage}

\renewcommand{\chaptermark}[1]{\markboth{{\em #1}}{}\markright{{\em #1}}}
\renewcommand{\sectionmark}[1]{\markright{\thesection \em \ #1}}
%\lhead[\thepage]{\rightmark}
%\cfoot{\protect\small\bf \fbox{PVS 2.3 DRAFT}}
%\cfoot{}
%\rhead[\leftmark]{\thepage}
\thispagestyle{empty}

\newpage
\renewcommand{\thepage}{ack}

\noindent\textbf{NOTE:} This manual is in the process of being updated.
Almost everything stated here is still correct, but incomplete due to the
many new features that have been introduced into PVS over the years.  The
release notes should be consulted for the most current information.

\vspace*{6in}\noindent
The initial development of PVS was funded by SRI International.
Subsequent enhancements were partially funded by SRI and by NASA
Contracts NAS1-18969 and NAS1-20334, NRL Contract N00014-96-C-2106,
NSF Grants CCR-9300044, CCR-9509931, and CCR-9712383, AFOSR contract
F49620-95-C0044, and DARPA Orders E276, A721, D431, D855, and E301.
\newpage
\pagenumbering{roman}
\setcounter{page}{1}

\tableofcontents
%\listoffigures

%\chapter{The PVS Specification Language}

%% Master File: language.tex
\addcontentsline{toc}{chapter}{\protect\numberline{}Preface}
\vspace{4in}
{\Huge\bf Preface}\linebreak
\vspace{.75in}

%\chapter{Preface}

This report presents a description of the \pvs\ specification language,
as implemented in Version 1.0 beta of the \pvs\ specification and
verification environment.  It is intended to provide a reference of all
of the features of the language, including the complete grammar, some
examples, and an informal semantics. This report is one of several
needed to effectively use \pvs.  Companion documents are devoted to the
use of the system~\cite{PVS:userguide}, the user of the
prover~\cite{PVS:prover}, a tutorial introduction~\cite{PVS:tutorial},
and a semantics~\cite{PVS:semantics}.

\memo{Give prerequisites to using \pvs.}

The \pvs\ system is the culmination of the effort of a large number of
people over many years, drawing heavily from the research and experience
gained from E{\sc
hdm}~\cite{EHDM:Userguide,EHDM:Language,EHDM:semantics,EHDM:supplement,EHDM:tutorial}.
The primary contributers to E{\sc hdm} in rough chronological order
were Michael Melliar-Smith, Richard Schwartz, Rob Shostak, Judith Crow,
Friedrich von Henke, Stan Jefferson, Rosanna Lee, John Rushby, Mark
Stickel, Natarajan Shankar, Sam Owre, David Cyrluk, Steven Phillips,
and Carl Witty.
%In addition to those named above, valuable contributions were made by
%Dorothy Denning, Brian Fromme, Allen van Gelder, Dwight Hare, Peter
%Ladkin, Sheralyn Listgarten, Jeff Miner, Paul Oppenheimer, Jeff
%Reninger, and Lorna Shinkle.
\pvs\ is primarily the work of John Rushby, Natarajan Shankar, Sam Owre,
Friedrich von Henke, David Cyrluk, and Carl Witty.

The present version of the \pvs\ Language Description was assembled by
Sam Owre, Natarajan Shankar, and John Rushby.



\cleardoublepage
\pagenumbering{arabic}
\setcounter{page}{1}

\setcounter{topnumber}{9}
\renewcommand{\topfraction}{.99}
\setcounter{bottomnumber}{9}
\renewcommand{\bottomfraction}{.99}
\setcounter{totalnumber}{10}
\renewcommand{\textfraction}{.01}
\renewcommand{\floatpagefraction}{.01}

% Document Type: LaTeX
% Master File: intro.tex

\chapter{Introduction}

PVS is a \emph{P}rototype \emph{V}erification \emph{S}ystem for the
development and analysis of formal specifications.  The PVS system
primarily consists of a specification language, a parser, a typechecker, a
prover, specification libraries, and various browsing tools.  This
document describes the specification language and is meant to be used as a
reference manual.  The \emph{PVS System Guide}~\cite{PVS:userguide} is to
be consulted for information on how to use the system to develop
specifications and proofs.  The \emph{PVS Prover Guide}~\cite{PVS:prover}
is a reference manual for the commands used to construct proofs.  The web
site \url{http://pvs.csl.sri.com} provides many useful links, including
various tutorials and examples.

In this section, we provide a brief summary of the PVS specification
language, enumerate the key design principles behind the language, and
discuss a simple \texttt{stacks} example.

\section{Summary of the PVS Language}

A PVS specification consists of a collection of \emph{theories}.
Each theory consists of a \emph{signature} for the type names and
constants introduced in the theory, and the axioms, definitions, and
theorems associated with the signature.  For example, a typical
specification for a queue would introduce the \texttt{queue} type and the
operations of \texttt{enq}, \texttt{deq}, and \texttt{front} with their
associated types.  In such a theory, one can either define a
representation for the \texttt{queue} type and its associated operations in
terms of some more primitive types and operations, or merely axiomatize
their properties.  A theory can build on other theories: for example, a
theory for ordered binary trees could build on the theory for
binary trees.  A theory can be \emph{parametric} in certain specified
types and values: as examples, a theory of queues can be parametric in
the maximum queue length, and a theory of ordered binary trees can be
parametric in the element type as well as the ordering relation.  It is
possible to place constraints, called \emph{assumptions}, on the
parameters of a theory so that, for instance, the ordering relation
parameter of an ordered binary tree can be constrained to be a total
ordering.

The PVS specification language is based on simply typed higher-order
logic.  Within a theory, \emph{types} can be defined starting from
\emph{base} types (Booleans, numbers, etc.) using type constructors such
as function, record, and tuple types.  The \emph{terms} of the language
can be constructed using, for example, function application, lambda
abstraction, and record or tuple constructions.

There are a few significant enhancements to the simply typed language
above that lend considerable power and sophistication to PVS.  New
uninterpreted base types may be introduced.  One can define a
\emph{predicate subtype} of a given type as the subset of individuals in a
type satisfying a given predicate: the subtype of nonzero reals is
written as \texttt{$\{$x:real | x /= 0$\}$}.  One benefit of such
subtyping is that when an operation is not defined on all the elements of
a type, the signature can directly reflect this.  For example, the
division operation on reals is given a type where the denominator is
constrained to be nonzero.  Typechecking then ensures that
division is never applied to a zero denominator.  Since the predicate used
in defining a predicate subtype is arbitrary, typechecking is undecidable
and may lead to proof obligations called \emph{type correctness
conditions} (TCCs).  The user is expected to discharge these proof
obligations with the assistance of the PVS prover.  The PVS type system
also features dependent function, record, and tuple type constructions.
There is also a facility for defining a certain class of abstract datatype
(namely well-founded trees) theories automatically.

\section{PVS Language Design Principles}

There are several basic principles that have motivated the design of
PVS which are explicated in this section.

\paragraph{Specification vs. Programming Languages.}
A specification represents requirements or a design whereas a program
text represents an implementation of a design.  A program can be seen as
a specification, but a specification need not be a program.  Typically,
a specification expresses \emph{what} is being
computed whereas a program expresses \emph{how} it is computed.  A
specification can be incomplete and still be meaningful whereas an
incomplete program will typically not be executable.  A specification
need not be executable; it may use high-level constructs, quantifiers
and the like, that need have no computational meaning.  However, there
are a number of aspects of programming languages that a specification
language should include, such as:
\begin{itemize}

\item the usual basic types: booleans, integers, and rational numbers

\item the familiar datatypes of programming languages such as arrays,
records, lists, sequences, and abstract datatypes

\item the higher-order capabilities provided by modern functional
programming languages so that extremely general-purpose operations can
be defined

\item definition by recursion

\item support for dividing large specifications into parameterized
modules

\end{itemize}

It is clearly not enough to say that a specification language shares some
important features of a programming language but need not be executable.
Any useful formal language must have a clearly defined
semantics\footnote{The PVS semantics are presented in a technical
report~\cite{PVS:semantics}.} and must be capable of being manipulated in
ways that are meaningful relative to the semantics.  A programming
language for example can be given a denotational semantics so that the
execution of the program respects its denotational meaning.  The reason
one writes a specification in a formal language is typically to ensure
that it is sensible, to derive some useful consequences from it, and to
demonstrate that one specification implements another.  All of these
activities require the notion of a justification or a proof based on the
specification, a notion that can only be captured meaningfully within the
framework of logic.

\paragraph{Untyped set theory versus higher-order logic}
\index{set theory}\index{higher-order logic}
Which logic should be chosen?  There is a wide variety of choices:
simple propositional logics, which can be classical or intuitionistic,
equational logics, quantificational logics, modal and temporal logics,
set theory, higher-order logic, etc.  Some propositional and modal
logics are appropriate for dealing with finite state machines where one
is primarily interested in efficiently deciding certain finite state
machine properties.  For a general purpose specification language,
however, only a set theory or a higher-order logic would provide the
needed expressiveness.  Higher-order logic requires strict typing to
avoid inconsistencies whereas set theory restricts the rules for forming
sets.  Set theory is inherently untyped, and grafting a typechecker onto
a language based on set theory is likely to be too strict and arbitrary.
Typechecking, however, is an extremely important and easy way of
checking whether a specification makes semantic sense (although 
for an opposing view, the reader is referred to a report by Lamport
and Paulson~\cite{Lamport&Paulson97}).  Higher-order
logic does admit effective typechecking but at the expense of an
inflexible type system.  Recent advances in type theory have made it
possible to design more flexible type systems for higher-order logic
without losing the benefits of typechecking.  We have therefore chosen
to base PVS on higher-order logic.

\paragraph{Total versus partial functions}
\index{function!total}\index{function!partial} In the PVS higher-order
logic, an individual is either a function, a tuple, a record, or the
member of a base type.  Functions are extremely important in higher-order
logic.  They are \emph{first-class} individuals, i.e., variables can range
over functions.  In general, functions can represent either \emph{total}
or \emph{partial} maps.  A total map from domain $A$ to range $B$ maps
each element of $A$ to some element of $B$, whereas a partial map only
maps some of the elements of $A$ to elements of $B$.  Most traditional
logics build in the assumption that functions represent total maps.
Partial functions arise quite naturally in specifications.  For example,
the division operation is undefined on a zero denominator and the
operation of popping a stack is undefined on an empty stack.

Some recent logics, notably those of VDM~\cite{Jones:VDM},
LUTINS~\cite{Farmer:functions}, RAISE~\cite{RAISE-tutorial},
Beeson~\cite{Beeson:book} and Scott~\cite{Scott79}, admit partial
functions.  In these logics, some terms may be \emph{undefined} by not
denoting any individuals.  Some of these logics have mechanisms for
distinguishing defined and undefined terms, while others allow
``undefined'' to propagate from terms to expressions and therefore must
employ multiple truth values.  In all these cases, the ability to
formalize partially defined functions comes at the cost of complicating
the deductive apparatus, even when the specification does not involve any
partial functions.  Though logics that allow partial functions are
extremely interesting, we have chosen to avoid partial functions in PVS.
We have instead employed the notion of a \emph{predicate
subtype}\index{predicate subtype}, a type that consists of those elements
of a given type satisfying a given predicate.  Using predicate subtypes,
the type of the division operator, for example, can be constrained to
admit only nonzero denominators.  Division then becomes a total operation
on the domain consisting of arbitrary numerators and nonzero denominators.
The domain of a \emph{pop} operation on stacks can be similarly restricted
to nonempty stacks.  PVS thus admits partial functions within the
framework of a logic of total functions by enriching the type system to
include predicate subtypes.  We find this use of predicate subtypes to be
significantly in tune with conventional mathematical practice of being
explicit about the domain over which a function is defined.

\section{An Example: \texttt{stacks}}\label{stacks-example}
\index{stacks example@\texttt{stacks} example}

In this section we discuss a specific example, the theory of
\texttt{stacks}, in order to give a feel for the various aspects of the
PVS language before going into detail.  Apart from the basic notation for
defining a theory, this example illustrates the use of type parameters at
the theory level, the general format of declarations, the use of predicate
subtyping to define the type of nonempty stacks, and the generation of
typechecking obligations.

\pvstheory{stacks-alltt}{{Theory \texttt{stacks}}}{stacks-alltt}

Figure~\ref{stacks-alltt} illustrates a theory for stacks of an arbitrary
type with corresponding stack operations.  Note that this is not the
recommended approach to specifying stacks; a more convenient and complete
specification is provided in Section~\ref{stacks-adt},
page~\pageref{stacks-adt}.

The first line introduces a theory named \texttt{stacks} that is
parameterized by a type \texttt{t} (the \emph{formal parameter} of
\texttt{stacks}).  The keyword \texttt{TYPE+} indicates that \texttt{t} is
a \emph{non-empty} type.  The uninterpreted (nonempty) type \texttt{stack}
is declared, and the constant \texttt{empty} and variable \texttt{s} are
declared to be of type \texttt{stack}.  The defined predicate
\texttt{nonemptystack?}~is then declared on elements of type
\texttt{stack}; it is \texttt{true} for a given \texttt{stack} element
iff\footnote{If and only if.} that element is not equal to
\texttt{empty}.\footnote{The \texttt{bool} type and \texttt{/=} operator
are declared in the \emph{prelude}, which is a large body of theories that
are preloaded into PVS.  This is described in Appendix~/ref{prelude}.}
The functions \texttt{push}, \texttt{pop}, and \texttt{top} are then
declared.  Note that the predicate \texttt{nonemptystack?}~is being used
as a type in specifying the signatures of these functions; any predicate
may be used where a type is expected simply by putting parentheses around
it.

The variables \texttt{x} and \texttt{y} are then declared, followed by the
usual axioms for \texttt{push}, \texttt{pop}, and \texttt{top}, which make
\texttt{push} a stack constructor and \texttt{pop} and \texttt{top} stack
accessors.  Finally, there is the theorem \texttt{pop2push2}, that can
easily be proved by two applications of the \texttt{pop\_push} axiom.

This simple theorem has an additional facet that shows up during
typechecking.  Note that \texttt{pop} expects an element of type
\texttt{(nonemptystack?)} and returns a value of type \texttt{stack}.
This works fine for the inner \texttt{pop} because it is applied to
\texttt{push}, which returns an element of type \texttt{(nonemptystack?)};
but the outer occurrence of \texttt{pop} cannot be seen to be type correct
by such syntactic means.  In cases like these, where a subtype is expected
but not directly provided, the system generates a \emph{type-correctness
condition} (TCC).  In this case, the TCC is
\begin{pvsex}
  pop2push2_TCC1: OBLIGATION
    FORALL (s: stack, x, y: t): nonemptystack?(pop(push(x, push(y, s))));
\end{pvsex}
and is easily proved using the \texttt{pop\_push} axiom.  The system keeps
track of all such obligations and will flag the unproved ones during proof
chain analysis.

Parameterized theories such as \texttt{stacks} introduce theory schemas,
where the type \texttt{t} may be instantiated with any other nonempty
type.  To use the types, constants, and formulas of the \texttt{stacks}
theory from another theory, the \texttt{stacks} theory must be imported,
with \emph{actual parameters} provided for the corresponding theory
parameters.  This is done by means of an \texttt{IMPORTING} clause. For
example, consider the theory \texttt{ustacks}.
\begin{session}
  ustacks : THEORY
   BEGIN
    IMPORTING stacks[int], stacks[stack[int]]

    si : stack[int]
    sos : stack[stack[int]] = push(si, empty)
   END ustacks
\end{session}
It imports stacks of integers and stacks of stacks of integers.  The constant
\texttt{si} is then declared to be a stack of integers, and the constant
\texttt{sos} is a stack of stacks of integers whose top element is
\texttt{si}.  Note that the system is able to determine which instance of
\texttt{push} and \texttt{empty} is meant from the type of the first
argument.  In general, the typechecker infers the type of an expression
from its context.  

The following chapters provide more details on the various features of the
language.  The lexical aspects of the language are explained in
Chapter~\ref{lexical}.  Chapter~\ref{declarations} describes declarations,
Chapters~\ref{types} and~\ref{expressions} describe type expressions and
expressions, and Chapter~\ref{theories} explains theories, theory
parameters, and the importing and exporting of names.  Theory
interpretaions and mappings are described in
Chapter~\ref{interpretations}.  Chapter~\ref{names} describes names and
name resolution, and Chapter~\ref{adts} details the datatype facility of
PVS.  Finally, Appendix~\ref{grammar} provides the grammar of the
language and Appendix~\ref{prelude} gives a brief overview of the theories
of the PVS prelude.

% Document Type: LaTeX
% Master File: language.tex
\chapter{The Lexical Structure}\label{lexical}

PVS specifications are text files, each composed of a sequence of lexical
elements which in turn are made up of characters.  The lexical elements of
PVS are the identifiers, reserved words, special symbols, numbers,
whitespace characters, and comments.

Identifiers\index{identifiers} are composed of letters, digits, and the
characters \texttt{\_} or \texttt{?}; they must begin with a letter.  They
may be arbitrarily long, constrained only by the limits imposed by the
underlying Common Lisp system.  Identifiers are case-sensitive;
\texttt{FOO}, \texttt{Foo}, and \texttt{foo} are different identifiers.
PVS strings contain any ASCII character: to include a \texttt{"} in the
string, use \texttt{\char'134 "} and to include a \texttt{\char'134} use
\texttt{\char'134\char'134}.

\pvsbnf{bnf-lexical}{Lexical Syntax}

The reserved words\index{reserved words} are shown in
Figure~\ref{reserved-words}.  Unlike identifiers, they are not
case-sensitive.  In this document, reserved words are always displayed in
upper case.  Note that identifiers may have reserved words embedded in
them, thus \texttt{ARRAYALL} is a valid identifier and will not be
confused with the two embedded reserved words.  The meaning of the
reserved words are given in the appropriate sections; they are collected
here for reference.

\begin{figure}[tb]
{\smaller\tt
\begin{tabular}{|*{5}{p{1.03in}}|}\hline
\input{keywords}
\hline
\end{tabular}}
\caption{\pvs\ Reserved Words}\label{reserved-words}
\end{figure}

The special symbols\index{special symbols} are listed in
Figure~\ref{special-symbols}.  All of these symbols are separators; they
separate identifiers, numbers, and reserved words.

\begin{figure}[tb]
\begin{center}
{\small\tt
\begin{tabular}{|*{6}{@{\hspace*{.2in}}c@{\extracolsep{.5in}}}@{\hspace*{.25in}}|}\hline
\input{operator-table}
\hline
\end{tabular}}
\end{center}
\caption{\pvs\ Special Symbols}\label{special-symbols}
\end{figure}

The whitespace characters are space, tab, newline, return, and newpage;
they are used to separate other lexical elements.  At least one whitespace
character must separate adjacent identifiers, numbers, and reserved words.

Comments\index{comments} may appear anywhere that a whitespace character
is allowed.  They consist of the `\texttt{\%}'\index{\%@\texttt{\%}} character
followed by any sequence of characters and terminated by a newline.

The \emph{definable} symbols are shown in table~\ref{definable-symbols}.
These keywords and symbols may be given declarations.  Some of them have
declarations given in the prelude.\footnote{In particular,
\texttt{\char38}, \texttt{*}, \texttt{+}, \texttt{-}, \texttt{/},
\texttt{/=}, \texttt{<}, \texttt{<<}, \texttt{<=}, \texttt{<=>},
\texttt{=}, \texttt{=>}, \texttt{>}, \texttt{>=}, \texttt{AND},
\texttt{IFF}, \texttt{IMPLIES}, \texttt{NOT}, \texttt{O}, \texttt{OR},
\texttt{WHEN}, \texttt{XOR}, \texttt{\char94}, and \texttt{\char126} are
declared there.  Note that many of these are overloaded, for example,
\texttt{\char94} has three different definitions.}  Any of these may be
(re)declared any number of times, though this may lead to ambiguities.
Such ambiguities may be resolved by including the theory name, actual
parameters,  and possibly the type as a coercion.

Symbols that are binary infix (\hyperlink{Binop}{\emph{Binop}}), for
example \texttt{AND} and \texttt{+}, may be declared with any number of
arguments.  If they are declared with two arguments then they may
subsequently be used in prefix or infix form.  Otherwise they may only be
used in prefix form.  Similarly for unary operators, and the \texttt{IF}
operator, which may be used in \texttt{IF-THEN-ELSE-ENDIF} form if
declared with three arguments.

Note that when typing the operators \texttt{/\\} or \texttt{\\/} outside
of a specification, the backslash may need to be doubled (or in rare
cases, quadrupled).  This is because it is commonly used as an ``escape''
character, and the character following may be interpreted specially.

The symbol pairs \lit{[|} and \lit{|]}, \lit{(|} and \lit{|)}, and
\lit{$\{$|} and \lit{|$\}$} are available as outfix operators.  They are
declared using \lit{[||]}, \lit{(||)}, and \lit{$\{$||$\}$}, respectively.
For example, with the declaration \texttt{[||]:\ [bool, int -> int]} the
outfix term \texttt{[| TRUE, 0 |]} is equivalent to the prefix form
\texttt{[||](TRUE, 0)}.

\begin{figure}[tb]
\begin{center}
{\small\tt
\begin{tabular}{|*{6}{@{\hspace*{.2in}}c@{\extracolsep{.4in minus .4in}}}@{\hspace*{.2in}}|}\hline
\input{opsym-table}
\hline
\end{tabular}}
\end{center}
\caption{\pvs\ Definable Symbols}\label{definable-symbols}
\end{figure}

% Document Type: LaTeX
% Master File: language.tex

\chapter{Declarations}\label{declarations}
\index{declaration|(pidx}

Entities of PVS are introduced by means of \emph{declarations}, which are
the main constituents of PVS specifications.  Declarations are used to
introduce types, variables, constants, formulas, judgements, conversions,
and other entities.  Most declarations have an \emph{identifier} and
belong to a unique theory.  Declarations also have a body which indicates
the \emph{kind} of the declaration and may provide a signature or
definition for the entity.  \emph{Top-level}
declarations\index{declaration!top-level} occur in the formal parameters,
the assertion section and the body of a theory.  \emph{Local}
declarations\index{declaration!local} for variables may be given, in
association with constant and recursive declarations and \emph{binding
expressions} (\eg\ involving \texttt{FORALL} or \texttt{LAMBDA}).
Declarations are ordered within a theory; earlier declarations may not
reference later ones.\footnote{Thus mutual recursion is not directly
supported.  The effect can be achieved with a single recursive function
that has an argument that serves as a switch for selecting between two or
more subexpressions.}

\index{exporting|(}\index{importing|(}
Declarations introduced in one theory may be referenced in another by
means of the \texttt{IMPORTING} and \texttt{EXPORTING} clauses.  The
\texttt{EXPORTING} clause of a theory indicates those entities that may be
referenced from outside the theory.  There is only one such clause for a
given theory.  The \texttt{IMPORTING} clauses provide access to the
entities exported by another theory.  There can be many \texttt{IMPORTING}
clauses in a theory; in general they may appear anywhere a top-level
declaration is allowed.  See Section~\ref{importings} for more details.
\index{importing|)}\index{exporting|)}

PVS allows the overloading\index{overloading} of declaration identifiers.
Thus a theory named \texttt{foo} may declare a constant \texttt{foo} and a
formula \texttt{foo}.  To support this \emph{ad hoc} overloading,
declarations are classified according to kind\index{declaration!kind}; in
PVS the primary kinds are \emph{type}\index{declaration!kind!type},
\emph{prop}\index{declaration!kind!prop},
\emph{expr}\index{declaration!kind!expr}, and
\emph{theory}\index{declaration!kind!theory}.  Type declarations are of
kind \emph{type}, and may be referenced in type declarations, actual
parameters, signatures, and expressions.  Formula declarations are of kind
\emph{prop}, and may be referenced in auto-rewrite declarations
(Section~\ref{auto-rewrite-decls}) or proofs (see the PVS Prover
Guide~\cite{PVS:prover}).  Variable, constant, and recursive definition
declarations are of kind \emph{expr}; these may be referenced in
expressions and actual parameters.  Newly introduced names need only be
unique within a kind, as there is no way, for example, to use an
expression where a type is expected.\footnote{There are a few exceptions,
for example the actual parameters of theories, since theories may be
instantiated with types or expressions.}

\pvsbnf{bnf-decls}{Declarations Syntax}
\index{syntax!declarations}

Declarations generally consist of an
\emph{identifier}\index{declaration!identifier}, an optional list of
\emph{bindings}\index{declaration!binding}, and a
\emph{body}\index{declaration!body}.  The body determines the kind of the
declaration, and the bindings and the body together determine the
signature and definition of the declared entity.  Multiple
declarations\index{declaration!multiple} may be given in compressed form
in which a common body is specified for multiple identifiers; for example
%
\begin{pvsex}
  x, y, z: VAR int
\end{pvsex}
In every case this is treated the same as the expanded form, thus the
above is equivalent to:
\begin{pvsex}
  x: VAR int
  y: VAR int
  z: VAR int
\end{pvsex}

In the rest of this chapter we describe declarations for types, variables,
constants, recursive definitions, macros, inductive and coinductive
definitions, formulas, judgements, conversions, libraries, and
auto-rewrites.  The declarations for theory parameters, importings,
exportings, and theory abbreviations are given in Chapter~\ref{theories}.
Figure~\ref{bnf-decls} gives the syntax for declarations.

\section{Type Declarations}\label{type-declarations}
\index{type declarations|(}

Type declarations are used to introduce new type names to the context.
There are four kinds of type declaration:

\begin{itemize}

\item \emph{uninterpreted type declaration}: \texttt{T:\ TYPE}
\index{uninterpreted type}\index{type!uninterpreted}

\item \emph{uninterpreted subtype declaration}: \texttt{S:\ TYPE FROM T}
\index{uninterpreted subtype}\index{type!uninterpreted subtype}

\item \emph{interpreted type declaration}: \texttt{T:\ TYPE =
int}\index{interpreted type}\index{type!interpreted}

\item \emph{enumeration type declarations}: \texttt{T:\ TYPE = \setb r,
g, b\sete} \index{enumeration types}\index{type!enumeration}

\end{itemize}

These type declarations introduce \emph{type names}\index{type!name}
that may be referenced in type expressions (see Section~\ref{types}).
They are introduced using one of the keywords
\keyword{TYPE}\index{type@\texttt{TYPE}},
\keyword{NONEMPTY\_TYPE}\index{type@\texttt{NONEMPTY\_TYPE}}, or
\keyword{TYPE+}\index{type+@\texttt{TYPE+}}.

\subsection{Uninterpreted Type Declarations}
\index{type!uninterpreted|(}

Uninterpreted types support abstraction by providing a means of
introducing a type with a minimum of assumptions on the type.  An
uninterpreted type imposes almost no constraints on an implementation of
the specification.  The only assumption made on an uninterpreted type
\texttt{T} is that it is disjoint from all other types, except for
subtypes of \texttt{T}.  For example,
\begin{pvsex}
  T1, T2, T3: TYPE
\end{pvsex}
%
introduces three new pairwise disjoint types.  If desired, further
constraints may be put on these types by means of axioms or assumptions
(see Section~\ref{formula-declarations} on
page~\pageref{formula-declarations}).

It should be emphasized that uninterpreted types are important in
providing the right level of abstraction in a specification.  Specifying
the type body may have the undesired effect of restricting the possible
implementations, and cluttering the specification with needless detail.

\index{type!uninterpreted|)}\index{uninterpreted type|)}


\subsection{Uninterpreted Subtype Declarations}
\index{uninterpreted subtype|(}

Uninterpreted subtype declarations are of the form
\begin{pvsex}
  s: TYPE FROM t
\end{pvsex}
\index{FROM@\texttt{FROM}}
This introduces an uninterpreted
\emph{subtype}\index{subtypes}\index{type!subtype} \texttt{s} of
the \emph{supertype}\index{supertype}\index{type!supertype}
\texttt{t}.  This has the same meaning as
\begin{pvsex}
  s_pred: [t -> bool]
  s: TYPE = (s_pred)
\end{pvsex}
%
in which a new predicate is introduced in the first line and the type
\texttt{s} is declared as a \emph{predicate} subtype in the second
line\footnote{This is described in Section~\ref{subtypes}
(page~\pageref{subtypes}).}.  No assumptions are made about uninterpreted
subtypes; in particular, they may or may not be empty, and two different
uninterpreted subtypes of the same supertype may or may not be disjoint.
Of course, if the supertypes themselves are disjoint, then the
uninterpreted subtypes are as well.

\index{uninterpreted subtype|)}

\subsection{Interpreted Type Declarations}
\index{interpreted type declarations|(}\index{type!interpreted|(}

Interpreted type declarations are primarily a means for providing names
for type expressions.  For example,
\begin{pvsex}
  intfun: TYPE = [int -> int]
\end{pvsex}
%
introduces the type name \texttt{intfun} as an abbreviation for the type
of functions with integer domain and range.  Because PVS uses
\emph{structural equivalence}\index{structural equivalence} instead of
\emph{name equivalence}\index{name equivalence}, any type expression
\texttt{T} involving \texttt{intfun} is equivalent to the type expression
obtained by substituting \texttt{[int -> int]} for \texttt{intfun} in
\texttt{T}.  The available type expressions are described in
Chapter~\ref{types} on page~\pageref{types}.

Interpreted type declarations may be given
parameters.\index{parameterized type names} For example, the type of
integer subranges may be given as
\begin{pvsex}
  subrange(m, n: int): TYPE = \setb{}i:int | m <= i AND i <= n\sete
\end{pvsex}
and \texttt{subrange} with two integer parameters may subsequently be used
wherever a type is expected.  Any use of a parameterized type must include
all of the parameters, so currying of the parameters is not allowed.  Note
that \texttt{subrange} may be overloaded to declare a different type in
the same theory without any ambiguity, as long as the number or type of
parameters is different.

\index{type!interpreted|)}\index{interpreted type declarations|)}


\subsection{Enumeration Type Declarations}\label{enum-types}
\index{enumeration types|(}\index{type!enumeration|(}

Enumeration type declarations are of the form
\begin{pvsex}
  enum: TYPE = \setb{}e_1,\ldots, e_n\sete
\end{pvsex}
%
where the \texttt{e\_i} are distinct identifiers which are taken to
completely enumerate the type.  This is actually a shorthand for the
datatype specification
\begin{pvsex}
  enum: DATATYPE
    e_1: e_1?
         \vdots
    e_n: e_n?
  END enum
\end{pvsex}
%
explained in Chapter~\ref{adts}.  Because of this, enumeration types may
only be given as top-level declarations, and are \emph{not} type
expressions.  The advantage of treating them as datatypes is that the
necessary axioms are automatically generated, and the prover has built-in
facilities for handling datatypes.

\index{type!enumeration|)}\index{enumeration types|)}

\index{type declarations|)}


\subsection{Empty versus Nonempty Types}
\label{emptytypes}
\index{nonempty type}
\index{empty type}
\index{type!nonempty|(}\index{type!empty|(}

As noted before, PVS allows empty types, and the term \emph{type} refers
to either empty or nonempty types.  Constants declared to be of a given
type provide elements of the type, so the type must be nonempty or there
is an inconsistency.  Thus whenever a constant is declared, the system
checks whether the type is nonempty, and if it cannot decide that it is
nonempty it generates an \emph{existence TCC}.\index{existence
TCC}\index{TCC!existence} This is the simple explanation, but it is made
somewhat complicated by the considerations of formal parameters,
uninterpreted versus interpreted type declarations, explicit declarations
of nonemptiness, and
\keyword{CONTAINING}\index{CONTAINING@\texttt{CONTAINING}} clauses on type
declarationss, as well as a desire to keep the number of TCCs generated to
a minimum, while guaranteeing soundness.  The details are provided below.

First note that having variables range over an empty type causes no
difficulties,\footnote{If the type \texttt{T} is empty, then the following
two equivalences hold:
\begin{pvsex}
  (FORALL (x: T): p(x)) IFF TRUE \quad \mbox{\textrm{and}} \quad (EXISTS (x: T): p(x)) IFF FALSE
\end{pvsex}}
so variable declarations and variable bindings never trigger the
nonemptiness check.

During typechecking, type declarations may indicate that the type is
nonempty, and constant declarations of a given type require that the type
be nonempty.  When a type is determined to be nonempty, it is marked as
such so that future checks of constants do not trigger more TCCs.  Below
we describe how type declarations are handled first for declarations in the
body of a theory, and then for type declarations that appear in the formal
parameters, as they require special handling.

\paragraph{Theory Body Type Declarations}

\begin{itemize}

\item Uninterpreted type or subtype declarations introduced with the
keyword \keyword{TYPE} may be empty.  Declaring a constant of that type
will lead to a TCC that is unprovable without further axioms.

\item Uninterpreted type declarations introduced with the keyword
\keyword{NONEMPTY\_TYPE}\index{nonempty_type@\keyword{NONEMPTY\_TYPE}}
or \keyword{TYPE+}\index{type+@\texttt{TYPE+}} are assumed to be nonempty.
Thus the type is marked nonempty.

\item Uninterpreted subtype declarations introduced with the keyword
\keyword{NONEMPTY\_TYPE} or \keyword{TYPE+} are assumed to be nonempty, as long as the
supertype is nonempty.  Thus the supertype is checked, and an existence
TCC is generated if the supertype is not known to be nonempty.  Then the
subtype is marked nonempty.

\item The type of an interpreted constant is nonempty, as the definition
provides a witness.

\item Interpreted type declarations introduced with the keyword
\keyword{TYPE} may be nonempty, depending on the type definition.

\item Any interpreted type declaration with a \keyword{CONTAINING} clause
is marked nonempty, and the \keyword{CONTAINING} expression is typechecked
against the specified type.  In this case no existence TCC is generated,
since the \keyword{CONTAINING} expression is a witness to the type.  Of
course, other TCCs may be generated as a result of typechecking the
\keyword{CONTAINING} expression.

\end{itemize}

\paragraph{Formal Type Declarations}

Only uninterpreted (sub)type declarations may appear in the formal
parameters list.

\begin{itemize}

\item Formal type declarations introduced with the \texttt{TYPE} keyword may
be empty.  This is handled according to the occurrences of constant
declarations involving the type.

\item If there is a constant declaration of that type in the formal
parameter list, then no TCCs are generated, since
any instance of the theory will need to provide both the type and a
witness.  The type is marked nonempty in this case.

\item If the type declaration is a formal parameter and a constant is
declared whose type involves the type, but is not the type itself (for
example, if the formal theory parameters are \texttt{[t:\ TYPE, f:\ [t ->
t]]}), then a TCC may be generated, and a comment is added to the TCC
indicating that an assuming clause may be needed in order to discharge the
TCC.  This TCC will be generated only if an earlier constant declaration
hasn't already forced the type to be marked nonempty.  Note that there are
circumstances in which the formal type may be empty but the type
expression involving that type is nonempty.  This is discussed further
below.

\end{itemize}

\subsection{Checking Nonemptiness}\label{nonemptiness-check}
\index{type!nonempty}
The typechecker knows a type to be nonempty under the
following circumstances:
\begin{itemize}

\item The type was declared to be nonempty, using either the
\keyword{NONEMPTY\_TYPE}\index{nonempty_type@\keyword{NONEMPTY\_TYPE}} or
the synonymous \keyword{TYPE+}\index{type+@\texttt{TYPE+}} keyword.  If the
type is uninterpreted, this amounts to an assumption that the type is
nonempty.  If the type has a definition, then an existence TCC is
generated unless the defining type expression is known to be nonempty.

\item The type was declared to have an element using a
\keyword{CONTAINING}\index{CONTAINING@\texttt{CONTAINING}} expression.

\item A constant was declared for the type.  In this case an existence TCC
is generated for the first such constant, after which the type is marked
as nonempty.

\item It was marked as nonempty from an earlier check.

\end{itemize}

Once an unmarked type is determined to be nonempty, it is marked by the
typechecker so that later checks will not generate existence TCCs.  In
addition, the type components are marked as nonempty.  Thus the types that
make up a tuple type, the field types of a record type, and the supertype
of a subtype are all marked.

It is possible for two equivalent types to be marked differently, for
example:
\begin{pvsex}
  t1: TYPE = \setb{}x: int | x > 2\sete
  t2: TYPE = \setb{}x: int | x > 2\sete
  c1: t1
\end{pvsex}
only marks the first type (\texttt{t1}).  Hence, it is best to name your types and
to use those names uniformly.

\index{type!empty|)}
\index{type!nonempty|)}

\section{Variable Declarations}
\index{variables|(}\index{declaration!variables|(}

Variable declarations introduce new variables and associate a type with
them.  These are \emph{logical} variables, not program variables; they
have nothing to do with state---they simply provide a name and associated
type so that binding expressions and formulas can be succinct.
Variables may not be exported.  Variable
declarations also appear in binding expressions such as \texttt{FORALL} and
\texttt{LAMBDA}.  Such local declarations ``shadow'' any earlier
declarations.  For example, in
\begin{pvsex}
  x: VAR bool
  f: FORMULA (FORALL (x: int): (EXISTS (x: nat): p(x)) AND q(x))
\end{pvsex}
%
The occurrence of \texttt{x} as an argument to \texttt{p} is of type
\texttt{nat}, shadowing the one of type \texttt{int}.  Similarly, the
occurrence of \texttt{x} as an argument to \texttt{q} is of type
\texttt{int}, shadowing the one of type \texttt{bool}.

\index{variables|)}\index{declaration!variables|)}

\section{Constant Declarations}\label{constants}
\index{constants|(}\index{declaration!constants|(}

Constant declarations introduce new constants, specifying their type and
optionally providing a value.  Since PVS is a higher order logic, the term
\emph{constant} refers to functions and relations, as well as the usual
(0-ary) constants.  As with types, there are both \emph{uninterpreted} and
\emph{interpreted} \index{constants!interpreted}%
\index{constants!uninterpreted} constants.  Uninterpreted constants make
no assumptions, although they require that the type be nonempty (see
Section~\ref{nonemptiness-check}, page~\pageref{nonemptiness-check}).
Here are some examples of constant declarations:
\begin{pvsex}
  n: int
  c: int = 3
  f: [int -> int] = (lambda (x: int): x + 1)
  g(x: int): int = x + 1
\end{pvsex}
%
The declaration for \texttt{n} simply introduces a new integer constant.
Nothing is known about this constant other than its type, unless further
properties are provided by \texttt{AXIOM}s.  The other three constants are
interpreted.  Each is equivalent to specifying two declarations: \eg\
the third line is equivalent to
\begin{pvsex}
  f: [int -> int]
  f: AXIOM  f = (LAMBDA (x: int): x + 1)
\end{pvsex}
%
except that the definition is guaranteed to form a \emph{conservative
extension}\index{conservative extension} of the theory.  Thus the
theory remains consistent after the declaration is given if it was
consistent before.

The declarations for \texttt{f} and \texttt{g} above are two different ways to
declare the same function.  This extends to more complex arguments, for
example
\begin{pvsex}
  f: [int -> [int, nat -> [int -> int]]] =
     (LAMBDA (x: int): (LAMBDA (y: int), (z: nat): (LAMBDA (w: int):
       x * (y + w) - z)))
\end{pvsex}
%
is equivalent to
\begin{pvsex}
  f(x: int)(y: int, z: nat)(w: int): int = x * (y + w) - z
\end{pvsex}
%
This can be shortened even further if the variables are declared first:
\begin{pvsex}
  x, y, w: VAR int
  z: VAR nat
  f(x)(y,z)(w): int = x * (y + w) - z
\end{pvsex}
%
Finally, a mix of predeclared and locally declared variables is possible:
\begin{pvsex}
  x, y: VAR int
  f(x)(y,(z: nat))(w: int): int = x * (y + w) - z
\end{pvsex}
%
Note the parentheses around \texttt{z:\ nat}; without these, \texttt{y} would
also be treated as if it were declared to be of type \texttt{nat}.

A construct that is frequently encountered when subtypes are involved is
shown by this example
\begin{pvsex}
  f(x: \setb{}x: int | p(x)\sete): int = x + 1
\end{pvsex}
%
There are two useful abbreviations for this expression.  In the first, we
use the fact that the type \texttt{\setb{}x:\ int | p(x)\sete} is equivalent to
the type expression \texttt{(p)} when \texttt{p} has type \texttt{[int ->
bool]}, and we can write
\begin{pvsex}
  f(x: (p)): int = x + 1
\end{pvsex}
%
The second form of abbreviation basically removes the set braces and the
redundant references to the variable, though extra parentheses are
required:
\begin{pvsex}
  f((x: int | p(x))): int = x + 1
\end{pvsex}
%
Which of these forms to use is mostly a matter of taste; in general,
choose the form that is clearest to read for a given declaration.

Note that functions with range type \texttt{bool} are generally referred
to as \emph{predicates}, and can also be regarded as relations or sets.
For example, the set of positive odd numbers can be characterized by a
predicate as follows:
\begin{pvsex}
  odd: [nat -> bool] = (LAMBDA (n: nat): EXISTS (m: nat): n = 2 * m + 1)
\end{pvsex}
%
PVS allows an alternate syntax for predicates that encourages a
set-theoretic interpretation:
\begin{pvsex}
  odd: [nat -> bool] = \setb{}n: nat | EXISTS (m: nat): n = 2 * m + 1\sete
\end{pvsex}

\index{constants|)}

\section{Recursive Definitions}\label{recursive-definitions}
\index{recursive definitions|(}

Recursive definitions are treated as constant declarations, except that
the defining expression is required, and a \emph{measure}\index{measure
function} must be provided, along with an optional well-founded order
relation.\index{well-founded order releation} The same syntax for
arguments is available as for constant declarations; see the preceding
section.

PVS allows a restricted form of recursive definition; mutual
recursion\index{recursion!mutual}\index{mutual recursion} is not allowed,
and the function must be \emph{total},\index{total function} so that the
function is defined for every value of its domain.  In order to ensure
this, recursive functions must be specified with a
\emph{measure}\index{measure}, which is a function whose signature matches
that of the recursive function, but with range type the domain of the
order relation, which defaults to \texttt{<} on \texttt{nat} or
\texttt{ordinal}\index{ordinal}\index{type!ordinal}.  If the order
relation is provided, then it must be a binary relation on the range type
of the measure, and it must be well-founded; a \emph{well-founded} \tcc\
\index{well-founded TCC}\index{TCC!well-founded} is generated if the order
is not declared to be well-founded.

Here is the classic example of the
\texttt{factorial}\index{factorial@\texttt{factorial}} function:
%
\begin{pvsex}
  factorial(x: nat): RECURSIVE nat =
    IF x = 0 THEN 1 ELSE x * factorial(x - 1) ENDIF
    MEASURE (LAMBDA (x: nat): x)
\end{pvsex}
%
The measure is the expression following the \texttt{MEASURE} keyword (the
optional order relation follows a \texttt{BY} keyword after the
measure).  This definition generates a \emph{termination
TCC};\index{TCC!termination}\index{termination TCC} a proof obligation
which must be discharged in order that the function be well-defined.  In
this case the obligation is
%
\begin{pvsex}
  factorial_TCC2: OBLIGATION
    FORALL (x: nat): NOT x = 0 IMPLIES x - 1 < x
\end{pvsex}

It is possible to abbreviate the given \texttt{MEASURE} function by
leaving out the \texttt{LAMBDA} binding.  For example, the measure
function of the factorial definition may be abbreviated to:
\begin{pvsex}
  MEASURE x
\end{pvsex}
The typechecker will automatically insert a lambda binding corresponding
to the arguments to the recursive function if the measure is not already
of the correct type, and will generate a typecheck error if this process
cannot determine an appropriate function from what has been specified.

A termination \tcc\ is generated for each recursive occurrence of the
defined entity within the body of the definition.\footnote{Some of these
may be subsumed by earlier TCCs, and hence will not be displayed with the
\texttt{M-x show-tccs} command.}  It is obtained in one of two ways.  If a
given recursive reference has at least as many arguments provided as
needed by the measure, then the \tcc\ is generated by applying the measure
to the arguments of the recursive call and comparing that to the measure
applied to the original arguments using the order relation.  The
\texttt{factorial} \tcc\ is of this form.  The context of the occurrence
is included in the \tcc; in this case the occurrence is within the
\texttt{ELSE} part of an \texttt{IF-THEN-ELSE} so the negated condition is
an antecedent to the proof obligation.

If the reference does not have enough arguments available, then the
reference is actually given a \emph{recursive signature}\index{recursive
signature} derived from the recursive function as described below.  This
type constrains the domain to satisfy the measure, and the termination
\tcc\ is generated as a \emph{termination-subtype}
\tcc.\index{termination-subtype TCC}\index{TCC!termination-subtype}
Termination-subtype \tccs\ are recognized as such by the occurrence of the
order in the goal of the \tcc.  For example, we could define a
substitution function for terms as follows.
\begin{pvsex}
  term: DATATYPE
  BEGIN
   mk_var(index: nat): var?
   mk_const(index: nat): const?
   mk_apply(fun: term, args: list[term]): apply?
  END term

  subst(x: (var?), y: term)(s: term): RECURSIVE term =
    (CASES s OF
      mk_var(i): (IF index(x) = i THEN y ELSE s ENDIF),
      mk_const(i): s,
      mk_apply(t, ss): mk_apply(subst(x, y)(t), map(subst(x, y))(ss))
     ENDCASES)
  MEASURE s BY <<
\end{pvsex}
Now the first recursive occurrence of \texttt{subst} has all arguments
provided, so the termination TCC is as expected.  The second occurrence
does not have enough arguments.  The recursive signature of that
occurrence is
\begin{pvsex}
  [[(var?), term] -> [\setb{}z1: term | z1 << s\sete -> term]]
\end{pvsex}
Hence the signature of \texttt{subst(x, y)} is \texttt{[\setb{}z1:\ term | z1 <<
s\sete -> term]}, and map is instantiated to \texttt{map[\setb{}z1:\ term | z1 <<
s\sete, term]}, which leads to the TCC
\begin{pvsex}
subst_TCC2: OBLIGATION
  FORALL (ss: list[term], t: term, s: term, x: (var?)):
    s = mk_apply(t, ss) IMPLIES every[term](LAMBDA (z: term): z << s)(ss);
\end{pvsex}
Note that this \texttt{map} instance could be given directly, just don't
make the mistake of providing \texttt{map[term, term]}, as this leads to a
TCC that says every \texttt{term} is \texttt{<<} \texttt{s}.
For the same reason, if the uncurried form of this definition is given,
then a lambda expression will have to be provided and the type will have
to include the measure, for example,
\begin{pvsex}
   subst(x: (var?), y, s: term): RECURSIVE term =
     (CASES s OF
       mk_var(i): (IF index(x) = i THEN y ELSE s ENDIF),
       mk_const(i): s,
       mk_apply(t, ss): mk_apply(subst(x, y, t),
                                 map(LAMBDA (s1: \setb{}z: term|z<<s\sete):
                                       subst(x, y, s1))(ss))
      ENDCASES)
   MEASURE s BY <<
\end{pvsex}

The recursive signature is generated based on the type of the recursive
function and the measure.  For curried functions, it may be that the
measure does not have the entire domain of the recursive function, but
only the first few.  For example, consider the measure for the function
\texttt{f}.
\begin{pvsex}
  f(r: real)(x, y: nat)(b: boolean): RECURSIVE boolean
    = ...
   MEASURE LAMBDA (r: real): LAMBDA (x, y: nat): x
\end{pvsex}
The type of the measure function is \texttt{[real -> [nat, nat -> nat]]},
which is a prefix of the function type.  In deriving the recursive
signature, the last domain type of the measure is constrained (using a
subtype) in the corresponding position of the recursive function type.  In
this case the recursive signature is
\begin{pvsex}
  [real -> [\setb{}z: [nat, nat] | z`1 < x\sete -> [boolean -> boolean]]]
\end{pvsex}
Note that the recursive signature is a dependent type that depends on the
arguments of the recursive function (\texttt{x} in this case), and hence
only applies within the body of the recursive definition.

The formal argument that typechecking the body of a recursive function
using the recursive signature is sound will appear in a future version of
the semantics manual, for now note that simple attempts to subvert this
mechanism do not work, as the following example illustrates.
\begin{pvsex}
  fbad: RECURSIVE [nat -> nat] = fbad
   MEASURE lambda (n: nat): n
\end{pvsex}
This leads an unprovable TCC.
\begin{pvsex}
  fbad_TCC1: OBLIGATION FORALL (x1: nat, x: nat): x < x1;
\end{pvsex}
The TCC results from the comaprison of the expected type \texttt{[nat ->
nat]} to the derived type \texttt{[\setb{}z:\ nat | z < x1\sete -> nat]}.  Remember
that in PVS domains of function types must be equal in order for the
function types to satisfy the subtype relation, and this is exactly what
the TCC states.

\pvstheory{f91-alltt}{Theory \texttt{f91}}{f91-alltt}
\index{f91@{\texttt{f91}}}

When a doubly recursive call is found, the inner recursive calls are
replaced by variables in the termination \tccs\ generated for the outer
calls.  For example, given the theory of Figure~\ref{f91-alltt} the
termination \tcc\ is

\begin{session}
f91_TCC5: OBLIGATION
  FORALL (i: nat,
          v: [i1:
               \setb{}z: nat |
                        (IF z > 101 THEN 0 ELSE 101 - z ENDIF) <
                         (IF i > 101 THEN 0 ELSE 101 - i ENDIF)\sete ->
               \setb{}j: nat | IF i1 > 100 THEN j = i1 - 10 ELSE j = 91 ENDIF\sete]):
    NOT i > 100 IMPLIES
     IF i > 100 THEN v(v(i + 11)) = i - 10 ELSE v(v(i + 11)) = 91 ENDIF;
\end{session}
where the inner calls to \texttt{f91} have been replaced by the
higher-order variable \texttt{v}, with the recursive signature as shown.
Since the obligation forces us to prove the termination condition for all
functions whose type is that of \texttt{f91}, it will also hold for
\texttt{f91}.  This example also illustrates the use of dependent types,
discussed in Section~\ref{dependent-types}.

\pvstheory{ackerman-alltt}{Theory \texttt{ackerman}}{ackerman-alltt}
\index{ackerman@{\texttt{ackerman}}}

In some cases the natural numbers are not a convenient measure; PVS
also provides the \texttt{ordinal}s, which allow recursion with measures up
to $\varepsilon_0$.  This is primarily useful in handling
lexicographical orderings.  For example, in the definition of the
Ackerman function in Figure~\ref{ackerman-alltt},\footnote{There are
ways of specifying \texttt{ackerman} using higher-order functionals, in
which case the measure is again on the natural numbers.} there are two
termination \tccs\ generated (along with a number of subtype \tccs).
The first termination \tcc\ is
\begin{pvsex}
  ack_TCC2:
    OBLIGATION
      (FORALL m, n:
        NOT m = 0 AND n = 0 IMPLIES ackmeas(m - 1, 1) < ackmeas(m, n))
\end{pvsex}
%
and corresponds to the first recursive call of \texttt{ack} in the body of
\texttt{ack}.  In this occurrence, it is known that \texttt{m $\neq$ 0}
and \texttt{n = 0}.  The remaining expression says that the measure
applied to the arguments of the recursive call to \texttt{ack} is less
than the measure applied to the initial arguments of \texttt{ack}.  Note
that the \texttt{<} in this expression is over the \texttt{ordinal}s, not
the \texttt{real}s.

\index{recursive definitions|)}


\section{Macros}\label{macro-declarations}
\index{macros|(}

There are some definitions that are convenient to use, but it's preferable
to have them expanded whenever they are referenced.  To some extent this
can be accomplished using auto-rewrites in the prover, but rewriting is
restricted.  In particular terms in types or actual parameters are not
rewritten; \texttt{typepred} and \texttt{same-name} must be used.  These
both require the terms to be given as arguments, making it difficult to
automate proofs.

The \texttt{MACRO} declaration is used to indicate definitions that are
expanded at typecheck time.  Macro declarations are normal constant
declarations, with the \texttt{MACRO} keyword preceding the
type.\footnote{This is similar to the \texttt{==} form of E\textsc{hdm}.}
For example, after the declaration
\begin{pvsex}
  N: MACRO nat = 100
\end{pvsex}
any reference to \texttt{N} is now automatically replaced by \texttt{100},
including such forms as \texttt{below[N]}.

Macros are not expanded until they have been typechecked.  This is because
the name overloading allowed by PVS precludes expanding during parsing.
TCCs are generated before the definition is expanded.
\index{macros|)}

\input{inductive_defs}

\section{Formula Declarations}\label{formula-declarations}
\index{formula declarations|(}\index{declaration!formulas|(}

Formula declarations introduce \emph{axioms}\index{axioms},
\emph{assumptions}\index{assumptions}, \emph{theorems}\index{theorems},
and \emph{obligations}\index{obligations}.  The identifier associated with
the declaration may be referenced in auto-rewrite declarations (see
Section~\ref{auto-rewrite-decls} and in proofs (see the \texttt{lemma} command
in the PVS Prover Guide~\cite{PVS:prover}).  The expression that makes up
the body of the formula is a boolean expression.  Axioms, assumptions, and
obligations are introduced with the keywords \texttt{AXIOM},
\texttt{ASSUMPTION}, and \texttt{OBLIGATION}, respectively.  Axioms may
also be introduced using the keyword \texttt{POSTULATE}\index{postulate}.
In the prelude postulates are used to indicate axioms that are provable by
the decision procedures, but not from other axioms.  Theorems may be
introduced with any of the keywords
\texttt{CHALLENGE}\index{claim@{\texttt{CHALLENGE}}},
\texttt{CLAIM}\index{claim@{\texttt{CLAIM}}},
\texttt{CONJECTURE}\index{conjecture@{\texttt{CONJECTURE}}},
\texttt{COROLLARY}\index{corollary@{\texttt{COROLLARY}}},
\texttt{FACT}\index{fact@{\texttt{FACT}}},
\texttt{FORMULA}\index{formula@{\texttt{FORMULA}}},
\texttt{LAW}\index{law@{\texttt{LAW}}},
\texttt{LEMMA}\index{lemma@{\texttt{LEMMA}}},
\texttt{PROPOSITION}\index{proposition@{\texttt{PROPOSITION}}},
\texttt{SUBLEMMA}\index{sublemma@{\texttt{SUBLEMMA}}}, or
\texttt{THEOREM}\index{theorem@{\texttt{THEOREM}}}.

Assumptions are only allowed in assuming clauses (see
Section~\ref{assuming}).  Obligations are generated by the system for
\tccs, and cannot be specified by the user.  Axioms are treated
specially when a proof is analyzed, in that they are not expected to
have an associated proof.  Otherwise they are treated exactly like
theorems.  All the keywords associated with theorems have the same
semantics, they are there simply to allow for greater diversity in
classifying formulas.

Formula declarations may contain free variables\index{free variables}, in
which case they are equivalent to the universal closure\index{universal
closure} of the formula.\footnote{The universal closure of a formula is
obtained by surrounding the formula with a \texttt{FORALL} binding
operator whose bindings are the free variables of the formula.  For
example, the universal closure of \texttt{p(x,y) => q(z)} is
\texttt{(FORALL x,y,z:\ p(x,y) => q(z))} (assuming \texttt{x}, \texttt{y}
and \texttt{z} resolve to variables).} In fact, the prover actually uses
the universal closure when it introduces a formula to a proof.  Formula
declarations are the only declarations in which free variables are
allowed.

\index{declaration!formulas|)}\index{formula declarations|)}

\input{judgements}

\input{conversions}

\input{libraries}

\input{auto-rewrite}

\index{declaration|)}

% Document Type: LaTeX
% Master File: language.tex

\chapter{Types}\label{types}
\index{type|(}

PVS specifications are strongly typed, meaning that every expression has
an associated type (although it need not be unique, more on this later).
The PVS type system is based on \emph{structural
equivalence}\index{structural equivalence} instead of \emph{name
equivalence}\index{name equivalence}, so types are closely related to
sets, in that two types are equal iff they have the same elements.
Section~\ref{type-declarations} describes the introduction of type names,
which are the simplest type expressions.  More complex type
expressions\index{type expressions} are built from these using \emph{type
constructors}\index{type constructors}.  There are type constructors for
\emph{subtypes}\index{subtypes}\index{type!subtype}, \emph{function
types}\index{function types}\index{type!function}, \emph{tuple
types}\index{tuple types}\index{type!tuple}, and \emph{record
types}\index{record types}\index{type!record}.  Function, record,
and tuple types may also be \emph{dependent}\index{dependent
types}\index{type!dependent}.  A form of \emph{type
application}\index{type application}\index{type!application} is
provided that makes it more convenient to specify parameterized subtypes.
There are also provisions for creating \emph{abstract datatypes},
described in Chapter~\ref{datatypes}.

Type expressions occur throughout a specification; in particular, they may
appear in theory parameters, type declarations, variable declarations,
constant declarations, recursive and inductive definitions, conversions,
and judgements.  In
addition, they may appear in certain expressions (coercions and local
bindings, see pages~\pageref{coercions} and~\pageref{binding-expressions},
respectively), and as actual parameters in names (page~\pageref{names}).
In the many examples which follow, type expressions will be presented in
the context of type declarations; but it must be remembered that they can
appear in any of the above places.

\pvsbnf{bnf-type-expr}{Type Expression Syntax}

\section{Subtypes}\label{subtypes}
\index{subtypes|(}\index{type!subtype|(}

Any collection of elements of a given type itself forms a type, called a
\emph{subtype}.  The type from which the elements are taken is called the
\emph{supertype}\index{supertype}.  The elements which form the subtype
are determined by a \emph{subtype predicate}\index{subtype predicate} on
the supertype.

Subtypes in PVS provide much of the expressive power of the language,
at the cost of making typechecking undecidable.  There are two forms of
subtypes.  The first is similar to the notation used to define a set:
\begin{pvsex}
  t: TYPE = \{x: s | p(x)\}
\end{pvsex}
%
where {\tt p} is a predicate on the type {\tt s}.\footnote{If {\tt x}
has been previously declared as a variable of type {\tt s}, then the
``{\tt :~s}'' may be omitted.} This has the usual set-theoretical
meaning, since types in PVS are modeled as sets.  Subtypes may also
be presented in an abbreviated form, by giving a predicate surrounded by
parentheses:
\begin{pvsex}
  t: TYPE = (p)
\end{pvsex}
%
This is equivalent to the form above.

Note that if the predicate \texttt{p} is everywhere false, then the type
is empty.  PVS supports empty types\index{empty type}\index{type!empty},
and the term \emph{type} is used to refer to any type, including the empty
type.  This is discussed in Section~\ref{type-declarations} (page~\pageref{type-declarations}).

Subtypes tend to make specifications more succinct and easier to read.
For example, in a specification such as
\begin{pvsex}
  FORALL (i:int):
    (i >= 0 IMPLIES (EXISTS (j:int): j >= 0 AND j > i))
\end{pvsex}
it is much more difficult to see what is being stated than in the
equivalent
\begin{pvsex}
  FORALL (i:nat): (EXISTS (j:nat): j > i))
\end{pvsex}
%
where {\tt nat} is defined in the prelude as
\begin{pvsex}
  naturalnumber: NONEMPTY\_TYPE = \{i:integer | i >= 0\} CONTAINING 0
  nat: NONEMPTY\_TYPE = naturalnumber
\end{pvsex}

Subtype constructors consist of a \emph{supertype}\index{supertype} and
an optional predicate on the supertype.  The primary property of a
subtype is that any element which belongs to the subtype automatically
belongs to the supertype.  In addition, functions defined on a type
automatically carry over to the subtype.

\index{TCC|(}

There are two \emph{type-correctness conditions} (\tccs) associated with
subtypes.  The first concerns \emph{empty types}\index{empty
type}\index{type!empty} as described in section~\ref{emptytypes}.  PVS
allows empty types as long as only variables range over them.  However,
allowing declarations of constants involving empty types leads to
inconsistencies.  Whenever a constant is declared, the typechecker checks
the types involved, and generates \emph{ existence} \tccs\ \index{existence
TCC}\index{TCC!existence} for those types which must be nonempty.  For
example,
%
\begin{pvsex}
  f: [int -> \{x:int | p(x)\}]
\end{pvsex}
leads to the \tcc\
\begin{pvsex}
  f_TCC1: OBLIGATION (EXISTS (x: int): p(x))
\end{pvsex}
These \tccs\ are recorded, so that the nonemptiness of a subtype need
be established only once in a theory.  However, the same \tcc\ may be
generated in different theories.  In particular, if a theory declares a
type but no constant of that type, then any theory which imports that
theory and declares a constant of that type will generate the nonempty
\tcc.  

A subtype may be guaranteed nonempty by providing a
\emph{witness},\index{witness} in which case no existence \tcc\ is
generated, though typechecking the witness itself may generate a \tcc.
The witness is provided using the {\tt
CONTAINING}\index{CONTAINING@\texttt{CONTAINING}} clause of a subtype
expression, as illustrated in the following:
\begin{pvsex}
  t: TYPE = \{x: int | 0 < x AND x < 10\} CONTAINING 1
\end{pvsex}
In this case a \tcc\ is generated with the witness in place of the
existential variable, resulting in the trivial \tcc\footnote{This \tcc\
will be proved automatically by PVS; see the
\cmd{typecheck-prove} command in the PVS System Guide~\cite{PVS:userguide}.}
\begin{pvsex}
  t_TCC1: OBLIGATION 0 < 1 AND 1 < 10
\end{pvsex}

The second \tcc\ associated with subtypes is the \emph{subtype}
\tcc,\index{subtype TCC}\index{TCC!subtype}, which come about from the use
of operations defined on subtypes which are applied to elements of the
supertype.  By this means partial functions may be handled directly,
without recourse to a partial term logic or some form of multi-valued
logic.  For instance, division in PVS is a total function, with signature
{\tt [real, nonzero\_real -> real]}.  So given the formula
\begin{pvsex}
  div_form: FORMULA (FORALL (x,y: int):
                      x /= y IMPLIES (x - y)/(y - x) = -1)
\end{pvsex}
%
the denominator is of type integer, but the signature for {\tt /} demands
a {\tt nonzero\_real}.  The typechecker thus generates a \emph{subtype}
\tcc\ whose conclusion is {\tt (y - x) /= 0}.  The premises of the \tcc\
are obtained from the expression's \emph{context}---the conditions which
lead to the {\tt /} operator---in this case {\tt x /= y}.  The \tcc\ is
then
\begin{pvsex}
  div_form_TCC1: OBLIGATION
    (FORALL (x,y: int): x /= y IMPLIES (y - x) /= 0)
\end{pvsex}
which is easily discharged by the prover.  In general, the context of an
expression is obtained from expressions involving {\tt IF-THEN-ELSE},
{\tt AND}, {\tt OR}, and {\tt IMPLIES} by translating to the {\tt
IF-THEN-ELSE} form.  Specifically,
\begin{center}
\begin{tabular}{|lc|} \hline
Expression & Context for {\tt E} \\ \hline
{\tt IF A THEN E ELSE C ENDIF} & {\tt A} \\
{\tt IF A THEN B ELSE E ENDIF} & {\tt NOT A} \\
{\tt A AND E} & {\tt A} \\
{\tt A OR E} & {\tt NOT A} \\
{\tt A IMPLIES E} & {\tt A} \\ \hline
\end{tabular}
\end{center}
Note that only these operators are treated this way; if, for example,
\texttt{IMPLIES} is overloaded it will not include the left-hand side in
the context for typechecking the right-hand side.  The \tccs\ generated
from the context of expression involving a subtype are sufficient, but not
necessary conditions which ensure that the value of the expression does
not depend on the value of functions applied outside their domain.

\index{TCC|)}

\index{type!subtype|)}\index{subtypes|)}

\section{Function Types}\label{function-types}
\index{function types|(}\index{type!function|(}

Function types have three equivalent forms:
\begin{itemize}
\item {\tt [t\(_1\), \ldots, t\(_n\) -> t]}

\item {\tt FUNCTION[t\(_1\), \ldots, t\(_n\) -> t]}

\item {\tt ARRAY[t\(_1\), \ldots, t\(_n\) -> t]}
\end{itemize}
%
where each {\tt t$_i$} is a type expression.  An element of this type is
simply a function whose domain is the sequence of types {\tt t$_1$},
\ldots, {\tt t$_n$}, and whose range is {\tt t}.  A function type is empty
if the range is empty and the domain is not.  There is no difference in
meaning between these three forms; they are provided to support different
intensional uses of the type, and may suggest how to handle the given type
when an implementation is created for the specification.

The two forms {\tt pred[t]}\index{pred@{{\tt pred}}} and {\tt
setof[t]}\index{setof@{{\tt setof}}} are both provided in the
prelude as shorthand for {\tt [t ->
bool]}.  There is no difference in semantics, as sets in
PVS are represented as predicates.  The different keywords are
provided to support different intentions; {\tt pred} focuses on
properties while {\tt setof} tends to emphasize elements.

A function type \texttt{[t$_1$,\ldots,t$_n$ -> t]} is a subtype of
\texttt{[s$_1$,\ldots,s$_m$ -> s]} iff \texttt{s} is a subtype of
\texttt{t}, $n = m$, and {\tt s$_i$} = {\tt t$_i$} for $1 \leq i \leq n$.
This leads to subtype TCCs (called \emph{domain mismatch
TCCs})\index{domain mismatch TCC}\index{TCC!domain mismatch} that state
the equivalence of the domain types.  For example, given
\begin{pvsex}
  p,q: pred[int]
  f: [\{x: int | p(x)\} -> int]
  g: [\{x: int | q(x)\} -> int]
  h: [int -> int]
  eq1: FORMULA f = g
  eq2: FORMULA f = h
\end{pvsex}
%
The following \tccs\ are generated:
\begin{pvsex}
eq1_TCC1: OBLIGATION
  (FORALL (x1: \{x : int | q(x)\}, y1 : \{x : int | p(x)\}) :
     q(y1) AND p(x1))

eq2_TCC1: OBLIGATION
  (FORALL (x1: int, y1 : \{x : int | p(x)\}) :
     TRUE AND p(x1))
\end{pvsex}

Section~\ref{conversion-examples} on page~\pageref{conversion-examples}
explains how the \texttt{restrict} conversion may be automatically applied
in some cases to eliminate the production of these TCCs.

\index{type!function|)}\index{function types|)}


\section{Tuple Types}\label{tuple-types}
\index{tuple types|(}\index{type!tuple|(}

Tuple types (also called product types) have the form {\tt [t$_1$, \ldots,
t$_n$]}, where the {\tt t$_i$} are type expressions.  Note that the 0-ary
tuple type is not allowed.  Elements of this type are tuples whose
components are elements of the corresponding type.  For example, {\tt (1,
TRUE, (LAMBDA (x:int): x + 1))} is an expression of type {\tt [int, bool,
[int -> int]]}.  Order is important.  Associated with every $n$-tuple type
is a set of projection functions: \texttt{`1}, \texttt{`2}, \ldots, (or
{\tt proj\_1}, \texttt{proj\_2}, \ldots) where the $i$th projection is of
type {\tt [[t$_1$, \ldots, t$_n$] -> t$_i$]}.  A tuple type is empty if
any of its component types is empty.  Function type domains and tuple
types are closely related, as the types {\tt [t$_1$,\ldots, t$_n$ -> t]}
and {\tt [[t$_1$,\ldots, t$_n$] -> t]} are equivalent; see
Section~\ref{tuple-exprs} for more details.

\index{type!tuple|)}\index{tuple types|)}

\section{Record Types}\label{record-types}
\index{record types|(}\index{type!record|(}

Record types are of the form {\tt [\# a$_1$:t$_1$, \ldots, a$_n$:t$_n$
\#]}.  The {\tt a$_i$} are called \emph{record accessors}\index{record
accessors} or fields and the {\tt t$_i$} are types.  Record types are
similar to tuple types, except that the order is unimportant and accessors
are used instead of projections.  Record types are empty if any of the
component types is empty.

\index{type!record|)}\index{record types|)}

\section{Dependent types}\label{dependent-types}
\index{dependent types|(}\index{type!dependent|(}

Function, tuple, and record types may be dependent; in other words, some
of the type components may depend on earlier components.  Here are some
examples:
\begin{pvsex}
  rem: [nat, d: \{n: nat | n /= 0\} -> \{r: nat | r < d\}]
  pfn: [d:pred[dom], [(d) -> ran]]
  stack: [\# size: nat, elements: [\{n:nat | n < size\} -> t] \#]
\end{pvsex}
The declaration for {\tt rem} indicates explicitly the range of the
remainder function, which depends on the second argument.  Function types
may also have dependencies within the domain types; \eg\ the second domain
type may depend on the first.  Note that for function and tuple dependent
types, local identifiers need be associated only with those types on which
later types depend.

The tuple type {\tt pfn} encodes partial functions as pairs consisting
of a predicate on the domain type and a function from the subtype
defined by that predicate to the range {\tt ran}.  If the second
component were given instead as a function of type {\tt [dom -> ran]},
then equality no longer works as intended.  For example, the absolute
value function {\tt abs} and the identity function {\tt id} are the same
on the domain {\tt nat}, so we would like to have
\begin{pvsex}
  ((LAMBDA (x:int):x >= 0),abs) = ((LAMBDA (x:int):x >= 0),id)
\end{pvsex}
%
but without the dependency this would be equivalent to {\tt abs = id}.

{\tt stack} encodes a stack as a pair consisting of a size and an array
mapping initial segments of the natural numbers to {\tt t}.  This is
similar to the {\tt pfn} example---in fact, if we were willing to use a
tuple instead of a record encoding, {\tt stack} could be declared as an
instance of the type of {\tt pfn}.

Another example, presented in~\cite{Cheng&Jones90} as a ``challenge'' to
specification languages without partial functions, is easily handled
with dependent types as shown below.
\begin{pvsex}
  subp(i:int,(j:int | i >= j)): RECURSIVE int =
       (IF (i=j) THEN 0 ELSE (subp(i, j+1)+1) ENDIF)
    MEASURE i - j
\end{pvsex}
However, some formulas that are valid with partial functions are not even
well-formed in PVS:
\begin{pvsex}
  subp_lemma: LEMMA subp(i, 0) = i OR subp(0, i) = i
\end{pvsex}
This generates unprovable TCCs.  In practice this is rarely a problem.

\index{type!dependent|)}\index{dependent types|)}

\index{type|)}

% Document Type: LaTeX
% Master File: language.tex

\chapter{Expressions}\label{expressions}
\index{expressions|(}

The PVS language offers the usual panoply of expression constructs,
including logical and arithmetic operators, quantifiers, lambda
abstractions, function application, tuples, a polymorphic
\texttt{IF-THEN-ELSE}, and function and record overrides.  Expressions may
appear in the body of a formula or constant declaration, as the predicate
of a subtype, or as an actual parameter of a theory instance.  The syntax
for PVS expressions is shown in Figures~\ref{bnf-expr} and~\ref{bnf-expr-aux}.

\pvsbnf{bnf-expr}{Expression syntax}

\pvsbnf{bnf-expr-aux}{Expression syntax (continued)}

\index{precedence|(} The language has a number of predefined operators
(although not all of these have a predefined meaning).  These are given in
Figure~\ref{precedenceops} below, along with their relative precedence
from lowest to highest.  Most of these operators are described in the
following sections.  \texttt{IN} is a part of \texttt{ LET} expressions,
\texttt{WITH} goes with override expressions, and the double colon
(\texttt{::}) is for coercion expressions.  The \texttt{o} operator is
defined in the prelude as the function composition operator.  Note that
most operators may be overloaded, see Chapter~\ref{lexical}
(page~\pageref{lexical}) for details.

\begin{figure}[htb]
\begin{center}{\small\tt
\begin{tabular}{|l|l|} \hline
{\rm Operators} & {\rm Associativity} \\ \hline
FORALL, EXISTS, LAMBDA, IN & None \\
\verb/|/ & Left \\
\verb/|-/, \verb/|=/ & Right \\
IFF, <=> & Right \\
IMPLIES, =>, WHEN & Right \\
OR, \verb|\/|, XOR, ORELSE & Right \\
AND, \&, \&\&, \verb|/\|, ANDTHEN & Right \\
NOT, \verb|~| & None \\
=, /=, ==, <, <=, >, >=, <<, >>, <<=, >>=, <|, |> & Left \\
WITH & Left \\
WHERE & Left \\
@, \# & Left \\
@@, \#\#, || & Left \\
+, -, ++, ~ & Left \\
*, /, **, // & Left \\
- & None \\
o & Left \\
:, ::, HAS\_TYPE & Left \\
\verb|[]|, <> & None \\
\verb|^|, \verb|^^| & Left \\
` & Left \\ \hline
\end{tabular}}
\end{center}\caption{Precedence Table}\label{precedenceops}
\end{figure}
\index{precedence|)}

\index{operator symbols|(}

Many of the operators may be overloaded by the user and retain their
precedence and form (\eg\ infix).  All of the infix operators may also be
given in prefix form; \texttt{x + 1} and \texttt{+(x,1)} are semantically equivalent.  Care must be taken in redefining these operators---if the
preceding declaration ends in an expression there could be an ambiguity.
To handle this situation the language allows declarations to be terminated
with a '\texttt{;}'.  For example,
\begin{pvsex}
  AND: [state, state -> state] = (LAMBDA a,b: (LAMBDA t: a(t) AND b(t)));
  OR: [state, state -> state] = (LAMBDA a,b: (LAMBDA t: a(t) OR b(t)));
\end{pvsex}
%
without the semicolon the second declaration would be seen as an infix
\texttt{OR} and the result would be a parse error.

Another common mistake when overloading operators with predefined meanings
is the assumption that overloading, for example, {\tt IMPLIES} automatically
provides an overloading for {\tt =>}.  This is not the case---they are distinct
operators (which happen to have the same meaning by default) and not syntactic
sugar.

\index{operator symbols|)}

\section{Boolean Expressions}\label{bool-exprs}
\index{boolean expressions}

The Boolean expressions include the constants \texttt{TRUE}\index{true@{\texttt{TRUE}}} and
\texttt{FALSE}\index{false@{\texttt{FALSE}}},
the unary operator \texttt{NOT}\index{not@{\texttt{NOT}}}, and
the binary operators \texttt{AND}\index{and@{\texttt{AND}}} (also written
\texttt{ \&}\index{\&}), \texttt{OR}\index{or@{\texttt{OR}}}, \texttt{
IMPLIES}\index{implies@{\texttt{IMPLIES}}}
(\texttt{=>}\index{=>@\texttt{=>}}),
\texttt{WHEN}\index{when@{\texttt{WHEN}}}, and
\texttt{IFF}\index{iff@{\texttt{IFF}}}
(\texttt{<=>}\index{<=>@\texttt{<=>}}).  The declarations for these are in
the \texttt{booleans} prelude theory.  All of these have their standard
meaning, except for \texttt{WHEN}, which is the converse of
\texttt{IMPLIES} (\ie\ $A$ \texttt{WHEN} $B$ $\equiv$ $B$ \texttt{IMPLIES}
$A$).

Equality\index{equality} (\texttt{=}\index{=}) and
disequality\index{disequality} (\texttt{/=}\index{/=}) are declared in the
prelude theories \texttt{equalities} and \texttt{notequal}.  They are both
polymorphic, the type depending on the types of the left- and right-hand
sides.  If the types are compatible, meaning that there is a common
supertype, then the (dis)equality is of the greatest common supertype.  Otherwise it is a type
error.  For example,
\begin{pvsex}
  S,T: TYPE
  s: VAR S
  t: VAR T
  eq1: FORMULA s = t
  i: VAR \setb{}x: int | x < 10\sete
  j: VAR \setb{}x: int | x > 100\sete
  eq2: FORMULA i = j
\end{pvsex}
%
\texttt{eq1} will cause a type error---remember that \texttt{S} and \texttt{T}
are assumed to be disjoint.  \texttt{eq2} is perfectly typesafe because
they have a common supertype \texttt{int} even though the subtypes have no
elements in common; the equality simply has the value \texttt{FALSE}.

When the equality is between terms of type \texttt{bool}, the semantics
are the same as for \texttt{IFF}.  There is a pragmatic difference in the
way the PVS prover processes these operators.  Equalities may be
used for rewriting, which makes for efficient proofs but is incomplete,
\ie\ the prover may fail to find the proof of a true formula.  On the other
hand the \texttt{IFF} form is complete, but may lead to a large number of
cases.  When in doubt, use equality as the prover provides commands
that turn an equality into an \texttt{IFF}.

%The decision to disallow \texttt{eq1} is a pragmatic one; the
%utility of such a declaration is questionable, and most likely the user
%has made an error in the specification.


\section{\texttt{IF-THEN-ELSE} Expressions}
\index{if-then-else@{\texttt{IF-THEN-ELSE}}}

The \texttt{IF-THEN-ELSE} expression \texttt{IF} {\em cond\/} \texttt{THEN} {\em
expr1\/} \texttt{ELSE} {\em expr2\/} \texttt{ENDIF} is polymorphic; its type is the
common type of {\em expr1\/} and {\em expr2\/}.  The {\em cond\/} must
be of type \texttt{boolean}.  Note that the \texttt{ELSE} part is not
optional as this is an expression, not an operational statement.  The
declaration for \texttt{IF} is in the \texttt{if\_def} prelude theory.  \texttt{
IF-THEN-ELSE} may be redeclared by the user in the same way as \texttt{
AND}, \texttt{OR}, etc.  Note that only \texttt{IF} is explicitly redeclared,
the \texttt{THEN} and \texttt{ELSE} are implicit.

Any number of \texttt{ELSIF} clauses may be present; they are translated into nested
\texttt{IF-THEN-ELSE} expressions.  Thus the expression
\begin{pvsex}
  IF A THEN B
  ELSIF C THEN D
  ELSE E
  ENDIF
\end{pvsex}
%
translates to
\begin{pvsex}
  IF A THEN B
  ELSE (IF C THEN D
        ELSE E
        ENDIF)
  ENDIF
\end{pvsex}

\section{Numeric Expressions}
\index{numeric expressions}

The numeric expressions include the \emph{numerals}\index{numerals} (0, 1,
2, \ldots), the unary operator \texttt{-}\index{-}, and the binary infix
operators \texttt{\char94}\index{\^}, \texttt{+}\index{+},
\texttt{-}\index{-}, \texttt{*}\index{*}, and \texttt{/}\index{/}.  The
numerals are all of type \texttt{real}\index{real@\texttt{real}}.
The typechecker has implicit judgements on numbers; \texttt{0} is known to
be \texttt{real}, \texttt{rat}, \texttt{int} and \texttt{nat}; all others
are known to be non zero and greater than zero.  The relational operators
on numeric types are \texttt{<}\index{<@\texttt{<}}, \texttt{
<=}\index{<=@\texttt{<=}}, \texttt{>}\index{>@\texttt{>}}, and
\texttt{>=}\index{>=@\texttt{>=}}.  The numeric operators and axioms are
all defined in the prelude.  As with the boolean operators, all of these
operators may be defined on new types and retain their original
precedences.

The numerals may also be treated as names, and
overloaded.\index{overloading numberals}\index{numerals!overloading} This
is particularly useful for defining algebraic structures such as groups
and rings, where it is natural to overload `\texttt{0}' and `\texttt{1}'.
Note that such use may include actual parameters, just as for names.  Thus
\texttt{groups[int].0} or \texttt{0[int]} might refer to the group zero
instantiated with the integer carrier set.

\section{Characters and String Expressions}
\index{string expressions}

String expressions are expressions enclosed in double quotes `\texttt{"}',
for example,
\begin{pvsex}
  "This is a string"
\end{pvsex}
Strings consist of eight bit ASCII characters.  To include control
characters or characters above the usual seven bits, use a back slash
`\verb|\|', as described in the following table.

\begin{tabular}{ll}
\verb|\a| & \verb|^G| (BEL) \\
\verb|\b| & \verb|^H| (backspace) \\
\verb|\f| & \verb|^L| (form feed) \\
\verb|\n| & \verb|^J| (new line) \\
\verb|\r| & \verb|^M| (carriage return) \\
\verb|\t| & \verb|^I| (horizontal tab) \\
\verb|\v| & \verb|^K| (vertical tab) \\
\verb|\"| & double quote \\
\verb|\\| & backslash \\
\verb|\x|NN & byte with hexadecimal value NN (2 digits) \\
\verb|\|NNN & byte with decimal value NNN (3 digits) \\
\verb|\0|NNN & byte with octal value NNN (3 digits) \\
\end{tabular}

Strings are finite sequences of characters, which in turn are represented
by a datatype.
\begin{pvsex}
  character: DATATYPE
   BEGIN
    char(code:below[256]):char?
   END character
\end{pvsex}
When a string is parsed, it is internally converted to a conversion of a
list of characters to a finite sequence.  The following lemm is thus
trivially true, because both sides are actually the same term.
\begin{pvsex}
  string_rep: LEMMA
    "foo" = list2finseq(cons(char(102),
                             cons(char(111),
                                  cons(char(111), null))))
\end{pvsex}
Note that there is no special notation for characters; this is because the
\texttt{extract1} conversion will automatically convert a string of length
one to a character.  Note also that because of the \texttt{finseq\_appl}
conversion, a specific character may be extracted from a string simply by
applying it.  For example the following will typecheck
\begin{pvsex}
  f: character = "f"
  char_test: LEMMA "foo"(0) = f
\end{pvsex}  

\section{Applications}
\index{application expressions}

Function application is specified as in ordinary mathematics; thus the
application of function \texttt{f} to expression \texttt{x} is denoted \texttt{
f(x)}.  Those operator symbols that are binary functions, and their
applications, may be written in prefix or the usual infix notation.  For
example, \texttt{(3 + 5) = (2 * 4)} may be written as \texttt{=(+(3,5),
*(2,4))}.

PVS supports higher-order types, so that functions may yield functions
as values or be curried\index{curried applications}.  For example, given
\texttt{f} of type \texttt{[int -> [int, int -> int]]}, \texttt{f(0)(2,3)}
yields an \texttt{int}.

If the application involves a dependent function type then the result
type of the application is substituted for accordingly.  For example,
\begin{pvsex}
  f: [a:int, b:\setb{}x:int | a < x\sete -> \setb{}y:int | a < y & y <= b\sete]
\end{pvsex}
the application \texttt{f(2,3)} is of type \texttt{\setb{}y:int | 2 < y \& y <=
3\sete}.  This application will also lead to the subtype \tcc\ \texttt{2 < 3}.

Application and tuple expressions have a special relation, due to the
type equivalence of \texttt{[t$_1$,\ldots,t$_n$ -> t]} and \texttt{
[[t$_1$,\ldots,t$_n$] -> t]}, see Section~\ref{tuple-exprs} for details.

\section{Binding Expressions}\label{binding-expressions}
\index{binding expressions}

The binding expressions are those which create a local scope for
variables, including the quantified expressions and
$\lambda$-expressions.  Binding expressions consist of an operator, a
list of bindings, and an expression.  The operator is one of the
keywords \texttt{FORALL}\index{forall@\texttt{FORALL}}, \texttt{
EXISTS}\index{exists@\texttt{EXISTS}}, or \texttt{LAMBDA}\index{lambda@{\texttt{LAMBDA}}}.\footnote{Set
expressions are also binding expressions; see Section~\ref{set-exprs} (page~\pageref{set-exprs}).}
The bindings specify the variables bound by the operator; each variable
has an id and may also include a type or a constraint.  Here is a
contrived example:
\begin{pvsex}
  x,y,z,d,e: VAR real
  ex1: AXIOM FORALL x,y,z: (x + y) + z = x + (y + z)
  ex2: AXIOM FORALL (x,y,z: nat): x * (y + z) = (x * y) + (x * z)
  ex3: AXIOM FORALL (n: num | n /= 0): EXISTS (x | x /= 0): x = 1/n
\end{pvsex}
%
In \texttt{ex1}, variables \texttt{x}, \texttt{y}, and \texttt{z} are all of type
\texttt{real}.  In \texttt{ex2} these same variables are of type \texttt{nat},
shadowing the global declarations.  \texttt{ex3} illustrates
the use of constraints; this is equivalent to the declaration
\begin{pvsex}
  ex3: AXIOM FORALL (n: \setb{}n: num | n /= 0\sete):
               EXISTS (x: \setb{}x | x /= 0\sete): x = 1/n
\end{pvsex}

Quantified expressions\index{quantified expressions} are introduced with
the keywords \texttt{FORALL} and \texttt{EXISTS}.  These expressions are
of type \texttt{boolean}.

Lambda expressions\index{lambda expressions} denote unnamed functions.
For example, the function which adds \texttt{3} to an integer may be
written
\begin{pvsex}
  (LAMBDA (x: int): x + 3)
\end{pvsex}
%
The type of this expression is the function type \texttt{[int ->
numfield]}.\footnote{\texttt{numfield} sits between \texttt{number} and
\texttt{real}, and is where the field operators are introduced.  See
Section~{prelude-numbers}.}  In addition, when the range is \texttt{bool},
a lambda expression may be represented as a set expression; see
Section~\ref{set-exprs}.

All of the binding expressions may involve dependent
types\index{dependent types} in the bindings, \eg
\begin{pvsex}
  FORALL (x: int), (y: \setb{}z: int | x < z\sete): p(x,y)
\end{pvsex}
%
Note that in the instantiation of such an expression during a proof will
generally lead to a subtype \tcc.  For example, substituting \texttt{e$_1$} for
\texttt{x} and \texttt{e$_2$} for \texttt{y} will lead to the \tcc\ \texttt{e$_1$ <
e$_2$}.\footnote{Such \tccs\ may never be seen, as they tend to be
proved automatically during a proof; more complicated examples may be
given, for which the prover would need help from the user.  In addition,
a false \tcc\ can show up, \eg\ substituting \texttt{2} for \texttt{x} and
\texttt{1} for \texttt{y}.  This means that the corresponding expression is
not type correct.}

Constant names may be treated as binding expressions by using a
\texttt{!}  suffix.  For example,
\begin{pvsex}
foo! (x : int) : e
\end{pvsex}
is equivalent to
\begin{pvsex}
foo( LAMBDA (x : int) : e)
\end{pvsex}

\section{\texttt{LET} and \texttt{WHERE} Expressions}
\index{let expressions@{\texttt{LET} expressions}}
\index{where expressions@{\texttt{WHERE} expressions}}

\texttt{LET} and \texttt{WHERE} expressions are provided for convenience,
making some forms easier to read.  Both of these forms provide local
bindings for variables that may then be referenced in the body of the
expression, thus reducing redundancy and allowing names to be provided for common subterms.
Here are two examples:
\begin{pvsex}
  LET x:int = 2, y:int = x * x IN x + y
  x + y WHERE x:int = 2, y:int = x * x
\end{pvsex}
%
The value of each of these expressions is 6.

\texttt{LET} and \texttt{WHERE} expressions are internally translated to
applications of lambda expressions; in this case both expressions
translate to
\begin{pvsex}
  (LAMBDA (x:int) : (LAMBDA (y:int) : x + y)(x * x))(2)
\end{pvsex}
%
These translations should be kept in mind when the semantics of these
expressions is in question.

The type declaration is optional, so the above could be written as
\begin{pvsex}
  LET x = 2, y = x * x IN x + y
  x + y WHERE x = 2, y = x * x
\end{pvsex}
In this case the typechecking of these expressions depends on whether
\texttt{x} and/or \texttt{y} have been previously declared as variables.
If they have, then those delarations are used to determine the type.
Otherwise, the right-hand side of the \texttt{=} is typechecked, and if it
is unambiguous is used to determine the type of the variable.  This is 
one way in which these expressions differ from their translation.
It is usually better to either reference a variable or give the type, as
the typechecker uses the ``natural'' type of the expression as the type of
the variable, which can lead to extra \tccs.

The \texttt{LET} expression has a limited form of pattern matching over
tuples.  An example is
\begin{pvsex}
  p: VAR [int, int]
  +(p): int = LET (m, n) = p IN m + n
\end{pvsex}
which is shorter than the equivalent
\begin{pvsex}
  p: VAR [int, int]
  +(p): int = LET m = p`1, n = p`2 IN m + n
\end{pvsex}


\section{Set Expressions}\label{set-exprs}

In PVS, sets of elements of a type \texttt{t} are represented as
predicates, \ie\ functions from \texttt{t} to \texttt{bool}.  The type of a
set may be given as \texttt{[t -> bool]}, \texttt{pred[t]}, or \texttt{
setof[t]}, which are all type equivalent.\footnote{The prelude theory
\texttt{defined\_types} also defines \texttt{PRED}, \texttt{predicate}, \texttt{
PREDICATE}, and \texttt{SETOF} as alternate equivalents.}
The choice depends wholly on the intended use of the type.
Similarly, a set may be given in the form \texttt{(LAMBDA (x:\ t):\
p(x))} or \texttt{\setb{}x:\ t | p(x)\sete}; these are equivalent
expressions.\footnote{In fact, internally they are represented by the
same abstract syntax, they simply print differently.} Note that the
latter form may also represent a type---this usually causes no
confusion as the context generally makes it clear which is expected.
The usual functions and properties of sets are provided in the prelude
theory \texttt{sets}.


\section{Tuple Expressions}\label{tuple-exprs}
\index{tuple expressions}

A tuple expression of the type \texttt{[t$_1$,\ldots,t$_n$]} has the form
\texttt{(e$_1$,\ldots,e$_n$)}.  For example, \texttt{(2, TRUE, (LAMBDA x:\ x +
1))} is of type \texttt{[nat, bool, [nat -> nat]]}.  0-tuples are not
allowed, and 1-tuples are treated simply as parenthesized expressions.
The following relation holds between function types and tuple types:
\begin{pvsex}
  [[t\(\sb{1}\),\ldots,t\(\sb{n}\)] -> t] \(\equiv\) [t\(\sb{1}\),\ldots,t\(\sb{n}\) -> t]
\end{pvsex}
%
This equivalence is most important in theory parameters; it allows one
theory to take the place of many.  For example the \texttt{functions}
theory from the prelude may be instantiated by the reference
\texttt{injective?[[int,int,int],int]}.  Applications of an element \texttt{f} of
this type include \texttt{f(1,2,3)}, \texttt{f((1,2,3))}, and \texttt{f(e)},
where \texttt{e} is of type \texttt{[int,int,int]}.

\section{Projection Expressions}\label{projection-exprs}
\index{projection expressions}

The components of an expression whose type is a tuple can be accessed
using the projection operators \texttt{`1}, \texttt{`2}, \ldots or
\texttt{PROJ\_1}, \texttt{PROJ\_2}, \ldots.  The former are preferred.
Like reserved words, projection expressions are case insensitive and may
not be redeclared.  For the most part, projection expressions are
analogous to field accessors for record types.  For example,
\begin{pvsex}
  t: [int, bool, [int -> int]]
  ft: FORMULA t`2 AND t`1 > t`3(0)
  ft_deprecated: FORMULA PROJ_2(t) AND PROJ_1(t) > (PROJ_3(t))(0)
\end{pvsex}

Projection expressions may be used without an argument as long as the
context determines the tuple type involved.  For example, in the following
it is obvious what tuple type is involved.
\begin{pvsex}
  F: [[[int, bool, [int -> int]] -> bool] -> bool]
  FP: FORMULA F(PROJ_2)
\end{pvsex}
Note that the \texttt{PROJ} keyword must be used in such cases, as, e.g.,
\texttt{`2} is not an expression.  In the following example we see that
the context does not provide enough information.
\begin{pvsex}
  PP: FORMULA PROJ_2 = PROJ_2
\end{pvsex}
To deal with such situations, the syntax for projections has been extended
to allow the tuple type to be provided.
\begin{pvsex}
  PP: FORMULA PROJ_2[[int, bool, [int -> int]]] = PROJ_2
\end{pvsex}
In this case only one of the operators needs to be annotated.  This looks
like a use of actual parameters, but it is not, as the \texttt{PROJ} is
not a name, and does not belong to a theory.


\section{Record Expressions}\label{record-expressions}
\index{record expressions}

Record expressions are of the form \texttt{(\# a$_1$ := e$_1$, \ldots,
a$_n$ := e$_n$ \#)}, which has type \texttt{[\# a$_1$:\ t$_1$, \ldots,
a$_n$:\ t$_n$ \#]}, where each \texttt{e$_i$} is of type \texttt{t$_i$}.
Partial record expressions are not allowed; all fields must be given.  If
it is desired to give a partial record, declare an uninterpreted constant
or variable of the record type, and use override expressions to specify
the given record at the fields of interest.  For example,
\begin{pvsex}
  rc: [# a, b : int #]
  re: [# a, b : int #] = rc WITH [`a := 0]
\end{pvsex}

The type of a record expression is determined by the type of its
components.  Thus \texttt{(\# a := 3, b := 2 \#)} is of type \texttt{[\# a,
b: real \#]}.  This means that a record expression is never of a dependent
record type directly, though it may be used where a dependent record is
expected, and \tccs\ may be generated as a result.  For example,
\begin{pvsex}
  R: TYPE = [# a: int, b: \setb{}x: int | x < a\sete #]
  r: R = (# a := 3, b := 4 #)
\end{pvsex}
%
leads to the (unprovable) \tcc\ \texttt{4 < 3}.

Record expressions may be introduced without introducing the record type
first, and the type of a record expression is determined by its
components, independently of any previously declared record type.  For
this reason record types do not automatically generate associated accessor
functions.

\section{Record Accessors}

The components of an expression of a record type are accessed using the
corresponding field name.  There are two forms of access.  For example if
\texttt{r} is of type \texttt{[\# x, y: real \#]}, the x-component may be
accessed using either \texttt{r`x} or \texttt{x(r)}.  The first form is
preferred as there is less chance for ambiguity.

As noted above, accessors are not stand-alone functions.  However, you can
define your own functions to provide this capability, and even use the
same name.  For example:
\begin{pvsex}
  point: TYPE = [# x, y: real #]
  x(p:point): real = p`x
  y(p:point): real = p`y
\end{pvsex}
Now \texttt{x} and \texttt{y} may be provided wherever a function is
expected.  Note that this means that a subsequent expression of the form
\texttt{x(p)} could be ambiguous, but the record field accessor is always
preferred, so in practice such ambiguities don't arise.

\section{Cotuple Expressions}\label{cotuple-expressions}
\index{cotuple expression}

Elements of cotuple types \texttt{[t$_1$ + \ldots + t$_n$]} are constructed
with the \emph{injection} operators \texttt{IN\_$i$} of type
\texttt{[t$_i$ -> [t$_1$ + \ldots + t$_n$]]}.  Thus if $e$ is of type
\texttt{t$_i$}, \texttt{IN\_$i$($e$)} is of the cotuple type.  If $x$ is
an element of a cotuple type, \texttt{IN?\_$i$($x$)} is a boolean that
tests if $x$ belongs to the $i^{th}$ component, and if it does,
\texttt{OUT\_$i$($x$)} returns the associated value of type
\texttt{t$_i$}.  Note that this is similar to a datatype of the form
\begin{pvsex}
  cotup: DATATYPE
   BEGIN
    IN_1(OUT_1: t\(\sb{1}\)): IN?_1
    \(\cdots\)
    IN_\(n\)(OUT_\(n\): t\(\sb{n}\)): IN?_\(n\)
   END cotup
\end{pvsex}
The differences are that cotuples are not recursive, do not generate all
the functions and axioms associated with datatypes, and allow for any
number of component types---using datatypes a new one would have to be
given for each arity.

The analogy works also for the \texttt{CASES} expression described in
Section~\ref{cases-expressions}.  This allows access to the values of a
cotuple element.  It has the form
\begin{pvsex}
  CASES \(e\) OF
    IN_1(x1): f\(\sb{1}\)(x1),
    \vdots
    IN_\(n\)(x\(n\)): f\(\sb{n}\)(x\(n\))
  ENDCASES
\end{pvsex}
where each \texttt{f$_i$} is an expression of type \texttt{[t$_i$ ->
$T$]}, and the common return type $T$ is the type of the \texttt{CASES}
expression.  For example, if \texttt{x} is of type \texttt{[int + bool +
[int -> int]}, the following expression will return a boolean value.
\begin{pvsex}
  CASES x OF
    IN_1(i): i > 0,
    IN_2(b): NOT b,
    IN_3(f): FORALL (n: int): f(f(n)) = f(n)
  ENDCASES
\end{pvsex}
If there are any missing components in the \texttt{CASES} expression, a
\emph{cases \tcc}\index{cases TCC}\index{TCC!cases} will be generated
stating that the cotuple expression must be one of the given selections,
unless there is an \texttt{ELSE} selection.

Like the projection operators \texttt{PROJ\_$i$}, the \texttt{IN\_$i$},
\texttt{OUT\_$i$} and \texttt{IN?\_$i$} operators make be disambiguated by
adding the cotuple type reference to the operator, for example,
\texttt{IN\_2[int + int](3)} or \texttt{IN?\_1[coT]}.  Note that although
they have the form of actual parameters, they are not, as these operators
are built in and not associated with any theory.  Also, for brevity, only
the cotuple type is given, not the full type of the operator.  There are a
number of axioms associated with cotuples that are built in to the PVS
typechecker and prover.


\section{Override Expressions}
\index{override expression}
\index{update expression}
\index{with expression}

Functions, tuples, records, and datatype elements may be ``modified'' by
means of the override expression.  The result of an override expression is
a function, tuple, record, or datatype element that is exactly the same as
the original, except that at the specified arguments it takes the new
values.  For example,
\begin{pvsex}
  identity WITH [(0) := 1, (1) := 2]
\end{pvsex}
%
is the same function as the \texttt{identity} function (defined in the
prelude) except at argument values \texttt{0} and \texttt{1}.  This is exactly
the same expression as either of
\begin{pvsex}
  (identity WITH [(0) := 1]) WITH [(1) := 2] {\rm or}
  (LAMBDA x: IF x = 1 THEN 2 ELSIF x = 0 THEN 1 ELSE identity(x))
\end{pvsex}

This order of evaluation ensures that functions remain total, and allows
for the possibility of expressions such as
\begin{pvsex}
  identity WITH [(c) := 1, (d) := 2]
\end{pvsex}
where \texttt{c} and \texttt{d} may or may not be equal.  If they are
equal, then the value of the override expression at the common argument is
\texttt{2}.

More complex overrides can be made using nested arguments; for example,
\begin{pvsex}
  R: TYPE = [# a: int, b: [int -> [int, int]] #]
  r1: R
  r2: R = r1 WITH [`a := 0, `b(1)`2 := 4]
\end{pvsex}
{\tt r2} is equivalent to
\begin{pvsex}
  (# a := 0,
     b := LAMBDA (x: int):
           IF x = 1
           THEN (r1`b(x)`1, 4)
           ELSE r2`b(x)
           ENDIF #)
\end{pvsex}

Updating a datatype element amounts to updating the accessor(s) associated
with a constructor.  For example, if \texttt{lst} is of type
\texttt{(cons?[nat])}, then \texttt{lst WITH [`car := 3]} returns a list
that is the same as \texttt{lst}, but whose first element is \texttt{3}.
If \texttt{lst} is given type \texttt{list[nat]}, then the same override
expression generates a \tcc\ obligation to prove that \texttt{lst} is a
\texttt{cons?}.  Because accessors may be both dependent and overloaded,
\tccs\ may get complicated.  For example,
\begin{pvsex}
  dt: DATATYPE
  BEGIN
   c0: c0?
   c1(x: int, a: \setb{}z: (even?) | z > x\sete, b: int): c1?
   c2(x: int, a: \setb{}n: nat | n > x\sete, c: int): c2?
  END dt
\end{pvsex}
If \texttt{d} is of type \texttt{dt}, the update expression \texttt{d WITH
[a := y]} leads to the \tcc
\begin{pvsex}
  f1_TCC1: OBLIGATION
    (c1?(d) AND even?(y) AND y > x(d)) OR
     (c2?(d) AND y >= 0 AND y > x(d));
\end{pvsex}

Another form of override expression is the maplet, indicated using
\texttt{|->} in place of \texttt{:=}.  This is used to extend the domain
of the corresponding element; for example, if \texttt{f:[nat -> int]} is
given, then \texttt{f WITH [(-1) |-> 0]} is a function of type
\texttt{[\setb{}i:int | i >= 0 OR i = -1\sete -> int]}.  This is especially useful
with dependent types, see Section~\ref{dependent-types}.  Domain extension
is also possible for record and tuple types; for example, \texttt{r1 WITH
[`c |-> 3]} is of type \texttt{[\# a:\ int, b:\ [int -> [int,int]], c:\ int
\#]}, and if \texttt{t1} is of type \texttt{[int, bool]}, then \texttt{t1
WITH [`3 |-> 1]} is of type \texttt{[int, bool, int]}.  It is an error to
extend a tuple type such that gaps are left, so \texttt{t1 WITH [`4 |->
1]} is illegal, though \texttt{t1 WITH [`3 |-> 1, `4 |-> 1]} is allowed.
Gaps would also be left if nested arguments were given, so \texttt{r1 WITH
[`c(0) |-> 0]} is also illegal.  It would have to be given as \texttt{r1
WITH [`c := LAMBDA (x:\ int):\ IF x = 0 THEN 0 ELSE $\cdots$ ENDIF]}, where
the gap $\cdots$ now has to be filled in.  Domain extension is not
possible for datatype elements, as a new datatype theory would need to be
generated for each such extension.

In the past, the two forms of assignment (using \texttt{:=} and
\texttt{|->}) were merely alternative notation, and domains would be
extended automatically whenever the typechecker could not determine that
the argument belonged to the domain.  In most cases, extending the domain
unnecessarily is harmless.  However, when terms get large, the types can
get cumbersome, slowing down the system dramatically.  Even worse, when
domains are extended and matched against a rewrite rule with the original
type, the match can fail, and the automatic rewrite will not be triggered.
For this reason, it is always best to use the maplet on function types
only when actually extending the domain.

\section{Coercion Expressions}\label{coercions}

Coercion expressions are of the form \texttt{expr ::\ type-expr}, indicating
that the expression \texttt{expr} is expected to be of type \texttt{
type-expr}.  This serves two purposes.  First, although PVS allows a
liberal amount of overloading, it cannot always disambiguate things for
itself, and coercion may be needed.  For example, in
\begin{pvsex}
  foo: int
  foo: [int -> int]
  foo: LEMMA foo = foo::int
\end{pvsex}
%
the coercion of \texttt{foo} to \texttt{int} is needed, because otherwise the
typechecker cannot determine the type.  Note that only one of the sides
of the equation needs to be disambiguated.

The second purpose of coercion is as an aid to typechecking; by
providing the expected type in key places within complex expressions,
the resulting \tccs\ may be considerably simplified.

\input{tables}

\index{expression|)}

% Document Type: LaTeX
% Master File: language.tex

\chapter{Theories}\label{theories}
\index{theories}

Specifications in \pvs\ are built from \emph{theories}, which provide
genericity, reusability, and structuring.  \pvs\ theories may be
parameterized.  A theory consists of a \emph{theory
identifier}, a list of formal \emph{parameters}, an \texttt{EXPORTING}
clause, an \emph{assuming part}, a \emph{theory body}, and an ending
id.  The syntax for theories is shown in Figure~\ref{bnf-theory}.

\pvsbnf{bnf-theory}{Theory Syntax}

Everything is optional except the identifiers and the keywords.  Thus
the simplest theory has the form
\begin{pvsex}
  triv : THEORY
    BEGIN
    END triv
\end{pvsex}

The formal parameters, assuming, and theory body consist of declarations
and \emph{importings}.  The various declarations are described in
Section~\ref{declarations}.  In this section we discuss the restrictions
on the allowable declarations within each section, the formal parameters,
the assuming part, and the exportings and importings.

The \texttt{groups} theory below illustrates these concepts.  It views a
group as a 4-tuple consisting of a type \texttt{G}, an identity element
\texttt{e} of \texttt{G}, and operations \texttt{o}\footnote{Recall that
\texttt{o} is an infix operator.} and \texttt{inv}.  Note the use of the
type parameter \texttt{G} in the rest of the formal parameter list.  The
assuming part provides the group axioms.  Any use of the \texttt{groups}
theory incurs the obligation to prove all of the \texttt{ASSUMPTION}s.
The body of the \texttt{groups} theory consists of two theorems, which can
be proved from the assumptions.

\pvstheory{groups-alltt}{Theory \texttt{groups}}{groups-alltt}

\section{Theory Identifiers}

The theory identifier introduces a name for a theory; as described in
Section~\ref{names}, this identifier can be used to help disambiguate
references to declarations of the theory.

In the \pvs\ system, the set of theories currently available to the
session form a \emph{context}.  Within the context theory names must be
unique.  There is an initial context available, called the prelude
(described in Appendix~\ref{prelude}), that provides, among other things,
the Boolean operators, equality, and the \texttt{real}, \texttt{rational},
\texttt{integer}, and \texttt{naturalnumber} types and their associated
properties.  The only difference between the prelude and user-defined
theories is that the prelude is automatically imported in every theory,
without requiring an explicit \rsv{IMPORTING} clause.

The end identifier must match the theory identifier, or an error is
signaled.


\section{Theory Parameters}\label{parameters}
\index{theory parameters|(}
\index{formal parameters|see{theory parameters}}

The theory parameters allow theory schemas to be specified.  This
provides support for \emph{universal polymorphism}\index{polymorphism}

Theory parameters may be types, subtypes, or constants, and importings may
be interspersed.  Theory parameters must have unique identifiers.  The
parameters are ordered, allowing later parameters to refer to earlier
ones.  This is another form of dependency, akin to dependent types (see
Section~\ref{dependent-types}).  A theory is \emph{ instantiated} from
within another theory by providing \emph{actual parameters}\index{actual
parameters} to substitute for the formals.  Actual parameters may occur in
importings, exportings, theory declarations, and names.  In each case they
are enclosed in braces (\texttt{[} and \texttt{]}) and separated with
commas.

The actuals must match the formals in number, kind, and (where
applicable) type.  In this matching process the importings, which
must be enclosed in parentheses, are ignored.  For example, given the
theory declaration
\begin{pvsex}
  T [t: TYPE,
     subt: TYPE FROM t
     (IMPORTING orders[subt]) <=: (partial_order?),
     c: subt,
     d: \setb{}x:subt | c <= x\sete]
\end{pvsex}
a valid instance has five actual parameters; an example is
\begin{pvsex}
  T[int, \setb{}x:nat | x < 10\sete, <=, 5, 6]
\end{pvsex}
%
Note that the matching process may lead to the generation of \emph{actual}
\tccs.\index{actual TCC}\index{TCC!actual}

\index{theory parameters|)}


\section{Importings and Exportings}\label{importings}

The importing and exporting clauses form a hierarchy, much like the
subroutine hierarchy of a programming language.

Names declared in a theory may be made available to other theories in the
same context by means of the \texttt{EXPORTING} clause.  Names exported by
a given theory may be imported into a second theory by means of the
\texttt{IMPORTING} clause.  Names that are exported from one theory are
said to be \emph{visible} to any theory which uses the given theory.  In
this section we describe the syntax of the \texttt{EXPORTING} and
\texttt{IMPORTING} clauses and give some simple examples.

\pvsbnf{bnf-exporting}{Importing and Exporting Syntax}


\subsection{The \texttt{EXPORTING} Clause}
\index{exporting@\texttt{EXPORTING}|(}

The \texttt{EXPORTING} clause specifies the names declared in the theory
which are to be made available to any theory \texttt{IMPORTING} it.  It may
also specify instances of the theories which it imported to be exported.
The syntax of the \texttt{EXPORTING} clause is given in
Figure~\ref{bnf-exporting}.

\noindent
The \texttt{EXPORTING} clause is optional; if omitted, it defaults to
\begin{alltt}
  EXPORTING ALL WITH ALL
\end{alltt}

Any declared name may be exported except for variable declarations and formal parameters.
When \texttt{ALL} is specified for the \emph{ExportingNames}, all
entities declared in the theory aside from the variables are exported.
If a list of names is specified, then these are exported.  Finally, when
a list of names follows \texttt{ALL BUT}, all names aside from these are exported.

Since PVS supports overloading, it is possible that the exported name will
be ambiguous.  Such names may be disambiguated by including the type, if
it is a constant, or by including one of the keywords \texttt{TYPE} or
\texttt{FORMULA}.  The keyword \texttt{TYPE} is used for any type
declaration, and \texttt{FORMULA} is used for any formula declaration
(including \texttt{AXIOM}s, \texttt{LEMMA}s, etc.)  If not disambiguated,
all declarations (except variables and formals) with the specified id will
be exported.

When names are specified they are checked for \emph{completeness}.
This means that when a name is exported all of the names on which the
corresponding declaration(s) depend must also be exported.  Thus, for
example, given the following declarations
\begin{alltt}
  sometype: TYPE
  someconst: sometype
\end{alltt}
it would be illegal to export \texttt{someconst} without also exporting
\texttt{sometype}.  Note that this check is unnecessary if exporting
\texttt{ALL} without \texttt{BUT}.

In some cases it is desirable (or necessary for completeness) to
export some of the instances of the theories which are used by the
given theory.  This is done by specifying a \texttt{WITH} subclause as a
part of the \texttt{EXPORTING} clause.  The \texttt{WITH} subclause may be
\texttt{ALL}, indicating that all instances of theories used by the given
theory are exported.  If \texttt{CLOSURE} is specified, then the
typechecker determines the instances to be exported by a \emph{
completion analysis}\index{completion analysis} on the exported
names.  Completion analysis determines those entities that are
directly or indirectly referenced by one of the exported
names.\footnote{Proofs are not used in completion analysis.} Finally,
a list of theory names may be given; in this case the theory names
must be complete in the sense that if an exported name refers to an
entity in another theory instance, then that theory instance must be
exported also.  Other theory instances may also be exported even if
not actually needed for completeness in this sense.  The \texttt{WITH}
subclause may only reference theory instances, \ie\ theory names with
actuals provided for all of the corresponding formal parameters.

As a practical matter, it is probably best not to include an
\texttt{EXPORTING} clause unless there is a good reason.  That way
everything that is declared will be visible at higher levels of the
\texttt{IMPORTING} chain.

\index{exporting@\texttt{EXPORTING}|)}

\subsection{\texttt{IMPORTING} Clauses}
\index{importings|(}

\texttt{IMPORTING} clauses import the visible names of another theory.
\texttt{IMPORTING} clauses may appear in the formal parameters list, the
assuming part, or the theory part of a theory.  In addition, theory
abbreviations implicitly import the theory name that they abbreviate (see
Section~\ref{theory-abbreviations}).

The names appearing in an \texttt{IMPORTING} or theory abbreviation specifies a
theory and optionally gives an instance of that theory, by providing
actual parameters corresponding to the formal parameters of the theory
used.  \texttt{IMPORTING}s are cumulative; entities made visible at some point
in a theory are visible to every declaration following.

An \texttt{IMPORTING} with actual parameters provided is said to be a \emph{
theory instance}.\index{theory instance} We use the same terminology for
an \texttt{IMPORTING} that refers to theory that has no formal parameters.
Otherwise it is referred to as a \emph{generic}\index{generic reference}
reference.

A single theory may appear in any number of \texttt{IMPORTING}s of another
theory, both instantiated and generic.  Obviously, any time there is
more than one \texttt{IMPORTING} of a given theory there is a chance for
ambiguity.  Section~\ref{names} discusses such ambiguities, explaining
how the system attempts to resolve them and how the user can
disambiguate in situations where the system cannot.

An \texttt{IMPORTING} forms a relation between the theory containing the
\texttt{IMPORTING} and the theory referenced.  The transitive closure of
the \texttt{IMPORTING} relation is called the \emph{importing chain} of a
theory.  The importing chain must form a directed acyclic graph; hence a
theory may not end up importing itself, directly or indirectly.
\index{importings|)}


\section{Theory Abbreviations}\label{theory-abbreviations}
\index{theory abbreviations}

A theory abbreviation introduces a new name for a theory instance,
providing an alternate means for referring to the instance.  For
example, given the declaration
\begin{pvsex}
  fsets: THEORY = sets[[integer -> integer]]
\end{pvsex}
where \texttt{sets} is a theory in which the function \texttt{member} is
declared, the name \texttt{sets[[integer -> integer]].member} may instead
be written as \texttt{fsets.member}.  The body of a theory abbreviation
must refer to a theory of the current context or the prelude, and the
actual parameters must be compatible (in number, kind, and type) with
the corresponding theory parameters\index{theory parameters}.

In addition to providing an abbreviation, such declarations do an
implicit \texttt{IMPORTING}\index{IMPORTING@\texttt{IMPORTING}} of the theory instance.
Theory abbreviations may not be exported.


\section{Assuming Part}\label{assuming}

The assuming part consists of top-level declarations and \texttt{IMPORTING}s.
The assuming part precedes the theory part, so the
theory part may refer to entities declared in the assuming part.  The
grammar for the assuming part is given in Figure~\ref{bnf-assuming}.

\pvsbnf{bnf-assuming}{Assuming Syntax}

The primary purpose of the assuming part is to provide constraints on the
use of the theory, by means of \texttt{ASSUMPTION}s.  These are formulas
expressing properties that are expected to hold of any instance of the
theory.  They are generally stated in terms of the formal parameters, and
when instantiated they become \emph{assuming} \tccs\ \index{assuming
TCC}\index{TCC!assuming} \texttt{OBLIGATION}s which must be discharged.  For
example, given the theory \texttt{groups} above, the importing
\begin{pvsex}
  IMPORTING groups[int, 0, +, -]
\end{pvsex}
generates the following obligations
\begin{pvsex}
  % Assuming TCC generated for  groups[int, 0, +, -]
  IMPORTING1_TCC1: OBLIGATION
    (FORALL (a: int), (b: int), (c: int):
         +(a, (+(b, c))) = +((+(a, b)), c))
  
  % Assuming TCC generated for  groups[int, 0, +, -]
  IMPORTING1_TCC2: OBLIGATION (FORALL (a: int):
         +(0, a) = a AND +(a, 0) = a)
  
  % Assuming TCC generated for  groups[int, 0, +, -]
  IMPORTING1_TCC3: OBLIGATION (FORALL (a: int):
         +(-(a), a) = 0 AND +(a, -(a)) = 0)
\end{pvsex}

Except for the variable declarations, the declarations of the assumings
are all externally visible.  
  
The dynamic semantics of an \emph{assuming} part of a theory is as
follows.  Internal to the theory, assumptions are used exactly as axioms
would be used.  Externally, for each import of a theory, the assumptions
have to be discharged (i.e., proved) with the actual parameters replacing
the formal parameters.  Note that in terms of the proof chain, every proof
in a theory depends on the proofs of the assumptions.

Assuming \tccs\ are generated when a theory is instantiated, which may or
may not occur when it is imported.  Thus if a theory with assumptions is
imported generically, the assuming \tccs\ are not generated until some
reference is instantiated.  If a theory instance is imported, then the
assuming \tccs\ precede the importing in the dynamic semantics.  Note that
this may not make sense, as the assumings may refer to entities that are
not visible until after the theory is imported.  Thus the following is
illegal.
\begin{pvsex}
  assuming_test[n: nat, m: {x:int | x < n}]: THEORY
  BEGIN
   ASSUMING
    rel_prime?(x, y: int): bool = EXISTS (a, b: int): x*a + y*b = 1
    rel_prime: ASSUMPTION rel_prime?(n,m)
   ENDASSUMING
  END assuming_test

  assimp: THEORY
  BEGIN
   IMPORTING assuming_test[4, 2]
  END assimp
\end{pvsex}
And leads to the following error message.
\begin{pvsex}
  Error: assumption refers to 
    assuming_test[4, 2].rel_prime?,
  which is not visible in the current theory
\end{pvsex}
There are a number of ways to solve this problem.  Perhaps the simplest is
to first import the theory generically, then import the instance.
\begin{pvsex}
   IMPORTING assuming_test
   IMPORTING assuming_test[4, 2]
\end{pvsex}
Now the reference to \texttt{rel\_prime?} makes sense in the assuming
\tcc\ generated for the second importing.

In this case, another solution is to simply define \texttt{rel\_prime?} as
a \emph{macro} (see Section~\ref{macro-declarations}).
\begin{pvsex}
  rel_prime?(x, y: int): MACRO bool = EXISTS (a, b: int): x*a + y*b = 1
\end{pvsex}
Of course, this will not work if the declaration in question is a
recursive or inductive definition.

Another solution is to provide the declaration in a theory that is
imported in both the theory with the assuming and the theory importing
that theory.
\begin{pvsex}
  rel_prime[y:int]: THEORY
  BEGIN
   rel_prime?(x: int): bool = EXISTS (a, b: int): x*a + y*b = 1
  END assth2

  assuming_test[n: nat, m: {x:int | x < n}]: THEORY
  BEGIN
   ASSUMING
    IMPORTING rel_prime[m]
    rel_prime: ASSUMPTION rel_prime?(n)
   ENDASSUMING
  END assuming_test2

  assuming_imp: THEORY
  BEGIN
   IMPORTING rel_prime[2], assuming_test[4, 2]
  END assuming_imp
\end{pvsex}
Now the reference to \texttt{rel\_prime?} in the assuming \tcc\ associated
with \texttt{assuming\_test[4, 2]} is the same as the previously imported
instance, so there is no problem.  In the theory \texttt{assuming\_imp},
\texttt{rel\_prime} may also be imported generically.  However, if
\texttt{rel\_prime} is not imported, or is imported with a different
parameter (e.g., \texttt{rel\_prime[3]}) then the above error is produced.


\section{Theory Part}

The theory part consists of top-level declarations and \texttt{IMPORTING}s.
Declarations are ordered; references may not be made to declarations
which occur later in the theory.  The theory part usually contains the
main body of the theory.  Assuming declarations are not allowed in the
theory part.  The grammar for the theory part is given in
Figure~\ref{bnf-theory-part}.

\pvsbnf{bnf-theory-part}{Theory Part Syntax}

% Document Type: LaTeX
% Master File: interpretations-final.tex
\documentclass[11pt,twoside,openright,titlepage]{cslreport}
\usepackage{relsize}
\usepackage{makebnf}
\usepackage{alltt}
%\usepackage{doublespace}

\usepackage{cite}
\usepackage{/homes/rushby/tex/oz}
%\usepackage{/homes/rushby/tex/cslrep}
\usepackage{url}
\usepackage{psfig}
\usepackage{times}
\usepackage{/homes/owre/tex/session}
\usepackage{boxedminipage}
\def\mapb{\char"7B\char"7B}
\def\mape{\char"7D\char"7D}
\def\setb{\char"7B}
\def\sete{\char"7D}
\newcommand{\specware}{{\sc Specware}}
\textwidth 5.5in
\oddsidemargin .65in
\evensidemargin .41in
\raggedbottom
\sloppy


\begin{document}
\begin{titlepage}
\title{\textbf{\larger Theory Interpretations in PVS}}
\author{Sam Owre \and N. Shankar
\date{April 2001}
\cslreportnumber{SRI-CSL-01-01}
\maketitle
\noindent
%\hspace*{-1in}
\raisebox{-0.8cm}[1cm][1cm]{\srilogo}
\acknowledge{Funded by NASA Langley Research Center contract numbers
NAS1-20334 and NAS1-0079 and DARPA/AFRL contract number F33615-00-C-3043.}
\end{titlepage}

\cleardoublepageblank
\pagenumbering{roman}

\begin{abstract}
\thispagestyle{plain}

We describe a mechanism for theory interpretations in PVS.  The
mechanization makes it possible to show that one collection of theories is
correctly interpreted by another collection of theories under a
user-specified interpretation for the uninterpreted types and constants.
A theory instance is generated and imported, while the axiom instances are
generated as proof obligations to ensure that the interpretation is valid.
Interpretations can be used to show that an implementation is a correct
refinement of a specification, that an axiomatically defined specification
is consistent, or that a axiomatically defined specification captures its
intended models.

In addition, the theory parameter mechanism has been extended with a
notion of \emph{theory as parameter} so that a theory instance can be
given as an actual parameter to an imported theory.  Theory
interpretations can thus be used to refine an abstract specification or to
demonstrate the consistency of an axiomatic theory.  In this report we
describe the mechanism in detail.  This extension is a part of PVS version
3.0, which will be publicly released in mid-2001.

\end{abstract}

\tableofcontents
\cleardoublepage
\setcounter{page}{0} 
\pagenumbering{arabic}

\chapter{Introduction}

Theory interpretations have a long history in first-order
logic~\cite{Shoenfield,Enderton,Monk76}\@.  They are used to show that the
language of a given source theory $S$ can be interpreted within a target
theory $T$ such that the corresponding interpretation of axioms of $S$
become theorems of $T$\@.  This demonstrates the consistency of $S$
relative to $T$, and also the decidability of $S$ modulo the decidability
of $T$\@.  Theories and theory interpretations have also become important
in higher-order logic and type theory with languages such as {\sc
Ehdm}~\cite{EHDM:manuals}, IMPS~\cite{Farmer:interpretations},
HOL~\cite{Windley92}, Maude~\cite{Maude}, Extended
ML~\cite{SannellaDT:essential-concepts97}, and
\specware~\cite{SrinivasJullig95}\@.  In these languages, theories are
used as structuring mechanisms for large specifications so that abstract
theories can be refined into more concrete ones through interpretation.
In this report, we describe a theory interpretation mechanism for the PVS
specification language.

Specification languages and programming languages usually have some
mechanism for packaging groups of definitions into modules.  Lisp and Ada
have \emph{packages}\@.  Standard ML has a module system consisting of
signatures, structures corresponding to a signature, and functors that map
between structures.  Packages can be made generic by allowing certain
declarations to serve as parameters that can be instantiated when the
package is imported.  Ada has \emph{generic} packages that allow
parameters.  SML \emph{functors} can be used to construct parametric
modules.  C++ allows \emph{templates}.

In specification languages, a \emph{theory} groups together related
declarations of constants, types, axioms, definitions, and theorems.  One
way of demonstrating the consistency of such a theory is by providing an
interpretation for the uninterpreted types and constants under which the
axioms are valid.  The definitions and theorems corresponding to a valid
interpretation can then be taken as valid without further proof as long as
they have been verified in the source theory.  The technique of
interpreting one axiomatic theory in another has many uses.  It can be
used to demonstrate the consistency or decidability of the former theory
with respect to the latter theory.  It can also be used to refine an
abstract theory down to an executable implementation.

Interpretations are also useful in showing that the axioms capture the
intended models.  For example, a clock synchronization algorithm was
developed in \textsc{Ehdm} and was later shown to be consistent using the
mappings, but it turned out that in one place $<$ was used instead of
$\leq$, and because of this a set of perfectly synchronized clocks was
actually disallowed by the model.  Using interpretations in this way is
similar to testing in allowing for the exploration of the space of models
for the theory.

Parametric theories in PVS share some of the features of theory
interpretations.  Such theories can be defined with formal parameters
ranging over types and individuals, for example,\footnote{Note that the
number 0 here is overloaded, and treated as an identifier.}
{\smaller\begin{alltt}
group[G: TYPE, + : [G, G -> G], 0: G, -: [G -> G]]: THEORY
  BEGIN
    \vdots
  END group
\end{alltt}}
An instance of the theory \texttt{group} can be imported by supplying
actual parameters, the type \texttt{int} of integers, integer addition
{\tt +}, zero \texttt{0}, and integer negation {\tt -}, corresponding to
the formal parameters, as in {\tt group[int, +, 0, -]}\@.  A theory can
include assumptions about the parameters that have to be discharged when
the actual parameters are supplied.  For example, the group axioms can
be given as assumptions in the \texttt{group} theory above.  However,
there are some crucial differences between parametric theories and theory
interpretations.  In particular, if axioms are always specified as
assumptions, then the theory can be imported only by discharging these
assumptions.  It is necessary to have separate mechanisms for importing a
theory with the axioms, and for interpreting a theory by supplying a valid
interpretation, that is, one that satisfies its axioms.

The PVS theory interpretation mechanism is quite similar to that for
theory parameterization.  The axiomatic specification of groups could
alternately be given in a theory
{\smaller\begin{alltt}
group: THEORY
 BEGIN
  G: TYPE+
  +: [G, G -> G]
  0: G
  -: [G -> G]
   \vdots
 END group
\end{alltt}}
The group axioms are declared in the body of the theory.  Such a theory
can be interpreted by writing \texttt{\smaller group\mapb{}G := int, + :=
+, 0 := 0, - := -\mape{}}\@.  Here the left-hand sides refer to the
uninterpreted types and constants of theory \texttt{group}, and the
right-hand sides are the interpretations.  This notation resembles that of
theory parameterization and is used in contexts where a theory is
imported.  The corresponding instances of the group axioms are generated
as proof obligations at the point where the theory is imported.  The
result is a theory that consists of the corresponding mapping of the
remaining declarations in the theory \texttt{group}\@.  This allows the
theory \texttt{group} to be used in other theories, such as rings and
fields, and also allows the theory \texttt{group} to be suitably
instantiated by group structures.

Theory interpretations largely subsume parametric theories in the sense
that the theory parameters and the corresponding assumings can instead be
presented as uninterpreted types and constants and axioms so that the
actual parameters are given by means of an interpretation.  However, a
parametric theory with both assumings and axioms involving the parameters
is not equivalent to any interpreted theory, as the parameters may be
instantiated without the need to prove the axioms.  It is also useful to
have parametric theories as a convenient way of grouping together all the
parameters that must be provided whenever the theory is used.  For
example, typical theory parameters such as the size of an array, or the
element type of an aggregate datatype such as an array, list, or tree, are
required as inputs whenever the corresponding theories are used.  While
this kind of parameterization can be captured by theory interpretations,
it would not capture the intent that these parameters are \emph{required}
inputs wherever the theory is used.  Furthermore, when an operation from a
parametric theory is used, PVS attempts to figure out the actual
parameters based on the context of its use.  It can do this because the
formal parameters are precisely delimited.  The corresponding inference is
harder for theory interpretations since there might be many possible
interpretations that are compatible with the context of the operations
use.

In addition to the uninterpreted types and constants in a source theory
$S$, the PVS theory interpretation mechanism can also be used to interpret
any theories that are imported into $S$ by means of the \texttt{THEORY}
declaration.  The interpretation of a theory declaration for $S'$ imported
within $S$ must itself be a theory interpretation of $S'$\@.  Two distinct
importations of a theory $S'$ within $S$ can be given distinct
interpretations.  A typical situation is when two theories $R_1$ and $R_2$
both import a theory $S$ as $S_1$ and $S_2$, respectively.  A theory $T$
importing both $R_1$ and $R_2$ might wish to identify $S_1$ and $S_2$
since, otherwise, these would be regarded as distinct within $T$\@.  This
can be done by importing an instance $S'$ of $S$ into $T$ and importing
$R_1$ with $S_1$ interpreted by $S'$ and $R_2$ with $S_2$ interpreted as
$S'$\@.  With theory interpretations, we have also extended parametric
theories in PVS to take theories as parameters.  For example, we might
have a theory \texttt{group\_homomorphism} of group homomorphisms that
takes two groups \texttt{G1} and \texttt{G2} as parameters as in the
declaration
\begin{alltt}
 group_homomorphism[G1, G2: THEORY group]: THEORY \ldots
\end{alltt}
The actual parameters for these theory formals must be
interpretations \texttt{G1'} and \texttt{G2'}
of the theory \texttt{group}\@.

Another typical requirement in a theory interpretation mechanism is the
ability to map a source type to some quotient with respect to an
equivalence relation over a target type.  For example, rational numbers
can be interpreted by means of a pair of integers corresponding to the
numerator and denominator, but the same rational number can have multiple
such representations.  We show how it is possible to define quotient types
in PVS and use these types to capture interpretations where the equality
over a source type is mapped to an equivalence relation over a target
type.

The implementation of theory interpretation in PVS is described in the
following chapters.  This report assumes the reader is already familiar
with the PVS language; for details see the PVS Language
Manual~\cite{PVS:language}.  Chapter 2 deals with mappings, explaining the
basic concepts and introduces the grammar.  Chapter 3 introduces theory
declarations and theories as parameters which allow any valid
interpretation of the formal parameter theory as an actual parameter.
Chapter 4 describes a new command for viewing theory instances.  Chapter 5
compares PVS interpretations with other systems, Chapter 6 describes
future work, and we conclude with Chapter 7.


\chapter{Mappings}\label{mappings}

Theory interpretations in PVS provide mappings for uninterpreted types and
constants of the \emph{source} theory into the current
(\emph{interpreting}) theory.  Applying a mapping to a source theory
yields an \emph{interpretation} (or \emph{target}) theory.  A mapping is
specified by means of the \emph{mapping} construct, which associates
uninterpreted entities of the source theory with expressions of the target
theory.  The mapping construct is an extension to the PVS notion of
``name''.  The changes to the existing grammar are given in
Figure~\ref{mapping-bnf}.

\begin{figure}
\setlength{\sessionboxwidth}{\linewidth}
\addtolength{\sessionboxwidth}{-\arrayrulewidth}
\addtolength{\sessionboxwidth}{-\tabcolsep}
\begin{boxedminipage}[b]{\sessionboxwidth}
\begin{bnf}

\production{TheoryName}
{\opt{Id \lit{@}} Id \opt{Actuals} \opt{Mappings}}

\production{Name}
{\opt{Id \lit{@}} IdOp \opt{Actuals} \opt{Mappings} \opt{\lit{.} IdOp}}

\production{Mappings}
{\lit{\mapb{}} \ites{Mapping}{,} \lit{\mape{}}}

\production{Mapping}
{MappingLhs MappingRhs}

\production{MappingLhs}
{IdOp \rep{Bindings} \opt{\lit{:} \brc{\lit{TYPE} \choice \lit{THEORY}
\choice TypeExpr}}}

\production{MappingRhs}
{\lit{:=} \brc{Expr \choice TypeExpr}}

\end{bnf}
\end{boxedminipage}
\caption{Grammar for Names with Mappings}\label{mapping-bnf}
\end{figure}

The mapping construct defines the basic translation, but to be a theory
interpretation the mapping must be consistent: if type \texttt{T} is
mapped to the type expression \emph{E}, then a constant \texttt{t} of type
\texttt{T} must be mapped to an expression \emph{e} of type \emph{E}.  In
addition, all axioms and theorems of the source theory must be shown to
hold in the target theory under the mapping.  Since the theorems are
provable from the axioms, it is enough to show that the translation of the
axioms hold.  Axioms whose translations do not involve any
uninterpreted types or constants of the source theory are converted to
proof obligations.  Otherwise they remain axioms.

Theory interpretation may be viewed as an extension of theory
parameterization.  Given a theory named \texttt{T}, the instance
\texttt{T[a$_1$,\ldots,a$_n$]\mapb{}c$_1$:= e$_1$,\ldots,c$_m$:=
e$_m$\mape{}} is the same as the original theory, with the \emph{actuals}
\texttt{a$_i$} substituted for the corresponding formal parameters, and
e$_i$ substituted for \texttt{c$_i$}, which must be an uninterpreted type
or constant declaration.  Declarations that appear in the target of a
substitution in the mapping are not visible in the importing theory.  Some
axioms are translated to proof obligations.  The substituted forms of any
remaining axioms, definitions, and lemmas are available for use, and are
considered proved if they are proved in the uninterpreted theory.

The following simple example illustrates the
basic concepts.
\begin{session}
th1[T: TYPE, e: T]: THEORY
 BEGIN
  t: TYPE+
  c: t
  f: [t -> T]
  ax: AXIOM EXISTS (x, y: t): f(x) /= f(y)
  lem1: LEMMA EXISTS (x:T): x /= e
 END th1
\end{session}
\begin{session}
th2: THEORY
 BEGIN
  IMPORTING th1[int, 0]
               \mapb{} t := bool,
                  c := true,
                  f(x: bool) := IF x THEN 1 ELSE 0 ENDIF \mape{}
  lem2: LEMMA EXISTS (x:int): x /= 0
 END th2
\end{session}
\noindent Here theory \texttt{th1} has both actual parameters and
uninterpreted types and constants, as well as an axiom and
a lemma.  Theory \texttt{th2} imports \texttt{th1}, making the
following substitutions:
\setlength{\jot}{-2pt}
\setlength{\abovedisplayskip}{0pt}
\setlength{\belowdisplayskip}{0pt}
{\smaller\begin{eqnarray*}
\texttt{T} & \leftarrow & \texttt{int} \\
\texttt{e} & \leftarrow & \texttt{0} \\
\texttt{t} & \leftarrow & \texttt{bool} \\
\texttt{c} & \leftarrow & \texttt{true} \\
\texttt{f} & \leftarrow & \texttt{LAMBDA (x:\ bool):\ IF x THEN 1 ELSE 0 ENDIF} \\
\end{eqnarray*}}
Note that the mapping for \texttt{f} uses an abbreviated form of
substitution.  Typechecking this leads to the following proof obligation.
\begin{session}
IMP_th1_ax_TCC1: OBLIGATION
  EXISTS (x, y: bool):
    IF x THEN 1 ELSE 0 ENDIF /= IF y THEN 1 ELSE 0 ENDIF;
\end{session}
This is simply the interpretation of the \texttt{ax} axiom and is easily
proved.  The lemma \texttt{lem1} can be proved from the axiom, and may
be used directly in proving \texttt{lem2} using the proof command
\texttt{(LEMMA "lem1")}.

Note that once the TCC has been proved, we know that \texttt{th1} is
consistent.  If we had left out the mapping for \texttt{f}, then the TCC
would not be generated, and the translation of theory \texttt{th1} would
still contain an axiom and not necessarily be consistent.

% Note that we used a lambda form in the axiom,
% rather than \texttt{f}.  This is because logically the generated proof
% obligation precedes the importing, which is only meaningful if the
% obligation is provable.  Hence \texttt{f} is not visible in the proof
% obligation and should not appear in any axiom of the theory.\footnote{We
% may allow this in future versions of PVS by automatically expanding
% non-recursive definitions as a part of substitution, treating them as a
% kind of macro.}  After the importing, of course, \texttt{f} is visible as
% seen in \texttt{lem2}.

% Note that mappings make theory parameters optional---they may be
% eliminated by moving the formal parameters to the theory body and turning
% assumptions into axioms.  The theory could then be instantiated using
% mappings instead of actual parameters.  Theory parameters are still
% useful, however.  First, they may be used to distinguish between the
% parameters to the system being specified and the entities defined by the
% system.  For example, in describing a protocol that works for any number
% of processes, it is more natural to make the number of processes a formal
% parameter rather than an uninterpreted constant.  Second, the
% typechecker can frequently infer the values of the actual parameters when
% a theory is imported generically, but mappings must be explicitly given.
% Although in principle the typechecker might be extended to infer mappings,
% it is hard to see how to do this efficiently.

One advantage to using mappings instead of parameters is that not all
uninterpreted entities need be mapped, whereas for parameters either all
or none must be given.  For example, consider the following theory.
\begin{session}
example1[T: TYPE, c: T]: THEORY
 BEGIN
  f(x: T): int = IF x = c THEN 0 ELSE 1 ENDIF
 END example1
\end{session}
\noindent It may be desirable to import this where \texttt{T} is always
\texttt{real}, and \texttt{c} is left as a parameter, but there is
currently no mechanism for this.  One could envision partial importings
such as \texttt{IMPORTING example1[real, \_]}, but it is not clear that
this is actually practical---in particular, the syntax for providing the
missing parameters is not obvious.  With mappings, on the other hand, we
can define \texttt{example1} as follows.
\begin{session}
example1: THEORY
 BEGIN
  T: TYPE
  c: T
  f(x: T): int = IF x = c THEN 0 ELSE 1 ENDIF
 END example1
\end{session}
\noindent Then we can refer to this theory from another theory as in the
following.
\begin{session}
example2: THEORY
 BEGIN
  th: THEORY = example1\mapb{}T := real\mape{}
  frm: FORMULA f\mapb{}c := 3\mape{} = f
 END example2
\end{session}
\noindent The \texttt{th} theory declaration just instantiates \texttt{T},
leaving \texttt{c} uninterpreted.  The first reference to \texttt{f} maps
\texttt{c} to \texttt{3}, whereas the second reference leaves it
uninterpreted though it is still a \texttt{real}.  Note that formula
\texttt{frm} is unprovable, since the uninterpreted \texttt{c} from the
second reference may or may not be equal to \texttt{3}.

As described in the introduction, an important aspect of mappings is the
support for quotient types.  In \textsc{Ehdm} this was done by
interpreting equality, but in PVS we instead define a theory of
equivalence classes, and allow the user to map constants to equivalence
classes under congruences.  For example, the \texttt{stacks} datatype
might be implemented using an array as follows.
\begin{session}
stack[t:TYPE]: DATATYPE
 BEGIN
  empty: empty?
  push(top:t, pop: stack): nonempty?
 END stack
\end{session}
\begin{session}\label{cstack}
cstack[t: TYPE+]: THEORY
 BEGIN
  cstack: TYPE = [# size: nat, elems: [nat -> t] #]
  cempty?(s: cstack): bool = (s`size = 0)
  some_t: t = epsilon(LAMBDA (x:t): true)
  cempty: (cempty?) =
    (# size := 0,
       elems := LAMBDA (n: nat): some_t #)
  cnonempty?(s: cstack): bool = (s`size /= 0)
  ctop(s: (cnonempty?)): t = s`elems(s`size - 1)
  cpop(s: (cnonempty?)): cstack = s WITH [`size := s`size - 1]
  cpush(x: t)(s: cstack): (cnonempty?) =
    (# size := s`size + 1,
       elems := s`elems WITH [(s`size) := x] #)
  ce: equivalence[cstack] =
    LAMBDA (s1, s2: cstack):
     s1`size = s2`size AND
     FORALL (n: below(s1`size)): s1`elems(n) = s2`elems(n)

  estack: TYPE = Quotient(ce)
  CONVERSION+ EquivClass(ce), rep(ce), lift(ce)
  \ldots
\end{session}
\texttt{Quotient}, \texttt{EquivClass}, and \texttt{rep} are defined in
the prelude theory \texttt{QuotientDefinition}, shown here in part.
\begin{session}
QuotientDefinition[T : TYPE] : THEORY
BEGIN
  R : VAR set[[T, T]]
  S : VAR equivalence[T]
  x, y, z : VAR T

  EquivClass(R)(x) : set[T] = { z | R(x, z) }
  \ldots
  Quotient(S) : TYPE =
    { P : set[T] | EXISTS x : P = EquivClass(S)(x) }
  \ldots
  rep(S)(P: Quotient(S)): T = choose(P)
  \ldots
END QuotientDefinition
\end{session}
The \texttt{lift} function is defined in the prelude theory
\texttt{QuotientExtensionProperties} as follows.
\begin{session}
QuotientExtensionProperties[X, Y : TYPE] : THEORY
BEGIN
  S : VAR equivalence[X]

  lift(S)(g : (PreservesEq[X, Y](S)))(P : Quotient(S)) : Y
    = g(rep(S)(P))
  \ldots
END QuotientExtensionProperties
\end{session}
This allows functions on concrete stacks to be lifted to functions on
equivalence classes, so long they satisfy the \texttt{PreservesEq}
relation, i.e., they produce the same values on \texttt{S}-equivalent
elements.

With these conversions in place, we can finish the specification of
\texttt{cstack} as follows.
\begin{session}
  \ldots
  IMPORTING stack[t]\mapb{} stack := estack,
                       empty? := cempty?,
                       nonempty? := cnonempty?,
                       empty := cempty,
                       top(s: (cnonempty?)) := ctop(s),
                       pop(s: (cnonempty?)) := cpop(s),
                       push(x: t, s: cstack) := cpush(x)(s) \mape{}
 END cstack
\end{session}
\noindent Here the source type \texttt{stack} is mapped to the quotient
type \texttt{estack} defined by the concrete equality \texttt{ce}.  The
\texttt{empty?} and \texttt{nonempty?} predicates are mapped to predicates
on \texttt{estack}s, using the \texttt{rep(ce)} conversion.  The
\texttt{empty} constructor is then mapped to its equivalence class.

\texttt{top}, \texttt{pop},

The mapping for \texttt{push} is more involved; \texttt{cpush} must first
be lifted in order to apply it to the abstract stack \texttt{s}.  This is
applied automatically by the conversion mechanism of PVS.  The application
of \texttt{lift} generates the proof obligation that \texttt{cpush}
preserves the equivalences, that is, it is a congruence.  This mapping
generates a large number of proof obligations, because the \texttt{stack}
datatype generates a \texttt{stacks\_adt} theory with a large number of
axioms, for example, extensionality, well-foundedness, and induction.

The PVS interpretations mechanism is much simpler to implement than the
one in \textsc{Ehdm}---equality is not a special case, but simply an
aspect of mapping a type to an equivalence class.  The technique of
mapping types to equivalence classes is quite useful, and captures the
notion of behavioral equivalence outlined
in~\cite{SannellaDT:essential-concepts97}.  In fact it is more general, in
that it works for any equivalence relation, not just those based on
observable sorts.


\chapter{Theory Declarations}

With the mapping mechanism, it is easy to specify a general theory and
have it stand for any number of instances.  For example, groups, rings,
and fields are all structures that can be given axiomatically in terms of
uninterpreted types and constants.  This works well when considering one
such structure at a time, but it is difficult to specify theories that
involve more than one structure, for example, group homomorphisms.
Importing the original theory twice is the same as importing it once, and
an attempted definition of a homomorphism would turn into an automorphism.
In this case what is needed is a way to specify multiple different
``copies'' of the original theory.  This is accomplished with \emph{theory
declarations}, which may appear in either the theory parameters or the
body of a theory.  A theory declaration in the formal parameters is
referred to as a \emph{theory as parameter}.\footnote{The term
\emph{theory parameter} refers to a parameter of a theory, so we use the
term \emph{theory as parameter} instead.}  Theory declarations allow
theories to be encapsulated, and instantiated copies of the implicitly
imported theory are generated.
\begin{figure}[!b]
\setlength{\sessionboxwidth}{\linewidth}
\addtolength{\sessionboxwidth}{-\arrayrulewidth}
\addtolength{\sessionboxwidth}{-\tabcolsep}
\begin{boxedminipage}[b]{\sessionboxwidth}
\begin{bnf}\smaller

\production{TheoryFormalDecl}
{TheoryFormalType \choice TheoryFormalConst \choice TheoryDecl}

\production{TheoryDecl}
{Id \lit{:} \lit{THEORY} TheoryDeclName}

\production{TheoryDeclName}
{\opt{Id \lit{@}} Id \opt{Actuals} \opt{TheoryDeclMappings}}

\production{TheoryDeclMappings}
{\lit{\mapb{}} \ites{TheoryDeclMapping}{,} \lit{\mape{}}}

\production{TheoryDeclMapping}
{MappingLhs TheoryDeclMappingRhs}

\production{TheoryDeclMappingRhs}
{MappingSubst \choice MappingDef \choice MappingRename}

\production{MappingSubst}
{\lit{:=} \brc{Expr \choice TypeExpr}}

\production{MappingDef}
{\lit{=} \brc{Expr \choice TypeExpr}}

\production{MappingRename}
{\lit{::=} \brc{IdOp \choice Number}}

\end{bnf}
\end{boxedminipage}
\caption{Grammar for Theory Declarations}\label{theory-parameter-bnf}
\end{figure}

For example, an (additive) group is normally thought of as a 4-tuple
consisting of a set $G$, a binary operator $+$, an identity element $0$,
and an inverse operator $-$ that satisfies the usual group axioms.  Using
theory interpretations, we simply define this as follows:
\begin{session}
group: THEORY
 BEGIN
  G: TYPE+
  +: [G, G -> G]
  0: G
  -: [G -> G]
  x, y, z: VAR G
  associative_ax: AXIOM FORALL x, y, z: x + (y + z) = (x + y) + z
  identity_ax: AXIOM FORALL x: x + 0 = x
  inverse_ax: AXIOM FORALL x: x + -x = 0 AND -x + x = 0
  idempotent_is_identity: LEMMA x + x = x => x = 0
 END group
\end{session}

As described in Chapter~\ref{mappings}, we can use mappings to create
specific instances of groups.  For example, {\smaller\begin{alltt}
group\mapb{}G := int, + := +, 0 := 0, - := -\mape{}
\end{alltt}}
\noindent is the additive group of integers, whereas
{\smaller\begin{alltt}
group\mapb{}G := nzreal, + := *, 0 := 1, - := LAMBDA (r:nzreal):\ 1/r\mape{}
\end{alltt}}
\noindent is the multiplicative group of nonzero reals.

This works nicely, until we try to define the notion of a group
homomorphism.  At this point we need two groups, both individually
instantiable.  We could simply duplicate the group specification, but
this is obviously inelegant and error prone.  Using theories as
parameters, we may define group homomorphisms as follows.
\begin{session}
group_homomorphism[G1, G2: THEORY group]: THEORY
 BEGIN
  x, y: VAR G1.G
  f: VAR [G1.G -> G2.G]
  homomorphism?(f): bool = FORALL x, y: f(x + y) = f(x) + f(y)
  hom_exists: LEMMA EXISTS f: homomorphism?(f)
 END group_homomorphism
\end{session}
\noindent Here \texttt{G1} and \texttt{G2} are theories as parameters to a
generic homomorphism theory that may be instantiated with two different
groups.  Hence we may import \texttt{group\_homomorphism}, for example, as
\begin{session}
IMPORTING group_homomorphism[group\mapb{}G := int, + := +, 0 := 0, - := -\mape{}
                             group\mapb{}G := nzreal, + := *, 0 := 1,
                                 - := LAMBDA (x: nzreal): 1/x\mape{}]
\end{session}

There is a subtlety here that needs emphasizing; \texttt{G1} and
\texttt{G2} are two \emph{distinct} copies of theory \texttt{group}.
For example, consider the addition of the following lemma to
\texttt{group\_homomorphism}.
\begin{session}
oops: LEMMA G1.0 = G2.0
\end{session}
\noindent If \texttt{G1} and \texttt{G2} are treated as the same
\texttt{group} theory, this is a provable lemma.  But then after the
importing given above we would be able to show that \texttt{0 = 1}.  Even
worse, the two different instances of groups may not even be type
compatible, so the \texttt{oops} lemma should not even typecheck.

We have solved this in PVS by expanding theories \texttt{G1} and
\texttt{G2} in place, within \texttt{group\_homomorphism}, as shown in
Figure~\ref{group_homo_ppe}.  Declarations within these expansions have
identifiers that guarantee they are distinct from each other and from the
original group theory.  Thus the \texttt{oops} lemma generates a type
error, as \texttt{G1.G} and \texttt{G2.G} are incompatible types, though
as they are uninterpreted they may later be mapped to compatible types.

The identifiers for a theory declaration are generated by prepending the
theory declaration identifier to each of the mapped declarations of the
source theory.  Hence for \texttt{G1}, all of the declarations of
\texttt{group} are mapped, but with \texttt{G1} prepended.  This can be
continued: if a declaration
\begin{alltt}
 gh: THEORY group_homomorphism
\end{alltt}
appears in another theory, then the type \texttt{gh.G1.G} will be created, etc.

This introduces new possibilities.  When expanding a theory the
mappings are substituted and the original declarations disappear.
However, it may be preferable to create definitions rather than
substitutions.  In addition, it is sometimes useful to simply rename the
types or constants of a theory.  For example, consider the following group
instance
\begin{session}
G1: THEORY = group\mapb{}G := int, + := +, 0 := 0, - := -\mape{}
\end{session}
\noindent which generates the following theory.
\label{group-instances-start}
\begin{session}
G1: THEORY
 BEGIN
  x, y, z: VAR int
  idempotent_is_identity: LEMMA x + x = x => x = 0
 END G1
\end{session}
To create definitions, use \texttt{=} instead of \texttt{:=}, as
in the following.
\begin{session}
G2: THEORY = group\mapb{}G = int, + = +, 0 = 0, - = -\mape{}
\end{session}
\noindent Now we get the following theory.
\begin{session}
G2: THEORY
 BEGIN
  G: TYPE+ = int
  +: [G, G -> G] = +
  0: G = 0
  -: [G -> G] = -
  x, y, z: VAR G
  idempotent_is_identity: LEMMA x + x = x => x = 0
 END G2
\end{session}
Finally, to simply rename the uninterpreted types and constants, use
\texttt{::=} as in the following.
\begin{session}
G3: THEORY = group\mapb{}G ::= MG, + ::= *, 0 ::= 1, - ::= inv\mape{}
\end{session}
\noindent The generated theory instance specifies multiplicative groups as
follows.
\begin{session}
G3: THEORY
 BEGIN
  MG: TYPE+
  *: [MG, MG -> MG]
  1: MG
  inv: [MG -> MG]
  x, y, z: VAR MG
  associative_ax: AXIOM FORALL x, y, z: x * (y * z) = (x * y) * z
  identity_ax: AXIOM FORALL x: x * 1 = x
  inverse_ax: AXIOM FORALL x: x * inv(x) = 1 AND inv(x) * x = 1
  idempotent_is_identity: LEMMA x * x = x => x = 1
 END G3
\end{session}
The right-hand side of a renaming mapping must be an identifier, operator,
or number, and must not create ambiguities within the generated theory.
Note that renamed declarations are still uninterpreted, and may themselves
be given interpretations, as in
\begin{session}
G3i: THEORY = G3\mapb{}MG := nzreal, * := *, 1 := 1,
                  inv := LAMBDA (r: nzreal): 1/r\mape{}
\end{session}

Finally, we can mix the different forms of mapping, to give a partial
mapping.
\begin{session}
G4: THEORY = group\mapb{}G = nzreal, + := *, 0 ::= one\mape{}
\end{session}
\noindent This generates the following theory instance.
\begin{session}
G4: THEORY
 BEGIN
  G: TYPE+ = nzreal;
  one: nzreal;
  -: [nzreal -> nzreal]
  x, y, z: VAR nzreal
  identity_ax: AXIOM FORALL (x: nzreal): x * one = x
  inverse_ax: AXIOM FORALL (x: nzreal):
                      x * -x = one AND -x * x = one
  idempotent_is_identity: LEMMA x * x = x => x = one
 END G4
\end{session}\label{group-instances-end}
Note that \texttt{associative\_ax} has disappeared---it has become a TCC
of the importing theory---whereas the other axioms are not so transformed
because they still reference uninterpreted types or constants.

With theories as parameters we have another situation in which mappings
are more convenient than theory parameters.  Many times the same set of
parameters is passed through an entire theory hierarchy.  If there are
assumings, then these must be copied.  For example, consider the
following theory.
\begin{session}
th[T: TYPE, a, b: T]: THEORY
 BEGIN
  ASSUMING
   A: ASSUMPTION a /= b
  ENDASSUMING
  ...
 END th
\end{session}
\noindent To import this theory, you simply provide a type and two
different elements of that type.  But suppose you wish to import this
theory from a theory that has the same parameters.  In this case the
assumption must also be copied, as there is otherwise no way to prove the
resulting obligation.  This can (and frequently does) lead to a tower of
theories, all with the same parameters and copies of the same assumptions,
as well as proofs of the same obligations.

There are ways around this, of course.  Most assumptions may be turned
into type constraints, as in the following.
\begin{session}
th[T: TYPE, a: T, b: \setb{}x: T | a /= x\sete{}]: THEORY
 ...
\end{session}
\noindent But this introduces an asymmetry in that \texttt{a} and
\texttt{b} now belong to different types, and the type predicate still
must be provided up the entire hierarchy.

Using a theory as a parameter, we may instead define \texttt{th} as
follows.
\begin{session}
th: THEORY
 BEGIN
  T: TYPE,
  a, b: T
  A: AXIOM a /= b
  ...
 END th
\end{session}
\noindent We then parameterize using this theory (which is implicitly
imported):
\begin{session}
th_1[t: THEORY th]: THEORY ...
\end{session}
\noindent We have encapsulated the uninterpreted types and constants into
a theory, and this is now represented as a single parameter.  Axiom
\texttt{A} is visible within theory \texttt{th\_1}, and no proof
obligations are generated since no mapping was given for \texttt{th}.  Now
we can continue defining new theories as follows.
\begin{session}
th_2[t: THEORY th]: THEORY IMPORTING th_1[t] ...
th_3[t: THEORY th]: THEORY IMPORTING th_2[t] ...
  \vdots
\end{session}
\noindent None of these generate proof obligations, as no mappings are
provided.

We may now instantiate \texttt{th\_n}, for example, with the following.
\begin{session}
IMPORTING th_n[th\mapb{}T := int, a := 0, b := 1\mape{}]
\end{session}
\noindent Now the substituted form of the axiom becomes a proof obligation
which, when proved, provides evidence that the theory \texttt{th} is
consistent.

% \chapter{Theory Declarations and Theory Abbreviations}

With the introduction of theories as parameters, it is natural to allow
theory declarations that may be mapped, in the same way that instances may
be provided for theories as parameters.  Thus the
\texttt{group\_homomorphism} may be rewritten as follows:
\begin{session}
group_homomorphism: THEORY
 BEGIN
  G1, G2: THEORY group
  x, y: VAR G1.G
  f: VAR [G1.G -> G2.G]
  homomorphism?(f): bool = FORALL x, y: f(x + y) = f(x) + f(y)
  hom_exists: LEMMA EXISTS f: homomorphism?(f)
 END group_homomorphism
\end{session}
\noindent Again, the choice between using theories as parameters or theory
declarations is really a question of taste, as they are largely
interchangeable.

As with theories as parameters, copies must be made for \texttt{G1} and
\texttt{G2}.  Note that this means that there is a difference between
theory abbreviations and theory declarations, as the former do not involve
any copying.  We decided to use the old form of theory abbreviation to
define theory declarations, and to extend the \texttt{IMPORTING} expressions to
allow abbreviations, as shown in Figure~\ref{importing-bnf}.  Thus instead of
\begin{session}
funset: THEORY = sets[[int -> int]]
\end{session}
\noindent which creates a copy of sets, use
\begin{session}
IMPORTING sets[[int -> int]] AS funset
\end{session}
\noindent which imports \texttt{sets[[int -> int]]} and abbreviates it as
\texttt{funset}.

\begin{figure}
\setlength{\sessionboxwidth}{\linewidth}
\addtolength{\sessionboxwidth}{-\arrayrulewidth}
\addtolength{\sessionboxwidth}{-\tabcolsep}
\begin{boxedminipage}[b]{\sessionboxwidth}
\begin{bnf}

\production{Importing}
{\lit{IMPORTING} \ites{ImportingItem}{,}}

\production{ImportingItem}
{TheoryName \opt{\lit{AS} Id}}

\end{bnf}
\end{boxedminipage}
\caption{Grammar for Importings}\label{importing-bnf}
\end{figure}

\chapter{Prettyprinting Theory Instances}

Mappings can get fairly complex, especially if actual parameters are
involved, and it may be desirable to see the specified theory instance
displayed with all the substitutions performed.  To support this, we have
provided a new PVS command: \texttt{prettyprint-theory-instance}
(\texttt{M-x ppti}).  This takes two arguments: a theory instance, which
in general is a theory name with actual parameters and/or mappings, and a
context theory, in which the theory instance may be typechecked.  The
simplest way to use this command is to put the cursor on the theory name
as it appears in a theory as parameter, theory declaration, or
importing---when the command is issued it then defaults to the theory
instance under the cursor and the current theory is the default
context theory.  For example, putting the cursor on
\texttt{group\_homomorphism} in the following and typing \texttt{M-x ppti}
followed by two carriage returns\footnote{The first uses the theory name
instance at the cursor, and the second uses the current theory as the
context.} generates a buffer named \texttt{group\_homomorphism.ppi}.
All instances of a given theory generate the same buffer name.
\begin{session}
IMPORTING group_homomorphism[\mapb{}G := int, + := +, 0 := 0, - := -\mape{}
                             \mapb{}G := nzreal, + := *, 0 := 1,
                               - := LAMBDA (x: nzreal): 1/x\mape{}]
\end{session}
\noindent This buffer has the following contents.
\begin{session}
% Theory instance for
  % group_homomorphism[groups\mapb{} G := int, + := +,
  %                             - := -, 0 := 0 \mape{},
  %                    groups\mapb{} G := nzreal, + := *,
  %                             - := (LAMBDA (x: nzreal): 1 / x),
  %                             0 := 1 \mape{}]
group_homomorphism_instance: THEORY
 BEGIN

  IMPORTING groups\mapb{} G := int, + := +, - := -, 0 := 0 \mape{}

  IMPORTING groups\mapb{} G := nzreal, + := *,
                     - := (LAMBDA (x: nzreal): 1 / x), 0 := 1 \mape{}

  x, y: VAR int

  f: VAR [int -> nzreal]

  homomorphism?(f): bool =
    FORALL (x: int), (y: int): f(x + y) = f(x) * f(y)

  hom_exists: LEMMA EXISTS (f: [int -> nzreal]): homomorphism?(f)
 END group_homomorphism_instance
\end{session}
The group instances shown on
pages~\pageref{group-instances-start}--\pageref{group-instances-end}
provide more examples of the output produced by
\texttt{prettyprint-theory-instance}.

\chapter{Comparison with Other Systems}

In this chapter we compare PVS theory interpretations to existing
programming and specification mechanisms of other systems.
The \textsc{Ehdm} system~\cite{EHDM:Language} has a notion of a mapping
module that maps a source module to a target module.  When a mapping
module is typechecked, a new module is automatically created that
represents the substitution of the interpretations for the body of the
source theory.  Equality is allowed to be mapped in \textsc{Ehdm}, in
which case it must be mapped to an equivalence relation.  In PVS, mappings
are provided as a syntactic component of names, and are essentially an
extension of theory parameters.  Equality is not treated specially, but is
handled by mapping a given type to a quotient type.

IMPS~\cite{Farmer:imps-cade,Farmer94} also supports theory
interpretations.  It is similar to \textsc{Ehdm} in that it has a special
\texttt{def-translation} form that takes a source theory, target
theory, sort association list, and constant association list, and generates a
theory translation.  Obligations may be generated that ensure that every
axiom of the source theory is a theorem of the target theory.  If these
are proved the translation is treated as an interpretation.  There is no
mechanism for mapping equality.  As with both PVS and \textsc{Ehdm},
defined sorts and constants of the source theory are automatically
translated.  A more detailed comparison between IMPS and an earlier
version of PVS appears in an unpublished report by
Kamm\"{u}ller~\cite{Kammuller:comparison}.

In Maude~\cite{Maude} and its precursor OBJ~\cite{OBJ:intro} it is
possible to
define \texttt{modules} that represent transition systems of a rewrite
theory whose states are equivalence classes of ground terms and whose
transitions are inference rules in \emph{rewriting logic}.  A given module
may import another module, either \texttt{protecting} it, which means that
the importing module adds no \emph{junk} or \emph{confusion}, or
\texttt{including} it, which imposes no such restrictions.  In addition to
modules, Maude has \emph{theories}, which are used to declare module
interfaces.  These may appear as module parameters, as in
$M[X_{1}::T_{1},\ldots,X_{n}::T_{n}]$, where the $X_{i}$ are \emph{labels}
and the $T_{i}$ are names of theories.  These theory parameters (source
theories) may be instantiated by target theories or modules using
\emph{views}, which indicate how each sort, function, class, and message
of the source theory is mapped to the target theory.  However, Maude
currently does not support the generation of proof obligations from source
theory axioms, so views are simply theory translations, not
interpretations.

The programming language Standard ML~\cite{ML-report} has a module
system where modules are given by \emph{structures} with a given
\emph{signature}, and parametric modules are \emph{functors} mapping
structures of a given signature to structures.  The PVS mechanism
of using theories as parameters resembles SML functors but for a
specification language rather than a programming language. 
Sannella and Tarlecki~\cite{SannellaDT:essential-concepts97} describe a
version of the ML module system in which there are \emph{specifications}
containing \emph{sorts}, \emph{operations}, and \emph{axioms}.  For
example, the signature of stacks is the following.
\begin{eqnarray*}
\emph{STACK} = & \textbf{sorts} & \emph{stack} \\
               & \textbf{opns} & \emph{empty} : \emph{stack} \\
               &               & \emph{push} : \texttt{int} \times \emph{stack} \rightarrow \emph{stack} \\
               &               & \emph{pop} :
                                 \emph{stack} \rightarrow \emph{stack} \\
               &               & \emph{top} :
                                 \emph{stack} \rightarrow \texttt{int} \\
               &               & \emph{is\_empty} :
                                 \emph{stack} \rightarrow \texttt{bool} \\
               & \textbf{axioms} & \emph{is\_empty}(\emph{empty}) =
                                   \texttt{true} \\
               &               &
               \forall\emph{s}:\emph{stack}.\forall\emph{n}:\texttt{int}.
                 \emph{is\_empty}(\emph{push}(\emph{n},\emph{s}))
                     = \texttt{false} \\
               &               &
               \forall\emph{s}:\emph{stack}.\forall\emph{n}:\texttt{int}.
                 \emph{top}(\emph{push}(\emph{n},\emph{s})) = \emph{n} \\
               &               &
               \forall\emph{s}:\emph{stack}.\forall\emph{n}:\texttt{int}.
                 \emph{pop}(\emph{push}(\emph{n},\emph{s})) = \emph{s} \\
\end{eqnarray*}
The following algebra is a \emph{realization} of the above specification
that corresponds to that of \texttt{cstack} on page~\pageref{cstack}.
{\smaller\begin{alltt}
  structure S2 : STACK =
      struct
          type stack = (int -> int) * int
          val empty = ((fn k => 0), 0)
          fun push (n, (f, i))
                = ((fn k => if k = i then n else f k), i+1)
          fun pop (f, i) = if i = 0 then (f, 0) else (f, i-1)
          fun top (f, i) = if i = 0 then 0 else f(i-1)
          fun is_empty (f, i) = (i=0)
\end{alltt}}
Note however, that the stacks \emph{empty} and
\emph{pop}(\emph{push}(\texttt{6},\emph{empty})) are not equal.  Thus they
distinguish the \emph{observable} sorts, in this case \texttt{int} and
\texttt{bool}, which are the only data directly visible to the user.  The
above two terms are not \emph{observable computations}, so it does not
matter that they are different.  In general, two different algebras are
\emph{behaviorally equivalent} if all observable computations yield the
same results. Note that choosing observable values based on sorts is a bit
coarse: for example, there may be two \texttt{int}-valued variables, one of
which is observable and one that represents an internal pointer.  Mapping
to equivalence classes is more general, as it is easy to capture
behavioral equivalence.

The induction theorem prover Nqthm~\cite{boyer-moore88,BoyerGoldschlag91}
has a feature called \texttt{FUNCTIONALLY-INSTANTIATE} that can be used to
derive an instance of a theorem by supplying an interpretation for some of
the function symbols used in defining the theorem.  The corresponding
instances of any axioms concerning these function symbols must be
discharged.  Such axioms can be introduced as conservative extensions as
definitions with the \texttt{DEFUN} declaration or through witnessed
constraints using the \texttt{CONSTRAIN} declaration, or they can be
introduced nonconservatively through an \texttt{ADD-AXIOM}
declaration.  While the functional instantiation mechanism is similar in
flavor to PVS theory interpretations, the underlying logic of Nqthm is a
fragment of first-order logic whose expressive power is more limited
than the higher-order logic of PVS.  In addition, Nqthm lacks types and
structuring mechanisms such as parametric theories.

The \specware{} language~\cite{SrinivasJullig95} employs theory
interpretations as a mechanism for the stepwise refinement of
specifications into executable code.  \specware{} has constructs for
composing specifications while identifying the common components, and for
compositionally refining specifications so that the refinement of a
specification can be composed from the refinement of its components.
Unlike PVS, \specware{} has the ability to incorporate multiple logics
and translate specifications between these logics.  A theory is an
independent unit of specification in PVS and hence there is no support for
composing theories from other theories.  However, the operations in
\specware{} can largely be simulated by means of theories and theory
interpretations in PVS.

In summary, theory interpretation has been a standard tool in
specification languages since the early work on HDM~\cite{HDM:Handbook}
and Clear~\cite{BURSTALL&GOGUEN}.  PVS implements theory interpretations
as a simple extension of the mechanism for importing parametric theories.
PVS theory interpretations subsume the corresponding capabilities
available in other specification frameworks.


\chapter{Future Work}

A number of interesting extensions may be contemplated for
the future.

\paragraph{Mapping of interpreted types and constants---}

There are two aspects: one is simply a convenience where, for
example, we might have a tuple type declaration \texttt{T: TYPE = [T1, T2,
T3]} and want to map it to \texttt{position: TYPE = [real, real, real]} by
simply giving the map \texttt{\mapb{}T := position\mape{}}.

The second aspect is where the mapping is between two different kinds, for
example mapping a record type to a function type.  This requires
determining the corresponding components as well as making explicit the
underlying axioms.  For example, record types satisfy extensionality, and
if they are mapped to a different type the implicit extensionality axiom
must be translated to a proof obligation.

\paragraph{Rewriting with congruences---}

In theory substitution, if a type is mapped to a quotient type then
equality over this type is mapped to equality over the quotient type.
If $T$ is an uninterpreted type, $\equiv$ an equivalence relation over
$T'$, and $T'/\equiv$ the quotient type, then \texttt{=[$T$]} is mapped to
\texttt{=[$T'/\equiv$]}, which is equivalent to $\equiv$.  An equational
formula thus still has the form of a rewrite.  However, to apply such a
rewrite one generally needs to do some lifting.  The following is a simple
example.
\begin{session}
th: THEORY
 BEGIN
  T: TYPE
  a, b: T
  f, g: [T -> T]
  \ldots \emph{Some axioms involving f, g, a, and b}
  lem: LEMMA f(a) = g(b)
 END th
th2: THEORY
 BEGIN
  ==(x, y: int): bool = divides(3, x - y)
  IMPORTING th\mapb{}T := E(==),
                a := equiv_class(==)(2),
                b := equiv_class(==)(1),
                f := LAMBDA (x: E(==)): equiv_class(rep(x) - 1),
                g := LAMBDA (x: E(==)): equiv_class(rep(x) - 2)\mape{}
  \ldots
 END th2
\end{session}
\noindent To rewrite with \texttt{lem}, \texttt{a} must first be lifted to
its equivalence class, then the rewrite is applied and the result is then
projected back using \texttt{rep}.  To do this requires some modification
to the rewriting mechanism of the prover.

\paragraph{Consistency Analysis---}

With a single independent theory such as groups, it is easy to generate a
mapping in which all axioms become proof obligations, and see directly
that the theory is consistent.  On the other hand, if many theories are
involved in which compositions of mappings are involved, this may become
quite difficult.  What is needed is a tool that analyzes a mapped theory
to see if it is consistent, and reports on any remaining axioms and
uninterpreted declarations.  This is similar in spirit to proof chain
analysis, but works at the theory level rather than for individual
formulas.

\paragraph{Semantics of Mappings---}

The semantics of theory interpretations needs to be formalized and added
to the PVS semantics report~\cite{PVS-Semantics:TR}.

\chapter{Conclusion}

Theory interpretations are used to embed an interpretation of an abstract
theory in a more concrete one.  In this way, they allow an abstract
development to be reused at the more concrete level.  Theory
interpretations can be used to refine a specification down to code.
Theory interpretations can also be used to demonstrate the consistency of
an axiomatic theory relative to another theory.

Parametric theories in PVS provide some but not all of the functionality
of theory interpretations.  In particular, they do not allow an abstract
theory to be imported with only a partial parameterization.  Theory
interpretations have been implemented in PVS version 3.0, which will be
released in mid-2001.  The current implementation allows the
interpretation of uninterpreted types and constants in a theory, as well
as theory declarations.  PVS has also been extended so that a theory may
appear as a formal parameter of another theory.  This allows related sets
of parameters to be packaged as a theory.  Quotient types have been
defined within PVS and used to admit interpretations of types where the
equality on a source type is treated as an equivalence relation on a
target type.

Theory interpretations have been implemented in PVS as an extension of the
theory parameter mechanism.  This way, theory interpretations are
an extension of an already familiar concept in PVS and can be used in
place of theory parameters where there is a need for greater
flexibility in the instantiation.  The proof obligations generated by
theory interpretations are similar to those for parametric theories
with assumptions.  

A number of extensions related to theory interpretations remain to be
implemented.  First, we plan to extend theory interpretations to the case
of interpreted types and constants.  This poses some challenges since
there are implicit operations and axioms associated with certain type
constructors.  Second, the rewriting mechanisms of the PVS prover need to
be extended to rewrite relative to a congruence.  This means that if we
are only interested in $f(a)$ up to some equivalence that is preserved by
$f$, then we could rewrite $a$ up to equivalence rather than equality.
Third, the PVS semantics have to be extended to incorporate theory
interpretations.  Finally, the PVS ground evaluator has to be extended to
handle theory interpretations.  Currently, the ground evaluator generates
code corresponding to a parametric theory and this code is reused with the
actual parameters used as arguments to the operations.  Theory
interpretations cannot be treated as arguments in this manner since there
is no fixed set of parameters; parameters can vary according to the
interpretation.  Also, non-executable operations can become executable as
a result of the interpretation.

In summary, we believe that theory interpretations are a significant
extension to the PVS specification language.  Our implementation of this
in PVS3.0 is simple yet powerful.  We expect theory interpretations to be
a widely used feature of PVS.

\newpage
\bibliographystyle{alpha}
\addcontentsline{toc}{chapter}{Bibliography}
\bibliography{../pvs}
\end{document}

% Master File: language.tex
% Document Type: LaTeX

\chapter{Name Resolution}\label{names}\label{resolution}

Names in PVS are used to denote theories, variables, constants, and
formulas.  New names are introduced by declarations.  The syntax of names
is given in Figure~\ref{bnf-names}.

\pvsbnf{bnf-names}{Name Syntax}

The simplest form of a name is an \emph{idop}, \ie\ an identifier or
operator symbol.  This is generally all that is needed, unless names are
overloaded.

The overloading of names, both from different theories and within a single
theory, is allowed as long as there is some way for the system to
distinguish references to them.  Names from different theories may be
distinguished by prefixing them with the theory name.  Within a theory,
all names of the same kind must be unique, except for expression kinds;
which need only be unique up to the signature.  This is because the
signature is enough to distinguish these declarations.  For example, if
\texttt{<} is declared to have signature \texttt{[bool,int -> bool]}, the
system will recognize from the context that \texttt{TRUE < 3} contains a
reference to this declaration, whereas \texttt{2 < 3} does
not.\footnote{Of course, this assumes that \texttt{TRUE} has not itself
been overloaded.}  If the use of the name is not enough to distinguish,
coercion may be used to specify the signature directly (see
page~\pageref{coercions}).  Theory parameters must be unique across all
kinds.

There are three possible forms for names (two for theory names, which
appear in \texttt{IMPORTING}s, \texttt{EXPORTING} \texttt{WITH}s, and
theory declarations).  Given a theory named \emph{theoryid}, with formal
parameters $f_1,\ldots,f_n$, that contains a declaration named \emph{id},
the following three forms may be used to reference the declaration in a
theory that imports \emph{theoryid}:
\begin{itemize}
\item \emph{theoryid}\texttt{[$a_1,\ldots,a_n$]}.\emph{id}

\item \emph{id}\texttt{[$a_1,\ldots,a_n$]}

\item \emph{id}
\end{itemize}
where the $a_i$ are expressions or type expressions that are compatible
with the formal parameters as described in Section~\ref{parameters}.  Note
that any of these forms may have \emph{mappings} immediately after the
actual parameters.  As described in Section~\ref{mappings}, these can be
viewed as an extension of the actuals.  Note also that theory names allow
different kinds of mappings.  The forms above are listed in order of
increasing likelihood of ambiguity---that is, names that are given with
just an \emph{id} are far more likely to produce an ambiguity than those
further up.  Note that even the top form may be ambiguous, as \emph{id}
may be declared more than once in \emph{theoryid}.  If this is the case,
then either the context will disambiguate the name or a type will have to
be supplied in the form of a coercion expression, \eg\
\texttt{\emph{id}::~nat}.  This kind of ambiguity is allowed only for
constants (including functions and recursive functions) and variables.

Names are resolved based on the expected type and the number and types of
arguments to which the name is applied.  The expected type is generally
determined from the context of the name, for example in
\begin{pvsex}
  c1: int = c2
\end{pvsex}
\texttt{c2} has expected type \texttt{int}.  For most expressions, this is
straight-forward, but applications create special problems.  For example,
in
\begin{pvsex}
  f: FORMULA c1 = c2
\end{pvsex}
we know that the equality (which \emph{is} an application) has range type
\texttt{boolean} since it is a formula, but this gives no information
about the types of the arguments.  We will first describe the simpler
situation, and then explain how names used as operators of an application
are resolved.

In general, the typechecker works by first collecting possible types for
the expressions, and then chooses from among the possible types using the
expected type, which is determined from the context of the expression.
The expected type is used to resolve ambiguities, but otherwise does not
contribute to the type of an expression.  Thus if \texttt{2 + 3}
typechecks, and \texttt{+} has not been redeclared, then it has type
\texttt{number\_field} regardless of its context.  However, for the purpose
of checking for TCCs, it may be treated as having a different type
depending on the expected type and the available judgements.

% Document Type: LaTeX
% Master File: language.tex

\chapter{Abstract Datatypes}\label{datatypes}\label{adts}

PVS provides a powerful mechanism for defining abstract datatypes.  This
mechanism is akin to, but more sophisticated than, the \emph{shell}
principle of the Boyer-Moore prover~\cite{Boyer-Moore79}).  A PVS datatype
is specified by providing a set of \emph{constructors} along with
associated \emph{accessors} and \emph{recognizers}.  When a datatype is
typechecked, a new theory is created that provides the axioms and
induction principles needed to ensure that the datatype is the initial
algebra defined by the constructors.

\pvsbnf{bnf-adts}{Datatype Syntax}

The syntax for PVS datatypes is given in Figure~\ref{bnf-adts}.  Datatypes
may appear at the \emph{top-level} as with theory declarations, or
\emph{in-line} as a declaration within a theory.\footnote{Enumeration
types are actually in-line datatypes---see Section~\ref{enum-types}.}
Typechecking a top-level datatype named \texttt{foo} causes the generation
of a new PVS file named \texttt{foo\_adt.pvs} containing up to three
theories as described below.  Typechecking an in-line datatype has the
effect of adding new declarations to the current theory, effectively
replacing the in-line datatype.  In-line datatypes are more restricted:
they may not have formal parameters or assuming parts, and they will not
generate the recursive combinators described below.  The declarations
generated for an in-line datatype may be viewed using the
\texttt{M-x~prettyprint-expanded} command (see the \emph{PVS System
Guide}~\cite{PVS:userguide}).

\section{A Datatype Example: \texttt{stack}}\label{stacks-adt}
An example of a datatype is \texttt{stack}:
\begin{session}
  stack[T: TYPE]: DATATYPE
   BEGIN
    empty: empty?
    push(top:T, pop:stack): nonempty?
   END stack
\end{session}
The \texttt{stack} datatype has two \emph{constructors}, \texttt{empty} and
\texttt{push}, that allow stack elements to be constructed.  For example,
the term \texttt{push(1, empty)} is an element of type \texttt{stack[int]}.
The \emph{recognizers} \texttt{empty?}\ and \texttt{nonempty?}\ are predicates
over the \texttt{stack} datatype that are true when their argument is
constructed using the corresponding constructor.  Given a \texttt{stack}
element that is known to be \texttt{nonempty?}, the \emph{accessors}
\texttt{top} and \texttt{pop} may be used to extract the first and second
arguments.

Typechecking the \texttt{stack} specification automatically creates a new
file \texttt{stack\_adt.pvs}, that contains the material found in
the next five figures.  This new file contains three theories:
\texttt{stack\_adt}, \texttt{stack\_adt\_map}, and
\texttt{stack\_adt\_reduce}.

\pvstheory{stack_adtA-alltt}{Theory \texttt{stack\_adt} (continues)}{stack_adtA-alltt}
\pvstheory{stack_adtB-alltt}{Theory \texttt{stack\_adt} (continues)}{stack_adtB-alltt}
\pvstheory{stack_adtC-alltt}{Theory \texttt{stack\_adt} (continues)}{stack_adtC-alltt}
\pvstheory{stack_adtD-alltt}{Theory \texttt{stack\_adt\_map}}{stack_adtD-alltt}
\pvstheory{stack_adtE-alltt}{Theory \texttt{stack\_adt\_reduce}}{stack_adtE-alltt}

The first theory \texttt{stack\_adt} is parametric in type \texttt{T}.
This is a specification of ``stacks of \texttt{T}'', where \texttt{T} may
be instantiated by any defined type when the stacks datatype is imported.
Thus ``stacks of integers'' as well as ``stacks of stacks of integers''
may be defined using this theory.  The first few lines of the theory
define the main type of stacks \texttt{stack}, the recognizers
\texttt{emptystack?} and \texttt{nonemptystack?}, the constructors
\texttt{empty} and \texttt{push}, and the accessors \texttt{top} and
\texttt{pop} are declared.

The \texttt{stack\_ord} function is defined, and an axiom provided for
it's definition.  This is provided instead of a disjointness axiom,
because the disjointness axiom becomes difficult to generate and use if
the number of constructors is large.  The disjointness comes from the fact
that the natural numbers are distinct.  The \texttt{ord} function is then
defined to return \texttt{0} on an empty stack and \texttt{1} on a
nonempty stack.  This is the same function as \texttt{stack\_ord}, but is
easier to use.

Then a series of axioms are given.  The
\texttt{stack\_empty\_extensionality} axiom states that there is only one
bottom element of the datatype: \texttt{empty}.
\texttt{stack\_push\_extensionality} states that any two stacks that have
the same \texttt{top} and \texttt{pop} (have the same components) are the
same.  The \texttt{stack\_push\_eta} axiom states that \texttt{pop}ping
and \texttt{push}ing the same element off and onto a stack results in a
stack identical to the original.  \texttt{stack\_top\_push} says that if
you \texttt{push} and element on a stack, you get that same element when
you \texttt{pop} it back off.  \texttt{stack\_pop\_push} says that pushing
something on a stack and then popping it back off results in the original
stack.

The \texttt{stack\_inclusive} axiom states that all stacks are either
\texttt{empty?} or \texttt{nonempty?}.  The PVS prover builds this axiom
in, so that it rarely needs be cited by a user.

\newpage
The next axiom, \texttt{stack\_induction}, introduces an induction formula
for stacks stating that any predicate $p$ of stacks that
\begin{enumerate}
\item holds for the empty stack (the base case), and
\item if $p$ holds for some stack then $p$ holds for the result of
\texttt{push}ing anything of the right type onto that stack (the induction
step),
\end{enumerate}
then $p$ holds for all stacks.

Then some useful functions are defined over stacks.  The stack predicate
\texttt{every} takes as arguments a predicate over \texttt{T} and a stack
and returns \texttt{TRUE} iff all elements on the stack satisfy the given
predicate.  \texttt{every} is introduced in both curried and uncurried
forms.  The stack predicate \texttt{some} is dual to \texttt{every},
returning \texttt{TRUE} iff there is some element on the stack that
satisfies the predicate.  The \texttt{subterm} predicate takes two stacks
and returns \texttt{TRUE} if and only if the first argument stack is a
subterm of the second.  That is, if the second stack consists of the first
stack with some (perhaps zero) elements pushed onto it.  The \texttt{<<}
predicate is the strict (irreflexive) \texttt{subterm} predicate.  Thus
for all stacks $s$, \texttt{subterm}$(s,s)$ holds, but for no stack $s$
does \texttt{<<}$(s,s)$ hold.  An alternative equivalent definition of
\texttt{<<} is as follows:
\begin{pvsex}
  <<(x: stack, y: stack): boolean = subterm(x,y) AND NOT x = y
\end{pvsex}
However, this definition is more awkward to use in a proof, as the
recursion is hidden in the definition of \texttt{subterm}.  For this
reason the definitions for \texttt{every}, \texttt{some},
\texttt{subterm}, and \texttt{<<}, are each defined as standalone
functions, though some of them could be defined in terms of the others.

The last four declarations of the theory \texttt{stack\_adt} are functions
which reduce a stack to a natural number or to an ordinal.  These
functions are useful for simplifying the proof of termination of
user-defined functions over stacks.  Recall that PVS requires recursive
functions to include a \emph{measure}, which is used to generate
termination conditions.  The primary use of the recursive combinator is to
allow measure functions to be specified.  The function
\texttt{reduce\_nat} takes a natural number and a function.  The natural
number is used for the empty stack, and then for each element on the
stack, the input function is applied to the element from the stack and the
current reduced natural number, returning a natural number.  The function
\texttt{reduce\_nat} returns the final natural number.  The function
\texttt{REDUCE\_nat} is analogous to \texttt{reduce\_nat}, except that the
reducing function is also given the entire contents of the stack.  This
version of reduction can be useful for complicated measures that involve,
for example, the number of repeated elements appearing on the stack.  The
simpler form of reduce is difficult to apply to such situations.  The
functions \texttt{reduce\_ordinal} and \texttt{REDUCE\_ordinal} are
analogous to \texttt{reduce\_nat} and \texttt{REDUCE\_nat} except that
they return ordinal numbers instead of natural numbers.  It is rare that a
termination argument requires the use of ordinals, so the simpler
\texttt{reduce\_nat} form is more often used.  This completes the
description of the \texttt{stack\_adt} theory.

The second theory in the file \texttt{stack\_adt.pvs} is
\texttt{stack\_adt\_map}.  This theory takes two types \texttt{T} and
\texttt{T1} as parameters, imports the \texttt{stack\_adt} theory, and
defines a mapping from \texttt{stacks[T]} to \texttt{stacks[T1]}.  The
higher-order \texttt{map} function takes a function \texttt{f} of type
\texttt{[T -> T1]}, and a stack of \texttt{T}, and returns a stack of
\texttt{T1} obtained by applying \texttt{f} to each element on the input
stack.  \texttt{map} is defined in both curried and uncurried forms.
\texttt{map} couldn't reside in the \texttt{stack\_adt} theory because
that theory has only one type parameter, while the \texttt{map} functions
require two: In order to construct and access stacks in two theories,
\texttt{map} must be parameterized in the two types.

Also in the \texttt{stack\_adt\_map} is a relational \texttt{every}
function.  It lifts a relation \texttt{R} between \texttt{T} and \texttt{T1},
to stacks of \texttt{T} and \texttt{T1}.  It is true if the stacks are the
same size, and corresponding elements satisfy \texttt{R}.

The third and final theory generated from \texttt{stack\_pvs} is
\texttt{stack\_adt\_reduce}.  This theory provides a generalized version
of \texttt{reduce\_nat} and \texttt{REDUCE\_nat}.  It takes as parameters
a type \texttt{T} and a range type \texttt{range}.  It defines a
generalized \texttt{reduce} which reduces stacks of \texttt{T} to elements
of \texttt{range}.  The functions \texttt{reduce\_nat},
\texttt{REDUCE\_nat}, \texttt{reduce\_ordinal}, and
\texttt{REDUCE\_ordinal} could have been defined using
\texttt{stack\_adt\_reduce}, but the direct definitions are provided for
additional user convenience.  The generalized \texttt{reduce} can be used
to provide evidence of termination of user-defined functions, but the
predefined versions such as \texttt{reduce\_nat} are easier to use in most
cases.

\section{Datatype Details}

In general, a datatype declaration has the form
\begin{pvsex}
  adt: DATATYPE WITH SUBTYPES S\(\sb{1}\), \ldots, S\(\sb{n}\)
    BEGIN
     cons\(\sb1\)(acc\(\sb{11}\): T\(\sb{11}\), \ldots, acc\(\sb{1{n\sb1}}\): T\(\sb{1{n\sb1}}\)): rec\(\sb1\) : S\(\sb{i\sb{1}}\)
     \vdots
     cons\(\sb{m}\)(acc\(\sb{m1}\): T\(\sb{m1}\), \ldots, acc\(\sb{1n\sb{m}}\): T\(\sb{1n\sb{m}}\)): rec\(\sb{m}\) : S\(\sb{i\sb{m}}\)
    END adt
\end{pvsex}
%
where the \texttt{cons$_i$} are the
\emph{constructors}\index{constructor}\index{datatype!constructor}, the
\texttt{acc$_{ij}$} are the
\emph{accessors}\index{accessor}\index{datatype!accessor}, the
\texttt{T$_{ij}$} are type expressions, and the \texttt{rec$_i$} are
\emph{recognizers}\index{recognizer}\index{datatype!recognizer}.  Each
line is referred to as a \emph{constructor
specification}\index{constructor specification}\index{datatype!constructor
specification}.  There are a number of restrictions enforced on
constructor specifications:
\begin{itemize}

\item The datatype identifier may not be used for a recognizer,
accessor, or subtype:\newline
($\texttt{adt} \not\equiv \texttt{rec}_i$ for all $i$, $\texttt{adt}
\not\equiv \texttt{acc}_{ij}$ for all $i$ and $j$, and $\texttt{adt}
\not\equiv \texttt{S}_i$ for all $i$).

\item The subtype names must be unique:
($i \neq j \Rightarrow \texttt{S}_i \not\equiv \texttt{S}_j$)

\item Each subtype name must be used at least once.

\item The constructor names must be unique:
($i \neq j \Rightarrow \texttt{cons}_i \not\equiv \texttt{cons}_j$).

\item The recognizer names must be unique:
($i \neq j \Rightarrow \texttt{rec}_i \not\equiv \texttt{rec}_j$).

\item No identifier may be used as both a constructor and a recognizer:\newline
($\texttt{cons}_i \not\equiv \texttt{rec}_j$ forall $i$ and $j$).

\item Duplicate accessor identifiers are not allowed within a single
constructor specification:
($j \neq k \Rightarrow \texttt{acc}_{ij} \not\equiv \texttt{acc}_{ik}$).

\end{itemize}

As seen in the \texttt{stack} example, datatypes may be recursive; this is
the case when the type of one or more of the accessors reference the
datatype.  In PVS, all such occurrences must be positive, where a type
occurrence \texttt{T} is positive in a type expression $\tau$ iff either
\begin{itemize}
\item $\tau\equiv \texttt{T}$.

\item $\tau\equiv \{x:\tau'|p(x)\}$ and the occurrence \texttt{T} is
positive in $\tau'$.

\item $\tau\equiv [{\tau_1} \rightarrow {\tau_2}]$ and the occurrence
\texttt{T} is positive in $\tau_2$\@.  For example, \texttt{T} occurs
positively in \texttt{sequence[T]} where \texttt{sequence[T]} is defined
in the PVS prelude as the function type \texttt{[nat -> T]}\@.

\item $\tau \equiv [\tau_1,\ldots, \tau_n]$ and the occurrence \texttt{T}
is positive in some $\tau_i$.

\item $\tau\equiv [\#\ l_1 : \tau_1, \ldots, l_n : \tau_n\ \#]$ and the occurrence \texttt{T} is positive in some $\tau_i$\@. 

\item $\tau\equiv \mbox{\emph{datatype}}[\tau_1,\ldots, \tau_n]$, where
\emph{datatype} is a previously defined datatype and the occurrence
\texttt{T} is positive in $\tau_i$, where $\tau_i$ is a \emph{positive
parameter} of \emph{datatype}\@.
\end{itemize}

When a top-level datatype is given with formal type parameters, they are
checked for whether their occurrences are all positive; this is used as
described above for any datatype that imports this one, as well as
determining some of the declarations described below.

When a datatype is typechecked, a number of new declarations are
generated:
\begin{itemize}

\item The datatype identifier is used to create an uninterpreted type
declaration.  In general, the term \emph{datatype} refers to this type.

\item Each recognizer is used to declare an uninterpreted subtype of the
datatype.

\item Each subtype identifier is used to declare an interpreted type that
is the disjunction of the types given by the recognizers that reference
the subtype identifier in the constructor specification.

\item Each constructor and accessor is used to generate a constant
declaration.

\item An \texttt{\emph{id}\_ord} uninterpreted function is created, and an
axiom \texttt{\emph{id}\_ord\_defaxiom} defines its values.  This is
provided instead of a disjointness axiom, because the disjointness axiom
becomes difficult to generate and use when the number of constructors is
large.

\item An \texttt{ord} function is generated that gives a zero-based number
to each constructor (e.g., \texttt{ord(null) = 0} and \texttt{cons(1,null)
= 1}).  This is mostly useful for enumeration types.

\item An extensionality axiom is generated for each constructor
specification.

\item An eta axiom is generated for each constructor specification
that has accessors.

\item For each accessor an axiom is created that says that the accessor
composed with the corresponding constructor returns the correct value; \eg\
\begin{pvsex}
  acc\(\sb{ij}\)(cons\(\sb{i}\)(e\(\sb{i1}\),\ldots, e\(\sb{i{m\sb{i}}})\) = e\(\sb{ij}\)
\end{pvsex}

\item An inclusive axiom is generated that says that every element of
the datatype belongs to at least one recognizer subtype.  This axiom is
not actually needed in practice as the prover checks for this directly.

\item Two induction schemes are provided for proving properties of the
datatype.

\item If there is at least one constructor with accessors,\footnote{Note
that enumeration types have no accessors.}  and there are positive type
parameters to the datatype, then \texttt{every} and \texttt{some}
functions are defined that provide a predicate on the datatype in terms of
the positive types.

\item The \texttt{subterm} and \texttt{<<} (irreflexive subterm) functions
are defined, and an axiom is generated that states that \texttt{<<} is
well-founded.  This allows it to be used as an ordering relation in
recursive function definitions.

\item If there is at least one constructor with
accessors,\addtocounter{footnote}{-1}\footnotemark{} the
\texttt{reduce\_nat}, \texttt{REDUCE\_nat}, \texttt{reduce\_ordinal}, and
\texttt{REDUCE\_ordinal} recursion combinators are defined.  These provide
a means for defining notions like the size or depth of a datatype term.

Note that accessor subtypes involving the datatype are
``lifted''.  The following example shows why.
\begin{pvsex}
  dt: DATATYPE
   BEGIN
    c0: c0?
    c1(a1: \setb{}x: list[dt] | length(x) > 0\sete): c1?
    c2(a2: \setb{}x: list[dt] | every(c0?)(x)\sete): c2?
   END dt
\end{pvsex}
Consider the \texttt{reduc\_nat} function.  The signature for the lifted
mapping function for \texttt{c1} and \texttt{c2} are the same:
\texttt{[list[nat] -> nat]}.  It's obvious the mapping function for
\texttt{c2} function could have the signature \texttt{[\setb{}x: list[nat]
| length(x) > 0\sete{} -> nat]}, but there is no obvious way to map
\texttt{c2} without lifting it.  Since it is not trivial to determine
which predicates map nicely, we lift them all.  In the future we may
provide heuristics that refine this.

\item If some type parameter is positive a \texttt{map} function is
generated in a separate theory.  Every positive type parameter in the
datatype is associated with a pair of \texttt{map} parameters, which form
the domain and range of a corresponding function argument.  Given a set of
such functions and a term of the datatype, \texttt{map} returns a term
that has the same structure, but with the ``leaf'' elements replaced by
the function values.

\item A separate theory is generated for the \texttt{reduce} and
\texttt{REDUCE} functions.  These generalize the \texttt{reduce} functions
above to an arbitrary range type.

\end{itemize}

Note that in the stack example, the \texttt{stack} type is nonempty, since
\texttt{empty} is an element of \texttt{stack} even if the parameter type
\texttt{T} is instantiated with an empty type.  However, there is no
requirement that a datatype be nonempty, though if it is imported and a
constant is declared to be of that type, a TCC will be generated as
described on page~\pageref{emptytypes} in section~\ref{emptytypes}.

The \texttt{stack\_adt} theory is parameterized in the type \texttt{T},
and introduces the uninterpreted type \texttt{stack}.  Under normal
circumstances, this would imply no relation between, for example,
\texttt{stack[nat]} and \texttt{stack[int]}.  However, since every
occurrence of \texttt{T} in the accessor types is positive, we can infer
that \texttt{stack[nat]} is a subtype of \texttt{stack[int]}.  In general,
given a type $T$ and a subtype $S \equiv \setb{}x:T | p(x)\sete$, then
\texttt{stack[$S$]} is treated the same as $\setb{}s:
\texttt{stack[}T\texttt{]} | \texttt{every}(p)(s)\sete$.  When a datatype
has a mix of positive and nonpositive type parameters, the subtype
relation only holds for the positive ones.  For example, in the datatype
\begin{session}
  dt[T1, T2: TYPE, c: T1]: DATATYPE
   BEGIN
    c(a1: T1, a2: [T2 -> T1]): c?
   END dt
\end{session}
\texttt{T1} is positive and \texttt{T2} is not, so \texttt{dt[nat, nat,
0]} is a subtype of \texttt{dt[int, nat, 0]}, but is not a subtype of
\texttt{dt[nat, int, 0]}, nor is it a subtype of \texttt{dt[nat, nat, 1]}.

More complex datatypes lead to correspondingly more complex declarations;
for example, in the following contrived datatype
\begin{session}
  adt1[t1,t2: TYPE, c:t1]: DATATYPE
   BEGIN
    bottom: bottom?
    c1(a11:t1, a12: [t2 -> int]): c1?
    c2(a21:adt1, a22:[nat -> adt1], a23: list[adt1]): c2?
    c3(a31:[list[int] -> adt1],
       a32:[# a: adt1, b: [int -> adt1] #],
       a33:[adt1, [set[int] -> adt1]]) : c3?
   END adt1
\end{session}
the curried \texttt{every} is generated as follows:
\begin{session}
  every(p: PRED[t1])(a1: adt1):  boolean =
      CASES a1
        OF bottom: TRUE,
           c1(c11_var, c12_var): p(c11_var),
           c2(c21_var, c22_var, c23_var):
             every(p)(c21_var) AND
              every(every(p))(c22_var) AND every[adt1](every(p))(c23_var),
           c3(c31_var, c32_var, c33_var):
                  (FORALL (x1: list[int]): every(p)(c31_var(x1)))
              AND every(p)(a(c32_var))
              AND FORALL (x: int): every(p)(b(c32_var)(x))
              AND every(p)(c33_var`1)
              AND FORALL (x: set[int]): every(p)(c33_var`2(x))
        ENDCASES;
\end{session}
Note that this is only defined for predicates over \texttt{t1}, since
the occurrence of \texttt{t2} in the constructor specification for
\texttt{c2} is not positive.

As with record types, constructor selectors may be dependent.  Here is a
simple example.
\begin{session}
  depdt: DATATYPE
   BEGIN
    b: b?
    c(x: int, y: \setb{}z: int | z < x\sete): c?
   END depdt
\end{session}

\section{Datatype Subtypes}

The \texttt{WITH SUBTYPES} keyword introduces a set of subtype names.
These are useful, for example, in defining the nonterminals of a language.
For example, we might try to describe a simple typed lambda calculus:
\begin{eqnarray*}
T & ::= & B \;|\; T \rightarrow T \\
E & ::= & x \;|\; \lambda x:T.E \;|\; E(E)
\end{eqnarray*}
This is difficult to express using datatypes without subtypes, but is
reasonably straightforward with them:\footnote{\texttt{TYPE},
\texttt{LAMBDA}, and \texttt{VAR} are PVS keywords, so variants are used
here.}
\begin{session}
tlc: DATATYPE WITH SUBTYPES typ, expr
 BEGIN
 base_type(n:nat): base_type? : typ
 fun_type(dom, ran: typ): fun_type? : typ
 expr_var(n:nat): expr_var? : expr
 lambda_expr(lvar:(expr_var?), ltype: typ, lexpr: expr)
                            : lambda_expr? : expr
 application(fun, arg: expr): application? : expr
 END tlc
\end{session}
In addition to the usual generated declarations, this generates
\begin{session}
  typ((x: tlc)): boolean = base_type?(x) OR fun_type?(x);
  typ: TYPE = \setb{}x: tlc | base_type?(x) OR fun_type?(x)\sete
  expr((x: tlc)): boolean =
     expr_var?(x) OR lambda_expr?(x) OR application?(x);
  expr: TYPE =
     \setb{}x: tlc | expr_var?(x) OR lambda_expr?(x) OR application?(x)\sete
\end{session}
immediately after the declarations generated for the recognizers, so they
may be referenced in the accessor types.  Note that only a single
induction scheme is generated.  To induct over a particular subtype,
extend the property of interest to the entire datatype so that it returns
true for everything else.


\section{\texttt{CASES} Expressions}\label{cases-expressions}
\index{cases expressions}

The \texttt{CASES} expression uses a simple form of pattern-matching on
abstract datatypes.  Patterns are of the form $c(x_1,\ldots, x_n)$ where
$c$ is an $n$-ary constructor and $x_1,\ldots, x_n$ is a list of distinct
variables.  Patterns here are simple so that certain logical properties of
the expression are easy to check.  Patterns are not defined in the grammar
but in the type rules, since the notion of a variable or a constructor is
only defined in the type rules.

For example, if \texttt{x} is of type \texttt{stack}, the cases expression
\begin{pvsex}
  CASES x OF
    empty : FALSE,
    push(y, z) : even?(y) AND empty?(z)
  ENDCASES
\end{pvsex}
is \texttt{TRUE} if \texttt{x} is a singleton even integer, and otherwise is
false.  This expression can be translated into
\begin{pvsex}
  IF empty?(x)
     THEN FALSE
     ELSE LET (y, z) = (car(x), cdr(x))
           IN even?(y) AND empty?(z)
  ENDIF
\end{pvsex}

The \texttt{CASES} expression also allows an \texttt{ELSE} clause, which
comes last and covers all constructors not previously mentioned in a
pattern.  If the \texttt{ELSE} clause is missing, and not all constructors
have been mentioned, then a \emph{cases TCC}\index{cases
TCC}\index{TCC!cases} is generated which states that the expression is not
any of the missing elements.  For example, if the \texttt{x} above is
declared to be a subtype of \texttt{stack} in which \texttt{empty} is
excluded, then the \texttt{empty} case can safely be left out, and a \tcc\
will be generated that obligates the user to prove that \texttt{x} is not
\texttt{empty}.  There is a trade-off here between simpler specifications
and simpler verifications; if the \texttt{empty} case is left in, then
there is no obligation to prove, but the extra case clutters up the
specification, and can mislead the reader into thinking that the
\texttt{empty} case is possible.  In general, we feel that the
specification should be as perspicuous as possible, even if it means a
little more work behind the scenes.


\appendix
\chapter{The Grammar}\label{grammar}

The complete \pvs\ grammar is presented in this Appendix, along with a
discussion of the notation used in presenting the grammar.

The conventions used in the presentation of the syntax are as follows.
\index{syntax!conventions}

\begin{itemize}

\item Names in {\it italics\/} indicate syntactic classes and
metavariables ranging over syntactic classes.

\item The reserved words of the language are
      printed in \lit{tt font, UPPERCASE}.

\item An optional part {\it A\/} of a clause is enclosed in square brackets:
\opt{{\it A\/}}.

\item Alternatives in a syntax production are separated by a bar
(``\choice''); a list of alternatives that is embedded in the right-hand
side of a syntax production is enclosed in brackets, as in

\begin{bnf}
\production{ExportingName}
{IdOp \opt{\lit{:} \brc{TypeExpr \choice \lit{TYPE} \choice \lit{FORMULA}}}}
\end{bnf}


\item Iteration of a clause {\it B\/} one or more times is indicated by
enclosing it in brackets followed by a plus sign: \ite{{\it B\/}};
repetition zero or more times is indicated by an asterisk instead of the
plus sign: \rep{{\it B\/}}.

\item A double plus or double asterisk indicates a clause separator; for
example, \reps{{\it B\/}}{,} indicates zero or more repetitions of the
clause {\it B} separated by commas.

\item Other items printed in tt font on the right hand side of
      productions are literals.  Be careful to distinguish where BNF
symbols occur as literals, \eg\ the BNF brackets \brc{} versus the
literal brackets \lit{\{\}}.

\end{itemize}

\subsubsection*{Specification}
\par\noindent
\spvsbnf{bnf-theory}

\subsubsection*{Importings and Exportings}
\par\noindent
\spvsbnf{bnf-exporting}

\subsubsection*{Assumings}
\par\noindent
\spvsbnf{bnf-assuming}

\subsubsection*{Theory Part}
\par\noindent
\spvsbnf{bnf-theory-part}

\subsubsection*{Declarations}
\par\noindent
\spvsbnf{bnf-decls}

\subsubsection*{Type Expressions}
\par\noindent
\spvsbnf{bnf-type-expr}

\subsubsection*{Expressions}
\par\noindent
\spvsbnf{bnf-expr}

\subsubsection*{Expressions (continued)}
\par\noindent
\spvsbnf{bnf-expr-aux}

\subsubsection*{Names}
\par\noindent
\spvsbnf{bnf-names}

\subsubsection*{Identifiers}
\par\noindent
\spvsbnf{bnf-lexical}

\subsubsection*{Datatypes}
\par\noindent
\spvsbnf{bnf-adts}

%% Derived from John Rushby's prelude.tex, modified for NFSS2
%
% define variants of the \LaTeX macro that avoid using \sc
% for use in headings
%

% Define fonts that work in math or text mode
\def\dwimrm#1{\ifmmode\mathrm{#1}\else\textrm{#1}\fi}
\def\dwimsf#1{\ifmmode\mathsf{#1}\else\textsf{#1}\fi}
\def\dwimtt#1{\ifmmode\mathtt{#1}\else\texttt{#1}\fi}
\def\dwimbf#1{\ifmmode\mathbf{#1}\else\textbf{#1}\fi}
\def\dwimit#1{\ifmmode\mathit{#1}\else\textit{#1}\fi}
\def\dwimnormal#1{\ifmmode\mathnormal{#1}\else\textnormal{#1}\fi}

\def\BigLaTeX{{\rm L\kern-.36em\raise.3ex\hbox{\small\small A}\kern-.15em
    T\kern-.1667em\lower.7ex\hbox{E}\kern-.125emX}}
\def\BoldLaTeX{{\bf L\kern-.36em\raise.3ex\hbox{\small\small\bf A}\kern-.15em
    T\kern-.1667em\lower.7ex\hbox{E}\kern-.125emX}}
%\def\labelitemi{$\bullet$}
\def\labelitemii{$\circ$}
\def\labelitemiii{$\star$}
\def\labelitemiv{$\diamond$}
\newcommand{\tcc}{{\small\small TCC}}
\newcommand{\tccs}{\tcc s}
\newcommand{\emacs}{{Emacs}}
\newcommand{\Emacs}{\emacs}
\newcommand{\ehdm}{{E{\small\small HDM}}}
\newcommand{\Ehdm}{\ehdm}
\newcommand{\tm}{$^{\mbox{\tiny TM}}$}
\newcommand{\hozline}{{\noindent\rule{\textwidth}{0.4mm}}}

\newcommand{\allclear}%
  {\mbox{\boldmath$\stackrel{\raisebox{-.2ex}[0pt][0pt]%
              {$\textstyle\oslash$}}{\displaystyle\bot}$}}

\newenvironment{private}{}{}

\newenvironment{smalltt}{\begin{alltt}\small}{\end{alltt}}

\newlength{\hsbw}

\newenvironment{session}%
  {\begin{flushleft}
   \setlength{\hsbw}{\linewidth}
   \addtolength{\hsbw}{-\arrayrulewidth}
   \addtolength{\hsbw}{-\tabcolsep}
   \begin{tabular}{@{}|c@{}|@{}}\hline 
   \begin{minipage}[b]{\hsbw}
   \begingroup\small\mbox{ }\\[-1.8\baselineskip]\begin{alltt}}
  {\end{alltt}\endgroup\end{minipage}\\ \hline 
   \end{tabular}
   \end{flushleft}}

\newenvironment{smallsession}%
  {\begin{flushleft}
   \setlength{\hsbw}{\linewidth}
   \addtolength{\hsbw}{-\arrayrulewidth}
   \addtolength{\hsbw}{-\tabcolsep}
   \begin{tabular}{@{}|c@{}|@{}}\hline 
   \begin{minipage}[b]{\hsbw}
   \begingroup\footnotesize\mbox{ }\\[-1.8\baselineskip]\begin{alltt}}%
  {\end{alltt}\endgroup\end{minipage}\\ \hline 
   \end{tabular}
   \end{flushleft}}

\newenvironment{spec}%
  {\begin{flushleft}
   \setlength{\hsbw}{\textwidth}
   \addtolength{\hsbw}{-\arrayrulewidth}
   \addtolength{\hsbw}{-\tabcolsep}
   \begin{tabular}{@{}|c@{}|@{}}\hline 
   \begin{minipage}[b]{\hsbw}
   \begingroup\small\mbox{ }\\[-0.2\baselineskip]}%
  {\endgroup\end{minipage}\\ \hline 
   \end{tabular}
   \end{flushleft}}

\newcommand{\memo}[1]%
  {\mbox{}\par\vspace{0.25in}%
   \setlength{\hsbw}{\linewidth}\addtolength{\hsbw}{-1.5ex}%
   \noindent\fbox{\parbox{\hsbw}{{\bf Memo: }#1}}\vspace{0.25in}}

\newcommand{\nb}[1]%
  {\mbox{}\par\vspace{0.25in}%
   \setlength{\hsbw}{\linewidth}\addtolength{\hsbw}{-1.5ex}%
   \noindent\fbox{\parbox{\hsbw}{{\bf Note: }#1}}\vspace{0.25in}}

\newcommand{\comment}[1]{}
\newcommand{\exfootnote}[1]{}
%\newcommand{\ifelse}[2]{#1}
\sloppy
\clubpenalty=100000
\widowpenalty=100000
%\displaywidowpenalty=100000
\setcounter{secnumdepth}{3} 
\setcounter{tocdepth}{3}
\setcounter{topnumber}{9}
\setcounter{bottomnumber}{9}
\setcounter{totalnumber}{9}
\renewcommand{\topfraction}{.99}
\renewcommand{\bottomfraction}{.99}
\renewcommand{\floatpagefraction}{.01}
\renewcommand{\textfraction}{.2}
\font\largett=cmtt10 scaled\magstep1
\font\Largett=cmtt10 scaled\magstep2
\font\hugett=cmtt10 scaled\magstep3


%\addcontentsline{toc}{chapter}{Bibliography}
\bibliographystyle{plain}
\bibliography{../pvs}

%\addcontentsline{toc}{chapter}{Index}   %% Put entry in T-O-C
%%\printindex  %% printindex makes extra call to "theindex"
{\smaller
\printindex
%% Document Type: LaTeX
% Master File: language.tex
\documentclass[12pt]{book}
\usepackage{alltt}
\usepackage{makeidx}
\usepackage{relsize}
\usepackage{boxedminipage}
\usepackage{url}
\usepackage{../../pvs}
\usepackage{../makebnf}
\usepackage[chapter]{tocbibind}
\usepackage{fancyvrb}
\usepackage[dvipsnames,usenames]{color}

\usepackage{amssymb}
\usepackage{mathpazo}
\usepackage{fontspec}
\setmainfont[Ligatures=TeX]{XITS}
\setmonofont{DejaVu Sans Mono}[Scale=MatchLowercase]
%\setmonofont{Free Mono}[Scale=0.8]
\usepackage[math-style=ISO]{unicode-math}
\renewcommand{\leadsto}{\rightsquigarrow}
%\setmathfont{XITS Math}

\topmargin -10pt
\textheight 8.5in
\textwidth 6.0in
\headheight 15 pt
\columnwidth \textwidth
\oddsidemargin 0.5in
\evensidemargin 0.5in   % fool system for page 0
\setcounter{topnumber}{9}
\renewcommand{\topfraction}{.99}
\setcounter{bottomnumber}{9}
\renewcommand{\bottomfraction}{.99}
\setcounter{totalnumber}{10}
\renewcommand{\textfraction}{.5}
\renewcommand{\floatpagefraction}{.1}
\usepackage{fancyhdr}
\pagestyle{fancy}
\raggedbottom

%\setcounter{secnumdepth}{1}

\index{type correctness condition|see{TCC}}
\makeindex

\usepackage[bookmarks=true,hyperindex=true,colorlinks=true,linkcolor=Brown,citecolor=blue,backref=page,pagebackref=true,plainpages=false,pdfpagelabels]{hyperref}

\input{../pvstex}

\begin{document}

\begin{titlepage}
\renewcommand{\thepage}{title}
\vspace*{1in}
\noindent
\rule[1pt]{\textwidth}{2pt}
\begin{center}
\newfont{\pvstitle}{cmss17 scaled \magstep4}
\textbf{\pvstitle PVS Language Reference}
\end{center}
\begin{flushright}
{\Large Version 7.1 {\smaller$\bullet$} August 2020}
\end{flushright}
\rule[1in]{\textwidth}{2pt}
\vspace*{2in}
\begin{flushleft}
S.~Owre\\
N.~Shankar\\
J.~M.~Rushby\\
D.~W.~J.~Stringer-Calvert\\
{\smaller\url{{Owre,Shankar,Rushby,Dave_SC}@csl.sri.com}}\\
{\smaller\url{http://pvs.csl.sri.com/}}
\end{flushleft}
\vspace*{1in}
\vbox{\hbox to \textwidth{{\Large SRI International\hfill}}%
\hbox to \textwidth{{\small\sf%
Computer Science Laboratory $\bullet$ 333 Ravenswood Avenue $\bullet$ Menlo Park CA 94025\hfil}}}
\end{titlepage}

\renewcommand{\chaptermark}[1]{\markboth{{\em #1}}{}\markright{{\em #1}}}
\renewcommand{\sectionmark}[1]{\markright{\thesection \em \ #1}}
%\lhead[\thepage]{\rightmark}
%\cfoot{\protect\small\bf \fbox{PVS 2.3 DRAFT}}
%\cfoot{}
%\rhead[\leftmark]{\thepage}
\thispagestyle{empty}

\newpage
\renewcommand{\thepage}{ack}

\noindent\textbf{NOTE:} This manual is in the process of being updated.
Almost everything stated here is still correct, but incomplete due to the
many new features that have been introduced into PVS over the years.  The
release notes should be consulted for the most current information.

\vspace*{6in}\noindent
The initial development of PVS was funded by SRI International.
Subsequent enhancements were partially funded by SRI and by NASA
Contracts NAS1-18969 and NAS1-20334, NRL Contract N00014-96-C-2106,
NSF Grants CCR-9300044, CCR-9509931, and CCR-9712383, AFOSR contract
F49620-95-C0044, and DARPA Orders E276, A721, D431, D855, and E301.
\newpage
\pagenumbering{roman}
\setcounter{page}{1}

\tableofcontents
%\listoffigures

%\chapter{The PVS Specification Language}

%\include{preface}
\cleardoublepage
\pagenumbering{arabic}
\setcounter{page}{1}

\setcounter{topnumber}{9}
\renewcommand{\topfraction}{.99}
\setcounter{bottomnumber}{9}
\renewcommand{\bottomfraction}{.99}
\setcounter{totalnumber}{10}
\renewcommand{\textfraction}{.01}
\renewcommand{\floatpagefraction}{.01}

\include{intro}
\include{lexical}
\include{declarations}
\include{types}
\include{expressions}
\include{theories}
\include{interpretations}
\include{names}
\include{adts}

\appendix
\include{grammar}
%\include{prelude}

%\addcontentsline{toc}{chapter}{Bibliography}
\bibliographystyle{plain}
\bibliography{../pvs}

%\addcontentsline{toc}{chapter}{Index}   %% Put entry in T-O-C
%%\printindex  %% printindex makes extra call to "theindex"
{\smaller
\printindex
%\input{language.ind}
}

\end{document}

%%% Local Variables: 
%%% TeX-command-default: "Make"
%%% mode: latex
%%% TeX-master: "language"
%%% End: 

}

\end{document}

%%% Local Variables: 
%%% TeX-command-default: "Make"
%%% mode: latex
%%% TeX-master: "language"
%%% End: 

}

\end{document}

%%% Local Variables: 
%%% TeX-command-default: "Make"
%%% mode: latex
%%% TeX-master: "language"
%%% End: 

% Document Type: LaTeX
% Master File: prover.tex
\documentclass[12pt,twoside]{book}
\usepackage{relsize,alltt,makeidx,url,boxedminipage,fancyheadings,tabularx}
%\usepackage{../../pvs}
\usepackage{../makebnf}
\usepackage[chapter]{tocbibind}

\ifpdf
\usepackage[pdftex,dvipsnames,usenames]{color}
\usepackage[bookmarks=true,hyperindex=true,colorlinks=true,linkcolor=Brown,citecolor=blue,backref=page,pagebackref=true,plainpages=false,pdfpagelabels]{hyperref}
\else
\usepackage[bookmarks=true,hyperindex=true]{hyperref}
\fi

\newcommand{\allttinput}[1]{\hozline{\smaller\smaller\smaller\begin{alltt}\input{#1}\end{alltt}}\hozline}
\topmargin -10pt
\textheight 8.5in
\textwidth 6.0in
\headheight 15 pt
\columnwidth \textwidth
\oddsidemargin 0.5in
\evensidemargin 0.5in   % fool system for page 0
\setcounter{topnumber}{9}
\renewcommand{\topfraction}{.99}
\setcounter{bottomnumber}{9}
\renewcommand{\bottomfraction}{.99}
\setcounter{totalnumber}{10}
\renewcommand{\textfraction}{.01}
\renewcommand{\floatpagefraction}{.01}
\raggedbottom

\font\largett=cmtt10 scaled\magstep2
\font\hugett=cmtt10 scaled\magstep4
\def\optl{{\smaller\sc {\smaller\smaller \&}optional}}
\def\rest{{\smaller\sc {\smaller\smaller \&}rest}}
\def\default#1{\textrm{[\texttt{#1}]}}
\def\bkt#1{{$\langle$#1$\rangle$}}
\newcommand{\ii}[1]{{\textit{\hyperpage{#1}}}}
\newcommand{\indextt}[1]{\index{#1@{\texttt{#1}}}} %just the index in tt form
\newcommand{\indtt}[1]{\texttt{#1}\index{#1@{\texttt{#1}}}}  %text+index in tt
\newcommand{\emacstt}[1]{\texttt{M-x~#1}\index{Emacs commands!#1@{\texttt{#1}}}}%index Emacs
\newcommand{\indexlargett}[1]{\index{#1@{\texttt{#1}}|ii}}
\newcommand{\indexlargettdoll}[1]{\index{#1@{\texttt{#1}}|ii}} %removed$
\newcommand{\indlargett}[1]{\texttt{#1}\indexlargett{#1}}%largett+index
\newcommand{\indttdol}[1]{\texttt{#1/\char36}\index{#1@{\texttt{#1}}}} %\index{#1@\texttt{#1\char36}}}
\newcommand{\indttdoll}[1]{\texttt{#1\char36}\index{#1\char36@{\texttt{#1\char36}}}}
\newcommand{\indttbang}[1]{\texttt{#1!}\index{#1"!@{\texttt{#1"!}}}}
\newcommand{\indttdbang}[1]{\texttt{#1!!}\index{#1"!"!@{\texttt{#1"!"!}}}}
\newcommand{\indskobang}{\index{skolem"!@{\texttt{skolem"!}}}}
\newcommand{\indskobangii}{\index{skolem"!@{\texttt{skolem"!}}|ii}}
\newcommand{\skobangdol}{\texttt{skolem!/\char36}\indskobang}
\newcommand{\skobang}{\texttt{skolem!}\indskobang}
\newcommand{\indbang}{\texttt{induct-and-rewrite!}\index{induct-and-rewrite"!@{\texttt{induct-and-rewrite"!}}}}
\newcommand{\thenat}{\texttt{then@}\index{then"@@{\texttt{then@}}}}
\newcommand{\carg}[1]{\textrm{\emph{#1}}\index{#1}}
\newcommand{\cargdflt}[2]{\carg{#1}\default{#2}}
\newenvironment{usage}[1]{\item[usage:\hspace*{-0.175in}]#1\begin{description}\setlength{\itemindent}{-0.2in}\setlength{\itemsep}{0.1in}}{\end{description}}
\newcommand{\prparagraph}[2]{\paragraph{#1}\label{#2}\index{#2}}
\newcommand{\prsubsection}[2]{
\subsection[{\largett #1}:  #2]
              {{\hugett #1}: \raggedright #2}\label{#1}\indexlargett{#1}}
\newcommand{\prdolsubsection}[2]{
\subsection[{\largett #1/\char36}:  #2]
              {{\hugett #1/\char36}: \raggedright #2}\label{#1}\indexlargett{#1}}%\indexlargettdoll{#1}}

\makeindex

\begin{document}
%\pagestyle{empty}

%\renewcommand{\baselinestretch}{2}
\newenvironment{display}{\begin{alltt}\small\tt\vspace{0.3\baselineskip}}{\vspace{0.3\baselineskip}\end{alltt}}
\newcommand{\normtt}[1]{{\obeyspaces \texttt{#1 }}}
\newenvironment{pagegroup}{}{}
%\newenvironment{smalltt}{\begin{alltt}\small\tt}{\end{alltt}}
\newenvironment{tdisplay}{\begin{alltt}\footnotesize\tt\vspace{0.3\baselineskip}}{\vspace{0.3\baselineskip}\end{alltt}}
%% Derived from John Rushby's prelude.tex, modified for NFSS2
%
% define variants of the \LaTeX macro that avoid using \sc
% for use in headings
%

% Define fonts that work in math or text mode
\def\dwimrm#1{\ifmmode\mathrm{#1}\else\textrm{#1}\fi}
\def\dwimsf#1{\ifmmode\mathsf{#1}\else\textsf{#1}\fi}
\def\dwimtt#1{\ifmmode\mathtt{#1}\else\texttt{#1}\fi}
\def\dwimbf#1{\ifmmode\mathbf{#1}\else\textbf{#1}\fi}
\def\dwimit#1{\ifmmode\mathit{#1}\else\textit{#1}\fi}
\def\dwimnormal#1{\ifmmode\mathnormal{#1}\else\textnormal{#1}\fi}

\def\BigLaTeX{{\rm L\kern-.36em\raise.3ex\hbox{\small\small A}\kern-.15em
    T\kern-.1667em\lower.7ex\hbox{E}\kern-.125emX}}
\def\BoldLaTeX{{\bf L\kern-.36em\raise.3ex\hbox{\small\small\bf A}\kern-.15em
    T\kern-.1667em\lower.7ex\hbox{E}\kern-.125emX}}
%\def\labelitemi{$\bullet$}
\def\labelitemii{$\circ$}
\def\labelitemiii{$\star$}
\def\labelitemiv{$\diamond$}
\newcommand{\tcc}{{\small\small TCC}}
\newcommand{\tccs}{\tcc s}
\newcommand{\emacs}{{Emacs}}
\newcommand{\Emacs}{\emacs}
\newcommand{\ehdm}{{E{\small\small HDM}}}
\newcommand{\Ehdm}{\ehdm}
\newcommand{\tm}{$^{\mbox{\tiny TM}}$}
\newcommand{\hozline}{{\noindent\rule{\textwidth}{0.4mm}}}

\newcommand{\allclear}%
  {\mbox{\boldmath$\stackrel{\raisebox{-.2ex}[0pt][0pt]%
              {$\textstyle\oslash$}}{\displaystyle\bot}$}}

\newenvironment{private}{}{}

\newenvironment{smalltt}{\begin{alltt}\small}{\end{alltt}}

\newlength{\hsbw}

\newenvironment{session}%
  {\begin{flushleft}
   \setlength{\hsbw}{\linewidth}
   \addtolength{\hsbw}{-\arrayrulewidth}
   \addtolength{\hsbw}{-\tabcolsep}
   \begin{tabular}{@{}|c@{}|@{}}\hline 
   \begin{minipage}[b]{\hsbw}
   \begingroup\small\mbox{ }\\[-1.8\baselineskip]\begin{alltt}}
  {\end{alltt}\endgroup\end{minipage}\\ \hline 
   \end{tabular}
   \end{flushleft}}

\newenvironment{smallsession}%
  {\begin{flushleft}
   \setlength{\hsbw}{\linewidth}
   \addtolength{\hsbw}{-\arrayrulewidth}
   \addtolength{\hsbw}{-\tabcolsep}
   \begin{tabular}{@{}|c@{}|@{}}\hline 
   \begin{minipage}[b]{\hsbw}
   \begingroup\footnotesize\mbox{ }\\[-1.8\baselineskip]\begin{alltt}}%
  {\end{alltt}\endgroup\end{minipage}\\ \hline 
   \end{tabular}
   \end{flushleft}}

\newenvironment{spec}%
  {\begin{flushleft}
   \setlength{\hsbw}{\textwidth}
   \addtolength{\hsbw}{-\arrayrulewidth}
   \addtolength{\hsbw}{-\tabcolsep}
   \begin{tabular}{@{}|c@{}|@{}}\hline 
   \begin{minipage}[b]{\hsbw}
   \begingroup\small\mbox{ }\\[-0.2\baselineskip]}%
  {\endgroup\end{minipage}\\ \hline 
   \end{tabular}
   \end{flushleft}}

\newcommand{\memo}[1]%
  {\mbox{}\par\vspace{0.25in}%
   \setlength{\hsbw}{\linewidth}\addtolength{\hsbw}{-1.5ex}%
   \noindent\fbox{\parbox{\hsbw}{{\bf Memo: }#1}}\vspace{0.25in}}

\newcommand{\nb}[1]%
  {\mbox{}\par\vspace{0.25in}%
   \setlength{\hsbw}{\linewidth}\addtolength{\hsbw}{-1.5ex}%
   \noindent\fbox{\parbox{\hsbw}{{\bf Note: }#1}}\vspace{0.25in}}

\newcommand{\comment}[1]{}
\newcommand{\exfootnote}[1]{}
%\newcommand{\ifelse}[2]{#1}
\sloppy
\clubpenalty=100000
\widowpenalty=100000
%\displaywidowpenalty=100000
\setcounter{secnumdepth}{3} 
\setcounter{tocdepth}{3}
\setcounter{topnumber}{9}
\setcounter{bottomnumber}{9}
\setcounter{totalnumber}{9}
\renewcommand{\topfraction}{.99}
\renewcommand{\bottomfraction}{.99}
\renewcommand{\floatpagefraction}{.01}
\renewcommand{\textfraction}{.2}
\font\largett=cmtt10 scaled\magstep1
\font\Largett=cmtt10 scaled\magstep2
\font\hugett=cmtt10 scaled\magstep3

\def\labelitemii{$\circ$}
\def\labelitemiii{$\star$}
\def\labelitemiv{$\diamond$}
\newcommand{\tcc}{{\small\small TCC}}
\newcommand{\tccs}{\tcc s}

%\renewcommand{\memo}[1]{\mbox{}\par\vspace{0.25in}\noindent\fbox{\parbox{.95\linewidth}{{\bf Memo: }#1}}\vspace{0.25in}}

\newcommand{\eg}{{\em e.g.\/},}
\newcommand{\ie}{{\em i.e.\/},}

\newcommand{\pvs}{PVS}

\newcommand{\ch}{\choice}
\newcommand{\rsv}[1]{{\rm\tt #1}}

\newcommand{\lpvstheory}[3]{\figurehead{\hozline\smaller\smaller\begin{alltt}}%
                           \figuretail{\end{alltt}\vspace{-0in}\hozline}%
                           \figurelabel{#3}\figurecap{#2}%
                           \begin{longfigure}\input{#1}\end{longfigure}}

\newcommand{\pvstheory}[3]
  {\begin{figure}[htb]\begin{boxedminipage}{\textwidth}%
   {\smaller\smaller\begin{alltt} \input{#1}\end{alltt}}\end{boxedminipage}%
   \caption{#2}\label{#3}\end{figure}}

\newcommand{\bpvstheory}[3]
  {\begin{figure}[b]\begin{boxedminipage}{\textwidth}%
   {\smaller\smaller\begin{alltt} \input{#1}\end{alltt}}\end{boxedminipage}%
   \caption{#2}\label{#3}\end{figure}}

\newcommand{\spvstheory}[1]
  {\vspace{0.1in}\par\noindent\begin{boxedminipage}{\textwidth}%
   {\smaller\smaller\begin{alltt} \input{#1}\end{alltt}}\end{boxedminipage}\vspace{0.1in}%
   }
%\newenvironment{spvstext}%
%  {\vspace{0.1in}\par\noindent\begin{boxedminipage}{\textwidth}%
%   \smaller\smaller\begin{alltt}}%
%  {\end{alltt}\end{boxedminipage}\vspace{0.1in}%
%   }


%  {\begin{boxedminipage}{\textwidth}{\smaller\smaller\begin{alltt}#1\end{alltt}}\end{boxedminipage}}
%\newenvironment{spvstheory}{\par\noindent\begin{boxedminipage}{\textwidth}\smaller\smaller\begin{alltt}}{\end{alltt}\end{boxedminipage}}

\newenvironment{pvsex}%
  {\setlength{\topsep}{0in}\smaller\begin{alltt}}%
  {\end{alltt}}

\newcommand{\pvsbnf}[2]
  {\begin{figure}[htb]\begin{boxedminipage}{\textwidth}%
   \input{#1}\end{boxedminipage}\caption{#2}\label{#1}\end{figure}}

\newcommand{\spvsbnf}[1]
  {\begin{boxedminipage}{\textwidth}\input{#1}\end{boxedminipage}}

\newcommand{\pidx}[1]{{\rm #1}} % primary index entry
\newcommand{\sidx}[1]{{\rm #1}} % secondary index entry
\newcommand{\cmdindex}[1]{\index{#1@\cmd{#1}}}
\newcommand{\icmd}[1]{\cmd{#1}\cmdindex{#1}}
\newcommand{\iecmd}[1]{\ecmd{#1}\cmdindex{#1}}
\newcommand{\buf}[1]{\texttt{#1}}
\newcommand{\ibuf}[1]{\buf{#1}\index{#1 buffer@\buf{#1} buffer}\index{buffers!\buf{#1}}}

\newenvironment{pvscmds}%
  {\par\noindent\smaller%
   \begin{tabular*}{\textwidth}{|l@{\extracolsep{\fill}}l@{\extracolsep{\fill}}l|}\hline%
     {\it Command} & {\it Aliases} & {\it Function}\\ \hline}%
  {\hline\end{tabular*}\vspace{0.1in}}

\newenvironment{pvscmdsna}%
  {\par\noindent\smaller%
   \begin{tabular*}{\textwidth}{|l@{\extracolsep{\fill}}l|}\hline%
     {\it Command} & {\it \,\,Function}\\ \hline}%
  {\hline\end{tabular*}\vspace{0.1in}}

\newcommand{\cmd}[1]{{\tt #1}}
\newcommand{\ecmd}[1]{{\tt M-x #1}}

\newcommand{\latex}{\LaTeX}                  %  LaTeX
\newcommand{\sun}{{S{\smaller\smaller UN}}}                 %  Sun
\newcommand{\sparc}{{S{\smaller\smaller PARC}}}             %  Sparc
\newcommand{\sunos}{{S{\smaller\smaller UN}OS}}             %  SunOS
\newcommand{\solaris}{{\em Solaris\/}}        %  Solaris
\newcommand{\sunview}{{S{\smaller\smaller UN}V{\smaller\smaller IEW}}} %SunView
\newcommand{\unix}{{U{\smaller\smaller NIX}}}               %  Unix
\newcommand{\lisp} {{\sc Lisp}}              %  Lisp
\newcommand{\gnu}{{Gnu Emacs}}           %  Gnu Emacs
\newcommand{\gnuemacs}{{Gnu Emacs}}      %  Gnu Emacs
\newcommand{\emacsl}{{Emacs-Lisp}}       %  Emacs Lisp
\newcommand{\shell}{{\sc Csh}}               %  C-shell

\newcommand{\update}[3]{#1\{#2\leftarrow #3\}}
\newcommand{\interp}[3]{\cal{M}\dlb {\tt #1 : #2 }\drb #3}
\newcommand{\myforall}[2]{(\forall{#1 .}\ #2)}
\newcommand{\myexists}[2]{(\exists{#1 .}\ #2)}
\newcommand{\mth}[1]{$ #1 $}
\newcommand{\labst}[2]{(\lambda{#1}.\ #2)}
\newcommand{\app}[2]{(#1\ #2)}
\newcommand{\problem}[1]{{\bf Exercise: } {\em #1}}
\newcommand{\rectype}[1]{[\# 1 \#]}
\newcommand{\recttype}[1]{{\tt [\# 1 \#]}}
\newcommand{\dlb}{\lbrack\!\lbrack}
\newcommand{\drb}{\rbrack\!\rbrack}
\newcommand{\cross}{\times}
\newcommand{\key}[1]{{\tt #1}}
\newcommand{\keyindex}[1]{\index{#1@\key{#1}}}
\newcommand{\ikey}[1]{\key{#1}\keyindex{#1}}
\newcommand{\keyword}[1]{{\smaller\texttt{#1}}}

\newenvironment{keybindings}%
  {\begin{center}\begin{tabular}{|l|l|}\hline Key & Function\\ \hline}%
  {\hline\end{tabular}\end{center}}
\def\rmif{\mbox{\bf if\ }}
\def\rmiff{\mbox{\bf \ iff \ }}
\def\rmthen{\mbox{\bf \ then }}
\def\rmelse{\mbox{\bf \ else }}
\def\rmend{\mbox{\bf end}}
\def\rmendif{\mbox{\bf \ endif}}
\def\rmotherwise{\mbox{\bf otherwise}}
\def\rmwith{\mbox{\bf \ with\ }}
\def\mapb{\char"7B\char"7B}
\def\mape{\char"7D\char"7D}
\def\setb{\char"7B}
\def\sete{\char"7D}

% ---------------------------------------------------------------------
% Macros for little PVS sessions displayed in boxes.
%
% Usage: (1) \setcounter{sessioncount}{1} resets the session counter
%
%        (2) \begin{session*}\label{thissession}
%             .
%              < lines from PVS session >
%             .
%            \end{session*}
%
%            typesets the session in a numbered box in ALLTT mode.
%
%  session instead of session* produces unnumbered boxes
%
%  Author: John Rushby
% ---------------------------------------------------------------------
\newlength{\hsbw}
\newenvironment{session}{\begin{flushleft}
 \setlength{\hsbw}{\linewidth}
 \addtolength{\hsbw}{-\arrayrulewidth}
 \addtolength{\hsbw}{-\tabcolsep}
 \begin{tabular}{@{}|c@{}|@{}}\hline 
 \begin{minipage}[b]{\hsbw}
% \begingroup\small\mbox{ }\\[-1.8\baselineskip]\begin{alltt}}{\end{alltt}\endgroup\end{minipage}\\ \hline
 \begingroup\sessionsize\vspace*{1.2ex}\begin{alltt}}{\end{alltt}\endgroup\end{minipage}\\ \hline
 \end{tabular}
 \end{flushleft}}
\newcounter{sessioncount}
\setcounter{sessioncount}{0}
\newenvironment{session*}{\begin{flushleft}
 \refstepcounter{sessioncount}
 \setlength{\hsbw}{\linewidth}
 \addtolength{\hsbw}{-\arrayrulewidth}
 \addtolength{\hsbw}{-\tabcolsep}
 \begin{tabular}{@{}|c@{}|@{}}\hline 
 \begin{minipage}[b]{\hsbw}
 \vspace*{-.5pt}
 \begin{flushright}
 \rule{0.01in}{.15in}\rule{0.3in}{0.01in}\hspace{-0.35in}
 \raisebox{0.04in}{\makebox[0.3in][c]{\footnotesize \thesessioncount}}
 \end{flushright}
 \vspace*{-.57in}
 \begingroup\small\vspace*{1.0ex}\begin{alltt}}{\end{alltt}\endgroup\end{minipage}\\ \hline 
 \end{tabular}
 \end{flushleft}}
\def\sessionsize{\footnotesize}
\def\smallsessionsize{\small}
\newenvironment{smallsession}{\begin{flushleft}
 \setlength{\hsbw}{\linewidth}
 \addtolength{\hsbw}{-\arrayrulewidth}
 \addtolength{\hsbw}{-\tabcolsep}
 \begin{tabular}{@{}|c@{}|@{}}\hline 
 \begin{minipage}[b]{\hsbw}
 \begingroup\smallsessionsize\mbox{ }\\[-1.8\baselineskip]\begin{alltt}}{\end{alltt}\endgroup\end{minipage}\\ \hline 
 \end{tabular}
 \end{flushleft}}
\newenvironment{spec}{\begin{flushleft}
 \setlength{\hsbw}{\textwidth}
 \addtolength{\hsbw}{-\arrayrulewidth}
 \addtolength{\hsbw}{-\tabcolsep}
 \begin{tabular}{@{}|c@{}|@{}}\hline 
 \begin{minipage}[b]{\hsbw}
 \begingroup\small\mbox{
}\\[-0.2\baselineskip]}{\endgroup\end{minipage}\\ \hline 
 \end{tabular}
 \end{flushleft}}
\newcommand{\memo}[1]{\mbox{}\par\vspace{0.25in}%
\setlength{\hsbw}{\linewidth}%
\addtolength{\hsbw}{-2\fboxsep}%
\addtolength{\hsbw}{-2\fboxrule}%
\noindent\fbox{\parbox{\hsbw}{{\bf Memo: }#1}}\vspace{0.25in}}
\newcommand{\nb}[1]{\mbox{}\par\vspace{0.25in}\setlength{\hsbw}{\linewidth}\addtolength{\hsbw}{-1.5ex}\noindent\fbox{\parbox{\hsbw}{{\bf Note: }#1}}\vspace{0.25in}}

%%% Local Variables: 
%%% mode: latex
%%% TeX-master: t
%%% End: 

%%
%
%    LaTeX  Macro File  /usr2/jcm/tex/macros.tex
%
%
\tracingonline=0 % shorter error messages (on screen)

%\long\def\comment#1{} % mulit-line comments
%\newcommand{\note}[1]{\fbox{#1}}

% font changes  (as function calls, scribe style)

\renewcommand{\i}[1]{{\it #1\/}}       % italics with space correction
%\newcommand{\emex}[1]{\/{\em #1}}   % emphasis in example, exercise, theorem, etc.
\renewcommand{\c}[1]{{\sc #1}}       % small caps (eliminates \c as cedilla)
%\newcommand{\r}[1]{{\rm #1}}            % for roman font in math mode
%\newcommand{\s}[1]{{\scr #1}}          % script (for use with tatex) 
\newcommand{\s}[1]{{\cal #1}}
\renewcommand{\b}[1]{{\bf #1}}           % bold face
\newcommand{\calg}[1]{{\cal #1}}          % caligraphic 

%\newcommand{\q}[1]{``#1''}    % matching quotes for in-line quotation

% numbered environments

%\newcounter{partcounter}
%\setcounter{partcounter}{0}
%\renewcommand{\part}[1]{\newpage \addtocounter{partcounter}{1}
%\noindent{\Large \bf Part \Roman{partcounter}. #1 } \\[1ex]}

\newtheorem{thm}{Theorem}[section]
\newtheorem{theorem}[thm]{Theorem}
\newtheorem{lemma}[thm]{Lemma}
\newtheorem{cor}[thm]{Corollary}
\newtheorem{corollary}[thm]{Corollary}
\newtheorem{claim}[thm]{Claim}
\newtheorem{prop}[thm]{Proposition}
\newtheorem{conj}[thm]{Conjecture}
\newtheorem{definition}[thm]{Definition}
\newtheorem{exercise}[thm]{Exercise}
\newtheorem{example}[thm]{Example}
\newtheorem{remark}[thm]{Remark}
\newtheorem{open}[thm]{Open Problem}

%\newcommand{\proof}{\\{\bf Proof.}\ }
\newenvironment{proof}{{\bf Proof. }}{\thmbox}

% axiom and inference rule (centered, in math mode,  with name at left)

\newcommand{\axiom}[2]
{\[ \hbox to \columnwidth
    { \rlap{$#2$} \hfil {$ #1 $} \hfil }
\]}

\newcommand{\infrule}[4]
{\[ \hbox to \columnwidth %\textwidth
    { \rlap{$#4$} \hfil $
      \frac {\strut\displaystyle #1 } {\strut\displaystyle #2 } \; \rlap{$#3$} \hfil $
      \hfil }
\]}
\newcommand{\infruletw}[4]
{\[ \hbox to \textwidth
    { \rlap{$#4$} \hfil $
      \frac {\strut\displaystyle #1 } {\strut\displaystyle #2 } \; \rlap{$#3$} \hfil $
      \hfil }
\]}

%\newcommand{\infrule}[4]
%{\[ \hbox to \columnwidth
%    { \rlap{$#4$} \hfil $
%      {{\displaystyle\strut #1}\over{\displaystyle\strut #2}}\quad\makebox[0pt][l]{\it #3} $
%      \hfil }
%\]}

% axiom and inference rule macros for use in tables, etc.
% presumed in math mode #1=axiom, #2=side condition
\newcommand{\Axiom}[2]{
{\displaystyle\strut #1}\qquad\makebox[0pt][l]{\it #2}
}
% presumed in math mode #1=top, #2=bottom, #3=side condition
\newcommand{\Infrule}[3]{
{{\displaystyle\strut #1}\over{\displaystyle\strut #2}}\;\mbox{\scriptsize$\bf #3$}
}

% sequence of #1's, numbered up to #2  (e.g., seq{x}{n} for x1, ..., xn   )

\newcommand{\seq}[2]{#1_{1} \ldots #1_{#2}}  
%\newcommand{\ie}{{\it i.e.}}
%\newcommand{\eg}{{\it e.g.}}
%\newcommand{\cf}{{\it c.f.\,}}

% common symbols

\newcommand{\union}{\cup}
\newcommand{\intersect}{\cap}
\newcommand{\subs}{\subseteq}
\newcommand{\el}{\in}
\newcommand{\nel}{\not\in}
\newcommand{\ns}{\emptyset}
\newcommand{\compose}{\circ}
\newcommand{\set}[2]{ \{\, #1 \,\mid\, #2 \,\}  } % set macro
\newcommand{\infinity}{\infty}
\newcommand{\pair}[1]{\langle #1 \rangle}
\newcommand{\tuple}[1]{\langle #1 \rangle}


\newcommand{\fa}{\forall}
\newcommand{\te}{\exists}
\newcommand{\imp}{\supset}
\newcommand{\implies}{\supset}
\newcommand{\ts}{\vdash}
\newcommand{\dts}{\models}

\newcommand{\aro}{\mathord\rightarrow} % see pages 154-155 of TeX manual
\newcommand{\paro}{\rightharpoonup} 
\newcommand{\karo}{\mathop\Rightarrow} % see pages 154-155 of TeX manual
%\newcommand{\cross}{\times}
%\newcommand{\dlb}{\lbrack\!\lbrack}
%\newcommand{\drb}{\rbrack\!\rbrack}
\newcommand{\mean}[1]{\lbrack\!\lbrack #1 \rbrack\!\rbrack}
\newcommand{\lam}{\lambda}
\newcommand{\subst}[2]{{}[#1/#2]}
\renewcommand{\dot}{\mathrel{\bullet}}
\newcommand{\Dinf}{D_{\infty}}
\newcommand{\bottom}{\perp}

\mathcode`:="603A  % treat : as punctuation instead of relation in math mode
\mathchardef\colon="303A	% relation colon

\newcommand{\eqdef}{\mathrel{:=}}
\newcommand{\Dom}{\mathop{\rm dom}}
\newcommand{\Pow}{\mathop{\rm Pow}}

\newcommand{\aequiv}{\equiv_\alpha}
\newcommand{\baro}{\buildrel \beta \over \rightarrow}
\newcommand{\earo}{\buildrel \eta \over \rightarrow}
\newcommand{\red}{\rightarrow\!\!\!\!\rightarrow}
\newcommand{\backred}{\leftarrow\!\!\!\!\leftarrow}
\newcommand{\bred}{\buildrel \beta \over \red}
\newcommand{\ered}{\buildrel \eta \over \red}
\newcommand{\conv}{\leftrightarrow}
\newcommand{\beconv}{\buildrel {\beta, \eta} \over\leftrightarrow}

% lambda calculus abbreviations
\newcommand{\letdec}[3]{\b{let\ } #1 = #2 \b{\ in\ } #3}
\newcommand{\letrec}[3]{\b{letrec\ } #1 = #2 \b{\ in\ } #3}

\newcommand{\thmbox}
   {{\ \hfill\hbox{%
      \vrule width1.0ex height1.0ex
   }\parfillskip 0pt }}
\newcommand{\qed}{\thmbox}


% make single spacing

\newcommand{\singlespace}{\renewcommand{\baselinestretch}{1}\@normalsize}
\newcommand{\etal}{{\em et al.}}

% bycase command for definition by cases (pg 49 of LaTeX)

\newcommand{\bycase}[1]
	{\left\{ \begin{array}{ll}  #1  \end{array} \right. }


% make \cite put blanks after the comma (use in alpha style)

%\def\@citex[#1]#2{\if@filesw\immediate\write\@auxout{\string\citation{#2}}\fi
%  \def\@citea{}\@cite{\@for\@citeb:=#2\do
%    {\@citea\def\@citea{,\penalty100\hskip2.5pt plus1.5pt minus.8pt}%
%       \@ifundefined{b@\@citeb}{{\bf ?}\@warning
%       {Citation `\@citeb' on page \thepage \space undefined}}
%\hbox{\csname b@\@citeb\endcsname}}}{#1}}

\newcommand{\Infrule}[3]{
{{\displaystyle\strut #1}\over{\displaystyle\strut #2}}\;\mbox{\scriptsize$\bf #3$}
}
\newcommand{\imp}{\supset}
\newcommand{\pair}[1]{\langle #1 \rangle}
\newcommand{\seq}[2]{#1_{1} \ldots #1_{#2}}  
\newcommand{\implies}{\supset}

\def\id#1{\hbox\texttt{#1}} %changing ids from roman to tt.
%\bibliographystyle{alpha}
%\input{title}
\vspace{4in}

\begin{titlepage}
\vspace*{1in}
\noindent
\rule[1pt]{\textwidth}{2pt}
\begin{center}
\newfont{\pvstitle}{cmss17 scaled \magstep4}
\textbf{\pvstitle PVS Prover Guide}
\end{center}
\begin{flushright}
{\Large Version 3.2 {\smaller$\bullet$} September 2004}
\end{flushright}
\rule[1in]{\textwidth}{2pt}
\vspace*{2in}
\begin{flushleft}
N.~Shankar\\
S.~Owre\\
J.~M.~Rushby\\
D.~W.~J.~Stringer-Calvert\\
{\smaller\url{{Owre,Shankar,Rushby,Dave_SC}@csl.sri.com}}\\
{\smaller\url{http://pvs.csl.sri.com/}}
\end{flushleft}
\vspace*{1in}
\vbox{\hbox to \textwidth{{\Large SRI International\hfill}}%
\hbox to \textwidth{{\small\sf
Computer Science Laboratory $\bullet$ 333 Ravenswood Avenue $\bullet$ Menlo Park CA 94025\hfill}}}
\end{titlepage}

\pagestyle{fancy}
\renewcommand{\chaptermark}[1]{\markboth{\emph{#1}}{}\markright{\emph{#1}}}
\renewcommand{\sectionmark}[1]{\markright{\thesection \em \ #1}}
\lhead[\thepage]{\rightmark}
%\cfoot{\protect\small\bf \fbox{PVS 2.3 DRAFT}}
\cfoot{}
\rhead[\leftmark]{\thepage}
\setcounter{secnumdepth}{1} 
\setcounter{tocdepth}{2}
\pagenumbering{roman}
\thispagestyle{empty}

\newpage

\vspace*{6in}\noindent
The initial development of PVS was funded by SRI International.
Subsequent enhancements were partially funded by SRI and by NASA
Contracts NAS1-18969 and NAS1-20334, NRL Contract N00014-96-C-2106,
NSF Grants CCR-9300044, CCR-9509931, and CCR-9712383, AFOSR contract
F49620-95-C0044, and DARPA Orders E276, A721, D431, D855, and E301.
\newpage\setcounter{page}{1}

\tableofcontents
\cleardoublepage
\pagenumbering{arabic}
\setcounter{page}{1}

\chapter{Introduction}

PVS stands for ``Prototype Verification System,'' and as the name suggests,
it is a prototype environment for specification and verification.  This document is a
reference manual for the commands employed in constructing proofs using the
PVS proof checker.  The \emph{PVS System Guide}~\cite{PVS:userguide} should
be consulted for information on how to use the system to develop
specifications and proofs.  The \emph{PVS Language
Reference}~\cite{PVS:language} describes the language of PVS\@. 

The primary purpose of PVS is to provide formal support for
conceptualization and debugging in the early stages of the life cycle of
the design of a hardware or software system.  In these stages, both the
requirements and designs are expressed in abstract terms that are not
necessarily executable.  We find that such abstract specifications are
best analyzed by attempting proofs of desirable consequences of the
specification.  Our own experience with PVS in this regard has been that
such attempted proofs of \emph{putative theorems\index{putative theorems}}
very quickly highlight even subtle errors and infelicities.  These would
be costly to detect and correct at later stages of the design life cycle.

The primary emphasis in the PVS proof checker is on supporting the
construction of readable proofs.  The  automation underlying
PVS serves to ensure that the process of verification yields human
insights that can be easily communicated to other humans, and
encapsulated for future verifications.  PVS therefore pays a lot of
attention to simplifying the process of developing, debugging,
maintaining, and presenting proofs.  In order to make proofs easier to
develop, the PVS proof checker provides a collection of powerful proof
commands to carry out propositional, equality, and arithmetic reasoning
with the use of definitions and lemmas.  These proof commands can be
combined to form \emph{proof strategies}.  To make proofs easier to
debug, the PVS proof checker permits proof steps to be undone, and it
also allows the specification to be modified during the course of a
proof.  To support proof maintenance, PVS allows proofs (and partial
proofs) to be edited and rerun.  Currently, the proofs generated by
PVS can be made presentable but they still fall short of being humanly
readable.   The readability of proofs will be one focus of future
enhancements to PVS\@.

PVS is meant to provide effective theorem proving support for a richly
expressive specification language.  The combination of an expressive logic
and a powerful theorem proving capability in PVS hinges on a careful
integration between the typechecker and the proof checker.  The
typechecker exploits the deductive power of the proof checker to
automatically discharge proof obligations generated by the typechecker.
These proof obligations, termed \emph{type correctness conditions} (or
TCCs\index{TCC}), arise for instance, when a term is typechecked against
an expected predicate subtype.  Such proof obligations can also arise as
subgoals during proof checking since the typechecker is frequently invoked
to check user-supplied expressions and quantifier instantiations.

The combination of direct control by the user for the higher levels of
proof development, and powerful automation for the lower levels, is also
somewhat unusual.  On the whole, PVS provides more automation than a
low-level proof checker (such as LCF\index{LCF}~\cite{LCF},
HOL\index{HOL}~\cite{Gordon:HOL88}, Nuprl\index{Nuprl}~\cite{Nuprl-book},
Automath\index{Automath}~\cite{deBruijn80}), and more control than a
highly automatic theorem prover (such as Otter\index{Otter}~\cite{Otter90}
or Nqthm\index{Nqthm}~\cite{Boyer-Moore79,Boyer-Moore88}).  Compared with
thoroughly automated theorem provers such as Nqthm\index{Nqthm}, the
deductive component of PVS may be considered a proof checker---but it
seems like a theorem prover to those accustomed to systems such as
HOL\index{HOL} which provide limited automation.  We reflect this
ambivalence by sometime referring to PVS as a theorem prover, and
sometimes as a proof checker.  The PVS proof checker is somewhat in the
spirit to the IMPLY\index{IMPLY} prover of Bledsoe\index{Bledsoe, Woodrow
W.} and his colleagues~\cite{Bledsoe74}.

While there are clearly many avenues for further improvement of the PVS
system, the combination of a highly expressive specification language and
a powerful interactive proof checking capability already yields a
productive verification environment.  There are a number of examples, both
big and small, that support this
observation---see~\cite{Owre-etal98:FM-TRENDS}.  A list of applications of
PVS and a bibliography of PVS related reports and papers is maintained at
the PVS web site at \url{http://pvs.csl.sri.com/}.



\section{PVS Proof Display and Construction}

We give a brief overview of the sequent-style proof representation used in
PVS since this is needed to understand the effect of the PVS proof
commands.  The PVS proof checker is interactive, but also supports a batch
mode in which proofs can be easily rerun.  The prover maintains a
\emph{proof tree}, and it is the goal of the user to construct a proof
tree which is complete, in the sense that all of the leaves are recognized
as true.  Each node of the proof tree is a \emph{proof goal} that follows
from its offspring nodes by means of a proof step.  Each proof goal is a
\emph{sequent} consisting of a sequence of formulas called
\emph{antecedents} and a sequence of formulas called \emph{consequents}.
Such a sequent is displayed as
\begin{center}
\begin{tabular}{ll}
  \texttt{\{-1\}} & $A_1$\\
  \texttt{\{-2\}} & $A_2$\\
  \texttt{[-3]} & $A_3$\\
 & \vdots\\
 \multicolumn{2}{l}\texttt{|-------}\\
  \texttt{\{1\}} & $B_1$\\
  \texttt{[2]} & $B_2$\\
  \texttt{\{3\}} & $B_3$\\
 & \vdots
\end{tabular}
\end{center}
where the $A_i$ and $B_j$ are PVS formulas collectively referred to
as \emph{sequent formulas}: the $A_i$ are the antecedents and the
$B_j$ are the consequents; the row of dashes serves to separate the
antecedents from the consequents.\footnote{In written text, sequents
may also be written as $A_1, A_2, A_3, \ldots\vdash B_1, B_2, B_3,
\ldots$} The sequence of antecedents or consequents (but not both) may
be empty.  The intuitive interpretation of a sequent is that the
conjunction of the antecedents  implies the disjunction of the
consequents, \ie\ $(A_1\wedge A_2\wedge A_3 \ldots) \supset (B_1 \vee
B_2 \vee B_3 \ldots)$.  The proof tree starts off with a root node of
the form $ \vdash A$, where $A$ is the theorem to be proved.   PVS
proof steps build a proof tree by adding subtrees to leaf nodes
as directed by the proof commands. 
It is easy to see that a sequent is \emph{true} if any antecedent is the same as any consequent, if any
antecedent is \emph{false}, or if any consequent is \emph{true}.
Other sequents can also be recognized as \emph{true}, using more
powerful inferences that will be described later.
Once a sequent is recognized as \emph{true}, that branch of the
proof tree is terminated.   The goal is to build a proof tree whose
branches have all been terminated in this way.

At any time in a PVS proof, attention is focussed on some sequent that is
a leaf node in the current proof tree---this is the sequent that is
displayed by the PVS prover while awaiting the user's command.  The
numbers in brackets, \eg\ \texttt{[-3]}, and braces, \eg\ \texttt{\{3\}},
before each formula in the displayed sequent are used to name the
corresponding formulas.  The formula numbers in square brackets (\eg\
\texttt{[-3]} above) indicate formulas that are unchanged in a subgoal
from the parent goal whereas the numbers in braces (\eg\ \{2\} in the
example above), serve to highlight those formulas that are either new or
modified from those of the parent sequent.

PVS \emph{interactive commands} allow the user to shift the focus (using
the \indtt{postpone} command) to a sibling of the current sequent (if any), or
to abandon (using the \indtt{fail} or \indtt{undo} command) a portion of the proof
containing the current sequent in order to return to some ancestor node
representing an earlier point in the proof.  PVS \emph{proof steps} cause a
subtree of sequents to ``grow'' from the current sequent, and shift the focus
to one of the leaves of the new subtree.  For example, one proof step (called
\indtt{split} in PVS) takes a sequent of the form \[ \Gamma \vdash A \wedge B
\] (where $\Gamma$ is any sequence of formulas) and creates the pair of child
sequents \[ \Gamma \vdash A \ \mbox{\ and\ }\ \Gamma \vdash B\] (\ie\ in order
to prove a conjunction, it is sufficient to prove each of the conjuncts
separately).

A PVS \emph{proof command} when applied to a sequent provides the
means to construct proof trees.  These commands can be used to introduce
lemmas, expand definitions, apply decision procedures, eliminate
quantifiers, and so on; they affect the proof tree, and are saved when
the proof is saved.  Proof commands may be invoked directly by the user,
or as the result of executing a strategy.   We refer to the
action resulting from a proof command as a \emph{proof step} or a \emph{proof
rule} and often use  these terms interchangeably.  

The proof commands that really define the PVS logic are called the
\emph{primitive rules}; they either recognize the current sequent as
true and terminate that branch of the proof tree, or they add one or
more child nodes to the current sequent and transfer the focus to one of
these children.  PVS \emph{strategies} are combinations of
proof-steps that can, in principle, add a subtree of any depth to the
current node (\ie\ the step may invoke substeps and so on).  On the
other hand, those proof steps called \emph{defined rules} (which can be
the result of invoking strategies with the \indtt{apply} control strategy)
silently prune those branches of the subtrees which they generate that
are recognized as true, and collapse all remaining interior nodes, so
that the subtree actually generated has depth zero (\ie\ the sequent is
recognized as true and this branch of the proof terminates) or one (\ie\
it simply adds children to the current node).

As mentioned earlier, some of the individual proof steps in the PVS prover
are extremely sophisticated and make heavy use of arithmetic and equality
decision procedures.  Various properties of function, record, tuple, and
cotuple types, and abstract datatypes, are also built into the operation
of the PVS prover.  The interplay between type information (from the
specification) and inference is also mechanized in a significant way by
PVS.\@ For example, if the function definition \[\textit{factorial}(n)
{\bf :\ recursive\ } \textit{nat} = \rmif n=0 \rmthen 1 \rmelse n\times
\textit{factorial}(n-1) \rmendif\] is used to expand the term
$\textit{factorial}(i+1)$, where $i$ is of type \emph{nat}, the PVS prover
will retrieve the type predicate for \emph{nat}, namely \[(\lambda n: n
\geq 0),\] instantiate it with $i$ and call the arithmetic decision
procedures to deduce that $i+1 \neq 0$, and thereby select just the
relevant branch of the definition to produce the result \[ (i+1)\times
\textit{factorial}(i+1-1). \]

Though there are only a few proof commands in PVS, many of these commands
are quite powerful and flexible.  It is wise to experiment with the
commands in order to more thoroughly understand how they work, and to
employ the more powerful commands whenever possible.  As with any
automated reasoning system, the form of the specification can
significantly affect the ease or difficulty of the accompanying proofs.
The specifier must demonstrate good taste in writing abstract
specifications, using definitions to name useful concepts, employing
types, subtypes, and abstract datatypes appropriately, and in stating
lemmas in their most useful forms.  In a system like PVS, it is quite easy
to carry out a less than elegant proof; the user must exercise enough
discipline to structure proofs so that they are less cluttered, easy to
read, and can be robustly rerun in the face of minor changes.  It is also
important to introduce useful lemmas as they arise in the proof, and
define strategies to encapsulate patterns of proof steps.

The remainder of this chapter summarizes how interactive PVS proof
attempts are initiated and terminated.  Chapter~\ref{example} contains a
small example proof to illustrate how the PVS proof checker is used.
Chapter~\ref{logic} gives a brief overview of the logical underpinnings of
PVS\@.  Chapter~\ref{commands} describes the syntax of the proof commands.
Chapter~\ref{strategy} is a guide to the PVS proof strategy language and
also gives examples of proof strategies and derived inference rules.

\section{Interaction Basics}\index{Interaction basics}\label{interaction}

The following paragraphs summarize how proof attempts are initiated,
abandoned, or interrupted, and how help information can be obtained.
Full details are presented in the PVS user guide~\cite{PVS:userguide}.

\prparagraph{Initiating a Proof Attempt}{Initiating Proofs}

A proof session is initiated from within an Emacs buffer containing a PVS
specification by using the Emacs command \emacstt{pr} with the cursor at
the formula that is to be proved.  If the formula has already been proved,
the user is asked whether the proof attempt should proceed.  If a proof or
partial proof for the relevant formula already exists, then the user is
asked if this proof should be rerun.  The Emacs command \emacstt{xpr} can
be used to initiate a proof that generates a Tcl/Tk display of the proof
structure as it is being developed.  To interactively rerun a proof, use
\emacstt{step-proof} (\emacstt{x-step-proof} to also generate a display).

\prparagraph{Exiting a Proof Attempt}{Exiting Proofs}

In the proof sessions shown below, all user input is displayed in
boldface.  The proof commands follow the \texttt{Rule?}\ prompt and the
remaining text is generated by PVS\@.  A proof attempt can be abandoned by
typing \texttt{q} or \indtt{quit} at the prompt.  At the end of a
successful or abandoned proof attempt, the user is queried as to whether
the resulting partial proof should be saved.  The timing characteristics
are displayed at the end.  The saved partial proof can be rerun in a
subsequent attempt so that the unfinished parts of the proof can be
completed.  Multiple proofs may be saved for a given formula, allowing new
proof approaches to be tried without losing earlier attempts.

\prparagraph{Getting Help}{Getting Help}

There are several ways of getting helpful information about the
proof checking commands.  The easiest way is to invoke
\emacstt{help-pvs-prover} or \emacstt{x-prover-commands}.  See
page~\pageref{help} for the interactive \indtt{help} rule.

\prparagraph{Interrupting Proofs}{Interrupting Proofs}

The proof checker can be interrupted when it is working on a command by
typing \indtt{C-c C-c}.  This places the system at a Lisp break, where it
is possible to interact with the underlying Lisp system.  Typing
\texttt{(restore)} at the break returns the system to the \texttt{Rule?}\
prompt corresponding to the last interaction.



\chapter{An Example Proof}\index{Proof example}
\label{example}

We consider a simple proof using induction to show that
when given two functions \texttt{f} and \texttt{g} on the natural numbers,
the sum of the first \texttt{n} values of \texttt{f} and \texttt{g}
is the same as the sum of the first \texttt{n} values of
the function \texttt{(LAMBDA n:~f(n) + g(n))}.
The theory \texttt{sum} below defines the summation operator \texttt{sum}
and states the desired theorem as \texttt{sum\_plus}.

\pvstheory{summation3}{\texttt{sum}}{sum}

In the first proof attempt described below,  \texttt{sum\_plus} is proved
using simple, low-level inference steps, and in the second proof attempt, the
same theorem is proved by invoking a single high-level proof strategy.

Once the proof is initiated\footnote{By placing the cursor on the formula
and typing \emacstt{prove}.  See the previous chapter or the System
Guide~\cite{PVS:userguide} for details on other ways to initiate proof
attempts.}, the main goal is displayed in the \texttt{*pvs*} buffer
followed by a \texttt{Rule?}\ prompt.  The user commands are typed in at
this prompt.  Note that the free \texttt{n} in the original formula has
been renamed to \texttt{n1}.  This is done both to make the formula less
confusing to read, and to ensure that variables are not inadvertently
captured.\footnote{PVS keeps internal pointers from variable references to
the bound variables, but if an expression is cut from the sequent and
pasted into a command argument, this information is lost and the results
can be confusing.}  The first command, \skobang{}, introduces Skolem
constants \texttt{f!1}, \texttt{g!1}, and \texttt{n1!1} for the
universally quantified variables in the theorem.  The second command,
\indtt{lemma}, introduces the induction scheme for natural numbers
\texttt{nat\_induction}\index{nat\_induction@{\texttt{nat\_induction}}} as
an antecedent formula.  This induction scheme is proved as a lemma in the
theory \indtt{naturalnumbers} in the PVS prelude.\footnote{The lemma name
can also be given in its full form \texttt{naturalnumbers.nat\_induction}
if the theory name is needed for disambiguation.  See the PVS language
manual~\cite{PVS:language} for more details of how name resolution is
performed.}

\clearpage
\noindent
\begin{smallsession}\indskobang\indextt{lemma}
sum_plus :  

  |-------
\{1\}   FORALL (f, g: [nat -> nat], n1: nat):
        sum(LAMBDA n: f(n) + g(n), n1) = sum(f, n1) + sum(g, n1)

Rule? (skolem!)
Skolemizing,
this simplifies to: 
sum_plus :  

  |-------
\{1\}   sum(LAMBDA n: f!1(n) + g!1(n), n1!1) =
       sum(f!1, n1!1) + sum(g!1, n1!1)

Rule? (lemma "nat_induction")
Applying nat_induction where 
this simplifies to: 
sum_plus :  

\{-1\}   FORALL (p: pred[nat]):
        (p(0) AND (FORALL j: p(j) IMPLIES p(j + 1))) IMPLIES
         (FORALL i: p(i))
  |-------
[1]   sum(LAMBDA n: f!1(n) + g!1(n), n1!1) =
       sum(f!1, n1!1) + sum(g!1, n1!1)
\end{smallsession}

The next step is to instantiate the induction scheme with a suitable
induction predicate.  This instantiation is supplied manually using
the \indtt{inst} command.   
\begin{smallsession}\indextt{inst}
Rule? (inst - "(LAMBDA n: sum((LAMBDA (n: nat): f!1(n) + g!1(n)), n)
          = sum(f!1, n) + sum(g!1, n))")
Instantiating the top quantifier in - with the terms: 
 (LAMBDA n: sum((LAMBDA (n: nat): f!1(n) + g!1(n)), n)
          = sum(f!1, n) + sum(g!1, n)),
this simplifies to:
\end{smallsession}
The effect of \indtt{inst} command is to generate a subgoal
where the universally quantified variable \texttt{p} has been
replaced by the given induction predicate.  After substitution the result
was beta reduced.
\begin{smallsession}
sum_plus :  

\{-1\}   (sum((LAMBDA (n: nat): f!1(n) + g!1(n)), 0) =
        sum(f!1, 0) + sum(g!1, 0)
        AND
        (FORALL j:
           sum((LAMBDA (n: nat): f!1(n) + g!1(n)), j) =
            sum(f!1, j) + sum(g!1, j)
            IMPLIES
            sum((LAMBDA (n: nat): f!1(n) + g!1(n)), j + 1) =
             sum(f!1, j + 1) + sum(g!1, j + 1)))
       IMPLIES
       (FORALL i:
          sum((LAMBDA (n: nat): f!1(n) + g!1(n)), i) =
           sum(f!1, i) + sum(g!1, i))
  |-------
[1]   sum(LAMBDA n: f!1(n) + g!1(n), n1!1) =
       sum(f!1, n1!1) + sum(g!1, n1!1)
\end{smallsession}

Applying the conjunctive splitting command \indtt{split} to the goal
yields three subgoals.  The first goal is to demonstrate that the
conclusion of the instantiated induction scheme implies the original
conjecture following the introduction of Skolem constants.  The second
subgoal is the base case, and the third subgoal is the induction step.
The first subgoal is easily proved by using the heuristic instantiation
command \indtt{inst?}.
\begin{smallsession}\indextt{split}
Rule? (split)
Splitting conjunctions,
this yields  3 subgoals: 
sum_plus.1 :  

\{-1\}   FORALL i:
        sum((LAMBDA (n: nat): f!1(n) + g!1(n)), i) =
         sum(f!1, i) + sum(g!1, i)
  |-------
[1]   sum(LAMBDA n: f!1(n) + g!1(n), n1!1) =
       sum(f!1, n1!1) + sum(g!1, n1!1)

Rule? (inst?)
Found substitution:
i gets n1!1,
Using template: sum((LAMBDA (n: nat): f!1(n) + g!1(n)), i) =
                 sum(f!1, i) + sum(g!1, i)
Instantiating quantified variables,

This completes the proof of sum_plus.1.
\end{smallsession}

The second subgoal, the base case, contains an irrelevant formula numbered
\texttt{2} which was only needed for the first subgoal proved
above.  This formula can be suppressed with the \indtt{hide} command.
The hidden formulas can be examined using the Emacs command
\emacstt{show-hidden-formulas}, and revealed or reintroduced into the
sequent using the \indtt{reveal} command. 
\begin{smallsession}\indextt{hide}
sum_plus.2 :  

  |-------
\{1\}   sum((LAMBDA (n: nat): f!1(n) + g!1(n)), 0) =
       sum(f!1, 0) + sum(g!1, 0)
[2]   sum(LAMBDA n: f!1(n) + g!1(n), n1!1) =
       sum(f!1, n1!1) + sum(g!1, n1!1)

Rule? (hide 2)
Hiding formulas:  2,
this simplifies to: 
sum_plus.2 :  

  |-------
[1]   sum((LAMBDA (n: nat): f!1(n) + g!1(n)), 0) =
       sum(f!1, 0) + sum(g!1, 0)
\end{smallsession}
 We are then left with the formula numbered \texttt{1} which is easily
proved by expanding the definition of \texttt{sum} using the
\indtt{expand} command.  Notice that this command uses the PVS decision
procedures to simplify the definition of \texttt{sum} and to reduce the
equality to \texttt{TRUE}\@.
\begin{smallsession}\indextt{expand}
Rule? (expand "sum")
Expanding the definition of sum,
this simplifies to: 
sum_plus.2 :  

  |-------
\{1\}   TRUE

which is trivially true.

\end{smallsession}

The remaining subgoal is the induction step.  It too contains the
irrelevant formula numbered \texttt{2} that is again suppressed using the
\indtt{hide} command.  
\begin{smallsession}\indextt{hide}
sum_plus.3 :  

  |-------
\{1\}   FORALL j:
        sum((LAMBDA (n: nat): f!1(n) + g!1(n)), j) =
         sum(f!1, j) + sum(g!1, j)
         IMPLIES
         sum((LAMBDA (n: nat): f!1(n) + g!1(n)), j + 1) =
          sum(f!1, j + 1) + sum(g!1, j + 1)
[2]   sum(LAMBDA n: f!1(n) + g!1(n), n1!1) =
       sum(f!1, n1!1) + sum(g!1, n1!1)

Rule? (hide 2)
Hiding formulas:  2,
this simplifies to: 
sum_plus.3 :  

  |-------
[1]   FORALL j:
        sum((LAMBDA (n: nat): f!1(n) + g!1(n)), j) =
         sum(f!1, j) + sum(g!1, j)
         IMPLIES
         sum((LAMBDA (n: nat): f!1(n) + g!1(n)), j + 1) =
          sum(f!1, j + 1) + sum(g!1, j + 1)
\end{smallsession}
Applying the \indtt{skosimp} command, which is
a compound of the \skobang{} and \indtt{flatten} commands,
the resulting simplified sequent contains an antecedent formula,
the induction hypothesis, and a consequent formula, the induction
conclusion.
\begin{smallsession}\indextt{skosimp}
Rule? (skosimp)
Skolemizing and flattening,
this simplifies to: 
sum_plus.3 :  

\{-1\}   sum((LAMBDA (n: nat): f!1(n) + g!1(n)), j!1) =
       sum(f!1, j!1) + sum(g!1, j!1)
  |-------
\{1\}   sum((LAMBDA (n: nat): f!1(n) + g!1(n)), j!1 + 1) =
       sum(f!1, j!1 + 1) + sum(g!1, j!1 + 1)
\end{smallsession}
If we apply the \indtt{expand} command selectively to expand those
occurrences of \texttt{sum} on the consequent side, we get a sequent that
is tautologously true.  Notice, once again, that the expand command makes
significant use of type information within the PVS decision procedures in
order to simplify not only the expanded definition of \texttt{sum} but
also the resulting equality between arithmetic expressions.
\begin{smallsession}\indextt{expand}
Rule? (expand "sum" +)
Expanding the definition of sum,
this simplifies to: 
sum_plus.3 :  

[-1]   sum((LAMBDA (n: nat): f!1(n) + g!1(n)), j!1) =
       sum(f!1, j!1) + sum(g!1, j!1)
  |-------
\{1\}   sum((LAMBDA (n: nat): f!1(n) + g!1(n)), j!1) =
       sum(f!1, j!1) + sum(g!1, j!1)

which is trivially true.

This completes the proof of sum_plus.3.

Q.E.D.


Run time  = 0.79 secs.
Real time = 38.10 secs.
\end{smallsession}

This successfully completes the proof attempt.  The CPU time
and wall clock time for the proof attempt are displayed above.

\section{The Example Proof Redone}

PVS has a language in which proof strategies can be written.  The essence
of the above proof can actually be captured as a strategy.  The strategy
\indbang{} invokes induction according to the scheme appropriate to the
given induction variable and then completes the proof by expanding the
functions used in the theorem and applying heuristic instantiation and the
decision procedures.
\begin{smallsession}
sum_plus :  

  |-------
\{1\}   FORALL (f, g: [nat -> nat], n1: nat):
        sum(LAMBDA n: f(n) + g(n), n1) = sum(f, n1) + sum(g, n1)

Rule? (induct-and-rewrite! "n1")
sum rewrites sum(LAMBDA n: f!1(n) + g!1(n), 0)
  to 0
sum rewrites sum(f!1, 0)
  to 0
sum rewrites sum(g!1, 0)
  to 0
sum rewrites sum(LAMBDA n: f!1(n) + g!1(n), 1 + j!1)
  to f!1(j!1) + g!1(j!1) + sum(LAMBDA n: f!1(n) + g!1(n), j!1)
sum rewrites sum(f!1, 1 + j!1)
  to f!1(j!1) + sum(f!1, j!1)
sum rewrites sum(g!1, 1 + j!1)
  to g!1(j!1) + sum(g!1, j!1)
By induction on n1 and rewriting,
Q.E.D.


Run time  = 0.85 secs.
Real time = 6.47 secs.
\end{smallsession}

Such high-level strategies might not always succeed.  When a strategy
fails to complete a proof, it is possible to continue proving the
resulting subgoals interactively using further proof commands, or to
backtrack (using \indtt{undo}) in order to try alternative proof commands.
When a strategy is invoked with a \texttt{\char36} suffix, e.g.,
\texttt{induct-and-rewrite!\char36}\index{induct-and-rewrite"!@\texttt{induct-and-rewrite"!\char36}},
the strategy is executed so that the expanded internal steps are visible.
This mode of strategy invocation provides more information and is useful
when debugging.

The Emacs command \emacstt{show-last-proof} can be used to get a summary
of the most recently completed proof.  Note that the Emacs commands
\emacstt{add-declaration} and \emacstt{modify-declaration} can be used to
alter the specification even while a proof is progress.  Since these
commands can affect the validity of a proof, proofs should always be rerun
after completion in order to confirm their validity.  During the course of
a proof, the Emacs command \emacstt{ancestry} displays the sequence of
goals leading back to the root goal of the proof, \emacstt{siblings}
displays the sibling subgoals of the current goal,
\emacstt{show-hidden-formulas} displays the hidden formulas of the current
sequent, and \emacstt{show-auto-rewrites} displays those rewrite rules
that are automatically applied.

\chapter{The Logic of PVS} \index{Logic (of PVS)}
\label{logic}


While using the PVS proof checker, it is useful to be aware of the
rules underlying the PVS logic.  The proof rules presented here
form the theoretical basis for PVS but are not the ones that
are directly implemented in the system.   There is, of course,
a great deal more to building an effective proof checker than merely
codifying the proof rules.

The following sections present the notation used throughout this
document, then the logical rules.  Note that to be complete we should
include the type rules; these are included in the Semantics
Report~\cite{PVS:semantics}.

\section{Notation}

PVS employs a sequent calculus.  We have already introduced the
notions of \emph{sequent}, \emph{antecedent}, and \emph{consequent}.
In the following, we will use the Greek letters $\Gamma$ and $\Delta$ to
represent (finite) sequences of formulas, and latin letters $A$, $B$, and
$C$ to represent individual formulas.  As usual, these can have indices.
\emph{Inference rules} are of the form
$$\Infrule{\Gamma_1\vdash\Delta_1\quad\cdots\quad\Gamma_n\vdash\Delta_n}
{\Gamma\vdash\Delta}{R}.$$ This says that if we are given a leaf of a
proof tree of the form $\Gamma\vdash\Delta$, then by applying the rule
named {\bf R}, we may obtain a tree with $n$ new leaves.

In the following, we will be using the usual logical notation for the
connectives, quantifiers, etc.  The following table relates them to
those used in PVS.\@

\begin{center}
\begin{tabular}{|c|c|}\hline
$\neg$ & \texttt{NOT}\\\hline
$\wedge$ & \texttt{AND}, \texttt{\&}\\\hline
$\vee$ & \texttt{OR}\\\hline
$\supset$ & \texttt{IMPLIES}, \texttt{=>}\\\hline
$\iff$ & \texttt{IFF}, \texttt{<=>}\\\hline
$\forall$ & \texttt{FORALL}\\\hline
$\exists$ & \texttt{EXISTS}\\\hline
$\lambda$ & \texttt{LAMBDA}\\\hline
\end{tabular}
\end{center}
Note that an expression of the form $A$ \texttt{WHEN} $B$ is equivalent to
$B\supset A$, and hence such expressions will not be explicitly
mentioned in the rules.  A PVS \texttt{IF} expression of the form
\begin{alltt}
  IF \(A\) THEN \(B\) ELSIF \(C\) THEN \(D\) ELSE \(E\) ENDIF
\end{alltt}
will be abbreviated  below as $\texttt{IF}(A,B,\texttt{IF}(C,D,E))$

\section{The Structural Rules}

The structural rules permit the sequent to be rearranged or weakened
via the introduction of new sequent formulas into the conclusion.  
All of the structural rules can be expressed in terms of the single
powerful weakening rule  shown below.
It allows a weaker statement to
be derived from a stronger one by adding either antecedent formulas or
consequent formulas.  The relation $\Gamma_1\subseteq\Gamma_2$ holds
between two lists when all the formulas in $\Gamma_1$ occur in the list
$\Gamma_2$.

\begin{center}
$\Infrule{\Gamma_1\vdash\Delta_1}
         {\Gamma_2\vdash\Delta_2} {W}$
\hspace{1cm}\mbox{\smaller\smaller if $\Gamma_1\subseteq\Gamma_2$ and $\Delta_1\subseteq\Delta_2$}
\end{center}

Both the Contraction and Exchange rules shown below are absorbed by the
above \emph{Weakening} rule.     The \emph{Contraction} rule
allows multiple occurrences of the same sequent formula
to be replaced by a single occurrence. 

\begin{center}
\begin{tabular}{ccc}
$\Infrule{A,A, \Gamma\vdash\Delta}
         {A, \Gamma\vdash\Delta} {C\vdash}$
& \hspace{0.5in} &
$\Infrule{\Gamma\vdash A, A, \Delta}
         {\Gamma\vdash A, \Delta} {\vdash C}$
\end{tabular}
\end{center}

The \emph{Exchange} rule asserts that the order of the formulas in the
antecedent and the consequent parts of the sequent is immaterial.  It can
be stated as

\begin{center}
\begin{tabular}{c@{\hspace{0.5in}}c}
$\Infrule{\Gamma_1, B, A,  \Gamma_2\vdash\Delta}
         {\Gamma_1, A, B, \Gamma_2\vdash\Delta} {X\vdash}$
&
$\Infrule{\Gamma\vdash\Delta_1, B, A, \Delta_2}
         {\Gamma\vdash\Delta_1, A, B, \Delta_2} {\vdash X}$
\end{tabular}
\end{center}


\section{The Propositional Rules}

The rules about conjunction, disjunction, implication, and negation are
quite straightforward.  The propositional axiom rule requires the notion
of two formulas $A$ and $B$ being syntactically equivalent modulo the
renaming of bound variables.  Thus, the syntactic equivalence: $(\forall x, y:
(\lambda z: f(y, z))(x) < y) \equiv (\forall z, x: (\lambda y: f(x,
y))(z) < x)$, holds.  The Propositional Axiom rule is then given as:

\begin{center}
$\Infrule{}{\Gamma, A\,\vdash\,B,\Delta}{Ax}$
\hspace{0.5in}\mbox{\smaller\smaller where $A\equiv B$}
\end{center}

The Cut rule can be seen as a mechanism for introducing a case-split
into a proof of a sequent $\Gamma\vdash\Delta$ to yield the subgoals
$\Gamma, A\vdash\Delta$ and $\Gamma\vdash A, \Delta$, which can be seen
as assuming $A$ along one branch and $\neg A$ along the other.

\begin{center}
$\Infrule{\Gamma,A \vdash\Delta\qquad\Gamma\vdash A, \Delta}
         {\Gamma\vdash\Delta}{Cut}$
\end{center}

There are two rules for each of the propositional connectives of
conjunction ($\wedge$), disjunction ($\vee$), implication ($\imp$), and
negation ($\neg$), corresponding to the antecedent and consequent
occurrences of these connectives.

\begin{center}
\begin{tabular}{c@{\hspace{0.5in}}c}
$\Infrule{A, B, \Gamma\vdash \Delta}
         {A\wedge B, \Gamma\vdash\Delta}{\wedge\vdash}$
&
$\Infrule{\Gamma\vdash A, \Delta\hspace{1cm}\Gamma\vdash B, \Delta}
         {\Gamma\vdash A\wedge B, \Delta} {\vdash\wedge}$
\\[0.3in]
$\Infrule{A, \Gamma\vdash \Delta\hspace{1cm}B, \Gamma\vdash \Delta}
         {A\vee B, \Gamma\vdash \Delta} {\vee\vdash}$
&
$\Infrule{\Gamma\vdash A, B, \Delta}
         {\Gamma\vdash A\vee B, \Delta} {\vdash\vee}$
\\[0.3in]
$\Infrule{B, \Gamma\vdash \Delta\hspace{1cm} \Gamma\vdash A,\Delta}
         {A\imp B, \Gamma\vdash \Delta} {\imp\,\vdash}$
&
$\Infrule{\Gamma, A \vdash B, \Delta}
         {\Gamma\vdash A\imp B, \Delta} {\vdash\,\imp}$
\\[0.3in]
$\Infrule{\Gamma\vdash A, \Delta}
         {\Gamma,\neg A \vdash \Delta} {\neg\vdash}$
&
$\Infrule{\Gamma, A \vdash \Delta}
         {\Gamma\vdash \neg A, \Delta} {\vdash\neg}$
\\
\end{tabular}
\end{center}

\section{The Equality Rules}
The rules for equality can be stated as below.  The rules of
transitivity and symmetry for equality can be derived from these rules.
The notation $A[e]$ is used
to highlight one or more 
occurrences of $e$ in the formula $A$.  The notation $\Delta[e]$ similarly highlights occurrences
of $e$ in $\Delta$. 

\begin{center}
\begin{tabular}{c@{\hspace{0.5in}}c}
$\Infrule{}{\Gamma\vdash a=b,
\Delta}{Refl}$\hspace{.3cm}\mbox{\smaller\smaller if $a \equiv b$}
&
$\Infrule{a=b, \Gamma[b]\vdash\Delta[b]}
         {a=b, \Gamma[a]\vdash\Delta[a]} {Repl}$
\end{tabular}
\end{center}


\section{The Quantifier Rules}
The quantifier rules are stated below.
The notation $A\{x \gets t\}$ represents the result of substituting the
term $t$ for all the free occurrences of $x$ in $A$ with the possible
renaming of bound variables in $A$ to avoid capturing any free variables
in $t$.  
In the ${\bf \vdash\forall}$ and
${\bf \exists\vdash}$ rules, $a$ must be a new constant that does not
occur in the conclusion sequent.  

\begin{center}
\begin{tabular}{c@{\hspace{0.5in}}c}
$\Infrule{\Gamma, A\{x\gets t\}\vdash \Delta}
         {\Gamma, (\forall x: A)\vdash\Delta} {\forall\vdash}$
&
$\Infrule{\Gamma\vdash A\{x\gets a\}, \Delta}
         {\Gamma\vdash (\forall x: A),\Delta} {\vdash\forall}$
%\hspace{.5cm}\mbox{\smaller\smaller where $a$ is a new constant}
\\[0.3in]
$\Infrule{\Gamma, A\{x\gets a\}\vdash \Delta}
         {\Gamma, (\exists x: A)\vdash\Delta} {\exists\vdash}$
%\hspace{.5cm}\mbox{\smaller\smaller where $a$ is a new constant}
&
$\Infrule{\Gamma \vdash A\{x\gets t\}, \Delta}
         {\Gamma\vdash (\exists x: A),\Delta} {\vdash\exists}$
\end{tabular}
\end{center}

\section{Rules for \indtt{IF}}

It is extremely useful to have the branching operation \indtt{IF} in the
language for expressing conditional expressions.  For each type
$\alpha$, there is an \indtt{IF} operation with the signature $[\texttt{bool}, \alpha, \alpha \rightarrow \alpha]$.    The
transformation of $A[e]$ to $A[b]$ represents the replacement of the
highlighted occurrences of $e$ in $A$ by $b$.  
Note that for the ${\indtt{IF}\uparrow\:\vdash}$ and
${\vdash\:\texttt{IF}\uparrow}$ rules, the  $A$ in
$B[\texttt{IF}(A, b, c)]$ must not contain any free variable occurrences
that are bound in $B[\texttt{IF}(A, b, c)]$.  
The inference rules for
\indtt{IF} are:

\begin{center}
\begin{tabular}{c@{\hspace{0.5in}}c}
$\Infrule{\Gamma, \texttt{IF}(A, B[b], B[c])\vdash \Delta}
         {\Gamma, B[\texttt{IF}(A, b, c)]\vdash\Delta}
                  % The empty \tt seems to be needed; o.w. the following
                  % \texttt{IF} is the wrong size
                  {\texttt{IF}\uparrow\:\vdash}$
&
$\Infrule{\Gamma\vdash \texttt{IF}(A, B[b], B[c]), \Delta}
         {\Gamma\vdash B[\texttt{IF}(A, b, c)], \Delta}
                  {\vdash \texttt{IF} \uparrow}$
\\[0.3in]
$\Infrule{\Gamma, A, B\vdash\Delta\hspace{1cm}\Gamma, \neg A, C\vdash\Delta}
         {\Gamma, \texttt{IF}(A, B, C)\vdash \Delta}
                  {\texttt{IF} \vdash}$
&
$\Infrule{\Gamma, A\vdash B,\Delta\hspace{1cm}\Gamma, \neg A \vdash C, \Delta}
         {\Gamma\vdash \texttt{IF}(A, B, C), \Delta}
                      {\vdash \texttt{IF}}$
\end{tabular}
\end{center}

The PVS proof checker is founded on the sequent calculus rules described
above, but the actual proof construction steps provided by PVS are very
different.  We will point out those commands which relate to the rules
given above in the {\bf notes} section of the command description.


\chapter{The PVS Proof Commands}\index{Proof commands}
\label{commands}

The sequent calculus inference rules displayed in Chapter~\ref{logic} form
the basis for the proof commands used to construct proofs with PVS\@.  The
PVS proof commands are however significantly more powerful than these
simple inference rules so as to make the proof construction process more
illuminating and less tedious.  Proof commands can be typed in by the user
at the \texttt{Rule?}\ prompt or they can be automatically applied by PVS
as part of a proof strategy.  A PVS proof command when applied to a goal
sequent either
\begin{enumerate}
  \item Succeeds in proving the goal sequent
  \item Generates one or more subgoal sequents
  \item Does nothing, which provides crucial control information to
the strategy mechanism
  \item Signals a failure that is propagated up the proof tree in order
to control proof search
  \item Postpones proof construction on the current goal sequent,
transferring focus to the next remaining subgoal.
\end{enumerate}

For example, in the list of commands below, a command like
\indtt{bddsimp}  succeeds in proving goal sequents that are
just propositionally true, whereas the \indtt{case} command typically
generates two or more subgoals.  The  \indtt{skip} command does nothing, but
many other commands can also have no effect on the state of the proof
particularly when the arguments to the command cause the parser or
typechecker to signal errors.   The \indtt{fail} command is the only
command that signals failure.  Failure is used either to backtrack or
to abandon a proof.  The \indtt{postpone} command is the only
command that causes the current subgoal to be postponed. 

The commands implemented by the PVS proof checker can be classified as:
\begin{enumerate}\raggedright
  \item Help: \indtt{help}.

  \item Annotation: \indtt{comment}, \indtt{label}, \indtt{unlabel}, and
  \indtt{with-labels}.

  \item Control: \indtt{fail}, \indtt{postpone}, \indtt{quit},
  \indtt{rewrite-msg-off}, \indtt{rewrite-msg-on},
  \indtt{set-print-depth}, \indtt{set-print-length},
  \indtt{set-print-lines}, \indtt{skip}, \indtt{skip-msg}, \indtt{trace},
  \indtt{track-rewrite}, \indtt{undo}, \indtt{untrace}, and
  \indtt{untrack-rewrite}.

  \item Structural rules: \indtt{copy}, \indtt{delete}, \indtt{hide},
  \indtt{hide-all-but}, and \indtt{reveal}.

  \item Propositional rules: \indtt{bddsimp}, \indtt{case}, \indtt{case*},
  \indtt{flatten}, \indtt{flatten-disjunct}, \indtt{iff}, \indtt{lift-if},
  \indtt{prop}, \indtt{propax}, \indtt{split}, and \indtt{merge-fnums}.

  \item Quantifier rules:
    \begin{enumerate}

    \item Existential: \indtt{inst}, \indtt{inst-cp}, \indtt{inst?},
    \indtt{instantiate}, and \indtt{instantiate-one}.

    \item Universal: \indtt{detuple-boundvars}, \indtt{generalize},
    \indtt{generalize-skolem-constants}, \indtt{skolem}, \skobang{},
    \indtt{skolem-typepred}, \indtt{skosimp}, and \indtt{skosimp*}.

    \end{enumerate}

  \item Equality rules: \indtt{beta}, \indtt{case-replace}, \indtt{name},
  \indtt{name-case-replace}, \indtt{name-replace}, \indtt{name-replace*},
  \indtt{replace}, \indtt{replace*}, and \indtt{same-name}.

  \item Rules for using definitions and lemmas:

\begin{enumerate}
\item Definition expansion: \indtt{expand}, and \indtt{expand*}.

\item Using lemmas: \indtt{forward-chain}, \indtt{forward-chain*},
\indtt{forward-chain@}, \indtt{forward-chain-theory}, \indtt{lemma},
\indtt{use} and \indtt{use*}.

\item Rewriting with definitions/lemmas: \indtt{rewrite},
\indtt{rewrite-lemma}, and \indtt{rewrite-with-fnum}.
\end{enumerate}

  \item Extensionality~rules: \indtt{apply-eta},
\indtt{apply-extensionality}, \indtt{decompose-equality}, \indtt{eta},
\indtt{extensionality}, \indtt{replace-eta}, and
\indtt{replace-extensionality}.

  \item Induction rules: \indtt{induct}, \texttt{induct-and-rewrite},
\indbang{}, \indtt{induct-and-simplify}, \indtt{measure-induct},
\indtt{measure-induct+}, \indtt{measure-induct-and-simplify},
\indtt{name-induct-and-rewrite}, \indtt{rule-induct},
\indtt{rule-induct-step}, \indtt{simple-induct}, and
\indtt{simple-measure-induct}.

    
  \item Rules for simplification using decision procedures and rewriting:
\indtt{assert}, \indtt{bash}, \indtt{both-sides}, \indtt{decide},
\indtt{do-rewrite}, \indtt{grind}, \indtt{grind-with-ext},
\indtt{grind-with-lemmas}, \indtt{ground}, \indtt{lazy-grind}, \indtt{record},
\indtt{reduce}, \indtt{reduce-with-ext}, \indtt{simplify},
\indtt{simplify-with-rewrites}, and \indtt{smash}.

  \item Installation and Removal of rewrite rules: \indtt{auto-rewrite},
\indttbang{auto-rewrite}, \indttdbang{auto-rewrite},
\indtt{auto-rewrite-defs}, \indtt{auto-rewrite-explicit},
\indtt{auto-rewrite-theories}, \indtt{auto-rewrite-theory},
\indtt{auto-rewrite-theory-with-importings}, \indtt{install-rewrites},
\indtt{stop-rewrite}, and \indtt{stop-rewrite-theory}.

  \item Making type constraints explicit: \indtt{all-typepreds},
\indtt{typepred}, and \indttbang{typepred}.

  \item Abstraction and Model Checking: \indtt{abstract},
  \indtt{abstract-and-mc}, \indtt{abs-simp},
   \indtt{model-check}, and \indtt{musimp}.

  \item Converting a strategy to a rule: \indtt{apply}.

  \item Default strategy: \indtt{default-strategy}.

  \item Strategies: \indtt{branch}, \indtt{checkpoint}, \indtt{else},
  \indtt{if}, \indtt{just-install-proof}, \indtt{let}, \indtt{query*},
  \indtt{quote}, \indtt{repeat}, \indtt{repeat*}, \indtt{rerun},
  \indtt{spread}, \indttbang{spread}, \indtt{spread@}, \indtt{then},
  \indtt{then@}, \indtt{time}, \indtt{try}, and
  \indtt{try-branch}


\end{enumerate}

\section{Formal and Actual Parameters of Rules}
Each of the proof commands takes a list of zero or more required and
optional parameters.  Each optional parameter has an associated default
value.  If the  \cargdflt{}{[default]} part of an optional parameter is
missing, it is taken to be \texttt{nil}.  A rule with its formal parameter
list is presented in the form:
\begin{center}
  \texttt{(\carg{\bkt{rulename} \bkt{required}$^{\textstyle *}$
        \optl\ \bkt{\cargdflt{optional}{default}}$^{\textstyle *}$
        \rest\ \bkt{argument}})}
\end{center}
The \optl\ and \rest\ are metalanguage keywords used in this reference
guide to indicate how the arguments are to 
be provided; they are never legal arguments themselves.\footnote{Those
with a background in Lisp will note the resemblance.  However, note that
\optl\ as used in PVS is a combination of \optl\ and {{\smaller\sc
{\smaller\smaller \&}key}}.} The \optl\ keyword indicates that the
arguments which follow it are \emph{optional}.  Such arguments may be
provided either by \emph{position} or by \emph{keyword}.  To provide
the argument by position, simply include values for all the preceding
arguments followed by the value of the argument in question.  This is
usually the most convenient way to use the commands.  Occasionally, you
will want the default taken for most of the optional arguments, and only
want to specify a different value for one near the end of the list.  In
this case, you may provide a pair of arguments, the first being the name
of the argument preceded by a colon, and the second the value for the
argument.  This will be made clear in the examples below.  The \rest\
keyword indicates that zero or more values may be provided for the
indicated argument and these are accumulated into a list.  The \rest\
argument can also by supplied as a list by keyword.  

Note that many proof rules have arguments indicating the number or numbers
of the sequent formulas where the rule is to be applied.  The syntactic
convention is that when a single such number is expected, we indicate the
argument as being an \emph{fnum} (for ``sequent formula number''), and
where a list of such numbers is expected, we indicate the argument as
\emph{fnums}.  Typically, a single number is acceptable where a list is
expected, and denotes the singleton list containing that number.  The list
of antecedent sequent formulas can be indicated by `\texttt{-}', the list
of consequent sequent formulas can be indicated by `\texttt{+}', and the
list of all sequent formulas can be indicated by `\texttt{*}'.  The use of
this notation will be illustrated in the examples below.  Lists here mean
a sequence of formula numbers separated by whitespace (space, tab or
carriage return) and surrounded by parentheses.  The \texttt{-},
\texttt{+}, \texttt{*} indicators are preferable to specific numbers since
they are more robust in the face of changes affecting the formula being
proved.

The value provided for a \emph{name}, \emph{expr}, or \emph{type}
argument should be a legal corresponding PVS expression enclosed in
a pair of string quotes (\texttt{"}).  As with \emph{fnums}, lists of
these may be expected when the argument is \emph{names}, \emph{exprs},
or \emph{types}.

 When interacting with the prover you are
essentially interacting with Lisp, and it is possible to give arguments
that are ill-formed enough to cause problems.  One such problem occurs
when parentheses or string quotes are unbalanced.  The immediate sign of
this is that the system does not respond.  To verify this, look on the
right hand side of the status line of the \texttt{*pvs*} buffer; it will
display \texttt{:ready} when waiting for a (complete) command, and \texttt{:run} when processing.  Until it says \texttt{:run}, you may freely edit
the command, even if it takes multiple lines.  Other keystrokes can
cause problems that break into Lisp (for example, typing a period at the
\texttt{Rule?}\ prompt).  If this happens, type \texttt{(restore)}, which
causes the focus to return to the last point of interaction 
with the \texttt{Rule?}\ prompt.
Rarely, the system will not respond correctly to the \texttt{(restore)}
function, in which case you will have to abort to the top level (by
typing \texttt{:reset}).  In this case, the proof attempt is lost and
you will have to start the proof over---though it still has the
previously saved proof attempt.

Here are some examples of rules with their formal parameter lists:
\begin{itemize}
\item \texttt{(lemma \carg{name} \optl\ \carg{subst})}
\item \texttt{(replace \carg{fnum} \optl\
      \cargdflt{fnums}{*} \cargdflt{dir}{LR} \carg{hide?}\ \carg{actuals?})}
\item \texttt{(delete \rest\ \carg{fnums})}
\end{itemize}
The following are possible invocations of the above proof rules:
\begin{itemize}
\item \texttt{(lemma "assoc")}
\item \texttt{(lemma "assoc" ("x" "1" "y" 2 "z" 3))}
\item \texttt{(lemma "assoc" :subst ("x" "1" "y" 2 "z" 3))}
\item \texttt{(replace -1)}
\item \texttt{(replace -1 :dir RL)}
\item \texttt{(replace -1 (2 -2 3) RL)}
\item \texttt{(delete)}
\item \texttt{(delete 1)}
\item \texttt{(delete -2 1 -3)}
\item \texttt{(delete :fnums (-2 1 -3))}
\end{itemize}



\subsection{Rules versus Strategies}
\index{Proof Strategies}
\index{Proof Rules}

A PVS proof command given at the \texttt{Rule?}\ prompt can either invoke a
rule or a strategy.  A rule in PVS is an atomic operation that typically
generates zero or more subgoals from the given goal.  A strategy need not
be atomic.  An application of a strategy expands into a number of atomic
steps.  The atomic proof steps resulting from this expansion of the
strategy are saved in the final proof and these are executed directly when
the proof is rerun.  The \indtt{apply} rule applies a given strategy as an
atomic step thus converting a strategy into a rule.   Rules are
either primitive or defined, and the defined rules are defined as
strategies but applied as atomic proof steps.  For example, the rule for
replacement using an antecedent equality, \indtt{replace}, is a primitive
rule whereas \indtt{rewrite} is a defined rule and is defined by a
strategy that uses the \indtt{replace} rule.  Several proof
commands that are rules have non-atomic analogues given by strategies:
\indtt{prop} is the atomic propositional simplification rule and \texttt{prop\char36} is the corresponding strategy.

In pragmatic terms, strategies should be used when the expanded proof is
of interest and  otherwise, rules should be used.  So for instance, \texttt{prop} should almost always be favored over \texttt{prop\char36} since the
details of propositional simplification are seldom interesting.
It is also useful to invoke the strategy corresponding to a rule 
in order to observe the inner workings of the strategy.  

PVS features a strategy language for defining new rules and strategies.
There is a corresponding interpreter for the strategy expressions defined
using this  language.  The strategy language contains constructs for
selecting among alternative proof strategies (\indtt{if}), for backtracking
(\indtt{try}), and for invoking Lisp code (\indtt{let}).  Strategies can also
be defined using recursion.


\subsection{Proof Checker Pragmatics}


It is not necessary to master all the proof commands in order to use the
PVS proof checker effectively.  It is advisable to learn the most
powerful commands first and try these and only rely on the simpler
commands when the powerful ones fail.  Broadly speaking, there are two
typical kinds of proofs: those that require induction and those that do
not.  For proofs by induction, one of the induction commands is usually
the first step, and \indtt{induct-and-simplify} is the most powerful and
useful of these.  The commands \indtt{induct-and-rewrite} and \indbang{}
are variants of \indtt{induct-and-simplify}.  

The \indtt{grind} command is usually a good way to complete a proof that
does not require induction, and only requires definition expansion, and
arithmetic, equality, and quantifier reasoning.  The behavior of
\indtt{grind} can be controlled through its various optional arguments,
particularly \texttt{if-match} and \texttt{defs}.   Simpler forms of
\indtt{grind} such as \indtt{bash}, \indtt{reduce}, and \indtt{smash} can
be used when \texttt{grind} becomes difficult to control.   The \indtt{grind}
command can sometimes instantiate existential strength quantifiers
prematurely, and when this happens, it is often more appropriate
to first apply \texttt{(grind :if-match nil)}, which performs all the
simplifications of \texttt{grind} except quantifier instantiation,
followed by \texttt{(grind)} to pick up the instantiations exposed by
the first \indtt{grind}.  

In a more interactive proof attempt, the initial step in a proof is
usually the introduction of Skolem constants, and the preferred and most
powerful form here is \indtt{skosimp*}.  Note that a universal quantifier 
is needed for induction and in such cases, \indtt{skosimp*} might
go too far and \skobang{}  or \indtt{skosimp} might be more appropriate.

The decision procedure command \indtt{assert} is used very frequently
particularly since it does simplification, automatic rewriting, and
records type information and the sequent formulas in the decision
procedure database for use in future simplifications.  The more
restrictive forms of \indtt{assert}, namely, \indtt{simplify},
\indtt{do-rewrite}, and \indtt{record}  also come in handy.
The command \indtt{simplify-with-rewrites} can be used to temporarily
install and apply rewrite rules using \indtt{assert}.  

The \indtt{inst?}\ command is the most powerful way to automatically
instantiate quantifiers of existential strength.  It has several options
to control the selection of suitable instances.   

The \indtt{bddsimp} command is the most efficient way to do propositional
simplification, but \indtt{prop} will do when efficiency is not important.
Propositional simplification has to be used with care since it can often
generate lots of subgoals that share the same proof.  The \indtt{flatten}
and \indtt{split} commands must be used to do the propositional
simplification more delicately.  The \indtt{case} command is very useful as
a way of introducing case splits into a proof.  The \indtt{lift-if} command
is typically needed to bring the case analyses in an expanded definition
to the surface of the sequent where it can be propositionally simplified.

Of the control commands, \indtt{postpone} is used to cycle through the
pending subgoals in the proof, and \indtt{undo} is  used to
recover from fruitless paths in a proof.

In addition to the above commands, it helps to be
familiar with the prelude theories which contain a lot
of useful background mathematics.  Advanced users wishing
to define their own proof strategies should examine the definitions
of the basic strategies supplied with PVS.\@  A file containing
these definitions is distributed with the system.  


We now describe each of these groups of rules.  The table at the beginning
of each group briefly summarizes the effect of each command and
indicates whether the rule is \emph{primitive} or \emph{defined}.
This distinction is not crucial to the use of the theorem prover.
The defined commands can be redefined by the user, but the primitive
commands capture the underlying PVS logic and therefore cannot be
changed.  Each rule is an atomic step in the proof.  There are
`glass-box' versions for some of the defined
rules where the rule is executed as a strategy.
The name for such a proof strategy is typically the rule
name with a `\texttt{\char36}' suffix. 
For example, the
glass-box version of \indtt{induct-and-simplify} is
\texttt{induct-and-simplify\char36}.   The documentation indicates
the presence of both the black-box and glass-box versions
of this rule by listing the name as \texttt{induct-and-simplify/\char36}.


\section{The Help Rule}
\begin{tabularx}{\textwidth}{|l|l|X|}\hline
\indtt{help} & \emph{primitive} & provide brief documentation\\\hline
\end{tabularx}

\prsubsection{help}{Help for Proof Commands}
\begin{description}
\item[syntax:] \texttt{(help \optl\ \cargdflt{name}{*})}

\item[effect:] Displays a brief description of a specific primitive proof
command, defined rule, or strategy, or of all of the rules, defined rules,
and strategies.  Apart from displaying help information, the \indtt{help}
rule behaves as a \texttt{(skip)}\indextt{skip}, \ie\ it has no effect on
the proof.

\begin{usage}{}
\item[\texttt{(help)} :] Displays help on all of the rules, defined rules,
and strategies
\item[\texttt{(help rules)} :] Displays help on all the primitive rules.
\item[\texttt{(help defined-rules)} :] Displays help on all of the defined rules.
\item[\texttt{(help strategies)} :] Displays help on all of the strategies.
\item[\texttt{(help skolem)} :] Displays help on the \texttt{skolem} rule.
\item[\texttt{(help prop\char36)} :] Displays help on the \texttt{prop\char36} strategy.
\end{usage}

\item[notes:] This command should only be used interactively.  It is
usually better to use the Emacs commands \emacstt{help-pvs-prover},
\emacstt{help-pvs-prover-command},\newline \emacstt{help-pvs-prover-strategy},
or \emacstt{help-pvs-prover-emacs}, since they display the information in
a buffer for repeated reference.  The Emacs commands
\emacstt{x-prover-commands} when used in conjunction with X Windows
displays a mousable window listing the prover commands.

\end{description}

\section{The Annotation Rules}
\index{Proof Rules!Annotation|)}


\begin{tabularx}{\textwidth}{|l|l|X|}\hline
\indtt{comment} & \emph{primitive} & attach comments to proof sequents\\\hline
\indtt{label} & \emph{primitive} & attach labels to sequent formulas\\\hline
\indtt{unlabel} & \emph{primitive} & remove labels from sequent formulas\\\hline
\indtt{with-labels} & \emph{defined} & label resulting formulas \\\hline
\end{tabularx}

\prsubsection{comment}{Attach Comments to a Proof Sequent}
\begin{description}
\item[syntax:] \texttt{(comment \carg{string})}

\item[effect:] Attaches a comment string to the current proof sequent that
is printed with preceding semi-colons above the sequent formulas.  This
comment string is also saved with the proof.  The \texttt{comment} command
can be nested within strategies and the comments are retained on the
subgoals generated by the strategy.

\begin{usage}{}
\item[\texttt{(comment "3rd induction case")} :] Prints the comment string
 between the sequent label and the sequent formulas.  
\end{usage}

\end{description}

\prsubsection{label}{Attach a Label to Sequent Formulas}
\begin{description}
\item[syntax:] \texttt{(label\ \carg{string-or-symbol} \carg{fnums} \optl\
\carg{push?})}

\item[effect:] It is often useful to group and label a collection of
related formulas in a proof sequent.  The \indtt{label} command is used
for this purpose.  Each sequent formula can have a set of labels, in
addition to its \emph{fnum}.  The labels are printed alongside the
\emph{fnum} whenever a proof sequent is displayed.  A label can be used
wherever an \emph{fnum} is expected.  A label can supplied as either a
symbol, e.g., \texttt{(label indhyp)} or a string, e.g., \texttt{(label
"indhyp")}, though it is stored internally as a symbol.  Labels are
automatically inherited by any subformulas of a sequent formulas that
appear through the application of an inference rule, e.g.,
\texttt{flatten} applied to a consequent formula $A \vee B$ labelled
\texttt{main} results in two sequent formulas $A$ and $B$ both labelled
\texttt{main}.

When \emph{push?}\ is \texttt{t}, the new label is added to any existing
labels on the formula.  Otherwise, the given label replaces any existing
ones.

\begin{usage}{}
\item[\texttt{(label uniqueness -3)} :] Labels the formula numbered
\texttt{-3} by the label \texttt{uniqueness}.

\item[\texttt{(label type-constraints (-1 -3 -4))} :] Labels the formulas
numbered \texttt{-1}, \texttt{-3}, and \texttt{-4} by the label
\texttt{type-constraints}.

\item[\texttt{(label antecedents - :push? t)} :] Adds the label
\texttt{type-constraints} in addition to any other labels on the formulas
numbered \texttt{-1}, \texttt{-3}, and \texttt{-4}.

\item[\texttt{(bddsimp type-constraints)} :] Applies BDD-based
propositional simplification to the formulas labelled
\texttt{type-constraints}.
\end{usage}

\item[errors:] The label may not be a number, or \texttt{nil},
\texttt{quote}, \texttt{*}, \texttt{+}, or \texttt{-}.

\item[notes:] Note that the \texttt{bddsimp} command does not retain
labels since there is no simple way to retain the connection between the
formula returned by BDD-simplification and its original parent formula.
\end{description}
\prsubsection{unlabel}{Remove Labels from Sequent Formulas}
\begin{description}
\item[syntax:] \texttt{(unlabel\ \optl\ \carg{fnums} \carg{label})}

\item[effect:] Removes specified \emph{label} (or all labels if none
specified) from the formulas in \emph{fnums}.  When \emph{fnums} is not
specified, label(s) are removed from all formulas.

\end{description}
\index{Proof Rules!Annotation|)}


\prsubsection{with-labels}{Label New Sequent Formulas}
\begin{description}
\item[syntax:] \texttt{(with-labels\ \carg{rule} \carg{labels} \optl\
\carg{push?})}

\item[effect:] Given a proof step \emph{rule} and a list of
list of \emph{labels} $((l_{11} \ldots)\ldots (l_{n1} \ldots))$,
if the rule generates $n$ subgoals, then the $j$'th new
sequent formula in the $i$'th subgoal is assigned the label
$l_{ij}$.  If there are more subgoals than label lists, then
the last label list is applied to the remaining subgoals.
In each pairing of new formulas with labels in a list, if there are
more formulas than labels, the last label is applied to the remaining
new formulas.  A singleton list of labels can be replaced by a
single label.

When \emph{push?}\ is \texttt{t}, the new label is added to any existing
labels on the formula.  Otherwise, the given label replaces any existing
ones.

\begin{usage}{}
\item[\texttt{(with-labels (flatten) ((l1 l2 l3)))}:]
Applies \texttt{flatten} rule to the current proof subgoal and labels
the new sequent formulas thus produced as \texttt{l1}, \texttt{l2},
and \texttt{l3}, respectively.

\item[\texttt{(with-labels (prop) ((l11 l12 l13) (l21 l22)))}:]
Applies the \texttt{prop} rule and labels the new formulas in the
first subgoal by labels \texttt{l11}, \texttt{l12}, and \texttt{l13}, and
the new formulas in any remaining subgoals are labelled by
labels \texttt{l21} and \texttt{l22}.

\item[\texttt{(with-labels (prop) "prop-formulas")}:] Labels all the
new sequent formulas resulting from the application of \texttt{prop}
by the label \texttt{prop-formulas}. 
\end{usage}

\end{description}


\section{The Control Rules}
\index{Proof Rules!Control|(}
\begin{tabularx}{\textwidth}{|l|l|X|}\hline
\indtt{fail} & \emph{primitive} & signal a failure\\\hline
\indtt{postpone}& \emph{primitive} & cause the current goal to be
left pending\\\hline
\indtt{quit} & \emph{primitive} & quit a proof attempt \\\hline
\indtt{rewrite-msg-off} & \emph{defined} & inhibit rewriting
commentary\\\hline
\indtt{rewrite-msg-on} & \emph{defined} & turn on rewriting
commentary\\\hline
\indtt{set-print-depth} & \emph{defined} & set the print-depth for formulas \\\hline
\indtt{set-print-length} & \emph{defined} & set the print-length for formulas \\\hline
\indtt{set-print-lines} & \emph{defined} & set the print-lines for formulas \\\hline
\indtt{skip} & \emph{primitive} & has no effect but is useful in defining
strategies\\\hline
\indtt{skip-msg} & \emph{defined} & like \emph{skip} but generate a
message\\\hline
\indtt{trace} & \emph{defined} & turn on tracing of proof
commands\\\hline
\indtt{track-rewrite} & \emph{defined} & explain why a rewrite was not
applied\\\hline
\indtt{undo} & \emph{primitive} & undo proof steps along a branch of
the proof\\\hline
\indtt{untrace} & \emph{defined} & turn off tracing of proof commands\\\hline
\indtt{untrack-rewrite} & \emph{defined} & turn off rewrite explanation\\\hline
\end{tabularx}

\prsubsection{fail}{Propagate Failure to the Parent}
\begin{description}
\item[syntax:] \texttt{(fail)}

\item[effect:] A failure signal is propagated to the parent proof goal.
If the parent goal is not able to act on this signal, it further
propagates the failure to its parent.  This rule, like \indtt{skip}, is
mainly employed in constructing strategies where it is used to control
backtracking.  Applying \indtt{fail} to the root sequent causes the proof
to be unsuccessfully terminated.  

\item[usage:] \texttt{(fail)}

\item[errors:]  No error messages are generated.

\item[notes:]  See the description of the \indtt{try} strategy in
page~\pageref{trypage} for examples of 
the use of \indtt{fail}.  
\end{description}


\prsubsection{postpone}{Go to Next Remaining Goal}
\begin{description}
\item[syntax:] \texttt{(postpone \optl\ \cargdflt{print?}{t})}

\item[effect:] Marks the current goal as pending to be proved and shifts
the focus to the next remaining goal.  By successively invoking \texttt{postpone} sufficiently often, it is possible to cycle back to the
original focus.  When \emph{print?}\ is \texttt{t} commentary is
suppressed. 

\item[usage:]   \texttt{(postpone)}

\item[errors:]  No error messages are generated.

\item[notes:]  The Emacs command \emacstt{siblings} shows the sibling
subgoals of the current subgoal in an emacs buffer.

\end{description}

\prsubsection{quit}{Terminate the Proof Attempt}
\begin{description}

\item[syntax:] \texttt{(quit)}

\item[effect:] Terminates the current proof attempt, and queries whether
the partial proof in progress should be saved.  This way, it is possible
to break and resume a long proof attempt by saving the partial proof and
rerunning it when the proof attempt is resumed.

\item[notes:] This strategy should only be used interactively.
\end{description}

\prsubsection{rewrite-msg-off}{Inhibit Rewriting Commentary}
\begin{description}

\item[syntax:] \texttt{(rewrite-msg-off)}

\item[effect:] In the default mode, automatic rewriting by commands such
as \texttt{assert} and \texttt{do-rewrite} generate a fairly verbose
commentary.  This can be entirely shut off by the \texttt{rewrite-msg-off}
command.  Behaves like a \texttt{skip} otherwise.

\item[notes:] Finer-grain control over the terseness of the
rewriting commentary is possible with the Emacs commands
\emacstt{set-rewrite-depth} and \emacstt{set-rewrite\-length}.
\end{description}

\prsubsection{rewrite-msg-on}{Turn On Rewriting Commentary}
\begin{description}

\item[syntax:] \texttt{(rewrite-msg-on)}

\item[effect:] The rewriting commentary turned off by
\texttt{rewrite-msg-off} can be restored by this command.  Behaves like a
\texttt{skip} otherwise.
\end{description}

\prsubsection{set-print-depth}{Set the Print Depth}
\begin{description}

\item[syntax:] \texttt{(set-print-depth \optl\ \carg{depth})}

\item[effect:] Sets the print depth for displaying formulas.  \emph{Depth}
must be a number or \texttt{nil}.  0 or \texttt{nil} means print the
entire formula, any other number causes terms below the given depth to be
elided.  Behaves like a \texttt{skip} otherwise.
\end{description}

\item[errors:] The label must be a number or \texttt{nil}.

\prsubsection{set-print-length}{Set the Print Length}
\begin{description}

\item[syntax:] \texttt{(set-print-length \optl\ \carg{length})}

\item[effect:] Sets the print length for displaying formulas.
\emph{Length} must be a number or \texttt{nil}.  0 or \texttt{nil} means
print the entire formula, any other number causes terms longer than the
given number to be elided.  Behaves like a \texttt{skip} otherwise.

\item[errors:] The label must be a number or \texttt{nil}.
\end{description}

\prsubsection{set-print-lines}{Set the Number of Print Lines}
\begin{description}

\item[syntax:] \texttt{(set-print-lines \optl\ \carg{lines})}

\item[effect:] Sets the number of print lines for displaying formulas.
\emph{Lines} must be a number or \texttt{nil}.  0 or \texttt{nil} means
print the entire formula, any other number causes only the first specified
number of lines of each formula of the sequent to be displayed.  Behaves
like a \texttt{skip} otherwise.

\item[errors:] The label must be a number or \texttt{nil}.
\end{description}

\prsubsection{skip}{Do Nothing}
\begin{description}
\item[syntax:] \texttt{(skip)}

\item[effect:] Has no effect on the proof.  The primary utility of \texttt{skip} is in writing strategies where a step is required to have no
effect unless some condition holds.  Typing \texttt{(skip)} in response to
a goal sequent returns the same proof state with a \texttt{"No change."}
message.

\item[usage:] \texttt{(skip)}

\item[errors:] No error messages are generated.
\end{description}


\prsubsection{skip-msg}{Do Nothing but Print}
\begin{description}
\item[syntax:] \texttt{(skip-msg \carg{msg} \optl\ \carg{force-printing?})}

\item[effect:] Has no effect on the proof but prints the given \emph{msg} string.  The main use of \texttt{skip-msg} is in generating error
messages from within strategies, typically as: \texttt{(if good?(input)
\ldots (skip-msg "Bad input."))}.

\begin{usage}{}
\item[\texttt{(skip-msg "Not enough terms given.")}]: Does nothing but
prints the error message \texttt{"Not enough terms given."}.

\item[\texttt{(skip-msg "Not enough terms given." t)}]:
Does nothing but prints the error message \texttt{"Not enough terms
given."} even when the \texttt{skip-msg} appear within an \texttt{apply}
where the printing of such messages is usually suppressed.

\item[]\smaller{\begin{alltt}
(let ((string (format nil
                "No such theory: ~a in current context." theory)))
  (skip-msg string)):
\end{alltt}}Builds the string \texttt{string}
for use within \texttt{skip-msg}. 
\end{usage}


\item[errors:] No error messages are generated.

\end{description}


\prsubsection{trace}{Trace Commands}
\begin{description}
\item[syntax:] \texttt{(trace \rest\ \carg{names})}

\item[effect:] Turns on the tracing of the proof commands named in \emph{names} so that any time any one of the named rules or strategies is used
in a proof, the entry into and exit out of such commands is
traced.  This makes it possible to check if the command is being properly
invoked and has the desired effect.   Behaves like a \texttt{skip} otherwise.


\item[usage:] \texttt{(trace assert inst?\ induct)}

\item[errors:] No error messages.

\item[notes:] \texttt{untrace} turns off tracing initiated by \texttt{trace}.
\end{description}



\prsubsection{track-rewrite}{Explain Failure of Rewrite Rules}
\begin{description}
\item[syntax:] \texttt{(track-rewrite \rest\ \carg{names})}

\item[effect:]  Explains why the attempt to apply a rewrite rule named in
\emph{names} was not applied.  The typical reasons are:
\begin{enumerate}
\item The expression being rewritten did not match the left-hand side of
the rewrite rule.

\item The match succeeded but generated type-correctness proof obligations
that could not be simplified to \texttt{TRUE}\@.  

\item The match succeeded but the corresponding conditions of the rewrite
rule could not be simplified to \texttt{TRUE}.

\item The match succeeded and the corresponding conditions did simplify to
\texttt{TRUE}, but the top-level conditional or \texttt{CASES} branch in the
corresponding right-hand side of the rewrite rule was not simplifiable.
This top-level conditional on the right-hand side should be simplifiable
in the case of recursive definitions and ordinary rewrite rules which are
not installed with the \texttt{always?}\ flag set to \texttt{t}.
\end{enumerate}

Other than setting up the names of the rewrite rules to be tracked during
simplification, \texttt{track-rewrite} behaves like a \texttt{skip}.  It has
no effect on the current proof sequent and is not saved as part of the
partial or completed proof.  


\item[usage:] \texttt{(track-rewrite "assoc" "append" "reverse\_append" )}:
Tracks the given rewrite rules during 
simplification and reports any failures corresponding to when the rewrites
are unsuccessful.

\item[errors:] No error messages.

\item[notes:] \indtt{untrack-rewrite} turns off the tracking of rewrite
rules initiated by track-rewrite.  
\end{description}


\prsubsection{undo}{Undo Proof to an Ancestor}
\begin{description}
\item[syntax:] \texttt{(undo \optl\ \carg{to}{1})}

\item[effect:] The undo command undoes the proof back to an ancestor
node of the current node as indicated by the \emph{to} argument.  The
user is then shown the sequent at that ancestor node, and asked for
verification.  The \emph{to} argument can either be:
\begin{enumerate}
\item A positive number indicating the number of levels in the proof tree
to be undone

\item A label, in which case the proof is undone to the lowest
occurrence of a sequent with that label above the current sequent
(since there can be many sequents in the proof with the same label; 
labels are only extended when there are multiple subgoals)

\item A proof rule or strategy in which case the proof is undone up to
the lowest occurrence of a sequent where the given rule was applied by
the prover or
the given strategy was supplied by the user, or

\item A rule name or strategy name, so that the proof is undone to the
lowest occurrence of a sequent where a rule with the given name was
applied by the prover or a strategy of the given name was supplied by the user.
\end{enumerate}
Undo applies its effects relative to the current node, not the last
command.  Thus undoing immediately after a branch has been proved or
postponed will not, in general, go back to the state of the proof tree
just before the last command.  However, undo can be used to undo
the effect of an undo command if invoked \emph{immediately} afterwards. 

\begin{usage}{}
\item[\texttt{(undo)}] : Undoes a single step of the proof.

\item[\texttt{(undo 3)}] : Undoes three steps in the proof.

\item[\texttt{(undo undo)}] : Undoes an undo, if it was the last command
executed.  If anything has been executed since the undo command, it is not
possible to undo the undo.

\item[\texttt{(undo (skolem 1))}] : Undoes back to the lowest ancestor node
where \texttt{(skolem 1)} was applied.

\item[\texttt{(undo skolem)}] : Undoes back to the lowest ancestor node
where a proof rule or strategy of the form \texttt{(skolem \ldots)} was
issued or applied.

\item[\texttt{(undo skolem!)}] : Undoes back to the lowest ancestor node
where \texttt{skolem!} was issued.

\item[\texttt{(undo "assoc.2")}] :  Undoes back to the lowest ancestor node
labelled with \texttt{assoc.2}, unless \texttt{assoc.2} labels the current node in
which case there is no change.
\end{usage}

\item[notes:]
\begin{itemize}
\item The \indtt{undo} command is only meant to be used
interactively.  Noninteractive strategies should use the \indtt{fail}
command for the same effect.

\item The Emacs command \emacstt{ancestry} shows the chain of proof
goals leading back to the root node of the proof.

\item A \texttt{(rerun)} command immediately after an \indtt{undo}
will cause the undone proof to be rerun and restored.
\end{itemize}
\end{description}

\prsubsection{untrace}{Disable Tracing of Commands}
\begin{description}
\item[syntax:] \texttt{(untrace \rest\ \carg{names})}

\item[effect:] Turns off the tracing of proof commands named
in \emph{names}, as initiated by \texttt{(trace)}.  Behaves like
a \texttt{skip} otherwise.

\item[usage:] \texttt{(untrace assert)}

\item[errors:] No error messages
\end{description}

\prsubsection{untrack-rewrite}{Disable Tracking of Rewrite Rules}
\begin{description}
\item[syntax:] \texttt{(untrack-rewrite \rest\ \carg{names})}

\item[effect:] Disables the tracking of rewrite rules invoked by
\indtt{track-rewrite}.  When untrack-rewrite is invoked with no arguments,
then tracking is discontinued for all currently tracked rewrite rules.
Other than removing the given \emph{names} from list of rewrite rules to
be tracked during simplification, \texttt{untrack-rewrite} behaves like a
\texttt{skip}.  It has no effect on the current proof sequent and is not
saved as part of the partial or completed proof.

\begin{usage}{}
\item[\texttt{(untrack-rewrite "assoc" "append" "reverse\_append" )}:]
Disables tracking on the given rewrite rules. 

\item[\texttt{(untrack-rewrite)}:]  Disables tracking on all currently
tracked rewrites rules.
\end{usage}

\item[errors:] No error messages.
\end{description}

\index{Proof Rules!Control|)}

\section{The Structural Rules}
\index{Proof Rules!Structural|(}

Sequent calculus based proof systems employ structural rules to rearrange
the formulas in a sequent.  The typical structural rules are described in
Chapter~\ref{logic}.  In PVS, the Exchange rule is entirely omitted since
the PVS proof commands already ignore the order of formula occurrences
in a sequent except for the use of formula numbers.  The Contraction rule
is not built into PVS and only appears in a limited form; the rule for
instantiating quantifiers of existential strength permits the copying of
these quantified formulas so that they can be reused, if needed.  Some use
of Contraction is already built into the rules so that the non-principal
formulas are shared between the premises of a rule.  The defined rule
\indtt{copy} also implements Contraction.  The Weakening rule is present
in PVS as the \indtt{delete} rule below.  The \indtt{hide} rule is a
more cautious form of \indtt{delete}, where certain sequent formulas can
be suppressed and recovered later in the proof using the \indtt{reveal}
rule.  The emacs command \emacstt{show-hidden-formulas} displays hidden
formulas along with their numbers.

\begin{tabularx}{\textwidth}{|l|l|X|}\hline
\indttdol{copy} & \emph{defined} & insert a copy of a sequent formula \\\hline
\indtt{delete} & \emph{primitive} &  delete selected formulas from a goal sequent\\\hline
\indtt{hide} & \emph{primitive} & temporarily hide selected formulas from the displayed goal \\\hline
\indtt{reveal} & \emph{primitive} & reveal hidden formulas\\\hline
\end{tabularx}

\prdolsubsection{copy}{Copy Selected Formula}
\begin{description}
\item[syntax:] \texttt{(copy \carg{fnum})}

\item[effect:] Inserts a copy of the sequent formula numbered \emph{fnum}
into the sequent.  If the given formula is an antecedent formula, then the
copy becomes the first antecedent formula, and if it is a consequent
formula, then the copy becomes the first consequent formula.

\begin{usage}{}
\item [\texttt{(copy -3)}] : Makes a copy of the formula numbered -3
and inserts it as the first antecedent formula.

\end{usage}

\item[errors:] {\bf Could not find formula number \emph{foo}}: attempt
to copy a formula which does not exist.
\end{description}

\prsubsection{delete}{Delete Selected Formulas}
\begin{description}
\item[syntax:]  \texttt{(delete \rest\ \carg{fnums})}

\item[effect:] Returns the subgoal that is the result of deleting all
of the sequent formulas in the current goal that are indicated by \emph{fnums}.  If there are no formulas in the sequent corresponding
to those indicated in \emph{fnums}, then the effect is that of a
\texttt{(skip)}.

\begin{usage}{}
\item [\texttt{(delete *)}] : Deletes every formula in the sequent yielding a
subgoal that is an empty sequent.  This invocation of the rule is not
advisable because the empty sequent is unprovable.

\item[\texttt{(delete +)}] :  Yields the subgoal where all the
consequent formulas in the current goal sequent have been deleted. 

\item[\texttt{(delete -)}] : Same as above with antecedent formulas.

\item[\texttt{(delete 2)}] : Yields the subgoal where formula number \texttt{2}
in the current subgoal is deleted.

\item[\texttt{(delete (-1 4 -3 2))}] : Yields the subgoal where formulas
numbered \texttt{-1}, \texttt{4}, \texttt{-3}, and \texttt{2} in the current subgoal
are deleted.

\item[\texttt{(delete -1 4 -3 2)}] : Same as above.
\end{usage}

\item[errors:] No error messages are generated.

\item[notes:] When in doubt, use \indtt{hide} instead of \indtt{delete}.
\end{description}

\prsubsection{hide}{Hide Selected Formulas}
\begin{description}
\item[syntax:] \texttt{(hide \rest\ \carg{fnums})}

\item[effect:] This is a more cautious version of \indtt{delete}.  The \texttt{hide} rule saves the deleted sequent formulas that are indicated by \emph{fnums} so that they can be restored to a descendant of the current sequent
by the \indtt{reveal} rule (see below).  Note that the non-copying version of
the \indtt{instantiate} rule and the \indtt{inst} rule automatically hide the
quantified formula so that quantifiers can be later reinstantiated
along the same branch of the proof if needed.


\begin{usage}{}
\item[\texttt{(hide 2)}] : Yields the subgoal sequent that results from hiding the
formula number \texttt{2} in the current goal sequent.

\item[\texttt{(hide (-1 4 -3 2))}] : Yields the subgoal sequent that results from
hiding the formulas numbered \texttt{-1}, \texttt{4}, \texttt{-3}, and \texttt{2} in
the current goal sequent.

\item[\texttt{(hide -1 4 -3 2)}] : Same as above.
\end{usage}

\item[errors:] No error messages are generated.

\item[notes:] Hidden formulas play no role in a proof until they are
revealed.  Thus in addition to eliminating ``clutter'' in the display,
they can also affect the performance of the ground prover which contains
decision procedures for equality and linear arithmetic.
\end{description}

\prdolsubsection{hide-all-but}{Hide Unselected Formulas}
\begin{description}
\item[syntax:] \texttt{(hide-all-but\ \optl\ \carg{keep-fnums} \cargdflt{fnums}{*})}

\item[effect:] This is a variant of the \indtt{hide} rule that hides all
the formulas indicated by \emph{fnums} except those indicated by \emph{keep-fnums}\@.  As with \indtt{hide}, hidden sequent formulas are saved
and can be restored to a descendant of the current sequent by the
\indtt{reveal} rule.

\begin{usage}{}
\item[\texttt{(hide-all-but (-1 -4) -)}] : Yields the subgoal sequent that
results from hiding all the antecendent formulas except the formulas
numbered \texttt{-1} and \texttt{-4} in the current sequent.

\item[\texttt{(hide-all-but * (-2 1))}] : Hides formulas numbered \texttt{-2} and
\texttt{1} in the current sequent.  

\item[\texttt{(hide-all-but :keep-fnums (-2 3))}] :  Hides all formulas except those
numbered \texttt{-2} and \texttt{3} in the current sequent.
\end{usage}

\item[errors:] No error messages are generated.

\end{description}


\prsubsection{reveal}{Reveal Hidden Formulas}
\begin{description}
\item[syntax:] \texttt{(reveal \rest\ \carg{fnums})}

\item[effect:] The Emacs command \emacstt{show-hidden-formulas} displays
the hidden formulas in sequent form, including the formula numbers that
may be used in \texttt{reveal}.  Invoking \indtt{reveal} yields a subgoal
that reintroduces the hidden formulas numbered \emph{fnums} into the
current sequent.  The formulas thus revealed are not removed from the list
of hidden formulas.

\begin{usage}{}
\item[\texttt{(reveal -2)}] : Reveals the formula numbered \texttt{-2} in
the sequent displayed by \emacstt{show-hidden-formulas}.

\item[\texttt{(reveal (-1 4 -3 2))}] : Reveals the formulas numbered \texttt{-1}, \texttt{4}, \texttt{-3}, \texttt{2} in the sequent displayed by
\emacstt{show-hidden-formulas}.

\item[\texttt{(reveal -1 4 -3 2)}] : Same as above.
\end{usage}

\item[errors:] No error messages are generated.
\end{description}
\index{Proof Rules!Structural|)}

\section{The Propositional Rules}
\index{Proof Rules!Propositional|(}

\begin{tabularx}{\textwidth}{|l|l|X|}\hline
\indtt{bddsimp} & \emph{primitive} & propositional simplification rule\\\hline
\indtt{case} & \emph{primitive} & introduce a case split (the Cut rule)\\\hline
\indttdol{case*} & \emph{defined} & introduce case splits \\\hline
\indtt{flatten} & \emph{defined} & disjunctive simplification\\\hline
\indtt{flatten-disjunct} & \emph{primitive} & controlled disjunctive simplification\\\hline
\indtt{iff} & \emph{primitive} & convert boolean equalities to \emph{if and only if} form\\\hline
\indtt{lift-if} & \emph{primitive} & the \texttt{IF}-lifting rule \\\hline
\indtt{merge-fnums} & \emph{defined} & combine sequent formulas\\\hline
\indttdol{prop} & \emph{defined} & propositional simplification rule \\\hline
\indtt{propax} & \emph{primitive} & the propositional axiom rule\\\hline
\indtt{split} & \emph{primitive} & the conjunctive splitting rule\\\hline
\end{tabularx}


\prsubsection{bddsimp}{Propositional Simplification using BDDs}
\begin{description}

\item[syntax:] \texttt{(bddsimp \optl\ \cargdflt{fnums}{*}
\carg{dynamic-ordering?}\ \cargdflt{irredundant?}{t})}

\item[effect:] Generates subgoals by applying propositional simplification
using an external package written in C and based on binary decision
diagrams (BDDs).  Each distinct atomic Boolean formula in the sequent is
converted into a literal and the top-level propositional structure is
translated into input that is accepted by the BDD package.  The result is
translated back into the a list of subgoal sequents.

The \emph{dynamic-ordering?}\ flag when set to \texttt{t}, allows the
BDD package to reorder literals to reduce the BDD size.

The \emph{irredundant?}\ flag, when set to \texttt{t}, normalizes the BDD so
that the generated subgoals are independant, \ie\ no subgoal is subsumed
by any of the others.  This is quite expensive, and large BDDs can take a
long time to process, but without it proofs may need to be repeated on
multiple subgoals.

\item[usage:]  \texttt{(bddsimp)}: Repeatedly applies the propositional rules to
all the formulas in the sequent to generate zero or more subgoals.

\texttt{(bddsimp +)}: Applies propositional rules to the consequent formulas.

\texttt{(bddsimp + T)}:  Uses the dynamic reordering heuristic to control BDD
size while applying propositional simplification to  the consequent
formulas.

\item[errors: ]   No error messages.

\end{description}


\prsubsection{case}{Case Analysis on Formulas}
\begin{description}
\item[syntax:] \texttt{(case \rest\ \carg{formulas})}

\item[effect:] If the current sequent is of the form
$\Gamma\vdash\Delta$, then the rule \texttt{(case $A_1$ \ldots $A_n$)}
generates the subgoals
\begin{eqnarray*}
A_n, \ldots, A_1, \Gamma & \vdash & \Delta \\
A_{n-1}, \ldots, A_{1},\Gamma & \vdash & A_n,\Delta \\
A_{n-2}, \ldots, A_{1}, \Gamma & \vdash & A_{n-1}, \Delta \\
& \vdots & \\
A_1, \Gamma & \vdash & A_2,\Delta \\
\Gamma & \vdash & A_1, \Delta
\end{eqnarray*}
Note that the \texttt{case} command generates $n+1$ subgoals given $n$
formulas.  This allows us to assume a formula or a collection of formulas
and subsequently prove these formulas to be true.  The formulas $A_i$ are
given as strings, \eg, \texttt{"x > 0 AND y > 0"}.  The given formulas
$A_1, \ldots, A_n$ are parsed and typechecked and are expected to be of
type \texttt{bool}.  The typechecking of these formulas could generate
additional subgoals corresponding to the type correctness conditions \eg\
\texttt{(case "(1/x) > 0")} would generate an additional subgoal with the
proof obligation requiring that \texttt{x /= 0}.  It is quite common for
this command to generate parser and typechecker errors which simply return
control back to the proof checker without affecting the state of the
proof.

This command is extremely useful for transforming an undesirable
expression $t$ into a more desirable form $s$ when this cannot be achieved
by the other proof commands.  The command \texttt{(case "$t$ = $s$")}
followed by \indtt{replace} can achieve the desired transformation.  This
combination is encapsulated in the defined rule \indtt{case-replace}
(page~\pageref{case-replace})\@.


\begin{usage}{}
\item[\texttt{(case)}] : same as a \texttt{(skip)}.

\item[\texttt{(case "x > 0")}] :  splits into two cases, assuming \texttt{x > 0} as
an antecedent on one branch of the proof and placing \texttt{x > 0} as a
proof obligation along the other branch of the proof.  There might be
additional branches if the typechecking of the formula in place of \texttt{x > 0} generates type correctness proof obligations.

\item[\texttt{(case "x > 0" "y > 0")}] : splits a goal $\Gamma\vdash\Delta$ into
three subgoals:
\begin{itemize}
\item $\texttt{y > 0}, \texttt{x > 0},  \Gamma\:\vdash\:\Delta$
\item $\texttt{x > 0}, \Gamma\:\vdash\: \texttt{y > 0}, \Delta$
\item $\Gamma\:\vdash\: \texttt{x > 0}, \Delta$
\end{itemize}
\end{usage}

\item[errors:] The \indtt{case} rule can generate the following error
messages:
\begin{description}

\item[{\bf No formulas given.}] This means the argument list was empty.

\item[{\bf Irrelevant free variables \ldots occur in formulas.}] No
sequent in a PVS proof can have free variables in it.  They can only
contain bound variables, ordinary constants, and Skolem constants.  The
formulas given as arguments to the
\indtt{case} rule must not contain any free variables.

\item[{\bf boolean expected here.}] This indicates that one of the given
formulas did not typecheck to the expected type \texttt{boolean}.

\item[{\bf Parser error: \ldots}] One of the given formulas did not
parse correctly.

\item[{\bf Typecheck error: \ldots}]The typechecking of the given
formulas failed in one of a variety of ways.

\end{description}

\item[notes:] The \indtt{case} rule corresponds to applications of the Cut rule.
This command is surprisingly useful for explicitly controlling case splits in
a proof, and for introducing assumptions that will eventually be discharged.
\end{description}

\prdolsubsection{case*}{Full Case Analysis on Formulas}
\begin{description}
\item[syntax:] \texttt{(case* \rest\ \carg{formulas})}

\item[effect:]  Like the \texttt{case} command, but performs a fully
branching case analysis.  
If the current sequent is of the form
$\Gamma\vdash\Delta$, then the rule \texttt{(case* $A_1$ \ldots $A_n$)}
generates the subgoals
\begin{eqnarray*}
A_n, \ldots, A_1, \Gamma & \vdash & \Delta \\
A_{n-1}, \ldots, A_{1},\Gamma & \vdash & A_n,\Delta \\
A_n, A_{n-2}, \ldots, A_{1}, \Gamma & \vdash & A_{n-1},\Delta \\
A_{n-2}, \ldots, A_{1}, \Gamma & \vdash & A_n, A_{n-1},\Delta \\
& \vdots &  \\
A_n, \Gamma & \vdash & A_{n-1}, \ldots, A_1, \Delta \\
\Gamma & \vdash & A_n, \ldots, A_1, \Delta
\end{eqnarray*}
 Note that the \texttt{case*} command generates $2^n$
subgoals given $n$ formulas.  
This allows us to assume a formula or a collection of formulas and
subsequently prove these formulas to be true.  

\begin{usage}{}
\item[\texttt{(case*)}] : same as a \texttt{(skip)}.

\item[\texttt{(case* "x > 0" "y > 0")}] : splits a goal $\Gamma\vdash\Delta$ into
four subgoals:
\begin{itemize}
\item $\texttt{y > 0}, \texttt{x > 0},  \Gamma\:\vdash\:\Delta$
\item $\texttt{x > 0}, \Gamma\:\vdash\: \texttt{y > 0}, \Delta$
\item $\texttt{y > 0}, \Gamma\:\vdash\: \texttt{x > 0}, \Delta$
\item $\Gamma\:\vdash\: \texttt{y > 0}, \texttt{x > 0}, \Delta$
\end{itemize}
\end{usage}

\item[errors:] The \indtt{case*} rule can generate the same
error message as the \indtt{case} command. 

\end{description}


\prsubsection{flatten}{Disjunctive Simplification}
\begin{description}
\item[syntax:] \texttt{(flatten \rest\ \carg{fnums})}

\item[effect:] A sequent formula is a disjunct if it is either an
antecedent formula of the form $\neg A$ or $A\wedge B$, or a consequent
formula of the form $\neg A$, $A\supset B$, or $A\vee B$.  Disjunctive
simplification transforms each indicated formula into a list of formulas
that contains no disjuncts by repeatedly transforming
\begin{enumerate}
\item An antecedent formula $\neg A$ into the consequent formula $A$

\item An antecedent formula $A\wedge B$ into the two antecedent formulas $A$
and $B$

\item A consequent formula $\neg A$ into the antecedent formula $A$

\item A consequent formula $A\supset B$ into the antecedent
formula $A$ and the consequent formula $B$

\item A consequent formula $A\vee B$ into the two consequent formulas
$A$ and $B$

\item An antecedent formula $A\iff B$ into the two antecedent
formulas $A\supset B$ and $B\supset A$.
\end{enumerate}
The \indtt{flatten} rule yields a subgoal where the indicated formulas in
the current goal are disjunctively simplified.    The rule behaves as a
\texttt{(skip)} if none of the indicated formulas can be disjunctively
simplified.  The current goal is proved if disjunctive simplification
yields an antecedent formula {\sc false} or a consequent formula {\sc
true}.

\begin{usage}{}
\item [\texttt{(flatten)}] : disjunctively simplifies every formula in the
current goal sequent yielding a subgoal that contains no disjuncts that are
sequent formulas.

\item[\texttt{(flatten 2)}] : disjunctively simplifies formula number \texttt{2} in the current goal sequent.

\item [\texttt{(flatten +)}] : disjunctively simplifies all the consequent
formulas in the current goal sequent. 

\item[\texttt{(flatten (-1 4 -2))}] : disjunctively simplifies the formulas
numbered \texttt{-1}, \texttt{4}, and \texttt{-2} in the current goal sequent.

\item[\texttt{(flatten -1 4 -2)}] : Same as above.
\end{usage}

\item[errors:] No error messages are generated.

\item[notes:] This command corresponds to repeated applications of the
inference rules $\wedge\vdash$, $\vdash\vee$, $\vdash\supset$,
$\neg\vdash$, and $\vdash\neg$ in Chapter~\ref{logic}.  Note that these
are all the propositional rules which do not cause branching.  Since
this command does not cause any branching, it is always safe to use and
generally makes the sequent easier to read.

\end{description}

\prsubsection{flatten-disjunct}{Controlled Disjunctive Simplification}
\begin{description}
\item[syntax:] \texttt{(flatten-disjunct \optl\ \carg{fnums} \carg{depth})}

\item[effect:] As per \texttt{(flatten)}, but with an optional \emph{depth}
argument which can be used to control the depth to which the top-level
disjuncts in a sequent formula are flattened.  If the depth is not given,
then the disjunctive simplification is carried out without any bound
on the depth.

\begin{usage}{}

\item[\texttt{(flatten-disjunct + :depth 2)}] : flattens the consequent
disjunctive formulas, but only up to a depth of 2.

\end{usage}
\item[errors:] No error messages are generated.

\end{description}

\prsubsection{iff}{Convert Boolean Equality to Equivalence}
\begin{description}
\item[syntax:] \texttt{(iff \rest\ \carg{fnums})}

\item[effect:] Yields a subgoal where any boolean equalities of the form
$A = B$, among the formulas in the current sequent that are indicated by
\emph{fnums} are converted to $A\iff B$.\footnote{Recall that
$\iff$ is written as \texttt{iff} or \texttt{<=>} in the raw PVS
language.} Treating all boolean equalities as equivalences is not a good
idea since that leads to a combinatorial explosion when the
propositional steps are applied, and  in many such cases,  equality
reasoning is
sufficient to complete the proof.

\begin{usage}{}

\item[\texttt{(iff)}] : same as \texttt{(iff *)}.  Converts any boolean
equalities among the sequent formulas into equivalences.  Behaves like
\texttt{(skip)} if there are no such boolean equalities.

\item[\texttt{(iff -3)}] : converts the formula numbered \texttt{-3} into an
equivalence.

\item[\texttt{(iff (4 2 -1))}] : converts the formulas number \texttt{4}, \texttt{2}, and \texttt{-1} into equivalences.

\item[\texttt{(iff 4 2 -1)}] : Same as above.
\end{usage}

\item[errors:]  No error messages are generated.
\end{description}


\prsubsection{lift-if}{Lift Embedded IF Connectives}
\begin{description}
\item[syntax:] \texttt{(lift-if \optl\ \carg{fnums} \cargdflt{updates?}{t})}

\item[effect:] In proving properties of programs, the proof often splits
up into cases according to the branching structure of the program.  This
branching structure is typically expressed using the
\texttt{IF}-connective or the \indtt{CASES} construct.  Since these
\indtt{IF} and \indtt{CASES} branches could occur embedded within the
formula, this branching structure must be lifted to the top level of the
formula where the propositional simplification steps can be applied (see
the \indtt{flatten} and \texttt{split} commands above).  The
\indtt{lift-if} rule lifts the leftmost-innermost contiguous \indtt{IF} or
\indtt{CASES} \emph{branching structure} out to the top level.  An example
of such a transformation is the rewriting of
$f(\texttt{IF}(A,B,\texttt{IF}(C,D,E)))$ to
$\texttt{IF}(A,f(B),\texttt{IF}(C,f(D),f(E)))$.  On the other hand,
$f(\texttt{IF}(\texttt{IF}(A, B, C), D, E))$ is transformed by
\indtt{lift-if} to $\texttt{IF}(A, f(\texttt{IF}(B, D, E)),
f(\texttt{IF}(C, D,E)))$, reflecting the selection of the conditionals of
the leftmost-innermost \texttt{IF}-expression.  Note that
$f($\texttt{IF}$(A,$ \texttt{IF}$(B, C, D),$ \texttt{IF}$(E, F, G)))$ is
transformed to $\texttt{IF}(A, \texttt{IF}(B, f(C), f(D)), \texttt{IF}(E,
f(F), f(G)))$ reflecting the preservation of the contiguous
\texttt{IF}-branching structure.  It is more effective to lift a
contiguous block of \indtt{IF} branches since it more accurately reflects
the case structure of the resulting argument and results in a more
efficient \texttt{IF}-expression (since the branches $B$ and $E$ in
previous example are kept independent).  Note that only conditionals
without bound variables can be lifted, and this is used as a criterion by
the \indtt{lift-if} rule in selecting the branching structure.  The
leftmost-innermost branching structure typically turns out to be the most
appropriate one.  If this choice of branching structure turns out to be
inappropriate, the \indtt{case} command (see below) can be used to carry
out the desired case analysis.

\label{if-lift-updates} Unless the \emph{update?}\ flag is \texttt{nil},
the \indtt{lift-if} command has been extended to extract the case
structure from an array or function update.  An expression of the form
\texttt{(f WITH [(x)(u) := 3, (y)(v) := 5])(z)(w)} is converted to
\begin{alltt}
   (IF z = y THEN
       (IF w = v THEN 5
        ELSIF z = x THEN (IF w = u THEN 3 ELSE f(z) ENDIF) 
        ELSE f(z)
        ENDIF)
    ELSIF z = x THEN 3
    ELSE f(z)
    ENDIF)
\end{alltt}
The resulting IF-expression is then lifted by the \indtt{lift-if} command.


\begin{usage}{}

\item[\texttt{(lift-if)}] : same as \texttt{(lift-if *)}.  Yields the subgoal
got by lifting the leftmost-innermost branching structure in each of the
formulas in the current sequent.  Applications where the operator is an
update are converted into IF-expressions which are also lifted.  

\item[\texttt{(lift-if +)}] : yields the subgoal got by lifting the
leftmost-innermost branching structure in each of the consequent formulas
in the current sequent.  

\item[\texttt{(lift-if -3)}] : lifts the branching structure from the formula
numbered \texttt{-3} in the current sequent.

\item[\texttt{(lift-if (-1 3 -2))}] : lifts the branching structure from the
formulas numbered \texttt{-1}, \texttt{3}, and \texttt{-2} in the current sequent.

\item[\texttt{(lift-if (-1 3 -2)) :updates? nil}] : Same as above without the
conversion of applications of updates to IF-expression form.  
\end{usage}

\item[errors:] No error messages are generated.

\item[notes:] This command roughly corresponds to the $\texttt{IF}\uparrow\vdash$
and $\vdash \texttt{IF} \uparrow$ rules of Chapter~\ref{logic}.
 The \indtt{simplify}, \indtt{record}, and \indtt{assert} commands do a limited
amount of ``if-lifting.''

\end{description}

\prdolsubsection{merge-fnums}{Combine Sequent Formulas}
\begin{description}
\item[syntax:] \texttt{(merge-fnums \carg{fnums})}

\item[effect:] If the sequent is of the form $A_1,\ldots,A_m\vdash
C_1,\ldots, C_n$, and \emph{fnums} picks out the sequent formulas
$A_i$, $A_j$, $C_k$, and $C_l$, then in the resulting sequent, these
formulas are replaced by the single formula $A_i \wedge A_j \supset C_k
\vee C_l$\@.   Presently, this command is mainly useful in defining the
\indtt{generalize} command which needs sequent formulas merged into
one formula so that a universal quantifier can be wrapped around  it.
Applying \texttt{merge-fnums} to a single formula has no effect.  

\begin{usage}{}
\item[\texttt{(merge-fnums (-1 -3 4))}:] Merges and replaces the sequent
formulas numbered \texttt{-1}, \texttt{-3}, and \texttt{4} by a single formula,
namely a consequent formula asserting the implication between the
conjunction of \texttt{-1} and \texttt{-3} and \texttt{4}\@.   
\end{usage}

\item[notes:] The \indttdol{merge-fnums} command is used in defining the
\texttt{generalize} strategy in order to collect together the formulas where
the term to be generalized by a universally quantified variable appears.
\end{description}

\prdolsubsection{prop}{Propositional Simplification}
\begin{description}

\item[syntax:] \texttt{(prop)}

\item[effect:] Carries out propositional simplification on the current
goal returning just those subgoals that are not propositional axioms and
do not have any top-level propositional connectives.  Using \indtt{prop}
indiscriminately could lead to a combinatorial explosion of cases caused
by splitting irrelevant conjunctions.  This in turn could lead to a
number of subgoals requiring identical proofs so \texttt{prop} should be used with some care.  The recursive definition
for \texttt{(prop)} is simply
\begin{alltt}
  (try (flatten) (prop) (try (split)(prop) (skip)))
\end{alltt}

\item[notes:] \indtt{prop} can be used for small-scale propositional
simplification.  For larger formulas, the primitive rule \indtt{bddsimp} which
uses a BDD-based propositional simplifier is usually more efficient.  

The \indtt{prop} rule should almost always be preferred to its strategy version
\texttt{prop\char36}.  
\end{description}

\prsubsection{propax}{Propositional Axioms}
\begin{description}
\item[syntax:] \texttt{(propax)}

\item[effect:] An application of \indtt{propax} either proves the sequent
or behaves like a \texttt{(skip)}.  Invoking \indtt{propax} proves sequents
of the form
\begin{enumerate}
\item $\ldots, \mbox{\sc false}, \ldots \vdash \Delta$,
\item $\Gamma\vdash\ldots, \mbox{\sc true}, \ldots$,
\item $\Gamma\vdash\ldots, t=t, \ldots$, or
\item $\ldots, A, \ldots\vdash\ldots, B, \ldots$, where the
sequent formulas $A$ and $B$ are syntactically equivalent (\ie\
identical upto the renaming of bound variables).
\end{enumerate}
The first two forms are not actually propositional axioms as described
in Chapter~\ref{logic}, but may easily be inferred.  The third form
above is actually an equality rule, but it is useful to group it with
the other propositional steps.  These forms correspond to sequents that
are structurally true, but it might be difficult to notice these forms
in a complicated looking sequent.

\item[usage:] \texttt{(propax)}

\item[errors:] No error messages are generated.

\item[notes:] It is important to note that the \indtt{propax} step is
automatically applied to every sequent that is ever generated in a
proof, so that there is never any need to actively invoke it.  It is
simply included here for the sake of completeness.

\end{description}

\prsubsection{split}{Conjunctive Splitting}
\begin{description}
\item[syntax:] \texttt{(split \optl\ \cargdflt{fnum}{*} \carg{depth})}

\item[effect:] Selects and splits a conjunctive formula in the current
goal sequent based on the information given in \emph{fnum}.  The \texttt{split} command splits every top-level conjunction in the selected
formula so that the resulting formulas are no longer conjunctions.  A
conjunctive formula $A$ in a goal sequent of the form $\Gamma,
A\vdash\Delta$ or $\Gamma\vdash A, \Delta$ is split by collecting lists
of antecedent and consequent formulas by recursively collecting
subformulas of $A$ as follows:
\begin{enumerate}
\item If $A$ is an antecedent formula of the form $B\vee C$, then
collect antecedent formulas $B$ and $C$

\item If $A$ is an antecedent formula of the form $B\supset C$, then
collect antecedent formula $C$ and consequent formula $B$

\item If $A$ is a consequent formula of the form $B\wedge C$, then
collect consequent formulas $B$ and $C$

\item If $A$ is a consequent formula of the form $B\iff C$, then
collect consequent formulas $B\supset C$ and $C\supset B$

\item If $A$ is an antecedent formula of the form $\texttt{IF}(B,C,D)$,
then collect antecedent formulas $B\wedge C$ and $\neg B\wedge D$

\item If $A$ is a consequent formula of the form $\texttt{IF}(B,C,D)$,
then collect consequent formulas $B\supset C$ and
$\neg B\supset D$.
\end{enumerate}
If this process yields the collection of antecedent formulas
$A_1,\ldots, A_m$ and the consequent formulas $B_1,\ldots, B_n$, then the
\indtt{split} command yields the subgoals $\Gamma, A_i\vdash\Delta$ for $0 <
i\leq m$, and $\Gamma\vdash B_j,\Delta$ for $0 < j \leq n$.  When there
is no conjunction in the current sequent as indicated by \texttt{fnum},
then the \indtt{split} rule behaves as a \texttt{(skip)}.

When the \emph{depth} argument is given, the top-level conjuncts are only
split to that given depth.  

\begin{usage}{}
\item[\texttt{(split)}] : (Same as \texttt{(split *)}.)  Yields the subgoals
resulting from splitting the first conjunction found in the current
sequent.  It is not easy to tell which conjunct would be split since a
sequent is internally represented as a list of formulas, but is
displayed in sequent form where the negated formulas appear in the
antecedent part, and the unnegated formulas appear in the consequent
part.  The \texttt{(split +)} and \texttt{(split -)} rules (see below) might
be more appropriate when such a confusion exists.

\item[\texttt{(split +)}] : splits the first consequent conjunctive formula
in the current goal sequent.

\item[\texttt{(split -)}] : splits the first antecedent conjunctive formula
in the current goal sequent.

\item[\texttt{(split -3)}] : splits the formula numbered \texttt{-3} in the
current sequent.

\item[\texttt{(split + :depth 2)}] : splits the top-level consequent conjuncts
to a maximum of two levels.
\end{usage}

\item[errors:] No error messages are generated.

\item[notes:]
\begin{itemize}
\item This command causes branching, so should be used with
caution; otherwise you will find yourself doing essentially the same
proof on many different branches.

\item The \indtt{split} command corresponds to the $\vdash\wedge$,
$\vee\vdash$, $\supset\vdash$, $\texttt{IF}\vdash$, and $\vdash\texttt{IF}$
inference rules given in Chapter~\ref{logic}.  These are all the
inference rules which cause branching.

\item To get the effect of repeatedly applying  \indtt{flatten} and \texttt{split}, use the \indtt{prop} or \indtt{bddsimp} commands.
\end{itemize}
\end{description}

\index{Proof Rules!Propositional|)}

\section{The Quantifier Rules}
\index{Proof Rules!Quantifier|(}

\begin{tabularx}{\textwidth}{|l|l|X|}\hline
\indttdol{detuple-boundvars} & \emph{defined}
  & distribute bound variables over tuple and record types \\\hline
\indttdol{generalize} & \emph{defined}
  & generalize a term by universal quantification \\\hline
\indttdol{generalize-skolem-constants} & \emph{defined}
  & generalize Skolem constants by universal quantification\\\hline
\indttdol{inst} & \emph{defined}
  & instantiate an existentially quantified formula without copying \\\hline
\indttdol{inst-cp} & \emph{defined}
  & copy and instantiate an existentially quantified formula \\\hline
\indttdol{inst?} & \emph{defined}
  & heuristically instantiate an existentially quantified formula \\\hline
\indtt{instantiate} & \emph{primitive}
  & instantiate an existentially quantified formula with a list of terms \\\hline
\indtt{instantiate-one} & \emph{defined}
  & instantiate an existentially quantified formula without duplicating results \\\hline
\indtt{skolem} & \emph{primitive}
  & \emph{Skolemize} a universal formula with specified names\\\hline
\skobangdol & \emph{defined}
  & \emph{Skolemize} a universal formula with generated names \\\hline
\indtt{skolem-typepred} & \emph{primitive}
  & \emph{Skolemize} a universal formula with generated names and include type constraints\\\hline
\indttdol{skosimp} & \emph{defined}
  & \emph{Skolemize} a universal formula with generated names and flatten \\\hline
\indttdol{skosimp*} & \emph{defined}
  & repeatedly \emph{Skolemize} with generated names and flatten \\\hline
\end{tabularx}

\prdolsubsection{detuple-boundvars}{Distribute Bound Variables}
\begin{description}

\item[syntax:] \texttt{(detuple-boundvars \optl\ \cargdflt{fnums}{*}
\carg{singles?})}

\item[effect:] A top-level sequent formula of the form \texttt{(FORALL
(x:\ [S1, S2]):\ g(x))} or \texttt{(FORALL (x:\ [\# s:\ S, t:\ T \#]):\
g(x))} is replaced by \texttt{(FORALL (x1:\ S1), (x2:\ S2):\ g(x1, x2))}
and \texttt{(FORALL (x1:\ S), (x2:\ T)):\ g((\# s := x1, t := x2 \#)))}\@.
This decomposition of tuple and record quantification is needed, for
example, to carry out an induction over one of the components.  Tuple
quantification can be introduced when instantiating parameterized theories
such as the \texttt{function} theory in the prelude.  This command is used
within the \indtt{measure-induct} strategy.

\end{description}


\prdolsubsection{generalize}{Universally Generalize a Term}
\begin{description}

\item[syntax:] \texttt{(generalize \carg{term} \carg{var} \optl\
\carg{type} \carg{fnums}{*} \cargdflt{subterms-only?}{t})}

\item[effect:] If the sequent is of the form
\texttt{a1(t), a2(t) $\vdash$ c1(t), c2(t)}, then applying the
\indtt{generalize} 
term \texttt{t} with variable \texttt{x} yields a sequent of the form
\texttt{$\vdash$ (FORALL x:~(a1(x) AND a2(x)) IMPLIES (c1(x) OR c2(x)))}\@.
More specifically, the \indtt{generalize} command collects together
all the sequent formulas from the given \emph{fnums} containing the
term \emph{term}, applies \indtt{merge-fnums} to obtain a single
formula which is then generalized by replacing  \emph{term}
by a universally quantified variable \emph{var}.

The \emph{type} option is to indicate that the universally quantified
variable should be bound with the given \emph{type}.  This is useful if
the term has a more specific type than is required of the generalization.

The generalization applies only to the subterms that are not within types
or actuals.  If a more sweeping generalization is needed, then the
\emph{subterms-only?}\ flag should be set to \texttt{nil}\@.

\end{description}

\prdolsubsection{generalize-skolem-constants}{Generalize from Skolem Constants}
\begin{description}

\item[syntax:] \texttt{(generalize-skolem-constants \optl\
\cargdflt{fnums}{*})}

\item[effect:] Applies universal generalization to the Skolem constants
that occur in the given \emph{fnums}\@.  Such a step is useful in
rearranging quantifiers by introducing skolem constants and generalizing
them over selected formulas.   

\end{description}

\prdolsubsection{inst}{Instantiate a Formula without Copying}
\begin{description}

\item[syntax:] \texttt{(inst \carg{fnum} \rest\ \carg{terms})}

\item[effect:]  This is simply a convenient form of the \indtt{instantiate}
rule where the quantified formula is not copied and the terms are
given in \rest\ form.

\item[usage:] \texttt{(inst - "a + 3" "\_" "b")}: Instantiates the first
universally quantified antecedent formula with exactly
three bound variables with \texttt{a + 3} for the first bound variable and
\texttt{b} for the third bound variable while leaving the second bound
variable uninstantiated.  
\end{description}

\prdolsubsection{inst-cp}{Copy and Instantiate a Formula}
\begin{description}

\item[syntax:] \texttt{(inst-cp \carg{fnum} \rest\ \carg{terms})}

\item[effect:]  This is simply a convenient form of the \indtt{instantiate}
rule where the quantified formula is copied and the terms are
given in \rest\ form.  
\end{description}

\prdolsubsection{inst?}{Instantiate a Formula by Matching}
\begin{description}

\item[syntax:] \texttt{(inst?~\optl\ \cargdflt{fnums}{*} \carg{subst}
\cargdflt{where}{*} \carg{copy?}\ \carg{if-match} \carg{polarity?}\
\newline \cargdflt{tcc?}{t})}

\item[effect:] This rule extends the capabilities of the \indtt{inst}
rule.  Here the given substitution \emph{subst} is used to select a
quantified formula of existential strength where the quantifier binds all
of the variables in \emph{subst}.  The rule then collects those atomic
subformulas or subterms in this formula that contain free occurrences of
all the outermost existentially quantified variables and tries to find a
match (extending \emph{subst}) for these in the sequent (or as
constrained by the \emph{where} argument).  These pattern subterms of the
quantified formula are collected as successive lists of templates
containing all the quantified variables, all but one of the quantified
variables, and so on.  The templates are essentially collected starting
from the leftmost-outermost one except for implications where the
templates for the conclusion part of the implication precede those from
the hypotheses.

In the default case when \emph{if-match} is \texttt{nil}, the first 
successful match (for all or some of the quantified variables)
for the first template with a successful match is used to generate one 
or more, partial or total instances of the chosen quantified formula.  
If a partial substitution is given but no match is found and \emph{if-match} is \texttt{nil}, then the rule goes ahead and instantiates
using the given partial substitution \emph{subst}.  If \emph{if-match}
is \texttt{t}, then the instantiation only takes place if the matching process
succeeds.  If \emph{if-match} is \texttt{all}, then the command 
returns all possible instantiations for all of the templates in the
first list of templates that yields a successful match.  Note that these
lists of templates are ordered by the number of quantified variables
that occur free in them.    If the \emph{if-match} flag is \texttt{best}, then the command chooses an
instantiation from the \texttt{all} case that generates the
fewest TCCs when typechecked.  If the \emph{if-match} flag is
\texttt{first*}, the command chooses all the instantiations of the first
successful template.

The \emph{polarity?}\ argument can be either \texttt{t} or \texttt{nil}\@.
When this is set to \texttt{t}, the \texttt{inst?}\ command is sensitive
to the polarity with which the patterns occur and it matches these
patterns only against expressions of opposing polarity.  The
polarity-sensitive matching pays attention to both boolean polarity (i.e.,
whether the expression occurs under an even or odd number of negations) as
well as arithmetic polarity (i.e., whether the expression occurs on the
lesser or greater side of an inequality.

If the \emph{tcc?}\ argument is \texttt{nil}, only instantiations that do
not lead to TCCs are selected.  There is no check to see if the TCCs are
true in the given context.

Sometimes a single \indtt{inst?}\ can only find a partial
instantiation where successive invocations of \indtt{inst?}\ 
can succeed in fully instantiating all of the quantified variables.

Note that if a bound variable name contains \texttt{\char36}, it is
sufficient to only give that part of the name preceding \texttt{\char36} in \emph{subst}.  The \emph{copy?}\ argument works exactly as in \indtt{quant} and
is used to retain a copy of the quantified formula.  Note that an
uncopied, quantified formula is automatically hidden.

\item[usage: ] \texttt{(inst?\ -1 ("x" 1) + T)}: Tries to instantiate the
quantified variables in the formula number \texttt{-1} by pattern-matching
against the subexpressions in the consequent formulas using
the given partial substitution where \texttt{x} is instantiated to \texttt{1}.
The first acceptable substitution found by pattern matching is returned.
A copy of the original formula is retained.

\texttt{(inst?\ - :if-match best)}: Tries to instantiate the first universally
quantified antecedent formula by pattern-matching the subexpressions
of this formula against all the formulas in the sequent.  
The best substitution, namely one that generates the fewest TCCs when
typechecked, is returned.

\item[errors: ] {\bf Given Substitution \ldots is not of the form \ldots:}
A substitution 
must be of the form \texttt{(\emph{var} \emph{term} \emph{var} \emph{term} \ldots)}.

{\bf Couldn't find a suitable quantified formula:}  No instantiable
formula found in the range specified by \emph{fnums}.

{\bf Couldn't find a suitable instantiation for any
quantified  formula.  Please provide partial instantiation: }
Pattern-matching was unable to instantiate the quantified variables so
that a further  hint in the form of a partial substitution might be
needed.

{\bf Given substitution \ldots
is not of the form: (<var> <term>...): }  An odd length substitution was
given.

{\bf The supplied terms should not contain free variables:}
Terms in given substitution should not contain free variables.

{\bf The types of the substituted variables contain free occurrences
of the following quantified variables: \ldots :} If the type of one
variable given in a substitution contains another quantified variable,
then that variable must also be instantiated in the substitution.

\end{description}


\prsubsection{instantiate}{Primitive Instantiation}
\begin{description}
\item[syntax:] \texttt{(instantiate \carg{fnum} \carg{terms} \optl\
\carg{copy?})}

\item[effect:] As the sequent calculus rules indicate, the universally
quantified formulas in the antecedent and the existentially quantified
formulas in the consequent are reduced by instantiating the quantified
variables with the terms that are being \emph{existentially
generalized} in the proof.  In an application of the \indtt{instantiate} rule,
\emph{fnum} is used to select the suitable quantified formula that is
either an antecedent formula of the form $(\forall x_1, \ldots, x_n :
A)$ or a consequent formula of the form $(\exists x_1, \ldots, x_n :
A)$.  The argument \emph{terms} provides the list of $n$ terms
$t_1,\ldots, t_n$ so that the chosen quantified formula is replaced by
$A[t_1/x_1,\ldots, t_n/x_n]$ in the generated subgoal.  Note that each
$t_i$ is typechecked to be of the type of $x_i$, and this typechecking
could generate additional goals corresponding to the type correctness
conditions.  As with \indtt{skolem}, it is possible to leave some
of the $x_i$ uninstantiated by supplying \texttt{"\_"} for the
corresponding $t_i$ in the \emph{terms} argument.

When the \emph{copy?}\ parameter is \texttt{t}, then a copy of the
quantified formula is saved in the subgoal sequent so that the
quantifier can be reused.  When the \emph{copy?}\ parameter is \texttt{nil}, the quantified formula is automatically hidden so that the
quantifier can be reused by revealing the hidden formulas using the \texttt{reveal} rule.

\begin{usage}{}

\item[\texttt{(instantiate * ("x + 3" "y - z"))}] : Finds the first formula in the
sequent that is either an antecedent formula of the form $(\forall x_1,
x_2 : A)$, or a consequent formula of the form $(\exists x_1, x_2 : A)$
and replaces this formula with $A[\texttt{x + 3}/x_1, \texttt{y - z}/x_2]$.

\item[\texttt{(instantiate - ("x + 3" "y - z"))}] : Searches for the first suitable
antecedent formula in the current sequent, and has the same effect as
the above invocation of \indtt{instantiate}.

\item[\texttt{(instantiate -3 ("x + 3" "y - z") T)}] :  Has the same
effect as above on the formula numbered \texttt{-3}, but also
makes a copy of the formula numbered \texttt{-3} provided it
is a formula of the form $(\forall x_1, x_2 : A)$.  

\item[\texttt{(instantiate -3 ("x + 3" "y - z") :copy?\ T)}] :  Same as above.
\end{usage}

\item[errors:]
\begin{description}

\item[{\bf No suitable quantified formula found}:] There was no formula
of existential strength in the range given by \emph{fnum} with the same
number of bound variables as the length of the supplied list of \emph{terms}.

\item[{\bf Expecting \ldots\ terms, but \ldots\ terms provided}:]\hfill Wrong
number of terms given in the \emph{terms} argument.

\item[{\bf The supplied terms should not contain free variables}:] There
can be no free variables in a PVS sequent and none are allowed to
sneak in through the \indtt{instantiate} rule.

\item[{\bf The types of the substituted variables contain free occurrences
\ldots}:]  If $x$ and $y$
are both bound by the top quantifier where the type of $y$ contains
free occurrences of $x$, then if the \indtt{instantiate} rule supplies a
substitution for $y$, it must also include a substitution for $x$.
\end{description}
In addition to the above, the \indtt{instantiate} rule can generate parser
and typecheck errors.

\item[notes:] The defined rule \indtt{inst} is generally preferred, as the
terms do not have to be in a list, and it avoids creating instances that
are already in the sequent.  In particular, \texttt{instantiate} is not
suitable for use in recursive strategies as it may simply generate the
same instance repeatedly.   To copy the quantified formula (so that it
may be instantiated with different terms) use the rule \indtt{inst-cp}.
In the related \indtt{inst?}\ rule, if the \emph{terms} argument is
missing then the rule attempts to find a suitable instantiation by
matching selected subterms of $A$ with subterms in the rest of the
sequent.  Note that \indtt{inst?}\ does not always succeed with the right
match.  It typically succeeds when there are very few possible matches.
The \indtt{instantiate} rule captures the $\forall\vdash$ and
$\vdash\exists$ sequent rules.

\end{description}

\prsubsection{instantiate-one}{Instantiate Existential Formula without Duplication}
\begin{description}
\item[syntax:] \texttt{(instantiate-one \carg{fnum} \carg{terms} \optl\
\carg{copy?})}

\item[effect:] Same as \texttt{instantiate}, but behaves as a
\texttt{skip} if the instantiation would yield a formula already in the
sequent.  \texttt{instantiate-one} may be used in \texttt{repeat} in
situations where \texttt{instantiate} would never terminate.
\end{description}

\prsubsection{skolem}{\emph{Skolemize} with Specified Names}
\begin{description}
\item[syntax:] \texttt{(skolem \carg{fnum} \carg{constants} \optl\
\carg{skolem-typepreds?})}

\item[effect:] If the formula in the current sequent indicated by \emph{fnum} is either an antecedent formula of the form $(\exists
x_1,\ldots, x_n : A)$ or a consequent formula of the form $(\forall x_1,
\ldots, x_n : A)$, and \emph{constants} is a list of new identifiers of
the form \texttt{($c_1$ \ldots $c_n$)}, then the \indtt{skolem} rule
generates a subgoal where the indicated formula has been replaced by
$A[c_1/x_1,\ldots, c_n/x_n]$.\footnote{This is slightly inaccurate since
in PVS, the type for the bound variable $x_i$ in $(\forall x_1,
\ldots, x_n : A)$ can contain free occurrences of $x_j$ where $j<i$
which is not reflected by the substitution instance $A[c_1/x_1,\ldots,
c_n/x_n]$.}   When \emph{fnum} is \texttt{*}, then the
first appropriate sequent formula (\ie\ either an antecedent existential
formula or a consequent universal formula binding $n$ variables) is
chosen for quantifier elimination.  The parameter \emph{fnum} can also be
either \texttt{+} or \texttt{-}.  If any of the $c_i$ are given as \texttt{"\_"},
then those bound variables are left alone and no corresponding skolem
constants are introduced.  If \emph{skolem-typepreds?}\ is \texttt{t}, then
typepreds will be introduced for the new constants.

\begin{usage}{}

\item[\texttt{(skolem * ("a3" "b3" "c3"))}] : The first suitable formula in
the sequent with the form $(\forall/\exists x_1, x_2, x_3 : A)$
is replaced by $A[\texttt{a3}/x_1, \texttt{b3}/x_2, \texttt{c3}/x_3]$.  As
mentioned earlier, it is not always easy to determine the first such
suitable formula in the sequent since the displayed sequent only
captures the correct order between and antecedent formulas and between
consequent formula.  The ordering of all formulas is not visible from the
displayed sequent.

\item[\texttt{(skolem - ("a3" "b3" "c3"))}] : Same as above, but for the
first suitable antecedent formula.

\item[\texttt{(skolem -3 ("a3" "b3" "c3"))}] : Same as above, but only for
formula number \texttt{-3} in the current sequent.

\item[\texttt{(skolem - ("a3" "\_" "c3"))}] : Replaces the first antecedent
formula of the form $(\exists x_1, x_2, x_3 : A)$ with $(\exists x_2 :
A)[\texttt{a1}/x_1, \texttt{a3}/x_3]$.

\end{usage}

\item[errors:]

\begin{description}

\item[{\bf No suitable quantified expression found}:] Either there is no
antecedent existentially quantified formula or consequent universally
quantified formula, or the list of skolem constants is of the wrong
length.

\item[{\bf Formula \ldots is not skolemizable}:] The indicated formula in
\emph{fnum} is not of the right form.

\item[{\bf Expecting \ldots skolem constant(s), but \ldots supplied}:]
The number of supplied skolem constants does not correspond to the
number of bound variables in the formula specified by \emph{fnum}.

\item[{\bf The supplied skolem constants must all be new names}:]\hfill
Either a skolem constant given as part of the \emph{constants} argument was
not a name or was already present in the context.

\item[{\bf Duplicate use of skolem constants}:] The list of skolem
constants contained a duplicate.

\item[\textbf{The types of the skolemized variables contain free \ldots}:]
This error is triggered when the top quantifier binds variables $x$ and
$y$ where the type of $y$ contains a free occurrence of $x$.  It is not
legal in this situation to supply a skolem constant for $y$ without
providing one for $x$.
\end{description}

\item[notes:] The \skobang{} rule  automatically
generates skolem constants and is usually an easier alternative to the
\indtt{skolem} rule.  The \indtt{skolem} rule is useful when adopting 
certain conventions about naming skolem constants within proof
strategies.  The \indtt{skolem} rule captures the $\exists\vdash$ and
$\vdash\forall$ sequent rules.  

\end{description}

\prdolsubsection{skolem!}{\emph{Skolemize} with Generated Names}

\begin{description}

\item[syntax:] \texttt{(skolem!\@ \optl\ \cargdflt{fnum}{*}
\carg{keep-underscore?})}

\item[effect:] Automatically generates Skolem names for the \emph{names} argument for the \indtt{skolem} rule.  These names have the
form \texttt{x!$n$} when the bound variable being named is \texttt{x}.  This
rule is a dangerous one to include in defined rules or within an \texttt{apply} rule since the names being generated could change when a proof is
rerun.  
When the \emph{fnum} argument is \texttt{*}, \texttt{+}, or \texttt{-}, the
first Skolemizable formula in this range is selected.
There is a small loss of robustness with the commands
\skobang{}, \indtt{skosimp}, and \indtt{skosimp*} since
any changes in the constants generated due to a change in the
formula or a reordering of proof steps can affect the behavior
of subsequent commands that explicitly mention these constants.

The \emph{keep-underscore?}\ flag when \texttt{t} ensures that
the bound variable \texttt{x\_1} is replaced by a skolem constant
of the form \texttt{x\_1!n} rather than the default \texttt{x!n} for
some number \texttt{n}\@.  
\end{description}

\prdolsubsection{skolem-typepred}{\emph{Skolemize} with Type Constraints}
\begin{description}

\item[syntax:] \texttt{(skolem-typepred  \optl\ \cargdflt{fnum}{*})}

\item[effect:] This is a variant of \indttbang{skolem}
where  the type constraints on the generated
skolem constants are introduced as antecedent formulas. 
\end{description}


\prdolsubsection{skosimp}{\emph{Skolemize} then Flatten}
\begin{description}

\item[syntax:] \texttt{(skosimp \optl\ \cargdflt{fnum}{*} \carg{preds?})}

\item[effect:] This is a short form for \texttt{(then (skolem \emph{fnum})
(flatten))}.  When \texttt{preds?}\ is \texttt{t}, \texttt{skolem-typepred} is used
in place of \texttt{skolem} so that the type constraints on the generated
skolem constants are introduced as antecedent formulas. 
\end{description}


\prdolsubsection{skosimp*}{Repeatedly \emph{Skolemize} then Flatten}
\begin{description}

\item[syntax:] \texttt{(skosimp* \optl\ \carg{preds?})}

\item[effect:] This is a short form for \texttt{(repeat (then (skolem)
(flatten)))}.  As with \texttt{skosimp}, when the \texttt{preds?}\ flag is \texttt{t}, the command \texttt{skolem-typepred} is used in place of \texttt{skolem}\@.
\end{description}


\index{Proof Rules!Quantifier|)}


\section{The Equality Rules}
\index{Proof Rules!Equality|(}

\begin{tabularx}{\textwidth}{|l|l|X|}\hline
\indtt{beta} & \emph{primitive}
  & reduce $\lambda$-, record, tuple, cotuple, update, and datatype redexes\\\hline
\indttdol{case-replace} & \emph{defined}
  & case-split and replace using an equation\\\hline
\indtt{name} & \emph{primitive}
  & introduce a name for an expression\\\hline
\indttdol{name-case-replace} & \emph{defined}
  & replace one expression with another, then rename \\\hline
\indttdol{name-replace} & \emph{defined}
  & replace an expression with a name\\\hline
\indttdol{name-replace*} & \emph{defined}
  & replace expressions with names\\\hline
\indtt{replace} & \emph{primitive}
  & replace using an equation\\\hline
\indtt{replace*} & \emph{primitive}
  & replace using equations\\\hline
\indtt{same-name} & \emph{primitive}
  & equate two constants with distinct but equal actuals\\\hline 
\end{tabularx}

\prsubsection{beta}{Beta Reduce}
\begin{description}
\item[syntax:] \texttt{(beta \optl\ \cargdflt{fnums}{*}
\carg{rewrite-flag} \cargdflt{let-reduce?}{t})}

\item[effect:] The \indtt{beta} rule rewrites certain expressions called
\emph{redexes} to their reduced forms.  The various forms of redexes
and their reduced forms are:
\begin{itemize}

\item $(\lambda x_1 \ldots x_n : e)(t_1,\ldots,t_n)$ reduces to
$e[t_1/x_1,\ldots,t_n/x_n]$.  Note that \texttt{LET} and \texttt{WHERE}
expressions are syntactic sugar for $\lambda$-redexes and will also
be beta-reduced by \texttt{beta}\@.  

\item \texttt{proj}$_i(t_1,\ldots,t_n)$ reduces to $t_i$.

\item $\textit{label}_i(\texttt{(\#} \emph{label}_1 \texttt{:=} t_1, \ldots,
                              \emph{label}_n \texttt{:=} t_n \texttt{\#)})$
reduces to $t_i$.

\item \texttt{out}$_i($\texttt{in}$_i(t))$ reduces to $t$.

\item \texttt{in}$_i($\texttt{out}$_i(t))$ reduces to $t$.

\item $\emph{accessor}_i(\emph{constructor}(t_1,\ldots, t_n))$ reduces
to $t_i$, where \emph{constructor} is an abstract datatype constructor,
and \emph{accessor}$_i$ is an accessor of that constructor.

\item \texttt{CASES $c(t_1,\ldots, t_n)$ OF
               \ldots, $c(x_1,\ldots, x_n)\ : e$,
               \ldots\ ENDCASES\hfill}
reduces to $e[t_1/x_1,\ldots,t_n/x_n]$, where $c$ is a constructor for
some datatype.

\item \texttt{\emph{label}$(r$ WITH [\ldots, \emph{label} := e,\ldots]$)$}
reduces to $e$, provided \texttt{\emph{label} := $e$} is the last
update of \emph{label} in the assignment above.

\item \texttt{$(f$ WITH [\ldots, $(i)$ := $e$, \ldots]$)(j)$} reduces to $e$ if
it can be shown that $i=j$, and the assignments following \texttt{$(i)$ := $e$} do not affect $f(j)$.  If it can be shown using the
decision procedures that none of the updates affect the value of $f$
at $j$, then the expression simply reduces to $f(j)$.

\end{itemize}

The \emph{rewrite-flag} argument is typically omitted.  When its value
is \texttt{lr}, it indicates that only the right-hand side of a formula
that is an equality should be simplified using beta-reduction.
Similarly, if the value of \emph{rewrite-flag} is \texttt{rl}, then only
the left-hand side of any equality is simplified.

The \emph{let-reduce?}\ flag indicates whether \texttt{LET} expressions
should also be reduced.

\begin{usage}{}
\item[\texttt{(beta)}] : Same as \texttt{(beta *)}.  Yields the subgoal obtained
by reducing all of the redexes in the current sequent.

\item[\texttt{(beta +)}] : Reduces all the redexes in the
consequent formulas in the current sequent.

\item[\texttt{(beta (-1 2 3))}] : Reduces all of the redexes in the formulas
numbered \texttt{-1}, \texttt{2}, and \texttt{3} in the current sequent.
\end{usage}

\item[errors:] There are no error messages generated.  If there are no
redexes to be reduced in the given range \emph{fnums}, then the message
\texttt{No suitable redexes found} is generated.

\item[notes:] The commands \indtt{assert} and \indtt{simplify}
also carry out beta-reduction among many other 
simplifications.  
\end{description}

\prdolsubsection{case-replace}{Introduce Equation and Replace}
\begin{description}
\item[syntax:] \texttt{(case-replace \carg{formula} \optl\ \carg{hide?})}

\item[effect:] This is defined as \texttt{(then@ (case \carg{formula})
(replace -1 :hide?\ \carg{hide?}))}.  It case splits on an equality (note
that a non-equality $A$ is interpreted as an equality $A = \mbox{\sc
true}$) and replaces the LHS by the RHS in the rest of the sequent in the
branch where the equality is an antecedent.  The \emph{hide?}\ flag
indicates that the equality should be hidden after the replacement in this
branch.

\item[usage:] \texttt{(case-replace "c = 0")}: Replaces \texttt{c} by
\texttt{0} in the current subgoal and generate a second subgoal with the
proof obligation \texttt{c = 0}\@.

\item[errors: ] This command can generate a parse or type error if the
given \emph{formula} is not well-formed.
\end{description}


\prsubsection{name}{Introduce a Name for an Expression}
\begin{description}
\item[syntax:] \texttt{(name \carg{name} \carg{expr})}

\item[effect:] Yields a subgoal where a formula of the form $\textit{expr}
= \textit{name}$ is added as a new antecedent formula.  This is typically
useful as a step towards generalizing a formula by replacing expressions
with constants (using the \indtt{replace} rule).  The given \emph{name}
must be new.  In addition to the new antecedent, a definition is generated
for the name, which may subsequently be expanded or rewritten.
It is also treated as an \texttt{AUTO\_REWRITE-}, so that it is not
expanded accidentally with the next \texttt{grind}, for example.

\item[usage:] \texttt{(name "d5" "(a + b + c)")} : Introduces the
antecedent formula \texttt{(a + b + c) = d5}.

\item[errors:]  This command can generate a parse or type error
if the given \emph{expr} is not well-formed.

{\bf \ldots is not a symbol:}  The \emph{name} argument must be a symbol.

{\bf \ldots is already declared: } The name argument must be a new name.
\end{description}

\prdolsubsection{name-case-replace}{Replace one Expression by Another,
then Rename}
\begin{description}
\item[syntax:] \texttt{(name-case-replace \carg{expr1} \carg{expr2}
\emph{name})}

\item[effect:]  This command creates the equality \texttt{\emph{expr1} =
\emph{expr2}}, does a \indtt{case-replace} on it, and then does a
\indtt{(name-replace \emph{name} \emph{expr2})}.
\end{description}

\prdolsubsection{name-replace}{Replace an Expression by a Name}
\begin{description}
\item[syntax:] \texttt{(name-replace \carg{name} \carg{expr} \optl\
\cargdflt{hide?}{t})}

\item[effect:]  This command is just the strategy
\texttt{(then@ (name \emph{name} \emph{expr}) (replace -1))}.
It replaces the expression \emph{expr} by \emph{name}
everywhere in the current sequent.  The equality between 
\texttt{name} and \texttt{expr} is hidden by default, unless
\emph{hide?}\ is \texttt{nil}\@. 
\end{description}

\prdolsubsection{name-replace*}{Replace Expressions by Names}
\begin{description}
\item[syntax:] \texttt{(name-replace* \carg{name-and-exprs} \optl\
\cargdflt{hide?}{t})}

\item[effect:]  This command is an iterated form of name-replace.

The \texttt{name-and-exprs} argument must be
of the form \texttt{($\pair{{\it name}_1}$ $\pair{{\it expr}_1}$
\ldots)}\@.  The command 
replaces each ${\it expr}_i$ in the current sequent with the corresponding
${\it name}_i$\@.  
\end{description}


\prsubsection{replace}{Replace using an Equation}
\begin{description}
\item[syntax:] \texttt{(replace \carg{fnum} \optl\ \cargdflt{fnums}{*}
\carg{dir} \carg{hide?}\ \carg{actuals?}\ \carg{dont-delete?})}

\item[effect:] The \texttt{replace} rule is typically used to rewrite some
selection of the formulas in the current sequent using an antecedent
equality formula of the form $l=r$.  The equality formula to be used is
indicated by the \emph{fnum} argument.  The targets of the rewrites are
listed in the \emph{fnums} argument.  When the \emph{direction} argument
is \texttt{rl} (denoting ``right-to-left''), the target occurrences of $r$
in the sequent are rewritten to $l$.  Otherwise, when the \emph{direction}
parameter is different from \texttt{rl}, the target occurrences of $l$ are
rewritten to $r$.  If \emph{fnum} is the number for an antecedent formula
$A$ that is not an equality, then the formula is regarded as an equality
of the form $A = \mbox{\sc true}$.  If \emph{fnum} is the number of a
consequent formula $A$, then the formula is regarded by \indtt{replace} as
an equality of the form $A = \mbox{\sc false}$.  Note that the formula
indicated by \emph{fnum} is unaffected by \indtt{replace}.

When \emph{hide?}\ is \texttt{t}, the formula indicated by \emph{fnum}
that is used for replacement is hidden using the \indtt{hide} command.
When \emph{actuals?}\ is \texttt{t}, the replacement is carried out within
actual parameters of names, including types.  Otherwise, the replacement
only occurs at the expression level.  When the \emph{dont-delete?}\ flag is
\texttt{t}, top-level sequent formulas are not deleted through being
replaced by \texttt{TRUE} or \texttt{FALSE}.

\begin{usage}{}

\item[\texttt{(replace -1)}] : If the formula numbered \texttt{-1} in the
current sequent has the form $l = r$, then this application of the
replace rule generates a subgoal where every occurrence of a term
syntactically equivalent to $l$ is replaced by $r$ in every formula of
the sequent other than in formula number \texttt{-1}.  If the formula
numbered \texttt{-1}, call it $A$, is not an equality, then it is treated
as being an equality of the form $A = \mbox{\sc true}$.

\item[\texttt{(replace -1 (-1 2 3) RL)}] : If the formula numbered \texttt{-1}
in the current sequent has the form $l = r$, then this application of
\indtt{replace} generates a subgoal where all the occurrences of $r$ in
the formulas numbered \texttt{2} and \texttt{3} are replaced by $l$.  Note that
formula number \texttt{-1} remains untouched.

\item[\texttt{(replace 2)}] : Yields the subgoal got by replacing all
occurrences of the formula numbered \texttt{2} in the rest of the current
sequent, by {\sc false}.  Note that in the sequent representation,
it is okay to use the negation of a consequent formula as an assumption.
\end{usage}

\item[errors:]
\begin{description}

\item[{\bf No sequent formula corresponding to \ldots}:] This means that
the \emph{fnum} argument was out of range and did not refer to a
formula in the current sequent.

\item[{\bf \ldots must be $*, +, -,$ an integer, or list of integers}:]
The given \emph{fnums} argument did not meet the
criterion listed in the error message.
\end{description}

\item[notes:] One open issue regarding the \indtt{replace} rule is whether
it is useful to have more refined control over the target occurrences of
the rewrite than is provided by the \emph{fnums} argument.
The defined rule \indtt{case-replace} is used to assume an
equality and apply it in the form of a replacement.  The \indtt{replace}
rule corresponds to the sequent rule {\bf Repl}.  

\end{description}

\prsubsection{replace*}{Replace using Equations}
\begin{description}
\item[syntax:] \texttt{(replace* \rest\ \carg{fnums})}

\item[effect:] Iteratively applies the \texttt{replace} command
to each of the formulas indicated by the numbers in \texttt{fnums}\@.
Unlike \texttt{replace}, only left-to-right replacement is possible
and the formulas used in the replacement cannot be hidden.
\end{description}

\prsubsection{same-name}{Equate Names with Distinct Actuals}
\begin{description}
\item[syntax:] \texttt{(same-name \carg{name1} \carg{name2} \optl\
\carg{type})}

\item[effect:] Two names such as $\texttt{cons[\{i:\ nat | i > 0\}]}$ and
$\texttt{cons[\{i:\ nat | i /= 0\}]}$ can be denotationally equal yet
syntactically distinct.  The command \indtt{same-name} can be used to
introduce an antecedent formula asserting the equality of two such names,
\eg, $\texttt{cons[\{i:\ nat | i > 0\}]} = \texttt{cons[\{i:\ nat | i /=
0\}]}$, while generating the proof obligations required to show that the
actuals coincide.  In the above case, the obligation would be to show that
\texttt{FORALL (i:\ nat):\ i > 0 IFF i /= 0}\@.

The \emph{type} argument can be used to disambiguate the \emph{name}
references in case it has been overloaded.

\item[errors:]
{\bf Argument \ldots is not a name:} Names can have a theory prefix
and/or actuals, but cannot be compound expressions.

{\bf Argument \ldots does not typecheck uniquely:} Need to supply
either theory prefixes and/or actuals for the name, or the \emph{type}
argument to disambiguate the name reference.

{\bf Argument \ldots must have actuals:} It does not make sense to use
\indtt{same-name} unless the names have actuals that can be
demonstrated to be equal.

{\bf Arguments \ldots and \ldots must have identical identifiers:}
Only the actuals can be syntactically different between \emph{name1} and
\emph{name2}\@.
\end{description}
\index{Proof Rules!Equality|)}

\section{Using Definitions and Lemmas}
\index{Proof Rules!Introducing Lemmas|(}

\begin{tabularx}{\textwidth}{|l|l|X|}\hline
\indtt{expand} & \emph{primitive}
  & expand (and simplify) a function definition\\\hline
\indttdol{expand*} & \emph{defined}
  & expand several function definitions\\\hline
\indttdol{forward-chain} & \emph{defined}
  & forward chain on an implication lemma\\\hline
\indttdol{forward-chain*} & \emph{defined}
  & forward chain on a list of lemmas repeatedly\\\hline
\indttdol{forward-chain@} & \emph{defined}
  & forward chain on a list of lemmas until one succeeds \\\hline
\indttdol{forward-chain-theory} & \emph{defined}
  & forward chain on formulas of a theory\\\hline
\indtt{lemma} & \emph{primitive}
  & introduce an axiom, lemma, or definition instance\\\hline
\indttdol{rewrite} & \emph{defined}
  & match and rewrite using a lemma or antecedent \\\hline
\indttdol{rewrite-lemma} & \emph{defined}
  & rewrite using an instance of lemma \\\hline
\indttdol{rewrite-with-fnum} & \emph{defined}
  & rewrite using an antecedent \\\hline
\indttdol{use} & \emph{defined}
  & introduce a lemma and instantiate/reduce \\\hline
\indttdol{use*} & \emph{defined}
  & introduce lemmas and instantiate/reduce \\\hline
\end{tabularx}

\prsubsection{expand}{Expand a Definition}
\begin{description}
\item[syntax:] \texttt{(expand \carg{function-name} \optl\
\cargdflt{fnum}{*} \carg{occurrence} \carg{if-simplifies}
\newline\hspace*{1in}
\carg{assert?})}

\item[effect:] Expands (and simplifies) the definition of \emph{name} at a
given \emph{occurrence}.  If \emph{occurrence} is not given, then all
instances of the definition are expanded.  The \emph{occurrence} is given
as a number $n$ referring to the $n${th} occurrence of the function symbol
counting from the left, or as a list of such numbers.  If the
\emph{if-simplifies} flag is \texttt{t}, then any expansion within a
sequent formula occurs only if the expanded form can be simplified (using
the decision procedures).  The \emph{if-simplifies} flag is needed to
control infinite expansions in case \indtt{expand} is used repeatedly
inside a strategy.  In the default case when \emph{assert?}\ is
\texttt{nil}, \indtt{expand} applies the \indtt{simplify} step with the
default settings to any sequent formula in which a definition is expanded.
When \emph{assert?}\ is \texttt{t}, \indtt{expand} applies the
\indtt{assert} version of \indtt{simplify} to any sequent formulas
affected by definition expansion.  The \emph{assert?}\ flag can also be
\texttt{none} in which case no simplification is applied to the sequent
formula following expansion.

If the \emph{function-name} is more than just an identifier, it is treated
as a pattern, and any unspecified part of the \emph{function-name} is
treated as matching anything.  Thus \texttt{th.foo} will match
\texttt{foo} only if it is from theory \texttt{th}, but will match any
instance or mapping of \texttt{th}.  \texttt{foo[int]} will match any
occurrence of \texttt{foo} of any theory, as long as it has a single
parameter matching \texttt{int}.  The \emph{occurrence} number counts only
the matching instances.

\begin{usage}{}

\item[\texttt{(expand "sum")}] : Expands the definition of \texttt{sum}
throughout the current sequent, whether they simplify or not.  The
resulting expressions are all simplified using decision procedures
and rewriting. 

\item[\texttt{(expand "sum" 1)}] : Expands \texttt{sum} throughout the formula
labeled \texttt{1}.

\item[\texttt{(expand "sum" 1 2)}] : Expands the second occurrence of \texttt{sum} in the formula labeled \texttt{1}.

\item[\texttt{(expand "sum" :if-simplifies t)}] : Expands those occurrences
of \texttt{sum} whose definitions can be simplified by means of the
decision procedures.  This is only relevant in the situation where the
definition is a \indtt{CASES} or \indtt{IF} expression.  The definition
expansion only occurs if such an expression  simplifies to one of
its branches.  


\item[\texttt{(expand "sum" :assert?\ T)}] : Expands \texttt{sum}, but
uses \indtt{assert} instead of \indtt{simplify} in the simplification process. 
\end{usage}

\item[errors:]
{\bf Occurrence \ldots\ must be nil, a positive number or a 
list of positive numbers:} Self-explanatory.

\item[notes:]  Typically, the defined rule \indtt{rewrite} can be used instead of
\indtt{expand} but \indtt{expand} has some advantages:
\begin{itemize}
\item \indtt{expand} is faster, since definitions are simple
(unconditional) equations.

\item \indtt{expand} does not require \emph{name} to be fully resolved;
it can use the occurrence to get the type information needed.

\item \indtt{expand} allows a specific \emph{occurrence}
or occurrences 
of a function symbol to be expanded.

\item \indtt{expand} can rewrite subterms containing variables that are
bound in some superterm, \eg, If $f(x)$ is defined as $g(h(x))$, then
\indtt{expand} would be able to rewrite $(\forall x. f(x) = 0)$ as
$(\forall x. g(h(x)) = 0)$, but \indtt{rewrite} would not.  
\end{itemize}

\end{description}

\prdolsubsection{expand*}{Expand Several Definitions}
\begin{description}
\item[syntax:] \texttt{(expand* \rest\ \carg{names})}

\item[effect:] An iterated version of \texttt{expand} that applies
the \texttt{expand} command to each of a list of function names.

\end{description}

\prdolsubsection{forward-chain}{Forward Chain}
\begin{description}
\item[syntax:] \texttt{(forward-chain \carg{lemma-or-fnum} \optl\
\carg{quiet?})}

\item[effect:] This rule is used to forward chain on the given lemma or
antecedent formula number.  If the given lemma or antecedent formula has
the form $A_1 \wedge \ldots\wedge A_n \supset C$, then this rule tries to
match the formulas $A_i$ against the antecedent formulas of the current
sequent.  If the match succeeds, the corresponding instance of $C$ is
added to the as an antecedent formula to current sequent.  If this
instance of $C$ is already in the current sequent, then the
\indtt{forward-chain} rule looks for other instances of the $A_i$ in the
current sequent.  An \emph{fnum} argument can be either \texttt{-},
\texttt{*}, or an antecedent formula number.  Backtracking messages are
printed unless \emph{quiet?}\ is \texttt{t}.

\end{description}

\prdolsubsection{forward-chain*}{Forward Chain Repeatedly}
\begin{description}
\item[syntax:] \texttt{(forward-chain \rest\ \carg{lemmas-or-fnums})}

\item[effect:] This rule is used to forward chain on the list of lemmas or
antecedent formula numbers (they can be freely mixed).  It 
invokes \indtt{forward-chain} on each element of the list until no more
changes occur.  If any element is successful, it starts over at the
beginning of the list.  Of course, there is no guarantee of termination.
\end{description}

\prdolsubsection{forward-chain@}{Forward Chain on a List}
\begin{description}
\item[syntax:] \texttt{(forward-chain@ \rest\ \carg{lemmas-or-fnums})}

\item[effect:] This rule is used to forward chain on a given list of
lemmas or formula numbers (they may be freely mixed) until one of them
succeeds.

\end{description}

\prdolsubsection{forward-chain-theory}{Forward Chain on a Theory}
\begin{description}
\item[syntax:] \texttt{(forward-chain \carg{theory})}

\item[effect:] This rule is used to forward chain on the lemmas of the
given theory.  It collects lemmas of the form $A_1 \wedge \ldots\wedge A_n
\supset C$ from the specified \emph{theory}, then invokes
\indtt{forward-chain*} to repeatedly forward chain until no more changes
occur.  Of course, it is easy to create lemmas for which forward chaining
will never terminate.

\end{description}

\prsubsection{lemma}{Introduce a Lemma}
\begin{description}
\item[syntax:] \texttt{(lemma \carg{name} \optl\ \carg{subst})}

\item[effect:] This rule introduces an instance of the lemma named \emph{name} corresponding to the substitutions supplied in \emph{subst} as
a new formula in the sequent.  Axioms, assumptions, and function
definitions are also seen as lemmas for the purpose of this rule.  There
can be several forms for the definition of a function.  For example,
there are four possible forms for the definition of a function $f$ such
that $f(x,y)(u)(v)$ is given to be $e$, for some term $e$:
\begin{enumerate}
\item $f(x,y)(u)(v) = e$

\item $f(x,y)(u) = (\lambda v : e)$

\item $f(x,y) = (\lambda u : (\lambda v : e))$

\item $f = (\lambda x, y : (\lambda u : (\lambda v : e)))$
\end{enumerate}
In such a situation, the \indtt{lemma} rule picks the first form if \emph{subst} contains substitutions for $x, y, u$, and $v$, and in general,
it picks the last definition (in the order of presentation above) in
which all the variables in the substitution \emph{subst} occur free in
the definition.

In using the \indtt{lemma} rule, \emph{name} must name a lemma that is
visible in the context of the statement being proved, and \emph{substs}
must be a list of substitutions of the form $(x_1\ t_1\ \ldots\ x_n\
t_n)$.  Let $A$ be the universally closed form of the lemma named \emph{name}, the lemma rule checks that each $x_i$ in \emph{subst} is a
\emph{substitutable} variable in $A$, \ie\ $A$ must have the form
$(\forall y_1,\ldots, y_m : B)$ and $x_i$ is either identical to some
$y_j$ or is substitutable in $B$.  PVS checks that each $t_i$ is
of the type of the corresponding $x_i$ and does not contain any free
variables.  Note that it is possible that there are many possible
instances of lemmas named \emph{name} either because these lemmas come
from different theories that are both visible in the current context or
from different instances of the same parametric theory.  The \indtt{lemma}
rule attempts to resolve such an ambiguity by using the information
given by the substitution \emph{subst}.  If this information is not
enough to unambiguously choose the named lemma, then the lemma name must
be supplied in a more complete form.  The form \emph{identifier[actuals]} for \emph{name} can be used in the case when the
generic theory where  \emph{name} occurs is unique, but the required
instantiation for the parameters of the theory is not obvious from the
context (\ie\ the sequent).  The
form \emph{theoryname.identifier[actuals]} is to be used to further
disambiguate the reference to the lemma.

The lemma rule can generate additional subgoals due to the type
correctness conditions.  As a consequence of the substitution,
this rule can also generate parsing and typechecking errors.

\begin{usage}
{\hspace*{0.2in}Consider the situation where the theory
\texttt{boolean\_props} contains a lemma named \texttt{assoc} stating the
associativity of conjunction, and the theory \texttt{listprops} with a
type parameter \texttt{t} has a lemma also named \texttt{assoc} stating
the associativity of \texttt{append}.  Both associativity lemmas are
stated in terms of the variables \texttt{x}, \texttt{y}, and \texttt{z}.}

\item[\texttt{(lemma "assoc")}] : This generates an error message asserting
that the given name could not be resolved, and behaves like \texttt{(skip)}, otherwise.

\item[\texttt{(lemma "boolean\_props.assoc")}] : Adds the formula
\begin{center}
\texttt{(FORALL x, y, z:\ ((x AND y) AND z) = (x AND (y AND z)))}
\end{center}
to the antecedent of the current sequent.

\item[\texttt{(lemma "assoc" ("x" "TRUE" "z" "FALSE"))}] : Adds the
formula
\begin{center}
\texttt{(FORALL  y:\ ((TRUE AND y) AND FALSE) = (TRUE AND (y AND FALSE)))}
\end{center}
to the ante\-ce\-dent of the current sequent.  Notice that the
substitution has been used to resolve the lemma name.

\item[\texttt{(lemma "assoc" ("x" "true" "z" "false" "y" "A"))}] : Adds the
formula
\begin{center}
\texttt{((TRUE AND A) AND FALSE) = (TRUE AND (A AND FALSE))}
\end{center}
to the antecedent of the current sequent.

\item[\texttt{(lemma "assoc[list[nat]]")}] : Adds the statement of the
associativity of \texttt{append} for lists of natural numbers to the
antecedent of the current sequent.
\end{usage}

\item[errors:]
\begin{description}
\item[{\bf The following are not possible variables \ldots}:]  One of the
expressions in a variable position in the given substitution was not a
name.  Check the substitution to see if it has the form \texttt{($x_1$
$t_1$ \ldots $x_n$ $t_n$)}, where the $x_i$ are all variable names.

\item[{\bf The form of a substitution is \ldots}:] This means that the
substitution argument was a list of odd length and did not have the form
\texttt{($x_1$ $t_1$ \ldots $x_n$ $t_n$)}.

\item[{\bf Irrelevant free variables \ldots in substitution}:] As with
other rules that introduce new expressions into the sequent, no free
variables can be allowed.

\item[{\bf Couldn't find a definition or a lemma named \ldots}:] There
is no lemma or definition with the given name in the current context.
Check the name.

\item[{\bf Unable to resolve \ldots relative to substitution}:] The
given name led to an ambiguity that could not be resolved from the given
substitutions.  Check the name or the substitutions, make the name more
explicit, or provide additional substitutions.
\end{description}

\item[notes:] The defined rules \indtt{rewrite} and \indtt{rewrite-lemma}
employ the \indtt{lemma} and \indtt{replace} rules to apply a lemma as a
rewrite rule.  The 
rules \indtt{rewrite}, \indtt{rewrite-lemma}, and \indtt{expand} are ways to
expand definitions without introducing any new antecedents.

The \indtt{lemma} rule is a form of the Cut rule where one of the
branches has been separately proved.
\end{description}


\prdolsubsection{rewrite}{Match/Rewrite with a Lemma or Antecedent}
\begin{description}

\item[syntax:] \texttt{(rewrite \carg{lemma-or-fnum} \optl\
\cargdflt{fnums}{*} \carg{subst} \cargdflt{target-fnums}{*}
\newline\hspace*{1in}
\cargdflt{dir}{lr} \cargdflt{order}{in} \carg{dont-delete?})}

\item[effect:] The \indtt{rewrite} rule extends \indtt{rewrite-lemma}.  It
tries to automatically determine the required substitutions by matching
the conclusion of the lemma against expressions in the formulas in
\emph{fnums}.  This rule is always to be preferred to
\indtt{rewrite-lemma} since it does the hard work of figuring out the
substitutions.  The \emph{target-fnums} corresponds to the \emph{fnums}
argument of \texttt{rewrite-lemma}.  If the \emph{order} argument is
\texttt{in} (which is the default), then the matching is to be done
inside-out, and if it is \texttt{out}, the matching is done outside in.
When the \emph{dont-delete?}\ flag is \texttt{t}, top-level sequent
formulas are not deleted through being replaced by \texttt{TRUE} or
\texttt{FALSE}.

\item[usage:]
\begin{description}
\item[\texttt{(rewrite "assoc")}: ] Finds and rewrites a single instance of
the lemma \texttt{assoc} throughout the current goal.

\item[\texttt{(rewrite "assoc" +)}: ]  Looks for a matching instantiation for
the lemma \texttt{assoc} in the consequent formulas but rewrites with
this lemma instance throughout the current goal.

\item[\texttt{(rewrite "assoc" + ("x" "A" "z" "B") -)}: ]
Looks for a matching instantiation in the consequent formulas
for the lemma \texttt{assoc} 
that extends the given substitution
but only rewrites with this instance over the antecedent formulas.

\item[\texttt{(rewrite "assoc" :dir RL :order OUT)}:]
Searches for a matching instance of the right-hand side of the
lemma \texttt{assoc} in a leftmost-outermost order and rewrites
the  instance of the right-hand side by the corresponding
instance of the left-hand side.
\end{description}

\item[errors:]
\begin{description}
\item[No resolution for \ldots:] No such lemma was found in the current
context.

\item[Substitution \ldots must be an even length list:] Self-explanatory.

\item[No sequent formulas for \ldots:] The \emph{fnums} argument is
incorrectly given.

\item[No matching instance for \ldots found:] The current goal does not
contain any instances of the rewritable part of the given lemma.
\end{description}

\item[notes:]  It is usually more effective to install a rewrite rule
using \indtt{auto-rewrite} (or its variants) than to use the
\indtt{rewrite} command.  The \indtt{rewrite} command is typically useful when
the rewrite generates a condition or a TCC proof obligation that cannot be
discharged automatically.  
\end{description}


\prdolsubsection{rewrite-lemma}{Match/Rewrite using a Lemma}
\begin{description}
\item[syntax:] \texttt{(rewrite-lemma \carg{lemma} \carg{subst}
\optl\ \cargdflt{fnums}{*} \cargdflt{dir}{lr} \carg{dont-delete?})}

\item[effect:] This is an extension of the \indtt{lemma} rule that carries
out rewriting given the required substitutions.  Here \emph{name} is
either the name for a lemma or a definition that can be used as a rewrite
rule or 
\emph{subst} is a substitution list of the form \texttt{($x_1$ $a_1$
\ldots\ $x_n$ $a_n$)}.  Each $t_i$ must be a term with no free variables,
and each $x_i$ is the identifier for a substitutable variable in \emph{name}, \ie\ one that is either a free variable or is universally
quantified at the block of universal quantifiers at the outermost level
of the formula.  The substitution list must provide substitutions for
all the substitutable variables, otherwise the rule will not carry out a
rewrite.  In the case of definitions of curried operators, this rule
picks the least curried form whose left-hand side includes all the
variables for which substitutions have been provided.  The formula
numbers in \emph{fnums} are the target formulas for where the rewriting
occurs.  The \emph{dir} is either \texttt{lr} (by default) indicate a
left-to-right use of the lemma as a rewrite rule, or \texttt{rl} for a
right-to-left use.  The \indtt{rewrite-lemma} rule also has some
capability of resolving the \emph{name} from the given
substitutions, \ie\ it tries to figure out the theory instance for the
lemma to be used.  When the \emph{dont-delete?}\ flag is \texttt{t},
top-level sequent formulas are not deleted through being replaced by
\texttt{TRUE} or \texttt{FALSE}.

\item[usage:] This command is similar to \indtt{rewrite} except that the
\emph{subst} argument is required and the entire substitution has to be
provided.

\item[notes:] This command is rarely used since the
\indtt{rewrite} command (which is defined in terms of \indtt{rewrite-lemma})
is almost always preferable.

\end{description}

\prdolsubsection{rewrite-with-fnum}{Match/Rewrite using an Antecedent}
\begin{description}

\item[syntax:] \texttt{(rewrite-with-fnum \carg{fnum} \optl\ \carg{subst}
\cargdflt{fnums}{*} \cargdflt{dir}{lr}
\newline\hspace*{1in}
\carg{dont-delete?})}

\item[effect:] Applies the \texttt{rewrite} command to an antecedent
formula indicated by \emph{fnum} which is used as a rewrite rule.  The
input substitution \emph{subst} is used to guide the matching process to
find a match that extends \emph{subst}\@.  The optional \emph{fnums}
argument is used to direct the command to look for matches within the
sequent formulas indicated by \emph{fnums}\@.  When \emph{dir} is
\texttt{lr}, the antecedent formula is used a rewrite rule in the
left-to-right direction, and when it is \texttt{rl}, the rewriting occurs
in the right-to-left direction.  When the \emph{dont-delete?}\ flag is
\texttt{t}, top-level sequent formulas are not deleted through being
replaced by \texttt{TRUE} or \texttt{FALSE}.

\end{description}

\prdolsubsection{use}{Introduce a Lemma and Instantiate/Reduce}
\begin{description}

\item[syntax:] \texttt{(use \carg{lemma} \optl\ \carg{subst}
\cargdflt{if-match}{best} \cargdflt{instantiator}{inst?}\
\newline\hspace*{1in}
\carg{polarity?}\ \carg{let-reduce?})}

\item[effect:] An extension of the \indtt{lemma} command where the formula
introduced is subject to repeated instantiation and beta-reduction using
the specified \emph{instantiator} (default \indtt{inst?})\ and
\indtt{beta} commands.  This is usually an effective alternative to the
\indtt{lemma} command.  The \emph{subst} argument is as in the
\indtt{lemma} command.  The \emph{let-reduce?}\ argument is as for the
\texttt{beta} command, and controls whether \texttt{LET} expressions are
reduced.

The \emph{instantiator} argument allows an instantiator to be provided; it
defaults to \texttt{inst?}, but may be any (user-defined) strategy that
performs instantiation.  The provided instantiator may accept the
\emph{if-match} and \emph{polarity?}\ arguments, which are as in the
\texttt{inst?}\ command.  The possible values for \emph{if-match} are:
\begin{itemize}
\item \texttt{all}: Find all instances of the first matching template in
the formula being instantiated.

\item \texttt{best}: Find the best instance (i.e., one that generates the
fewest TCCs) for the first matching template.

\item \texttt{t}: Ignore the partial substitution given unless some
matching template was found.

\item \texttt{nil}: Apply the partial substitution even if none of the
templates yielded a match.
\end{itemize}
\end{description}

\prdolsubsection{use*}{Introduce Lemmas and Instantiate/Reduce}
\begin{description}

\item[syntax:] \texttt{(use* \rest\ \carg{names})}

\item[effect:]  An iterated form of the \indtt{use} command which
applies \indtt{use} with default arguments to a sequence of lemmas.
\end{description}

\index{Proof Rules!Introducing Lemmas|)}


\section{Using Extensionality}
\index{Proof Rules!Extensionality|(}

\begin{tabularx}{\textwidth}{|l|l|X|}\hline
\indttdol{apply-eta} & \emph{defined}
  & use eta form extensionality \\\hline
\indttdol{apply-extensionality} & \emph{defined}
  & use extensionality to prove equality \\\hline
\indttdol{decompose-equality} & \emph{defined}
  & reduce equality to component equalities \\\hline
\indttdol{eta} & \emph{defined}
  & introduce eta version of extensionality\\\hline
\indtt{extensionality} & \emph{primitive}
  & introduce extensionality axiom scheme\\\hline
\indttdol{replace-eta} & \emph{defined}
  & replace using eta\\\hline
\indttdol{replace-extensionality} & \emph{defined}
  & replace using extensionality \\\hline
\end{tabularx}


\prdolsubsection{apply-eta}{Apply Eta Form of Extensionality}
\begin{description}

\item[syntax:] \texttt{(apply-eta \carg{term} \optl\ \carg{type})}

\item[effect:] This rule is an extension of the \indtt{extensionality}
rule.  Given a succedent in the form of an equation $l = r$, where the
type of $l$ and $r$ has a corresponding extensionality axiom scheme, \texttt{apply-extensionality} will generate a new succedent that is the result
of using \indtt{replace-extensionality} on $l$ and $r$.

\end{description}

\prdolsubsection{apply-extensionality}{Apply Extensionality}
\begin{description}

\item[syntax:] \texttt{(apply-extensionality \optl\ \cargdflt{fnum}{+}
\carg{keep?}\ \carg{hide?})}

\item[effect:] This rule is an extension of the \indtt{extensionality}
rule.  Given a succedent in the form of an equation $l = r$, where the
type of $l$ and $r$ has a corresponding extensionality axiom scheme,
\texttt{apply-extensionality} will generate a new succedent that is the
result of using \indtt{replace-extensionality} on $l$ and $r$.

If the \emph{keep?}\ flag is set to \texttt{t}, the antecedent equality
introduced by this command is retained in the resulting goal sequent.

If the \emph{hide?}\ flag is set to \texttt{t}, the equality formula
to which the apply-extensionality command has been applied, is hidden
in the resulting sequent.  It is more typical to require this formula to
be hidden than not, which means that the default value of \texttt{nil} for this
flag is poorly chosen.  
\end{description}

\prdolsubsection{decompose-equality}{Reduce Equality to Component Equalities}
\begin{description}
\item[syntax:] \texttt{(decompose-equality \optl\ \cargdflt{fnum}{*} \cargdflt{hide?}{t})}

\item[effect:]  Decomposes an antecedent or consequent equality of the
form \texttt{t1 = t2} where the terms are of function, record, tuple, or
a datatype constructor type.  If the terms are of function type,
the decomposition returns the universal quantification
\texttt{(FORALL x:\ t1(x) = t2(x))}\@.  If the terms are of record type, 
the decomposition returns the conjunction of equalities of the individual
fields of the terms.  The decomposition is similar for tuple types.
The decomposition on datatype constructors returns the equalities on the
corresponding accessor fields.  If the equality is on the consequent side,
or is a disequality on the antecedent side, then the
\indtt{decompose-equality} rule is the same as \texttt{apply-extensionality}\@.    
\end{description}

\prdolsubsection{eta}{Introduce Eta Axiom Scheme}
\begin{description}
\item[syntax:] \texttt{(eta \carg{type})}

\item[effect:] This is a variant of the \indtt{extensionality} rule where
the \emph{eta} version of the axiom scheme is introduced as an antecedent
formula.

\item[usage:]
\begin{description}
\item[\texttt{(eta "[nat, nat -> nat]")}: ] Introduces the antecedent formula
\begin{alltt}
 (FORALL (u\_2: [[nat, nat] -> nat]):
         LAMBDA (x\_3: [nat, nat]): u\_2(x\_3) = u\_2)
\end{alltt}

\item[\texttt{(eta "[\# a:\ nat, b:\ int \#]")}: ]  Introduces the antecedent
formula
\begin{alltt}
(FORALL (r\_8: [# a: nat, b: int #]):
  (# a := a(r\_8), b := b(r\_8) #) = r\_8)
\end{alltt}

\item[\texttt{(eta "(cons?[nat])")}] :
Introduces the antecedent formula
\begin{alltt}
(FORALL (cons?_var: (cons?[nat])):
         cons(car(cons?_var), cdr(cons?_var)) = cons?_var)
\end{alltt}
\end{description}
\item[errors:] {\bf No suitable eta formula for given type:}
Self-explanatory. 
\end{description}


\prsubsection{extensionality}{Introduce Extensionality Axiom}
\begin{description}
\item[syntax:] \texttt{(extensionality \carg{type})}

\item[effect:] The \indtt{extensionality} rule is similar to the \texttt{lemma} rule in that it introduces an extensionality axiom for the given
\emph{type} as an antecedent formula.  An extensionality axiom can be
generated corresponding to function, record, and tuple types, and
constructor subtypes of PVS abstract datatypes.

\begin{usage}{}
\item[ \texttt{(extensionality "[nat, nat -> nat]")}] :
Yields a subgoal got by adding an antecedent formula of the form
\begin{alltt}
  (FORALL (f, g: [nat, nat -> nat]) :
      (FORALL (i, j: nat) : f(i,j) = g(i,j))
         IMPLIES f = g)
\end{alltt}
to the current sequent.

\item[\texttt{(extensionality "[nat, int]")}] :
Adds an antecedent formula of the form
\begin{alltt}
  (FORALL (u : [nat, int]), (v : [nat, int]):
      proj\_1(u) = proj\_1(v) AND proj\_2(u) = proj\_2(v)
   IMPLIES u = v)
\end{alltt}

\item[\texttt{(extensionality "[\# a:\ nat, b:\ int \#]")}] :
Adds an antecedent formula of the form
\begin{alltt}
  (FORALL (r : [\# a: nat, b: int \#]),
          (s : [\# a: nat, b: int \#]) :
      a(r) = a(s) AND b(r) = b(s)
   IMPLIES r = s)
\end{alltt}

\item[\texttt{(extensionality "(cons?[nat])")}\footnotemark] :
Adds an antecedent formula of the form
\begin{alltt}
  (FORALL (x: (cons?[nat])), (y: (cons?[nat])) :
      car(x) = car(y) AND cdr(x) = cdr(y)
   IMPLIES x = y)
\end{alltt}
The extensionality rule applied to other constructor subtypes
for PVS datatypes behaves similarly.
\end{usage}
\footnotetext{\texttt{cons?}\ is defined in the PVS prelude; use the
command \emacstt{view-prelude-theory} on the theory \texttt{list\_adt} for its
definition.}

\item[errors:] In addition to parsing and typechecking errors,
the following error messages are generated:
\begin{description}
\item[{\bf The following irrelevant free variables occur in \ldots\ }:] As
with the other rules, no free variables can be introduced into a proof
sequent through a rule application.

\item[{\bf Could not find a suitable extensionality axiom for \ldots\ }:]
This means that there is no extensionality axiom for the given type
expression.

\item[{\bf Could not find ADT extensionality axiom for \ldots\ }:] The
given type was a subtype of a PVS datatype but not a constructor
subtype as is required for generating an extensionality axiom.
\end{description}

\item[notes:] The defined rule \indtt{apply-extensionality} makes it
possible to directly apply the extensionality scheme to show two terms
to be equal.  The defined rule \indtt{replace-extensionality}
uses extensionality to replace one term by another.  
\end{description}

\prdolsubsection{replace-eta}{Replace using Eta}
\begin{description}

\item[syntax:] \texttt{(replace-eta \carg{term} \optl\ \carg{type} \carg{keep?})}

\item[effect:] This rule extends the \indtt{eta} rule.  The eta axiom
scheme is instantiated with the given term, which is then used in
a replace command.  A specific type for term may be specified where
typechecking the term may give rise to ambiguity.

When \emph{keep?}\ is \texttt{t}, the instantiated eta axiom scheme that is
introduced as an antecedent is retained in the goal sequent, and
otherwise, it is discarded.  
\end{description}

\prdolsubsection{replace-extensionality}{Replace using Extensionality}
\begin{description}

\item[syntax:] \texttt{(replace-extensionality \carg{expr1} \carg{expr2}
\optl\ \carg{expected} \carg{keep?})}

\item[effect:] This rule is an extension of the \indtt{extensionality}
rule.  It uses the relevant extensionality axiom scheme to demonstrate
the equality of \emph{expr1} and \emph{expr2} and replaces \emph{expr1} in the sequent with \emph{expr2}.  In some cases, the optional
\emph{expected} type might have to be supplied to resolve any
ambiguities in the typechecking of the given expressions.

When \emph{keep?}\ is \texttt{t}, the extensionality scheme that is
introduced as an antecedent is retained in the goal sequent, and
otherwise, it is discarded.  
\end{description}

\section{Applying Induction}
\begin{tabularx}{\textwidth}{|l|l|X|}\hline
\indttdol{induct} & \emph{defined}
  & induct over a variable\\\hline
\indttdol{induct-and-rewrite} & \emph{defined}
  & induct and rewrite\\\hline
\indbang{} & \emph{defined}
  & induct and rewrite with definitions\\\hline
\indttdol{induct-and-simplify} & \emph{defined}
  & induct and rewrite/simplify \\\hline
\indttdol{measure-induct} & \emph{defined}
  & support for measure induction\\\hline
\indttdol{measure-induct+} & \emph{defined}
  & measure induction\\\hline
\indttdol{measure-induct-and-simplify} & \emph{defined}
  & measure induction with simplification\\\hline
\indttdol{name-induct-and-rewrite} & \emph{defined}
  & induct on a named scheme and rewrite\\\hline
\indttdol{rule-induct} & \emph{defined}
  & induction on inductive relation\\\hline
\indttdol{rule-induct-step} & \emph{defined}
  & support for induction on inductive relations\\\hline
\indttdol{simple-induct} & \emph{defined}
  & introduce an instance of an induction scheme\\\hline
\indttdol{simple-measure-induct} & \emph{defined}
  & introduce an instance of a measure induction scheme\\\hline
\end{tabularx}
\prdolsubsection{induct}{Invoke Induction}
\begin{description}

\item[syntax:] \texttt{(induct \carg{var} \optl\ \cargdflt{fnum}{1}
\carg{name})}

\item[effect:] The formula indicated by \emph{fnum} must be a
universally quantified, consequent formula.  The variable name \emph{var} must be quantified at the outermost level of this formula.
As with the substitutions in \indtt{inst?}, it is sufficient to give
that part of the variable name preceding the last \texttt{\_} mark, if there
is one in the bound variable name and it is followed by a number.  The
bound variable must be of type, i.e., must include as a supertype,
\texttt{nat} or a PVS abstract datatype for the induction scheme to be
selected automatically.  The induction scheme corresponding to the
type is then instantiated with an induction predicate constructed
from  the formula \emph{fnum} and the
resulting base and induction subgoals are generated.  The induction
scheme can also be explicitly provided by naming it using the optional
\emph{name} argument.  Typically, this name will have to be fully
instantiated with the actual theory parameters.  Note that
user-supplied induction schemes must have a form similar to the
induction schemes in the prelude or those generated by the abstract
datatype mechanism: \texttt{(FORALL (p:\ pred[T]):\ {\it induction subgoal}
IMPLIES {\it goal})}, where \texttt{p} is to be instantiated
by the induction predicate.

\begin{usage}{}
\item[\texttt{(induct "i")}] :
Given that \texttt{i} is of type \texttt{nat}, and the formula numbered \texttt{1}
has the form \texttt{(FORALL \ldots, i, \ldots: \emph{p}(\ldots, i,
\ldots))},
we get the instantiation of the natural number induction scheme with
the induction predicate \texttt{(LAMBDA i:\ (FORALL \ldots: \emph{p}(\ldots,
i, \ldots)))}\@.  The resulting formula is then beta-reduced and
simplified into the base and induction subcases.  If the type of
\texttt{i} is a subtype of \texttt{nat} such as, say, \texttt{(even?)}, then
the subtype predicate is added to the induction predicate to get
\texttt{(LAMBDA (i:\ nat):\ even?(i) IMPLIES (FORALL \ldots: \emph{p}(\ldots,
i, \ldots)))}\@.  If \texttt{i} has type that is a datatype such as binary trees or
lists, then the induction scheme for that datatype is used by default.

\item[\texttt{(induct "x" :fnum 2 :name "below\_induction[N]")}] :
 Employs the induction scheme named \texttt{below\_induction} and
instantiates it with a predicate taken from the sequent
formula number \texttt{2}\@.  
\end{usage}

\item[errors:]
\begin{description}
\item[{\bf Could not find suitable induction scheme}] : The \texttt{simple-induct} rule invoked by this step failed to find an induction
scheme for the given variable \emph{var}, \emph{fnum}, and \emph{name}\@.  Check that \emph{var} occurs as an outermost universally
quantified variable whose type contains no free occurrences of other bound
variables, and that has a supertype that matches what is required by the
(default or named) induction scheme.

\item[{\bf No formula corresponding to \ldots}] : A bad \emph{fnum} was
given. 
\end{description}


\end{description}

\prdolsubsection{induct-and-rewrite}{Induct then Rewrite}
\begin{description}

\item[syntax:] \texttt{(induct-and-rewrite \carg{var} \optl\
\cargdflt{fnum}{1} \rest\ \carg{rewrites})}

\item[effect:] This command has been superseded by the more general
\indtt{induct-and-simplify} but is retained for backward compatibility.
It employs \indtt{induct} on \emph{variable} and \emph{fnum} to select,
instantiate, and introduce an induction scheme, and then uses the rewrite
rules given in \emph{rewrites} to simplify the resulting base and
induction cases employing \indtt{skosimp*}, \indtt{assert},
\indtt{lift-if}, \indtt{inst?}, and \indtt{bddsimp}\@.

\begin{usage}{}
\item[\texttt{(induct-and-rewrite "x" 1 "append" "reverse")}] :
Introduces an instance of the induction scheme according to variable \texttt{x} obtained by instantiating it with the predicate formed from formula
number \texttt{1}, simplifies the resulting base and induction cases using
skolemization, rewriting, decision procedures, if-lifting, and heuristic
instantiation.

\item[\texttt{(induct-and-rewrite "x" :rewrites ("append" "reverse"))}] :
Same as above.  
\end{usage}

\end{description}

\prdolsubsection{induct-and-rewrite!}{Induct then Rewrite with Definitions}
\index{induct-and-rewrite"!@{\texttt{induct-and-rewrite"!}}|ii}
\begin{description}
\item[syntax:] \texttt{(induct-and-rewrite!\ \carg{var} \optl\ \cargdflt{fnum}{1} \rest\ \carg{rewrites})}

\item[effect:]  This is a variant of \indtt{induct-and-rewrite}
which automatically  uses all the definitions in the given sequent
in its rewriting/simplification.  These definitions are used in
their \texttt{!} form so that explicit definitions are always rewritten
regardless of whether the right-hand sides are simplifiable.
Additional rewrite rules can be given can be give using the
\emph{rewrites} argument.  The usage is similar to
\indtt{induct-and-rewrite}\@.  

\end{description}

\prdolsubsection{induct-and-simplify}{Induct and Rewrite/Simplify}
\begin{description}
\item[syntax:] \texttt{(induct-and-simplify \carg{var} \optl\
\cargdflt{fnum}{1} \carg{name} \cargdflt{defs}{t}
\newline\hspace*{1in}
\cargdflt{if-match}{best} \carg{theories} \carg{rewrites} \carg{exclude}
\cargdflt{instantiator}{inst?})}

\item[effect:] This is an extremely useful proof command for directing
proofs involving induction followed by simplification.  It uses
\indtt{install-rewrites} to install the rewrites in \emph{defs},
\emph{theories} and \emph{rewrites} while excluding those in
\emph{exclude}, invokes the the \indtt{induct} command on \emph{var},
\emph{fnum} and \emph{name} to get the base and induction cases, which are
then simplified by repeated application of \indtt{skosimp*},
\indtt{assert}, \indtt{lift-if}, \indtt{bddsimp}, and the specified
\emph{instantiator} (default \indtt{inst?}) which is controlled by the
\emph{if-match} argument.

The \emph{instantiator} argument allows an instantiator to be provided; it
defaults to \texttt{inst?}, but may be any (user-defined) strategy that
performs instantiation.  The provided instantiator may accept the
\emph{if-match} argument, which is as in the \texttt{inst?}\ command.

\begin{usage}{}
\item[]\texttt{(induct-and-simplify "i" :defs !\ :theories "real\_props"
:rewrites "assoc" :exclude ("div\_times" "add\_div"))}:
If \texttt{i} has type \texttt{nat}, then the natural number induction
scheme is instantiated with a predicate constructed from sequent
formula \texttt{1}, and the resulting cases are simplified using
definitions in the given sequent (unconditionally expanding
explicit definitions), the rewrites in the  prelude theory \texttt{real\_props} but excluding 
\texttt{div\_times} and \texttt{add\_div}, and the rewrite rule \texttt{assoc}.

\item[\texttt{(induct-and-simplify "A" :defs nil :if-match nil)}]:
Inducts according the induction scheme given by the type of \texttt{A}
and then simplifies without rewriting any of the definitions or
rewrite rules, and does not perform any heuristic instantaion.
The \texttt{:if-match} nil option is useful when the heuristic instantiation
fails to find the right quantifier substitutions.
\end{usage}

\end{description}

\prdolsubsection{measure-induct}{Support for Measure Induction}
\begin{description}
\item[syntax:] \texttt{(measure-induct \carg{measure} \carg{vars} \optl\
\cargdflt{fnum}{1} \carg{order}
\newline\hspace*{1in}
\carg{skolem-typepreds?})}

\item[effect:] This is actually a helper command; \texttt{measure-induct+}
is the preferred way to invoke measure induction.  The
\texttt{measure-induct} command takes a \emph{measure} expression and a
list of the \emph{induction} variables \emph{vars} in which the measure is
defined.  These variables must occur universally quantified in the formula
numbered \emph{fnum}.  This list of variables is needed in order to
unambiguously identify those universally quantified variables on which the
measure is defined.  As with \indtt{induct}, the \texttt{measure-induct}
command forms the induction predicate by lambda-abstracting the formula
over the variables given in \emph{vars}.  The measure function is also
obtained by lambda-abstracting the given \emph{measure} over the variables
in \emph{vars}.  The
\texttt{measure\_induction}\index{measure\_induction@{\texttt{measure\_induction}}}
scheme from the prelude is then instantiated with domain type of the
measure, the measure, and the ordering on the range of the measure.  The
well-founded ordering is taken by default to be \texttt{<} on natural
numbers or ordinals unless a different ordering is given through the
\emph{order} argument.  The \indtt{lemma} rule is used to introduce the
measure induction scheme instantiated with the selected induction
predicate.  The work so far is actually carried out by the
\indtt{simple-measure-induct} proof step.  The \indtt{measure-induct} step
then beta-reduces, simplifies, and instantiates the measure induction
lemma to discharge the goal sequent and in the process generates an
induction subgoal consisting of an antecedent induction hypothesis and a
consequent induction conclusion.  If \emph{skolem-typepreds?}\ is \emph{t},
then typepreds are introduced for any introduced skolem constants.

The problem with \texttt{measure-induct} is that the arrangement of
quantifiers in the induction hypothesis is unhelpful.  If the formula
numbered \emph{fnum} has the form \texttt{(FORALL x, w:\ p(x, w))} where
the induction variable is \texttt{x} and the measure is \texttt{m}, the
induction predicate is \texttt{(LAMBDA x:\ (FORALL w:\ p(x, w)))}, and the
resulting induction hypothesis has the form \texttt{(FORALL x:\ m(x) <
m(c) IMPLIES (FORALL w:\ p(x, w)))}.  This form nests the universal
quantification on \texttt{w} which might be useful in guiding the
instantiation of \texttt{x}\@.  Therefore a more useful form of the
induction hypothesis is with quantification rearranged as \texttt{(FORALL
x, w:\ m(x) < m(c) IMPLIES p(x, w))}\@.  This rearrangement is carried out
by \indtt{measure-induct+}\@.  See \indtt{measure-induct+} for usage.

\end{description}

\prdolsubsection{measure-induct+}{Measure Induction}
\begin{description}
\item[syntax:] \texttt{(measure-induct+ \carg{measure} \carg{vars} \optl\
\cargdflt{fnum}{1} \carg{order}
\newline\hspace*{1in}
\carg{skolem-typepred?})}

\item[effect:] This is the preferred way to invoke measure induction.  The
\texttt{measure-induct+} command takes a \emph{measure} expression and a
list of the \emph{induction} variables \emph{vars} in which the measure is
defined.  These variables must occur universally quantified in the formula
numbered \emph{fnum}.  The \indtt{measure-induct+} command invokes
\indtt{measure-induct} to introduce the measure induction scheme
instantiated with the selected induction predicate, and then to
beta-reduce, simplify, and instantiate the measure induction lemma to
discharge the goal sequent and in the process generates an induction
subgoal consisting of an antecedent induction hypothesis and a consequent
induction conclusion.  The well-founded ordering is taken by default to be
\texttt{<} on natural numbers or ordinals unless a different ordering is
given through the \emph{order} argument.  If \emph{skolem-typepreds?}\ is
\texttt{t}, then typepreds are introduced for any introduced skolem
constants.

The problem with \indtt{measure-induct} is that the arrangement of
quantifiers in the induction hypothesis is unhelpful.  If the formula
numbered \emph{fnum} has the form \texttt{(FORALL x, w:\ p(x, w))} where
the induction variable is \texttt{x} and the measure is \texttt{m}, the
induction predicate is \texttt{(LAMBDA x:\ (FORALL w:\ p(x, w)))}, and the
resulting induction hypothesis has the form \texttt{(FORALL x:\ m(x) <
m(c) IMPLIES (FORALL w:\ p(x, w)))}.  This form nests the universal
quantification on \texttt{w} which might be useful in guiding the
instantiation of \texttt{x}\@.  A more useful form of the induction
hypothesis is with quantification rearranged as \texttt{(FORALL x, w:\
m(x) < m(c) IMPLIES p(x, w))}\@.  This rearrangement is carried out by
\indtt{measure-induct+}\@.  The command
\indtt{measure-induct-and-simplify} is similar to
\indtt{induct-and-simplify} but uses \texttt{measure-induct+} instead of
\indtt{induct}.

\begin{usage}{}
\item[\texttt{(measure-induct+ "length(x) + length(y)" ("x" "y") 2)}]:
Applies the instance 
of measure induction with measure \texttt{length(x) + length(y)} on the
universally quantified variables \texttt{x} and \texttt{y} in the formula
numbered \texttt{2} to return a goal with an induction conclusion and an
induction hypothesis.

\item[\texttt{(measure-induct+ "m(x)" "x" :order "smaller?")}]:
Applies measure induction on the measure \texttt{m(x)} in the
universally quantified variable \texttt{x} and the well-founded ordering
relation \texttt{smaller?}\ on the range of \texttt{m}\@.  Note that this will
generate a TCC subgoal where the well-foundedness of the ordering relation
has to be established. 
\end{usage}

\end{description}

\prdolsubsection{measure-induct-and-simplify}{Measure Induct and Simplify}
\begin{description}
\item[syntax:] \texttt{(measure-induct-and-simplify \carg{measure}
\carg{vars} \optl\ \cargdflt{fnum}{1} \carg{order}
\newline\hspace*{1in}
\carg{expand}
\cargdflt{defs}{t} \cargdflt{if-match}{best} \carg{theories}
\carg{rewrites} \carg{exclude}
\newline\hspace*{1in}
\cargdflt{instantiator}{inst?}\
\carg{skolem-typepreds?})}

\item[effect:] Invokes the appropriate instance of measure induction using
\indtt{measure-induct+}, skolemizes the resulting induction conclusion,
and expands the definitions listed in the \emph{expand} argument to then
generate the corresponding cases.  The resulting subgoals are then
simplified and the induction hypothesis is instantiated by the
\emph{instantiator} (default \texttt{inst?}) and each subgoal is subject
to further propositional splitting and simplification based on rewriting
and decision procedures.  This command is very similar to
\indtt{induct-and-simplify} but employs measure induction and uses the
\emph{expand} argument to guide the case analysis.  If multiple instances
of the induction hypothesis are needed, the \emph{if-match} argument can
be given as \texttt{first*} to obtain all instances of the first matching
template in the quantified formula, or \texttt{all} to obtain all matches
for all templates.  If \emph{skolem-typepreds?}\ is \emph{t}, then
typepreds are introduced for any introduced skolem constants.

The \emph{instantiator} argument allows an instantiator to be provided; it
defaults to \texttt{inst?}, but may be any (user-defined) strategy that
performs instantiation.  The provided instantiator may accept the
\emph{if-match} argument, which is as in the \texttt{inst?}\ command.

\begin{usage}{}
\item[]\texttt{(measure-induct-and-simplify "length(x) + length(y)" ("x" "y")
\newline :expand "merge" :theories "merge\_sort")}:
This command could for instance be used to try to prove that the merge of
two ordered lists is ordered.  It invokes measure induction on the
sum of the lengths of the two lists, then expands the definition
of \texttt{merge}, and then repeatedly simplifies, instantiates, and rewrites
(using the theory \texttt{merge\_sort}.

\item[]\texttt{(measure-induct-and-simplify "length(x) + length(y)" ("x"
"y") \newline :expand "quicksort" :if-match first* :theories
"quicksort")}: \newline Since there are multiple recursive calls in the
recursive case of \texttt{quicksort}, the \texttt{first*} option to
\emph{if-match} is used.
\end{usage}

\end{description}

\prdolsubsection{name-induct-and-rewrite}{Induct on a Named Scheme and Rewrite}
\begin{description}
\item[syntax:] \texttt{(name-induct-and-rewrite \carg{var} \optl\
\cargdflt{fnum}{1} \carg{name}
\newline\hspace*{1in}
\rest\ \carg{rewrites})}

\item[effect:]  Subsumed by \indtt{induct-and-simplify}.
This command was a variant of a \texttt{induct-and-rewrite} that could
be told to use a particular induction scheme using the \emph{name}
argument.   

\end{description}

\prdolsubsection{rule-induct}{Induct on a (Co)Inductive Relation}
\begin{description}
\item[syntax:] \texttt{(rule-induct \carg{rel} \optl\ \cargdflt{fnum}{*}
\carg{name})}

\item[effect:] Applies the induction scheme given by an inductive relation
\emph{rel} to a sequent of one of the forms $$ \ldots \vdash (\forall x_1,
\ldots, x_n: {\it rel}(x_1, \ldots, x_n) \supset \ldots)$$ or, $$ \ldots,
{\it rel}(c_1, \ldots, c_n), \ldots \vdash \ldots$$, or applies the
coinduction scheme given by a coinductive relation \emph{rel} to a sequent
of one of the forms $$ \ldots \vdash (\forall x_1, \ldots, x_n: \ldots
\supset {\it rel}(x_1, \ldots, x_n))$$ or, $$ \ldots \vdash \ldots, {\it
rel}(c_1, \ldots, c_n), \ldots$$ The generated (co)induction schemes
corresponding to a (co)inductive relation \emph{rel} are
\texttt{\emph{rel}\_weak\_induction} and \texttt{\emph{rel}\_induction}
(or \texttt{\emph{rel}\_weak\_coinduction} and
\texttt{\emph{rel}\_coinduction}).

This command applies repeated skolemization and flattening to the
specified \emph{fnum} (or the first positive, skolemizable consequent or
negative, skolemizable antecedent) before invoking the command
\indtt{rule-induct-step} on the resulting subgoal.  The strategy uses the
weak induction scheme by default but can be told to use strong induction
by supplying \texttt{\emph{rel}\_induction} (or
\texttt{\emph{rel}\_coinduction}) as the \emph{name} argument.

\end{description}

\prdolsubsection{rule-induct-step}{Support for Rule (Co)Induction}
\begin{description}
\item[syntax:] \texttt{(rule-induct-step \carg{rel} \optl\
\cargdflt{fnum}{-} \carg{name})}

\item[effect:] Subsumed by \indtt{rule-induct}.  Applies the (co)induction
scheme given by a (co)inductive relation \emph{rel} to a sequent of the form:
$$ \ldots, {\it rel}(c_1, \ldots, c_n), \ldots \vdash \ldots$$
or
$$ \ldots \vdash \ldots, {\it rel}(c_1, \ldots, c_n), \ldots$$

Searches for a sequent formula of the form ${\it rel}(c_1,\ldots, c_n)$
but this formula can also be given explicitly as \emph{fnum}.  The
(co)induction predicate is formulated using all the sequent formulas
containing the constants $c_1$ to $c_n$\@.  The strategy uses the weak
induction scheme by default but can be told to use strong induction by
supplying \texttt{\emph{rel}\_induction}
(\texttt{\emph{rel}\_coinduction}) as the \emph{name} argument.

\end{description}

\prdolsubsection{simple-induct}{Introduce Induction Scheme}
\begin{description}
\item[syntax:] \texttt{(simple-induct \carg{var} \carg{fmla} \optl\
\carg{name})}

\item[effect:]  Subsumed by \indtt{rule-induct}\@.  
Applies the induction scheme given by an inductive
relation \emph{rel} to a sequent of the form:
$$ \ldots, {\it rel}(c_1, \ldots, c_n), \ldots \vdash \ldots$$

Searches for an antecedent formula of the form ${\it rel}(c_1,\ldots,
c_n)$ but this formula can also be given explicitly as \emph{fnum}.  The
induction predicate is formulated using all the sequent formulas
containing the constants $c_1$ to $c_n$\@.  The strategy uses the weak
induction scheme by default but can be told to use strong induction by
supplying \texttt{\emph{rel}\_induction} as the \emph{induction}
argument.

\end{description}

\prdolsubsection{simple-measure-induct}{Introduce Measure Induction Scheme}
\begin{description}
\item[syntax:] \texttt{(simple-measure-induct \carg{measure} \carg{vars} \optl\
\cargdflt{fnum}{1} \carg{order})}

\item[effect:]  Selects and insert as an antecedent, an instance of measure
induction with measure \emph{measure} containing only free variables from
\emph{vars} using formula \emph{fnum} to formulate an induction predicate.
Uses \emph{order} as the well-founded relation.  If the order is not
specified, it defaults to the \texttt{<} relation on nats or ordinals.

\item[usage:] \texttt{(simple-measure-induct "i+j" ("i" "j"))}:
Inserts measure induction on the measure \texttt{i + j} in the variables \texttt{i} and \texttt{j}\@.  

\end{description}


\section{Simplification with Decision Procedures and Rewriting}

The commands here are very powerful, but the names are not very intuitive,
and the term ``simplification'' is overloaded.  The \indtt{simplify}
command is the basis for most of the automation in the prover.  The
\indtt{assert}, \indtt{do-rewrite}, and \indtt{record} commands are simply
invocations of \indtt{simplify} with particular parameters.  The following
table give an idea of the relationship between the progressively more
powerful commands.

\begin{center}
\begin{tabular}{|l|l|}\hline
\textbf{Command} & \textbf{Added Commands}\\ \hline\hline
\indtt{smash} & \indtt{bddsimp}, \indtt{assert}, \indtt{lift-if}\\ \hline
\indtt{bash} & \indtt{inst?}, \indtt{skolem-typepred}, \indtt{flatten}\\ \hline
\indtt{reduce} & \indtt{replace*}\\ \hline
\indtt{grind} & \indtt{install-rewrites}\\ \hline
\end{tabular}
\end{center}

Each command in the left column includes all the added commands of the
commands above it, though the actual invocation and control of these may
be very different.  For example, though \indtt{grind} invokes
\indtt{reduce} directly, beforehand it invokes \indtt{bddsimp},
\indtt{assert}, and \indtt{replace*}.  And \indtt{bash} is defined
independently of \indtt{smash}.

\begin{tabularx}{\textwidth}{|l|l|X|}\hline
\indttdol{assert} & \emph{defined}
  & use decision procedures to simplify sequent formulas \\\hline
\indttdol{bash} & \emph{defined}
  & \indtt{smash} with quantifier heuristics \\\hline
\indttdol{both-sides} & \emph{defined}
  & apply an operation to both sides of an inequality chain \\\hline
\indtt{decide} & \emph{primitive}
  & run decision procedures without simplification\\\hline
\indttdol{do-rewrite} & \emph{defined}
  & apply installed automatic rewrites\\\hline
\indttdol{grind} & \emph{defined}
  & install rewrites and repeatedly \indtt{reduce} \\\hline
\indttdol{grind-with-ext} & \emph{defined}
  & \indtt{grind} with \indtt{apply-extensionality} \\\hline
\indttdol{grind-with-lemmas} & \emph{defined}
  & \texttt{grind} using lemmas \\\hline
\indttdol{ground} & \emph{defined}
  & propositional and ground simplification \\\hline
\indttdol{lazy-grind} & \emph{defined}
  & \texttt{grind} postponing instantiations \\\hline
\indttdol{record} & \emph{defined}
  & record assumptions for the decision procedures\\\hline
\indttdol{reduce} & \emph{defined}
  & \indtt{bash} repeatedly with replacements\\\hline
\indttdol{reduce-with-ext} & \emph{defined}
  & \indtt{reduce} with extensionality\\\hline
\indtt{simplify} & \emph{primitive}
  & Boolean simplification using decision procedures\\\hline
\indttdol{simplify-with-rewrites} & \emph{defined}
  & install rewrites, simplify, and stop rewrites\\\hline
\indttdol{smash} & \emph{defined}
  & propositional and ground simplification with IF-lifting\\\hline
\end{tabularx}

\prsubsection{assert}{Simplify Using Decision Procedures}

\begin{description}

\item[syntax:] \texttt{(assert \optl\ \cargdflt{fnums}{*}
\carg{rewrite-flag} \carg{flush?}\ \carg{linear?}\ \carg{cases-rewrite?}\
\newline\hspace*{1in}
\cargdflt{type-constraints?}{t} \carg{ignore-prover-output?}\ 
\carg{let-reduce?}\
\newline\hspace*{1in}
\carg{quant-simp?} \carg{implicit-typepreds?})}

\item[effect:] The \indtt{assert} rule is a combination of \indtt{record},
\indtt{simplify}, \indtt{beta}, and \indtt{do-rewrite}.  The use of
decision procedures for equalities and linear inequalities is perhaps the
most significant and pervasive part of PVS@.  These procedures are
invoked to prove trivial theorems, to simplify complex expressions
(particularly definitions), and even to perform matching.  These decision
procedures, originally due to Shostak, employ congruence closure for
equality reasoning, and they also perform linear arithmetic reasoning over
the natural numbers and reals.  They deal solely with \emph{ground}
formulas, namely those that contain no quantifiers.  While they primarily
deal with linear arithmetic, \ie\ expressions of the form \texttt{2*x +
3*y <= 4*z}, there are some modest extensions for dealing with expressions
involving nonlinear subterms using simplifications such as \texttt{(x +
y)*(x - y) = (x*x - y*y)} and simplifications involving division such as
\texttt{x*(y/x) = y}.

The assert rule employs the decision procedures to either simplify
formulas or to assert the formula to the data-structures employed by the
decision procedures.  The assert rule can have one of three effects for
a given formula named in \emph{fnums}:
\begin{enumerate}

\item It can have no visible effect on the formula but could have
asserted the formula to the congruence closure data-structures
employed by the decision procedures.  Antecedent formulas are recorded
as being true, and consequent formulas are recorded as being false.
only those formulas that do not contain branching (\texttt{if} or \texttt{cases}) or propositional structure (unless they are within a
quantifier or a \texttt{lambda} binding) are asserted into the database
since such structures are likely to need further simplification before
they can be processed by the decision procedures.
Unless the \emph{type-constraints?}\ is set to \texttt{nil},  any subtype
constraints on subexpressions of the formulas processed by \texttt{assert} are also automatically recorded by the decision procedures for
the commands \indtt{assert}, \indtt{record}, \indtt{simplify}, and \texttt{do-rewrite}.  The new assertions in the data-structures remain valid
for any descendant proof node of the current sequent and are
automatically employed when assert is invoked at the lower nodes.

\item If the decision procedures succeed in demonstrating a
contradiction from the formula as asserted, then the entire sequent is
regarded as being proved.  If there are no assertable or simplifiable
formulas among those listed in \emph{fnums}, then the rule behaves as a
\texttt{(skip)}.  In every remaining case, a subgoal is generated.

 \item It can simplify the formula by carrying out boolean
simplification, simplify \texttt{if}-expressions by attempting to
reducing the condition part to {\sc true} or to {\sc false}, and
rewrite expressions using the rewrite rules provided by the \texttt{auto-rewrite} and \indtt{auto-rewrite-theory} rules below (see \texttt{do-rewrite}).  In order for any automatic rewrites to take effect, it
must be the case that the conditions of the instance of the rewrite
rule should all simplify to {\sc true}, as should any type correctness
conditions generated by typechecking an instantiating term with
respect to the type of the variable that it instantiates.
As a simple check to prevent such rewrites from looping,
if a rewrite rule rewrites $l$ to $r$ and $r$ is either a \texttt{cases}
or an \texttt{if} 
expression, then the top-most conditional of these expressions
is treated as if it were another condition in a conditional rewrite
rule.  In other words, if $r$ is of the form \texttt{if a then b else c
endif}, then $a$ must simplify to \texttt{true} or \texttt{false}, and
similarly, if $r$ has the form \texttt{cases e of $p_1$ : $c_1$, \ldots
endcases} then $e$ should match one of the patterns $p_i$.  
The simplifications carried out by \indtt{assert} also include various
obvious datatype simplifications and all of the beta-reductions.  the
resulting simplified formula, if suitable, is asserted to the
data-structures.
\end{enumerate}

When the \emph{rewrite-flag} is \texttt{lr}, only the right-hand side of
any equality formula is simplified since simplifying the left-hand side of
a formula to be used by \indtt{replace} in the process of rewriting (see
strategies \indtt{rewrite} and \indtt{rewrite-lemma} below) could cause
the replace to fail.  correspondingly, when the \emph{rewrite-flag} is
\texttt{rl}, only the left-hand side of an equality formula is rewritten.
If the \emph{flush?}\ argument is \texttt{t}, then the existing database
used by the decision procedures is flushed.  This database can get fairly
large in the course of a proof thereby decreasing the efficiency of the
assert rule.  The \emph{flush?}\ flag should be used with caution.  The
decision procedures apply a modest amount of non-linear arithmetic
simplification to multiplication and division expressions.  This can
sometimes get in the way.  the \emph{linear?}\ argument can be set to
\texttt{t} in order to prevent such simplifications.

The \texttt{cases-rewrite?}\ must be set to \texttt{t} in order for
simplification to occur within a branch of a CASES-expression.

The \texttt{type-constraints?}\ flag must be set to \texttt{nil} to
avoid asserting subtype constraints of subexpressions occurring in the
sequent.  If there are several large expressions with subtype constraints,
this phase of simplification can be quite slow.

The ground prover when confronted with a non-convex assertion returns a
disjunction of assertions that are equivalent to the input assertion.  The
simplification commands examine these outputs to see if every path through
them yields a contradiction.  This step can in some cases be expensive.
To avoid the examination of the ground prover outputs the
\texttt{ignore-prover-output?}\ flag can be set to \texttt{nil}.

The \emph{let-reduce?}\ flag indicates whether \texttt{LET} expressions
should also be reduced.

The \emph{quant-simp?}\ flag indicates that quantifier expressions of the
form, for example, \texttt{EXISTS x: x = a AND P(x)} or \texttt{FORALL y:
y = a IMPLIES P(y)} should be simplified to \texttt{P(a)}.  This is not
always desirable because type information may be lost, and the quantified
formula may be useful for doing induction.



\begin{usage}{}

\item[\texttt{(assert)}] : Same as \texttt{(assert *)}.  proves, simplifies, or
asserts all of the formulas in the sequent employing the decision
procedures.

\item[\texttt{(assert -1 lr)}] : Simplifies the right-hand side of the first
antecedent formula, since the \emph{rewrite-flag} is set to \texttt{lr}
(meaning ``left-to-right'').

\item[\texttt{(assert (-1 3 4))}] : Proves, simplifies, or asserts the
formulas numbered \texttt{-1}, \texttt{3}, and \texttt{4} in the sequent.

\item[\texttt{(assert + :flush?\ t)}] : Flushes the existing database of
assertions and asserts the consequent formulas in the current sequent.
\end{usage}

\item[errors:] No error messages are generated.

\item[notes:] One significant point about \indtt{assert} is that it can be sensitive to
the order in which the formulas are asserted.  It is sometimes necessary
to apply \indtt{assert} more than once in order to obtain the desired
effect.  For example, if the formula $a$ is asserted before the formula
$b$, but $b$ is used to simplify $a$, then \indtt{assert} would have to be
reapplied in order to effect this simplification.  Another reason that
\indtt{assert} might need to be repeated is that the subtype constraints
on subexpressions of a formula are silently collected recorded after
\indtt{assert} has processed the formula.  Simplifications
that rely on these constraints can be missed in the first pass of \texttt{assert}.  

Another point about the \indtt{assert} rule is that is that from the
point of view of the control mechanism used for strategies, \texttt{assert} almost always succeeds so that a strategy like \texttt{(repeat
(assert))} is certain to get into an infinite loop.  The typical way
to get around this is to have \indtt{assert} follow some other step \emph{step1} that is guaranteed to not repeat indefinitely, as in:
\begin{alltt}
       (repeat (try \emph{step1} (assert) (skip)))
\end{alltt}
The \indtt{try} strategy ensures that the \indtt{assert} step is invoked
on the subgoals generated by  \emph{step1}.

\end{description}

\prdolsubsection{bash}{\texttt{smash} with Quantifier Heuristics}

\begin{description}
\item[syntax:] \texttt{(bash \optl\ \cargdflt{if-match}{t}
\cargdflt{updates?}{t} \carg{polarity?}\ \cargdflt{instantiator}{inst?}
\newline\hspace*{1in}
\cargdflt{let-reduce?}{t} \carg{quant-simp?})}

\item[effect:] This command executes \indtt{assert}, \indtt{bddsimp}, the
\emph{instantiator} (default \indtt{inst?}), \indtt{skolem-typepred},
\indtt{flatten}, and \indtt{lift-if}, in that order.  This command is the
core of the \indtt{reduce} command which in turn is the workhorse of
\indtt{grind}.

The \emph{instantiator} argument allows an instantiator to be provided; it
defaults to \texttt{inst?}, but may be any (user-defined) strategy that
performs instantiation.  The provided instantiator may accept the
\emph{if-match} and \emph{polarity?}\ arguments, which are as in the
\texttt{inst?}\ command.  Thus the \emph{if-match} option can be
\texttt{nil}, \texttt{t}, \texttt{all}, or \texttt{best} for no, some,
all, or the best instantiation, respectively.  Note that the
\emph{instantiator} precedes \indtt{skolem-typepred}, so that matches that
are in the original sequent are preferred to those that are exposed by
Skolemization.  This has the drawback that quite often, it is better to
instantiate following Skolemization.  To avoid eager instantiation, set
\emph{if-match} to \texttt{nil}.

If the \emph{updates?}\ option is \texttt{nil}, update applications are not
if-lifted.

When the \emph{polarity?}\ flag is \texttt{t}, the \indtt{inst?}\ command
matches templates against complementary subexpressions.

The \emph{let-reduce?}\ flag indicates whether \texttt{LET} expressions
should also be reduced.

The \emph{quant-simp?}\ flag indicates that quantifier expressions of the
form, for example, \texttt{EXISTS x: x = a AND P(x)} or \texttt{FORALL y:
y = a IMPLIES P(y)} should be simplified to \texttt{P(a)}.  This is not
always desirable because type information may be lost, and the quantified
formula may be useful for doing induction.

\end{description}

\prdolsubsection{both-sides}{Apply Operation to Both Sides of Inequality}

\begin{description}
\item[syntax:] \texttt{(both-sides \carg{op} \carg{term} \optl\
\cargdflt{fnum}{1})}

\item[effect:] Here \emph{fnum} is used to find a chained conjunction of
inequalities of the form \texttt{e1 <= e2 AND e2 <= e3 AND e3 <= e4}.  If
the \emph{term} argument is \texttt{t}, the command replaces the above
chain with \texttt{e1 op t <= e2 op t AND e2 op t <= e3 op t AND e3 op t
<= e4 op t}.  If the equivalence between this chain and the previous one
doesn't simplify to \texttt{TRUE} using \indtt{assert} and
\indtt{do-rewrite} with respect to the prelude theory
\texttt{real\_props}, then a proof obligation is generated.
\end{description}

\prsubsection{decide}{Decide without Simplification}

\begin{description}
\item[syntax:] \texttt{(decide \optl\ \cargdflt{fnums}{*})}

\item[effect:] This provides a way to directly invoke the decision
procedures without the simplification inherent in \indtt{simplify}.  It is
not often needed, but can be useful in comparing different decision
procedures, or in trying to determine exactly what a given decision
procedure decides.

\end{description}

\prdolsubsection{do-rewrite}{Apply Installed Automatic Rewrites}

\begin{description}
\item[syntax:] \texttt{(do-rewrite \optl\ \cargdflt{fnums}{*}
\carg{rewrite-flag} \carg{flush?}\ \carg{linear?}\
\newline\hspace*{1in}
\carg{cases-rewrite?}\ \cargdflt{type-constraints?}{t})}

\item[effect:]
This command is another fragment of \indtt{assert}.  It is used to
automatically carry out the rewrites specified by \indtt{auto-rewrite} and
\indtt{auto-rewrite-theory}.  This command uses simplification
as defined by \indtt{assert} 
to discharge the hypotheses of conditional rewrite rules and any
type correctness proof obligations that arise from the use of a rewrite
rule, and also to simplify the rewritten result.  The other arguments are
as in \indtt{assert}\@.  

\begin{usage}{}
  \item[\texttt{(do-rewrite)}] :  Applies the rewrite rules to all the
formulas in the sequent.

  \item[\texttt{(do-rewrite (-1 -3 2))}] : Applies rewrite rules to 
the formulas \texttt{-1}, \texttt{-3}, and \texttt{2}. 
\end{usage}

\item[notes:] No new information is recorded into the data structures
used by the decision procedures, except for the subtype constraints
on subexpressions processed by \indtt{do-rewrite}.  


\end{description}

\prdolsubsection{grind}{Install Rewrites and Repeatedly \texttt{reduce}}
\begin{description}
\item[syntax:] \texttt{(grind \optl\ \cargdflt{defs}{!}\ 
\carg{theories} \carg{rewrites} \carg{exclude} \cargdflt{if-match}{t}
\newline\hspace*{1in}
\cargdflt{updates?}{t} \carg{polarity?}\ 
\cargdflt{instantiator}{inst?}\ \cargdflt{let-reduce?}{t}
\newline\hspace*{1in}
\carg{quant-simp?}\ \carg{no-replace?})}

\item[effect:] This is a catch-all strategy that is frequently used to
automatically complete a proof branch or to apply all the obvious
simplifications till they no longer apply.  The strategy first applies
\indtt{install-rewrites} to install the given theories and rewrite rules
along with all the relevant definitions in the given subgoal.  It then
applies \indtt{bddsimp} followed by \indtt{assert} (similar to
\indtt{ground}) to carry out the first level of simplification.  This is
followed by \indtt{replace*} to carry out all the equality replacements.
This is followed by \indtt{reduce} which repeatedly applies \indtt{bash}
(which invokes \indtt{assert}, \indtt{bddsimp}, the \emph{instantiator}
(default \indtt{inst?}), \indtt{skolem-typepred}, \indtt{flatten}, and
\indtt{lift-if}) followed by \indtt{replace*}\@.

The options to \indtt{grind} can be used to carefully guide its behavior.

The \emph{defs}, \emph{theories}, \emph{rewrites}, and \emph{exclude}
arguments are as in \indtt{install-rewrites}.

The \emph{updates?}\ option is as in \indtt{bash} and \indtt{reduce}.

The \emph{instantiator} argument allows an instantiator to be provided; it
defaults to \texttt{inst?}, but may be any (user-defined) strategy that
performs instantiation.  The provided instantiator may accept the
\emph{if-match} and \emph{polarity?}\ arguments, which are as in the
\texttt{inst?}\ command. Note that by setting \emph{if-match} to
\texttt{nil}, one can avoid the eager instantiation behavior of
\indtt{grind}.  A second \indtt{grind} can then be used to pick up the
instantiations exposed by the first instantiation-free \indtt{grind}.

The \emph{let-reduce?}\ flag indicates whether \texttt{LET} expressions
should also be reduced.

The \emph{quant-simp?}\ flag indicates that quantifier expressions of the
form, for example, \texttt{EXISTS x: x = a AND P(x)} or \texttt{FORALL y:
y = a IMPLIES P(y)} should be simplified to \texttt{P(a)}.  This is not
always desirable because type information may be lost, and the quantified
formula may be useful for doing induction.

The \emph{no-replace?}\ flag indicates whether \indtt{replace*} is
invoked.

\end{description}

\prdolsubsection{grind-with-ext}{\texttt{grind} with Extensionality}
\begin{description}
\item[syntax:] \texttt{(grind-with-ext \optl\ \cargdflt{defs}{!}\ 
\carg{theories} \carg{rewrites} \carg{exclude}
\newline\hspace*{1in}
\cargdflt{if-match}{t} \cargdflt{updates?}{t} \carg{polarity?}\ 
\cargdflt{instantiator}{inst?}\
\newline\hspace*{1in}
\cargdflt{let-reduce?}{t} \carg{quant-simp?}\ \carg{no-replace?})}

\item[effect:] This is like \indtt{grind}, but includes calls to
\indtt{apply-extensionality}.  The arguments are exactly as in
\indtt{grind}.  This is particularly useful when reasoning about functions
(e.g., sets).
\end{description}

\prdolsubsection{grind-with-lemmas}{\texttt{grind} using Lemmas}
\begin{description}
\item[syntax:] \texttt{(grind-with-lemmas \optl\
\cargdflt{lazy-match?}{t} \cargdflt{if-match}{t} \carg{polarity?}\ 
\newline\hspace*{1in}
\cargdflt{defs}{!}\ \carg{rewrites} \carg{theories} \carg{exclude}
\cargdflt{updates?}{t} \rest\ \carg{lemmas})}

\item[effect:] This is like \indtt{grind}, but does a combination of
\texttt{(lemma)} and \texttt{(grind)}; if \emph{lazy-match?}\ is \texttt{t}, postpones instantiations to follow a first round of simplification.
\end{description}

\prdolsubsection{ground}{Propositional and Ground Simplification}
\begin{description}
\item[syntax:] \texttt{(ground \optl\ \cargdflt{let-reduce?}{t}
\carg{quant-simp?})}

\item[effect:] This command invokes propositional simplification followed
by \texttt{assert}.  It is useful in obtaining simplified forms of the
cases arising from propositional simplification.  These simplifications
include those given by \indtt{assert}, namely, the various boolean,
datatype, and arithmetic simplifications, beta-reduction, simplification
using ground decision procedures, and rewriting with respect to the
installed rewrite rules.

The \emph{let-reduce?}\ flag indicates whether \texttt{LET} expressions
should also be reduced.

The \emph{quant-simp?}\ flag indicates that quantifier expressions of the
form, for example, \texttt{EXISTS x: x = a AND P(x)} or \texttt{FORALL y:
y = a IMPLIES P(y)} should be simplified to \texttt{P(a)}.  This is not
always desirable because type information may be lost, and the quantified
formula may be useful for doing induction.

\end{description}

\prdolsubsection{lazy-grind}{\texttt{grind} Postponing Instantiation}
\begin{description}
\item[syntax:] \texttt{(lazy-grind \optl\ \cargdflt{if-match}{t}
\cargdflt{defs}{!}\ \carg{rewrites} \carg{theories} \carg{exclude}
\newline\hspace*{1in}
\cargdflt{updates?}{t} \cargdflt{let-reduce?}{t} \carg{quant-simp?})}

\item[effect:] This is like \indtt{grind}, but postpones heuristic
instantiation until after simplification.  It is essentially \texttt{(then
(grind :if-match nil) (reduce))}.  All arguments are as in \indtt{grind}.
\end{description}


\prdolsubsection{record}{Record Assumptions for the Decision Procedures}
\begin{description}
\item[syntax:] \texttt{(record \optl\ \cargdflt{fnums}{*}
\carg{rewrite-flag} \carg{flush?}\ \carg{linear?}\ \carg{cases-rewrite?}\ 
\newline\hspace*{1in}
\cargdflt{type-constraints?}{t} \carg{ignore-prover-output?})}

\item[effect:] The decision procedures maintain efficient data structures
where the assumptions that are true in the current context are recorded.
The record command is used to add more assumptions to these data
structures.  The only assumptions that can be recorded are those that do
not contain any embedded \texttt{IF}, \texttt{CASES}, or boolean
structure.  The assumptions are antecedent formulas and negations of
consequent formulas.  It is possible for a \indtt{record} command to prove
the sequent if the assumptions are found to be contradictory.  The other
arguments are as in \indtt{assert}.

\begin{usage}{}
  \item[\texttt{(record)}] : records all the assumption formulas in the
sequent into the data structures used by the PVS decision procedures.

  \item[\texttt{(record (-1 -3 2))}] : records the assumptions from
formulas \texttt{-1}, \texttt{-3}, and \texttt{2}. 
\end{usage}

\item[notes:]  a formula is simplified before it is recorded (see \texttt{simplify} and \indtt{assert} below) so that it is possible for \texttt{record} to record an
assumption to contain \texttt{if}, \texttt{cases} or boolean structure that is
eliminated by simplification.   the command \indtt{assert} subsumes \texttt{record} but its behavior is more difficult to explain.

\end{description}

\prdolsubsection{reduce}{\texttt{bash} Repeatedly with Replacements}
\begin{description}
\item[syntax:] \texttt{(reduce \optl\ \cargdflt{if-match}{t}
\cargdflt{updates?}{t} \carg{polarity?}\ \cargdflt{instantiator}{inst?}
\newline\hspace*{1in}
\cargdflt{let-reduce?}{t} \carg{quant-simp?}\ \carg{no-replace?})}

\item[effect:] This command is the main workhorse of the \indtt{grind}
command.  It applies \indtt{bash} followed by \indtt{replace*} in a loop
until neither command has any effect.

The \emph{updates?}\ option is also as in the \indtt{bash} command
and must be set to \texttt{nil} in order avoid the automatic if-lifting
of update applications.

The \emph{instantiator} argument allows an instantiator to be provided; it
defaults to \texttt{inst?}, but may be any (user-defined) strategy that
performs instantiation.  The provided instantiator may accept the
\emph{if-match} and \emph{polarity?}\ arguments, which are as in the
\texttt{inst?}\ command. Note that by setting \emph{if-match} to
\texttt{nil}, one can avoid the eager instantiation behavior of
\indtt{reduce}.  A second \indtt{reduce} can then be used to pick up the
instantiations exposed by the first instantiation-free \indtt{reduce}.

The \emph{let-reduce?}\ flag indicates whether \texttt{LET} expressions
should also be reduced.

The \emph{quant-simp?}\ flag indicates that quantifier expressions of the
form, for example, \texttt{EXISTS x: x = a AND P(x)} or \texttt{FORALL y:
y = a IMPLIES P(y)} should be simplified to \texttt{P(a)}.  This is not
always desirable because type information may be lost, and the quantified
formula may be useful for doing induction.

The \emph{no-replace?}\ flag indicates whether \indtt{replace*} is
invoked.


\end{description}

\prdolsubsection{reduce-with-ext}{\texttt{reduce} with Extensionality}
\begin{description}
\item[syntax:] \texttt{(reduce-with-ext \optl\ \cargdflt{if-match}{t}
\cargdflt{updates?}{t} \carg{polarity?}\
\newline\hspace*{1in}
\cargdflt{instantiator}{inst?}\ \cargdflt{let-reduce?}{t}
\carg{quant-simp?}\ \carg{no-replace?})}

\item[effect:] This command is the main workhorse of the
\indtt{grind-with-ext} command.  It applies \indtt{bash} followed by
\indtt{apply-extensionality} and \indtt{replace*} in a loop until neither
command has any effect.  The arguments are as in \indtt{reduce}.

\end{description}

\prsubsection{simplify}{Simplify using Decision Procedures}

\begin{description}
\item[syntax:] \texttt{(simplify \optl\ \cargdflt{fnums}{*}
\carg{record?}\ \carg{rewrite?}\ \carg{rewrite-flag} \carg{flush?}\
\newline\hspace*{1in}
\carg{linear?}\ \carg{ cases-rewrite?}\
\cargdflt{type-constraints?}{t}
\newline\hspace*{1in}
\carg{ignore-prover-output?}\ 
\cargdflt{let-reduce?}{t} \carg{quant-simp?})}

\item[effect:] \indtt{simplify} is a primitive command used in the
definition of \indtt{assert}, \indtt{do-rewrite}, and \texttt{record}.
\indtt{simplify} includes the arguments to \indtt{assert} along
with two flags: \emph{record?}, and \emph{rewrite?}.  For the
\indtt{assert} command, \emph{record?}\ and \emph{rewrite?}\ must be
\texttt{t}.  To get the \indtt{do-rewrite} command, \emph{record?}\ must
be \texttt{ nil} and \emph{rewrite?}\ must be \texttt{t}.  To get the
\indtt{record} command, \emph{record?}\ must be \texttt{t}, and
\emph{rewrite?}\ must be \texttt{nil}.

The other flags have already been documented with the \indtt{assert}
command.  

Simplification works by maintaining database of currently recorded
information which is then used to simplify and record further information.
The ground decision procedures can be used to decide if a given formula
(that is, a boolean expression) is true or false (or not known to be
either) with respect to the current database and relative to theories such
as those of equality over uninterpreted function symbols and linear
arithmetic.  In a sequent of the form $\seq{a}{m}\vdash\seq{b}{n}$, the
$a_i$ are simplified and recorded as being true, and the $b_i$ are
simplified and recorded as being false.  The simplifications are described
below.  The recording process can yield a refutation in which case the
sequent has been proved.

The theories handled by the ground decision procedures include:
\begin{enumerate}
\item The theory of equality with uninterpreted functions symbols.  This
would enable it to prove a sequent of the form \texttt{x = f(x) $\vdash$
f(f(f(x))) = x}.

\item Quantifier-free linear arithmetic equalities and inequalities, \eg,
\texttt{x < 2*y, y < 3*z $\vdash$ 3*x < 18*z}.  Note that \texttt{x},
\texttt{y}, and \texttt{z} are implicitly universally quantified.


\item Quantifier-free integer linear arithmetic, \eg, \texttt{i > 1, 2*i <
5 $\vdash$ i = 2}.  This procedure is incomplete since the decision
problem for this theory is not known to be polynomial.

\item Arrays and functions with updates.  Examples include:
\begin{enumerate}
\item  \texttt{$\vdash$ f with [(s) := f(s)] = f},
\item  \texttt{$\vdash$ (f with [(s) := x])(s) = x}, and
\item  \texttt{ r/=s $\vdash$ (f with [(s) := x])(s) = f(s)}
\end{enumerate}
\end{enumerate}

The simplifications carried out by \indtt{simplify} are represented
by means of $\longrightarrow$, and  include:
\begin{enumerate}
\item {\bf Beta reduction:}  Examples of such redexes and the
corresponding reductions are:
\begin{itemize}
\item Lambda redex: \texttt{(lambda x : x * x)(2)} $\longrightarrow$ 2 * 2
\item Record redex : \texttt{b((\# a:= 1, b:= 2, c:= 3 \#)) $\longrightarrow$
2}
\item Tuple redex :   \texttt{proj\_2((1, 2, 3))  $\longrightarrow$ 2}
\item Function update redex: For function \texttt{f},
\begin{eqnarray*}
 \texttt{(f\ WITH\ [(i) := 3])(i)} &\Longrightarrow & \texttt{3}\\
   \texttt{(f\ WITH\ [(0) := 3])(1)} &\Longrightarrow & \texttt{f(1)}
\end{eqnarray*}
\item Record update redex: For record \texttt{r},
\begin{eqnarray*}
\texttt{a(r\ WITH\ [(a) := 3])}  & \Longrightarrow& \texttt{3}\\
\texttt{a(r\ WITH\ [(b) := 2])} & \Longrightarrow & \texttt{a(r)}
\end{eqnarray*}
\item Cotuple redex:
\begin{eqnarray*}
\texttt{in\_2(out\_2(x))} & \Longrightarrow & \texttt{x}\\
\texttt{out\_2(in\_2(x))} & \Longrightarrow & \texttt{x}\\
\texttt{in\_2?(in\_2(x))} & \Longrightarrow & \texttt{TRUE}\\
\texttt{in\_1?(in\_2(x))} & \Longrightarrow & \texttt{FALSE}
\end{eqnarray*}
\item Datatype redex: \texttt{car(cons(1, null)) $\Longrightarrow$ 1}
\item Recognizer redex:
\begin{eqnarray*}
\texttt{cons?(null)} & \Longrightarrow & \texttt{FALSE}\\
\texttt{cons?(cons(1, null))} & \Longrightarrow & \texttt{TRUE}
\end{eqnarray*}
\item Subtype redex:  \texttt{even?(i) $\Longrightarrow$ TRUE}, if \texttt{even?}
is one of the subtype predicates in the type of \texttt{i}.
\end{itemize}

\item {\bf Arithmetic simplifications:}  If \texttt{a}, \texttt{b}, \texttt{c} are
arbitrary arithmetic expressions, the following are examples of
simplifications that are
carried out by the \indtt{simplify} command :
\begin{eqnarray*}
 \texttt{ a + 0} &  \Longrightarrow & \texttt{a}\\
 \texttt{ a + 2*a } & \Longrightarrow & \texttt{3*a}\\
 \texttt{1 + a + 3 } & \Longrightarrow & \texttt{a + 4}\\
 \texttt{a*(b + c) } & \Longrightarrow & \texttt{a*b + a*c}\\
 \texttt{0 * a } & \Longrightarrow & \texttt{0}\\
 \texttt{1 * a } & \Longrightarrow & \texttt{a}\\
 \texttt{a + b = b + c } & \Longrightarrow & \texttt{a = c}
\end{eqnarray*}
In addition, sums and products are ordered into a canonical form.

\item {\bf Conditional simplification:}  If \texttt{A} is a formula and   
\texttt{a} and \texttt{b} are expressions, the following are examples of
simplifications applied by \indtt{simplify}:
\begin{eqnarray*}
 \texttt{(IF\ TRUE\ THEN\ a\ ELSE\ b\ ENDIF) } & \Longrightarrow & \texttt{a}\\
 \texttt{ (IF\ FALSE\ THEN\ a\ ELSE\ b\ ENDIF) } & \Longrightarrow & \texttt{b}\\
 \texttt{(IF\ A\ THEN\ b\ ELSE\ b\ ENDIF) } & \Longrightarrow & \texttt{b}\\
 \texttt{(IF\ A\ THEN\ a\ ELSE\ b\ ENDIF) } & \Longrightarrow & \texttt{(IF\ A'\ THEN\ a'\
ELSE\ b')}
\end{eqnarray*}
where: 
\begin{array}[t]{rcl}
 \texttt{A}  \Longrightarrow  \texttt{A}'\\
 \texttt{a}  \Longrightarrow  \texttt{a}' & \mbox{assuming} & \texttt{A}'\\
 \texttt{b} \Longrightarrow \texttt{b}' & \mbox{assuming} & \texttt{NOT A}'
\end{array}

\begin{eqnarray*}
 \texttt{(CASES\ null\ OF\ null:\ a,\ cons(x,y):\ b\ ENDCASES)} & \Longrightarrow& \texttt{a}
\end{eqnarray*}

\item Datatype simplifications:
\begin{eqnarray*}
 \texttt{cons?(a)} \Longrightarrow \texttt{TRUE}, & \texttt{if} &
   \texttt{null?(a)} \Longrightarrow \texttt{FALSE}\\
 \texttt{cons?(a)} \Longrightarrow \texttt{FALSE}, &\texttt{if}&
   \texttt{null?(a)} \Longrightarrow \texttt{TRUE}
\end{eqnarray*}
The datatype simplifications might look circular but the simplifier
uses the decision procedures to check for each recognizer applied to an
expression having a datatype 
as a type, whether the result is true or false or unknown.

\item Boolean simplification:  If the decision procedures determine a
boolean expression to be \texttt{TRUE} or \texttt{FALSE} in the logical context
in which the expression occurs, then it is simplified accordingly.  
If \texttt{A} is a boolean expression, the
following are examples of other simplifications carried out by \indtt{simplify}:
\begin{eqnarray*}
  \texttt{A\ AND\ TRUE\ } & \Longrightarrow & \texttt{\ A}\\ 
  \texttt{A\ AND\ FALSE\ } & \Longrightarrow & \texttt{\ FALSE}\\
  \texttt{A\ OR\ TRUE\ } & \Longrightarrow & \texttt{\ TRUE}\\
  \texttt{A\ OR\ FALSE\ } & \Longrightarrow & \texttt{\ A}\\
  \texttt{TRUE\ IMPLIES\ A\ } & \Longrightarrow & \texttt{\ A}\\
  \texttt{FALSE\ IMPLIES\ A\ } & \Longrightarrow & \texttt{\ TRUE}\\
  \texttt{NOT\ (NOT\ A)\ } & \Longrightarrow & \texttt{\ A}\\
  \texttt{NOT\ TRUE\ } & \Longrightarrow & \texttt{\ FALSE}\\
  \texttt{NOT\ FALSE\ } & \Longrightarrow & \texttt{\ TRUE}\\
  \texttt{a = a\ } & \Longrightarrow & \texttt{\ TRUE}\\
  \texttt{(FORALL\ x:\ TRUE)\ } & \Longrightarrow & \texttt{\ TRUE}\\
  \texttt{(EXISTS\ x:\ FALSE)\ } & \Longrightarrow & \texttt{\ FALSE}
\end{eqnarray*}
\item Quantifier simplifications: some quantified expressions are now
simplified, including the following examples.

\begin{tabular}{rcl}
\texttt{(EXISTS x:\ x = 5)} & $\Longrightarrow$ & \texttt{TRUE}\\
\multicolumn{3}{l}{\texttt{(EXISTS x, y, z:\ x = y + z AND f(x, y, z))}} \\
  & $\Longrightarrow$ & \texttt{(EXISTS y, z:\ f(y + z, y, z))}\\
\texttt{(EXISTS (x:\ T):\ TRUE)} & $\Longrightarrow$ & \texttt{TRUE}\\
\texttt{(FORALL (x:\ T):\ FALSE)} & $\Longrightarrow$ & \texttt{FALSE}\\
\end{tabular}

The last two simplifications only happen when the type \texttt{t} is known
to be nonempty.
\item Rewriting:  The simplifications include conditional rewriting with
respect to the rewrite rules installed by means of \indtt{auto-rewrite},
\indttbang{auto-rewrite}, 
\indtt{auto-rewrite-theory}, \indtt{auto-rewrite-theories}, etc.
The current ordered set of rewrite rules can be viewed using the
PVS Emacs command \emacstt{show-auto-rewrites}.
Rewriting only occurs when the \indtt{rewrite} flag for the \indtt{simplify}
command is \texttt{t}.  If $A$ and $B$ are boolean expressions, and $a$ and
$b$ are expressions, then rewrite rules can be of one of the following
forms:
\begin{enumerate}
\item $a = b$
\item $A\ \texttt{IMPLIES}\ a = b$
\item $A\ \texttt{IMPLIES}\ B$ 
\end{enumerate}
In cases 1 and 2, the left-hand side (lhs) is $a$ and the right-hand side
(rhs) is $b$\@.   In cases 2 and 3, the condition of the rewrite rule is
$A$, whereas for case~1, the condition is \texttt{TRUE}\@.  In case~3, the
lhs is $B$ and the rhs is \texttt{TRUE}\@.

If $a'$ is an instance of the lhs of a rewrite rule so that $b'$ is the
corresponding instance of the rhs and $A'$ is an instance of the
condition, then rewriting simplifies $a'$ to $b''$ provided
$A'$ simplifies to \texttt{TRUE} and $b'$ simplifies to $b''$\@.
If $b'$ is of the form \texttt{(IF $B$ THEN $c$ ELSE $d$ ENDIF)}, then
$B$ must simplify to either \texttt{TRUE} or \texttt{FALSE} for the
rewrite rule to be applicable,   
unless the rewrite rule has been installed with the \texttt{always?}\ flag set
to \texttt{t}\@.   The same constraint also applies if $b'$ is
a \texttt{CASES} expression.  In applying a rewrite rule, the lhs of the
rewrite rule, say $a$, is matched against the expression to be simplified.
If the match succeeds, the typechecking of the instantiation could
generate TCCs.  The simplification process is applied to the TCCs which  
must simplify to \texttt{TRUE} before the rewrite rule is applied.
\end{enumerate}


\begin{usage}{}
  \item[\texttt{(simplify)}] :  Tries to simplify all the formulas in the
sequent using the PVS decision procedures, beta-reduction, boolean
simplification, datatype simplification, arithmetic simplification, and
rewriting.  

  \item[\texttt{(simplify (-1 -3 2))}] : Simplifies the 
formulas \texttt{-1}, \texttt{-3}, and \texttt{2}.

  \item[\texttt{(simplify (-1 -3 2) :flush?\ T)}] : Flushes the decision
procedure database and then simplifies the formulas \texttt{-1}, \texttt{-3},
and \texttt{2}.  This database can sometimes contain information that
interferes with the expected simplification and it helps to flush the
database.

  \item[\texttt{(simplify (-1 -3 2) :linear?\ T)}] :  The PVS decision
procedures have a modest ability to handle nonlinear multiplication and
division.  This invocation of \texttt{simplify} causes simplification
to occur with this capability turned off.

  \item[\texttt{(simplify -1 :rewrite-flag RL)}] :  The rewrite flag
can take on values LR (``left-to-right'') or RL (``right-to-left'')
to indicate that left-hand side (for LR) or right-hand side (for RL)
of the indicated formula should be
left undisturbed by simplification.  This is needed in case the next step
involves a syntactic replacement.
\end{usage}

\item[notes:] The \indtt{simplify} commands and their variants can be
placed within \indtt{repeat} strategies without the danger of creating an
infinitely looping strategy.

The command \indtt{track-rewrite} can be used to obtain diagnostic
information about the progress of rewriting,  and the command
\indtt{untrack-rewrite} can be used to turn off the printing of this
information.  
\end{description}

\prdolsubsection{simplify-with-rewrites}{Install Rewrites, Simplify, and Stop Rewrites}
\begin{description}
\item[syntax:] \texttt{(simplify-with-rewrites \optl\ \cargdflt{fnums}{*}
\carg{defs} \carg{theories} \carg{rewrites}
\newline\hspace*{1in}
\carg{exclude-theories} \carg{exclude})}

\item[effect:] Installs rewrites (according to \emph{defs}) from
\emph{theories} and \emph{rewrites}, excluding those rewrites from
\emph{exclude} or \emph{excluded-theories}, applies \texttt{(assert
\emph{fnums})}, and then turns off all the installed rewrites.  The
arguments other than \emph{fnums} are all as in \indtt{install-rewrites}.

\end{description}

\prdolsubsection{smash}{Propositional/Ground Simplification with IF-lifting}
\begin{description}
\item[syntax:] \texttt{(smash \optl\ \cargdflt{updates?}{t}
\cargdflt{let-reduce?}{t} \carg{quant-simp?})}

\item[effect:] This command is a more powerful version of \indtt{ground}.
It is essentially an iterated application of \indtt{bddsimp} (whereas the
propositional simplification in \indtt{ground} is similar to that of
\indtt{prop}), \indtt{assert}, and \indtt{lift-if}.

The \emph{updates?}\ flag can be set to \texttt{nil} in order to avoid
if-lifting update applications (see page~\pageref{if-lift-updates}).

The \emph{let-reduce?}\ flag indicates whether \texttt{LET} expressions
should also be reduced.

The \emph{quant-simp?}\ flag indicates that quantifier expressions of the
form, for example, \texttt{EXISTS x: x = a AND P(x)} or \texttt{FORALL y:
y = a IMPLIES P(y)} should be simplified to \texttt{P(a)}.  This is not
always desirable because type information may be lost, and the quantified
formula may be useful for doing induction.

\end{description}

\section{Installing and Removing Rewrite Rules}
\begin{tabularx}{\textwidth}{|l|l|X|}\hline
\indtt{auto-rewrite} & \emph{primitive}
  & install a list of formulas as lazy rewrite rules\\\hline
\indttbang{auto-rewrite} & \emph{defined}
  & install a list of formulas as eager rewrite rules\\\hline
\indttdbang{auto-rewrite} & \emph{defined}
  & install a list of formulas as macro rewrite rules\\\hline
\indtt{auto-rewrite-defs} & \emph{defined}
  & install relevant definitions as rewrite rules\\\hline
\indtt{auto-rewrite-explicit} & \emph{defined}
  & install relevant definitions as eager rewrite rules\\\hline
\indtt{auto-rewrite-theories} & \emph{defined}
  & install declarations of theories as rewrite rules\\\hline
\indtt{auto-rewrite-theory} & \emph{defined}
  & install declarations of a theory as rewrite rules\\\hline
\multicolumn{3}{|l|}{\indtt{auto-rewrite-theory-with-importings}}\\
  & \emph{defined}
  & install declarations of a theory and its importings as rewrite rules\\\hline
\indtt{install-rewrites} & \emph{defined}
  & install rewrites from names and theories \\\hline 
\indtt{stop-rewrite} & \emph{primitive}
  & disable automatic rewrite rules \\\hline
\indtt{stop-rewrite-theory} & \emph{defined}
  & disable automatic rewrites from a theory\\\hline
\end{tabularx}

\prsubsection{auto-rewrite}{Install Lazy Rewrite Rules}
\begin{description}

\item[syntax:] \texttt{(auto-rewrite \rest\ \carg{names})}

\item[effect:] The definitions, lemmas, and antecedent formulas named (or
numbered) in \emph{names} are made available for the descendant nodes of
the current proof node as automatic rewrite rules to be used for
simplification by the \texttt{assert} rule.  Only formulas of a certain
special form can be considered as automatic rewrite rules.  An automatic
rewrite rule is either:
\begin{itemize}

\item An \emph{equality rewrite rule} of the form $l = r$ or $l \iff r$
or an atomic Boolean proposition, where $l$ is any expression that
is not an individual variable, and the set of free variables in $r$ is a
subset of the free variables of $l$

\item An unquantified conditional rewrite rule of the form $A\supset B$
where $B$ is either an equality or an unquantified conditional rewrite
rule and the set of free variables in $A$ is a subset of the free
variables in $B$, or

\item A quantified rewrite rule of the form $(\forall x_1,\ldots, x_n :
A)$, where $A$ has the form of a rewrite rule.
\end{itemize}
In addition, a generic rewrite rule that is installed without actual
theory parameters must be such that all the formal parameters occur in the
left-hand side expression  $l$\@. 

All rewrite rules have a left-hand side $l$, a right-hand side $r$, and a
condition $H$ so that any matching instance $l'$ of the left-hand side is
replaced by the corresponding instance $r'$ of the right-hand side
provided the corresponding instance $H'$ of the condition can be
simplified to \texttt{TRUE}\@.  ($H$ is usually the left-hand side of an
implication, e.g. $H \implies l = r$, or the condition of an $IF$ or
$CASES$ expression.)

There are three kinds of automatic rewrite rules: \emph{lazy},
\emph{eager}, and \emph{macros}.  At a given sequent, each installed
rewrite rule can belong to at most one of these classes.

There are two forms of the \emph{names} argument.  In the preferred form,
each name is a \emph{rewrite-name-or-fnum}:

\begin{bnf}
\production{rewrite-name-or-fnum}{fnum \choice rewrite-name}
\production{fnum}
  {\opt{\lit{-}} Number \opt{\lit{!} \opt{\lit{!}}}}
\production{rewrite-name}
  {Name \opt{\lit{!} \opt{\lit{!}}} \opt{\lit{:} \brc{TypeExpr \choice FormulaName}}}
\end{bnf}

Here a lazy rule has no exclamation marks, an eager rule has one, and a
macro has two.  For a rewrite name, a type expr or formula name allows a
specific rewrite to be specified in the presence of overloading.

In the second form of \emph{names} argument some of the names may be
parenthesized, and the depth of parenthesization indicates whether it is
lazy, eager, or a macro.  If the name appears as unparenthesized as
\texttt{"assoc"}, then it is lazy.  If it is singly parenthesized, as in
\texttt{("assoc")}, then it is eager.  If it is doubly parenthesized, as
in \texttt{(("assoc"))}, then it is a macro.  The single and doubly
parenthesized forms can include multiple unparenthesized names.  If a
given name is overloaded in a theory, there is no way in this form to
indicate a specific declaration.

For example, for the command
\begin{alltt}
  (auto-rewrite "A" "B!" "1!" "C" "-3!!" "D!!")
\end{alltt}
\texttt{A} and \texttt{C} are lazy rewrites, \texttt{B} and formula number
\texttt{1} are eager rewrites, and formula number \texttt{-3} and
\texttt{D} are macro rewrites.  In the deprecated form, this may be given
as
\begin{alltt}
  (auto-rewrite "A" ("B" "1") "C" (("-3" "D")))
\end{alltt}
The two forms may not be mixed, and the parenthesized form may be
disallowed in the future.

Lazy rules are the default where if the right hand side is a conditional
or \texttt{CASES} expression, the rewrite rule is not triggered unless the 
top-level condition simplifies to \texttt{TRUE} or \texttt{FALSE} or the
top-level \texttt{CASES} expression is resolved.  All recursive definitions
can only be lazy rewrite rules since there is a possibility that the
rewriting might loop following the recursion.  In an eager rewrite rule,
the rewrite rule is applied regardless of the consequence of
simplifications on the right-hand side instance.  When rewriting with
function definitions, both lazy and eager rewrite rules work with
left-hand sides that, when curried, are in their fully applied form.  That
is, if a function definition allows the left-hand side forms $f$, $f(x,
y)$, $f(x, y)(z)$, then $f(x, y)(z)$ is the only valid left-hand side to a
lazy or eager rewrite rule form of this definition.  Macros on the other
hand always rewrite any occurrence of $f$ so that $f$ is rewritten to
$(\lambda (x, y): (\lambda z: \ldots))$ and $f(a, b)$ is rewritten to
$(\lambda z: \ldots)[a/x, b/y]$\@.


It is preferable that the names of 
lemmas  from imported theories be completely specified, \ie\ with
all the actual theory
parameters explicitly given.  Otherwise, the rewrite rule is known
as \emph{generic} and the actual parameters are instantiated
when the left-hand side is matched.  Not all rewrite rules contain
instances of all the actuals, and such rewrite rules are not installed
in their generic form.

For rewrite rules installed from antecedent formulas, an internal name is
generated that can be used for turning off the rewrite rule using
\indtt{stop-rewrite} or one of its variants.

It is important to note that \indtt{auto-rewrite} has no visible effect
on the sequent, but  
it affects the subsequent behavior of \indtt{assert} in any lower branch
of the proof.  The scope of an \indtt{auto-rewrite} declaration is
restricted to the branch of the proof below it.

\item[usage:] \texttt{(auto-rewrite "append[nat]" "append[int]"
"length[nat]")}

\item[errors:] Due to the presence of actual parameters, \texttt{auto-rewrite} can generate parsing and typechecking errors, in addition
to those listed below.
\begin{description}

\item[{\bf No resolutions for \ldots}:] The system was unable to find
declarations corresponding to the given names.

\item[{\bf Can't rewrite using \ldots: LHS key \ldots is bad.}]
Rewrite rules are arranged by a key which in the case of an application
is the left-most operator name or expression type such as a
record constructor, tuple constructor, update expression,
tuple projection, record field access, cases expression, and lambda,
forall, and exists expressions.   Note that having too many rewrite rules
attached to a single key can slow down rewrite lookup.  

\item[{\bf RHS free variables must be contained in the LHS free
variables}:] Since the matching for rewriting is done using the left-hand
side of the rewrite rule, there should be no free variables left in the
formula that are not instantiated during such a match.

\item[{\bf Hypothesis free variables must be contained in the LHS free
variables}:] See explanation above.

\item[]{\bf Theory \ldots is generic; No actuals given; Free parameters in
the LHS of rewrite must contain all theory formals.}:   Rewrite rules
from generic theories are allowed as long as it is possible to
extract the actual parameters by matching against the left-hand side.

\end{description}

\item[notes:] \indtt{stop-rewrite} turns off automatic rewrite rules,
\indtt{auto-rewrite-theory} turns an entire theory into rewrite rules, and
\indtt{stop-rewrite-theory} turns off theory rewriting.  The Emacs command
\emacstt{show-auto-rewrites} displays the currently active rewrite rules
in a separate buffer and \indtt{track-rewrite} can be used to
explain why rewrite rules did not perform the expected rewriting.  Rewrite
rules that are already installed are not affected by
\emacstt{modify-declaration} and might therefore need to be reinstalled.
\end{description}

\prdolsubsection{auto-rewrite!}{Install Eager Rewrite Rules}
\index{auto-rewrite"!@{\texttt{auto-rewrite"!}}|ii}

\begin{description}
\item[syntax:] \texttt{(auto-rewrite!\ \rest\ \carg{names})}

\item[effect:] This is just a macro for the eager case of
\indtt{auto-rewrite}\@.  The convenience it offers over
\indtt{auto-rewrite} is that the arguments can be given in {\rest} form. 
\end{description}

\prdolsubsection{auto-rewrite!!}{Install Macro Rewrite Rules}
\index{auto-rewrite"!"!@{\texttt{auto-rewrite"!"!}}|ii}

\begin{description}
\item[syntax:] \texttt{(auto-rewrite!!\ \rest\ \carg{names})}

\item[effect:] This is just an abbreviation for the macro case of
\indtt{auto-rewrite} where the arguments can be given in {\rest} form. 
\end{description}

\prdolsubsection{auto-rewrite-defs}{Install Relevant Definitions as Rewrites}

\begin{description}
\item[syntax:] \texttt{(auto-rewrite-defs \optl\ \carg{explicit?}\ \carg{always?}\ \carg{exclude-theories}) }

\item[effect:]  Installs all the definitions used directly or indirectly in the
current sequent as auto-rewrite rules.  If the \emph{explicit?}\ flag is T, the
recursive definitions are not installed and only the explicit definitions are.
If \emph{always?}\ is \texttt{!!}, the explicit definitions are installed as
macros.   
If this flag is \texttt{t}, then the explicit definitions are installed as
eager rewrite rules.  Otherwise, all definitions are installed as lazy
rewrite rules.  (See \indtt{auto-rewrites}.)

The \emph{exclude-theories} takes a list of theories and the definitions
in these theories will not be expanded.  

The \indtt{install-rewrites} command should always be preferred over any of
the specialized rewrite installation commands. 
\end{description}

\prdolsubsection{auto-rewrite-explicit}{Install Relevant Definitions as
Eager Rewrites}

\begin{description}
\item[syntax:] \texttt{(auto-rewrite-explicit \optl\ \carg{always?})}

\item[effect:] Installs all and only the explicit definitions used
directly or indirectly in the current sequent as auto-rewrite rules.  If
\emph{always?}\ is \texttt{!!}, the explicit definitions are installed as
macros.  If this flag is \texttt{t}, then the explicit definitions are
installed as eager rewrite rules.  Otherwise, all definitions are
installed as lazy rewrite rules.  (See \texttt{auto-rewrite}.)

The \indtt{install-rewrites} command should always be preferred over any of
the specialized rewrite installation commands.
\end{description}

\prdolsubsection{auto-rewrite-theories}{Install Rewrites of Theories}

\begin{description}

\item[syntax:] \texttt{(auto-rewrite-theories \rest\ \carg{theories})}

\item[effect:] Applies the \indtt{auto-rewrite-theory} command to each of
the theories in the list \emph{theories}\@.   Each entry in the list of
theories can be a theory name with or without actuals or a list of
arguments in the form accepted by \indtt{auto-rewrite-theory}.

This command is subsumed by \indtt{install-rewrites}.
\end{description}

\prdolsubsection{auto-rewrite-theory}{Install Rewrites of a Theory}

\begin{description}

\item[syntax:] \texttt{(auto-rewrite-theory \carg{name} \optl\
\carg{exclude} \carg{defs} \carg{always?}\ \carg{tccs?})}

\item[effect:] Installs an entire theory or only (explicit) definitions
if \emph{defs} is \texttt{t} (\texttt{explicit}) as auto-rewrites.  In the
case of a parametric theory, unless the \emph{defs} flag is \texttt{t} or
\texttt{explicit}, the actual parameters must be given.  If \emph{always?}
is \texttt{t} the rewrites are installed so that any rewrite other
than a recursive definition always takes effect (see
\texttt{auto-rewrite!}).  If \emph{always?}\ is set to \texttt{!!}, then
the non-recursive definitions are always rewritten even when only a few of
the curried arguments have been provided.)  Declarations named in
\emph{exclude} are not introduced and any current rewrite rules in the
\emph{exclude} list are disabled.  By default, TCCs in the theory are
excluded but they can be included when the \emph{tcc?}\ flag is \texttt{t}.

\begin{usage}{}

\item[]\texttt{(auto-rewrite-theory "sets[nat]" "sets[rational]"
"list\_adt[nat]")} : Declares all those definitions and formula from the
listed modules that can be viewed as rewrite rules to be automatic
rewrite rules.

\item[\texttt{(auto-rewrite-theory "agreement")}] : If we are proving a
lemma names \texttt{main} in the theory \texttt{agreement}, then all the
rewrite rules preceding \texttt{main} are declared as automatic rewrite
rules to be used by \indtt{assert}.
\end{usage}

\item[errors:] Apart from parsing and typechecking errors, the following
error messages are possible:
\begin{description}

\item[{\bf Could not find theory \ldots}:] The named theory does not
appear in the current context.  Add the theory to the \texttt{IMPORTING}
list if needed.

\item[{\bf \ldots is not a fully instantiated theory}:] Provide the
relevant actual parameters.
\end{description}
   
\end{description}

\prdolsubsection{auto-rewrite-theory-with-importings}{Install Rewrites
of a Theory and Its Importings}

\begin{description}

\item[syntax:] \texttt{(auto-rewrite-theory-with-importings \carg{name}
\newline\hspace*{1in}
\optl\ \carg{exclude-theories} \carg{importchain?}\ \carg{exclude}
\carg{defs}
\newline\hspace*{1in}
\carg{always?}\ \carg{tccs?})}

\item[effect:] Installs rewrites in theory \emph{name} along with any
theories imported by \emph{name}.  The full import chain of theories can
be installed by supplying the \emph{importchain?}\ flag as \texttt{t}.
Theories named in \emph{exclude-theories} are ignored.  The other
arguments are similar to those of \texttt{auto-rewrite-theory} and apply
uniformly to each of the theories to be installed.

\item[errors:] Same as for \texttt{auto-rewrite-theories}
\end{description}

\prdolsubsection{install-rewrites}{Install Rewrites from Names
and Theories}

\begin{description}
\item [syntax:] \texttt{(install-rewrites \optl\ \carg{defs}
\carg{theories} \carg{rewrites} \carg{exclude-theories}
\newline\hspace*{1in}
\carg{exclude})} 

\item[effect:]  This is the most powerful way to install rewrite rules
and essentially subsumes all the other ways of installing rewrite rules.

The \emph{defs} argument can be
\begin{itemize}
\item \texttt{nil}:  To avoid installing the
definitions relevant to the current sequent.
\item \texttt{t}, \texttt{!}, or \texttt{!!}: To install all definitions as lazy, eager,
or macro rewrites, respectively.
\item  \texttt{explicit}, \texttt{explicit!}, or \texttt{explicit!!}: Only the
explicit definitions are installed as lazy, eager, or macro rewrites,
respectively.
\end{itemize}

The \emph{theories} argument is a list of theories whose declarations
are meant to be used as rewrite rules.  Each entry in the list is either a
theory name, with or without 
actual parameters, or an argument list in the format expected by
\indtt{auto-rewrite-theory}\@.

The \emph{rewrites} argument is a list of names of rewrite rules to be
installed.

The \emph{exclude-theories} argument is a list of theories that have to
be excluded when installing rewrite rules from theories.

The \emph{exclude} argument is a list of names of rewrite rules that are
removed from the list of installed rewrite rules.  
\end{description}


\prsubsection{stop-rewrite}{Disable Automatic Rewrites}
\begin{description}

\item[syntax:] \texttt{(stop-rewrite \rest\ \carg{names})}

\item[effect:] Turns off those automatic rewrite rules named in \emph{names} that were turned on by either \indtt{auto-rewrite} or \texttt{auto-rewrite-theory}.

\item[usage:] \texttt{(stop-rewrite "append[nat]" "append[int]"
"length[nat]")} : Turns off automatic rewriting of these specific
rewrite rules regardless of whether they were added using \texttt{auto-rewrite} or \indtt{auto-rewrite-theory}.

\item[errors:] {\bf \ldots is not an auto-rewrite}: This is a helpful
message rather than an error.

\end{description}


\prdolsubsection{stop-rewrite-theory}{Disable Automatic Rewrites from a Theory}

\begin{description}

\item[syntax:] \texttt{(stop-rewrite-theory \rest\ \carg{names})}

\item[effect:] Turns off any rewrite rules from the theories named in
\emph{names} that were turned on by either \indtt{auto-rewrite} or \texttt{auto-rewrite-theory}.  Any theory listed in \emph{names} must either be
fully instantiated, or name a defined constant, or be the theory for
the current proof.  Note that for a rewrite rule to be turned off,
the name should be given in the same form to \indtt{stop-rewrite} as in
the corresponding \indtt{auto-rewrite}.   command

\item[usage:] \texttt{(stop-rewrite-theory "sets[nat]" "sets[rational]"
"list\_adt[nat]")} : Turns off any rewrite rules in the listed
theories.  As with \indtt{auto-rewrite-theory}, the theory names should
be fully instantiated.

\item[errors:] Same as \texttt{auto-rewrite-theory}.
   
\end{description}
\index{Proof Rules!Extensionality|)}

\section{Making Type Constraints Explicit}
\index{Proof Rules!Type Constraints|(}

\begin{tabularx}{\textwidth}{|l|l|X|}\hline
\indtt{all-typepreds} & \emph{defined}
  & make type constraints of subexpressions explicit\\\hline
\indtt{typepred} & \emph{primitive}
  & make type constraints of expressions explicit\\\hline
\indttbang{typepred} & \emph{primitive}
  & make all type constraints of expressions explicit\\\hline
\end{tabularx}

\prsubsection{all-typepreds}{Make Type Constraints of Subexpressions Explicit}
\begin{description}

\item[syntax:] \texttt{(all-typepreds \optl\ \cargdflt{fnums}{*})}

\item[effect:] This provides the type constraints for all subexpressions
of the given formulas.  The type constraints are only collected for
subexpressions without free variables, and which contain an expandable
definition or a propositional operator; otherwise the type constraint will
not be useful, since the decision procedures already handle the other
possibilities.
\end{description}

\prsubsection{typepred}{Make Type Constraints of Expressions Explicit}
\begin{description}

\item[syntax:] \texttt{(typepred \rest\ \carg{exprs})}

\item[effect:] If \emph{exprs} is a list of expressions $e_1, \ldots
e_n$, then for each $e_i$, and for each type-constraint predicate $p$ in
the type of $e_i$, an antecedent formula of the form $p(e_i)$ is
introduced.  A predicate $p$ is a type-constraint predicate in a type
$\{ x : \tau | q(x) \}$ if either $p \equiv q$ or $p$ is a
type-constraint predicate in $\tau$.

\begin{usage}{}

\item[\texttt{(typepred "i+j")}] : If \texttt{i} and \texttt{j} are natural
numbers, then the antecedent formula \texttt{i + j >= 0} is added to the
current sequent along with the assertions \texttt{integer\_pred(i + j)},
\texttt{rational\_pred(i + j)}, and \texttt{real\_pred(i + j)}.

\item[\texttt{(typepred "cons(i, null)")}] : Adds the antecedent formula\linebreak
\texttt{cons?[nat](cons(i, null))} to the current sequent.

\end{usage}

\item[errors:] Apart from parsing and typechecking errors,
\indtt{typepred} can generate two errors:
\begin{description}

\item[{\bf Given expression does not typecheck uniquely}:] This means
that there was some type ambiguity in the given expression that can be
resolved by providing theory names, actuals, and/or explicit coercions.

\item[{\bf Irrelevant free variables in \ldots}:] As with many other
rules, no free variables can be introduced into a sequent in a PVS
proof.
\end{description}

\item[notes:] Type predicates are automatically made available to the
decision procedures in most cases.  This command is needed when the
predicates involve definitions or propositional operators.

It is important to provide the right (sub)term to \texttt{typepred}.  For
example, given the curried function
\begin{alltt}
  f: {g: [y: T -> {z: T | R(y, z)}] | P(g)}
\end{alltt}
\texttt{(typepred \"f\")} yields \texttt{P(f)}, while \texttt{(typepred
\"f(a)\")} yields \texttt{R(a, f(a))}.  If you want both, try
\indtt{all-typepreds}.
   
\end{description}

\prsubsection{typepred!}{Make All Type Constraints of Expressions Explicit}
\index{typepred"!@{\texttt{typepred"!}}|ii}
\begin{description}

\item[syntax:] \texttt{(typepred!\ \carg{exprs} \optl\ \carg{all?})}

\item[effect:]  The only difference between \indtt{typepred} and
\indttbang{typepred} is that the \emph{exprs} argument is not given in
{\rest} form and the \texttt{all?}\ flag can be set to \texttt{t} to get
all the type predicates for subtypes of the type \texttt{number}.  The
\indtt{typepred} command only returns the type predicates up to the
natural number constraint.  
\end{description}
\index{Proof Rules!Type Constraints|)}

\section{Abstraction and Model Checking}
\index{Proof Rules!Model Checking|(}

\begin{tabularx}{\textwidth}{|l|l|X|}\hline
\indtt{abstract} & \emph{defined}
  & Boolean abstraction of expressions\\\hline
\indtt{abstract-and-mc} & \emph{defined}
  & Boolean/data abstraction followed by model-checking\\\hline
\indtt{abs-simp} & \emph{primitive}
  & Boolean abstraction \\\hline
\indtt{model-check} & \emph{defined}
  & CTL model checker\\\hline
\indtt{musimp} & \emph{primitive}
  & mu-calculus model checker \\\hline
\end{tabularx}


%
% abstract
%


\prdolsubsection{abstract}{Create Abstraction}
\begin{description}
\item[syntax:]  \texttt{(abstract 
\carg{cstate}
\carg{astate}
\carg{amap}
\optl\ 
\carg{theories}
\carg{rewrites}
\newline\hspace*{1in}
\carg{exclude}
\cargdflt{strategy}{(assert)}
\carg{feasible}
\carg{verbose?})}
\item[effect:]  This command invokes the 
command \indtt{abstract} to construct an abstraction 
of the mu-calculus formulas in the given goal.

The mu-calculus formulas in the goal that contain quantification
over the concrete state type \emph{cstate} are abstracted with
respect to both predicate and data abstraction maps given in \emph{amap}.
The result is a corresponding mu-calculus formula over the abstract
state type \emph{astate}.  Both \emph{cstate} and \emph{astate} are
expected to be type expressions, and \emph{amap} is a list of
pairs of field-name (from the record type \emph{astate})
and a function from the concrete state type \emph{cstate} to
the type (which currently must be a boolean or a scalar type)
corresponding to the 
field-name in the type \emph{astate}\@. 
The optional arguments \emph{theories}, \emph{rewrites}, and
\emph{exclude}
are as in \indtt{install-rewrites}\@.  The \emph{strategy} argument
takes a proof command that is used to discharge the proof obligations that
arise in the construction of the abstraction.  The default strategy
is \texttt{(assert)}.  Various forms of \indtt{grind} are also suitable
though significantly more expensive in terms of time.   
The \texttt{feasible}
argument takes a predicate in the abstract state \emph{astate}
that characterizes the feasible abstract states.  This is needed
when the abstracted formulas contain quantifiers of existential strength. 
Each abstract state corresponds to a set of concrete states, but the
latter set might be empty leading to an existential formula that is
satisfiable at the abstract level but not at the concrete level.
Finally, the \emph{verbose?}\ flag prints out an extensive listing
of the proof obligations generated during abstraction and the
success or failure of the proof effort.
\end{description}

\prdolsubsection{abstract-and-mc}{Abstract and Model Check}
\begin{description}
\item[syntax:]  \texttt{(abstract-and-mc
\carg{cstate}
\carg{astate}
\carg{amap}
\optl\ 
\carg{theories}
\carg{rewrites}
\newline\hspace*{1in}
\carg{exclude}
\cargdflt{strategy}{(assert)}
\carg{feasible}
\carg{verbose?})}
\item[effect:]  This command constructs an abstraction 
of the mu-calculus formulas in the given goal using
\indtt{install-rewrites} to install various rewrites,
\indtt{assert} to apply these rewrites and other simplifications,
and \indtt{abs-simp} to actually construct the abstraction. 

The mu-calculus formulas in the goal that contain quantification
over the concrete state type \emph{cstate} are abstracted with
respect to both predicate and data abstraction maps given in \emph{amap}.
The result is a corresponding mu-calculus formula over the abstract
state type \emph{astate}.  Both \emph{cstate} and \emph{astate} are
expected to be type expressions, and \emph{amap} is a list of
pairs of field-name (from the record type \emph{astate})
and a function from the concrete state type \emph{cstate} to
the type (which currently must be a boolean or a scalar type)
corresponding to the 
field-name in the type \emph{astate}\@. 
The optional arguments \emph{theories}, \emph{rewrites}, and
\emph{exclude}
are as in \indtt{install-rewrites}\@.  The \emph{strategy} argument
takes a proof command that is used to discharge the proof obligations that
arise in the construction of the abstraction.  The default strategy
is \texttt{(assert)}.  Various forms of \indtt{grind} are also suitable
though significantly more expensive in terms of time.   
The \texttt{feasible}
argument takes a predicate in the abstract state \emph{astate}
that characterizes the feasible abstract states.  This is needed
when the abstracted formulas contain quantifiers of existential strength. 
Each abstract state corresponds to a set of concrete states, but the
latter set might be empty leading to an existential formula that is
satisfiable at the abstract level but not at the concrete level.
Finally, the \emph{verbose?}\ flag prints out an extensive listing
of the proof obligations generated during abstraction and the
success or failure of the proof effort.
\end{description}


\prdolsubsection{abs-simp}{Create Boolean abstraction}
\begin{description}
\item[syntax:] \texttt{(abs-simp 
\carg{cstate}
\carg{astate}
\carg{amap}
\optl\ 
\cargdflt{strategy}{(assert)}
\carg{feasible}
\newline\hspace*{1in}
\carg{verbose?})}
\item[effect:]  This is the primitive proof command
used to construct an abstraction 
of the mu-calculus formulas in the given goal.

The mu-calculus formulas in the goal that contain quantification
over the concrete state type \emph{cstate} are abstracted with
respect to both predicate and data abstraction maps given in \emph{amap}.
The result is a corresponding mu-calculus formula over the abstract
state type \emph{astate}.  Both \emph{cstate} and \emph{astate} are
expected to be type expressions, and \emph{amap} is a list of
pairs of field-name (from the record type \emph{astate})
and a function from the concrete state type \emph{cstate} to
the type (which currently must be a boolean or a scalar type)
corresponding to the 
field-name in the type \emph{astate}\@. 
  The \emph{strategy} argument
takes a proof command that is used to discharge the proof obligations that
arise in the construction of the abstraction.  The default strategy
is \texttt{(assert)}.  Various forms of \indtt{grind} are also suitable
though significantly more expensive in terms of time.   
The \texttt{feasible}
argument takes a predicate in the abstract state \emph{astate}
that characterizes the feasible abstract states.  This is needed
when the abstracted formulas contain quantifiers of existential strength. 
Each abstract state corresponds to a set of concrete states, but the
latter set might be empty leading to an existential formula that is
satisfiable at the abstract level but not at the concrete level.
Finally, the \emph{verbose?}\ flag prints out an extensive listing
of the proof obligations generated during abstraction and the
success or failure of the proof effort.
\end{description}

\prdolsubsection{model-check}{CTL Model Checker}

\begin{description}
\item[syntax:] \texttt{(model-check \optl\ \cargdflt{dynamic-ordering?}{t}
\cargdflt{cases-rewrite?}{t} \carg{defs}
\newline\hspace*{1in}
\carg{theories} \carg{rewrites}
\carg{exclude} \carg{irredundant?})}

\item[effect:]  This command is still quite experimental.
It has the effect of rewriting with respect to the theories that
define the CTL operators in terms of the mu-calculus, and then applying
\indtt{musimp}, the mu-calculus model checker to the result.

The \emph{dynamic-ordering?}\ flag can be set to \texttt{nil} to turn off the
dynamic reordering of variables in order to reduce BDD size.

The \emph{cases-rewrite?}\ flag can be set to \texttt{nil} to avoid rewriting
and simplification within unresolved selections within a \texttt{CASES}
expression for the sake of efficiency.

The commands \emph{defs}, \emph{theories}, \emph{rewrites}, and \emph{exclude}, are used exactly as in \indtt{install-rewrites} to set
up rewrite rules for use in simplification prior to model checking.

The model checker invoked by \indtt{ musimp} uses the same BDD package
as the \indtt{bddsimp} command.  The results of Boolean simplification
and model checking are returned in sum-of-products form as a
disjunction of conjunction of literals.  Some of the disjuncts
might be redundant but generating a minimal set of disjunctions is
expensive.  The flag \emph{irredundant?}\ when set to \texttt{t}
allows a less expensive redundant sum-of-products to be returned as the
result.

The model checker either verifies the goal, returns a collection of
subgoals that serve as counterexamples to the given mu-calculus assertion,
or gives an indication that the result cannot be translated.  The
counterexamples correspond to the set of states for which the mu-calculus
assertion fails.

\begin{usage}{}
\item[\texttt{(model-check :theories ("transitions" "props")
:irredundant?\ T)}]: Translates the given sequent containing CTL or
mu-calculus assertions into a Boolean mu-calculus, invokes a BDD-based
symbolic model checker on this, and either proves the result or returns a
collection of subgoals.

\item[\texttt{(model-check :dynamic-ordering? nil)}]: Invokes CTL/mu-calculus
model checking procedure on the subgoal but with dynamic reordering of
BDDs disabled.  The dynamic reordering tries to reduce the size of the
BDDs but can be expensive in terms of time. 
\end{usage}
\end{description}

\prsubsection{musimp}{Mu-Calculus Model Checker}

\begin{description}
\item[syntax:] \texttt{(musimp \optl\ \cargdflt{fnums}{*}
\carg{dynamic-ordering?}\ \carg{irredundant?}\ \carg{verbose?})}

\item[effect:] This command is primarily used in the \indtt{model-check}
strategy to invoke the BDD-based, symbolic mu-calculus model checker.  A
glaring weakness of this model checker is that it does not generate a
counterexample trace.  The outcome of the command is usually a collection
of subgoals corresponding to the states that violate the mu-calculus
formula given by the sequent formulas selected using \emph{fnums}.

When set to \texttt{t} the \emph{irredundant?}\ flag computes
the disjunctive normal form of the result (which can take quite a bit of
time).  The \emph{verbose?}\ flag controls the amount of information
printed, and is mostly useful for debugging.

\end{description}

\index{Proof Rules!Model Checking|)}

\section{Converting Strategies to Rules}
\index{Proof Strategies!Converting to Rules}
\index{Proof Rules}

\begin{tabularx}{\textwidth}{|l|l|X|}\hline
\indtt{apply} & \emph{primitive}
  & apply a proof strategy in a single atomic step\\\hline
\end{tabularx}


\prsubsection{apply}{Make Proof Strategies Atomic}
\begin{description}
\item[syntax:] \texttt{(apply \carg{strategy}\ \optl\ \carg{comment}
\carg{save?}\ \carg{time?}\ \carg{timeout})}

\item[effect:] The \indtt{apply} rule takes an application of a proof
\emph{strategy} and applies it as a single atomic step that generates
those subgoals left unproved by the proof strategy.  The \indtt{apply}
rule is frequently used when one wishes to employ a proof strategy but is
not interested in the details of the intermediate steps.  A number of
defined rules employ \indtt{apply} to suppress trivial details.

The optional \emph{comment} field can be used to provide a format string
to be used as commentary while printing out the proof.  If the
\texttt{save?}\ flag is set to \texttt{t}, the \texttt{apply} step is
saved even if the applied strategy results in no change to the proof.
This is useful if, for example, the command within the apply uses the
\texttt{lisp} command to change a Lisp variable for use elsewhere in the
proof.  The \texttt{time?}\ flag when \texttt{t} is used to return timing
information regarding the applied step.

The \emph{timeout} may be set to an integer to indicate that the apply
step should give up after the specified number of seconds.  If it succeeds
before then, it is treated exactly as an \texttt{apply}, and removes the
\emph{timeout} argument from the saved proof, so that it will succeed in
the same way even if the proof is subsequently rerun on a slower machine.
If it fails, the proof state is restored, and the \emph{timeout} argument
is retained.

\begin{usage}{}

\item[\texttt{(apply (then (skolem 2 ("a4" "b5")) (beta) (flatten)}]
\texttt{"Skolemizing and beta-reducing")} : 
The \indtt{then} strategy performs each of the steps given by its
arguments in sequence.  Wrapping this strategy in an \indtt{apply} ensures
that the intermediate steps in the sequence are hidden.
 The given commentary string is printed out as part of
the proof.

\item[\texttt{(apply (try (skolem!) (flatten) (ground)))}] : This applies
a strategy that applies \texttt{(skolem!)} to the current goal, and if that
``succeeds,'' applies \texttt{(flatten)} to the resulting subgoals, and
otherwise it applies \texttt{(ground)} to the current goal.  The above
rule carries out this strategy in an atomic step and returns the
resulting subgoals.

\item[\texttt{(apply (grind) :save? T :time? T)}] Applies the
\texttt{grind} strategy but saves the step even when \texttt{grind} has
no effect, and returns timing information. 
\end{usage}

\item[errors:] No error messages are generated by apply.
\end{description}

\section{Using Default Strategies}

\begin{tabularx}{\textwidth}{|l|l|X|}\hline
\indtt{default-strategy} & \emph{defined}
  & invoke default strategies\\\hline
\end{tabularx}


\prdolsubsection{default-strategy}{Invoke Default Strategies}
\begin{description}
\item[syntax:] \texttt{(default-strategy)}

\item[effect:] The \texttt{default-strategy} is a strategy that makes it
easy to invoke user-defined strategies as defaults.  It looks for a
strategy of the name \texttt{\emph{th}-strategy}, where \texttt{\emph{th}}
is the current theory (i.e., the theory containing the formula being
proved).  If that is not defined, it looks for a strategy named
\texttt{context-strategy}, and if that is not found, it invokes
\texttt{(grind)}.  This strategy is the default for many of the commands
that apply proofs across many formulas; see the System Guide for details.

\end{description}

\pagebreak

\chapter{Proof Strategies}\label{strategy}
\index{Proof Strategies|(}
\index{Strategies|see{Proof Strategies}}
\index{Rules|see{Proof Rules}}


We have so far described the primitive proof rules employed by PVS to
construct proofs.  Since it would be moderately tedious to construct
proofs using only the primitive proof rules, there is a simple language
for defining more powerful proof rules and proof strategies for
combining proof rules.  A \emph{proof strategy} is intended to capture
patterns of inference steps.  A \emph{defined proof rule} is a strategy
that is applied in a single atomic step so that only the final effect of
the strategy is visible and the intermediate steps are hidden from the
user.  There are four basic forms for constructing strategies or proof
rules: recursion, let, backtracking, and the conditional form.  Lisp
expressions can be employed in constructing strategies as shown below.
There is a special purpose interpreter for strategy expressions.  An
advanced user would  need to study the interpreter.  The crucial
aspect of PVS commands is that the arguments are not evaluated.  We
use a substitution model of evaluation.  Lisp code can only appear in
the conditional of an \indtt{if} strategy and in the bindings of a \texttt{let} strategy.

\section{Global Variables used in Strategies}

The following global variables are kept current with each proof state
and can be used within strategies.  

\noindent
\begin{tabularx}{\textwidth}{|l|X|}
\hline
\indtt{*ps*} & Current proof state \\\hline
\indtt{*goal*} & Goal sequent of current proof state \\\hline
\indtt{*label*} & Label of current proof state  \\\hline
\indtt{*par-ps*} & Current parent proof state \\\hline
\indtt{*par-label*} & Label of current parent \\\hline
\indtt{*par-goal*} & Goal sequent of current parent \\\hline
\indtt{*+*} & Consequent sequent formulas \\\hline
\indtt{*-*} & Antecedent sequent formulas \\\hline
\indtt{*new-fmla-nums*} &  Numbers of new formulas in current sequent
\\\hline
\indtt{*current-context*} & Current typecheck context \\\hline
\indtt{*module-context*} & Context of current module \\\hline
\indtt{*current-theory*} & Current theory \\\hline 
\end{tabularx}

\section{Data Structures}

We now document the various operations on PVS data structures for terms,
formulas, 
and proof goals that are needed for writing nontrivial PVS proof
strategies.  PVS data structures are defined as classes in the Common Lisp
Object Sysem (CLOS).  Each class is defined by indicating its slots.
Classes can be defined as subclasses of one or more \emph{superclasses}
by introducing the additional slots.  For example, the proof state
that is the root node of a proof is defined as a subclass of an ordinary
proof state that contains an extra slot for referring to the formula
declaration corresponding to the 
proof.  Data objects corresponding to a class are called \emph{instances}\@.  If a Lisp term $t$ has instance $v$ as its value, then
\texttt{(show \(t\))} displays the slot values of $v$\@.  With PVS
data structures, if value $v$ is an instance of class \texttt{c}, then
\texttt{c?}\ is the recognizer corresponding to the class so that
\texttt{(c?~\(v\))} is \texttt{t}.  Furthermore, if \texttt{c} is a subclass of
class \texttt{b}, then \texttt{(b?~\(v\))} is also \texttt{t}\@.  If \texttt{s} is a
slot name in class \texttt{c}, then \texttt{(s \(v\))} returns the corresponding
slot value in $v$\@.    A slot value is destructively updated by
\texttt{(setf (s \(v\)) \(u\))}, which sets the slot value of slot \texttt{s}
in $v$ to $u$\@.  An instance can be nondestructively copied and updated
by \texttt{(copy \(v\) 's1 \(u_1\) 's2 \(u_2\))}, which returns a copy of
$v$ with slot \texttt{s1} set to $u_1$ and \texttt{s2} set to $u_2$\@.
There is a lazy form of copy where \texttt{(lcopy \(v\) 's1 \(u_1\) 's2
\(u_2\))}  creates a new copy only when the updates actually change
the slot values.

The class \texttt{proofstate} of proof states consists of a number of slots.
These slots include:

\noindent
\begin{tabularx}{\textwidth}{|l|X|}\hline
\indtt{label} & Displayed label of proof state\\\hline
\indtt{current-goal} & Sequent part of proof state \\\hline
\indtt{current-rule} & Rule being applied to current proof state \\\hline
\indtt{alist} & Decision procedure data structures \\\hline
\indtt{done-subgoals} & List of proof states of completed subgoals
\\\hline
\indtt{pending-subgoals} & List of processed but incomplete subgoals
\\\hline
\indtt{remaining-subgoals} & List of unprocesses subgoals\\\hline
\indtt{current-subgoal} & Proof state currently being processed \\\hline
\indtt{subgoalnum} & Number of current proof state as subgoal of parent
\\\hline
\indtt{context} & Current typecheck context  \\\hline
\indtt{parent-proofstate} & Parent proof state \\\hline
\indtt{justification} & Proof of subtree \\\hline
\indtt{current-auto-rewrites} & Current rewrite rules \\\hline
\end{tabularx}

Only a few of these are really relevant for writing strategies,
and these are typically the ones that are already captured in
global variables.

The global variable \texttt{*ps*} is always bound to the currently active
proof goal.  Each proof goal is an instance of class \texttt{proofstate}\@.
The sequent corresponding to the proof goal is kept in the global variable
\indtt{*goal*} and appears in the \indtt{current-goal} slot of the
proofstate.   The current sequent is an instance of the
class \texttt{sequent} which has the slots:

\noindent
\begin{tabularx}{\textwidth}{|l|X|}\hline
\indtt{s-forms} & List of active sequent  formulas \\\hline
\indtt{hidden-s-forms} & List of hidden sequent formulas \\\hline
\end{tabularx}

A sequent formula is of class \texttt{s-formula} and the
main slot here is \texttt{formula} so that if \texttt{sf} is a sequent formula,
\texttt{(formula sf)} is the expression corresponding to the formula.
This expression is a negation in the case of an antecedent formula.

Typical formulas are either negations, disjunctions, conjunctions,
implications, equalities, equivalences, conditional expressions,
arithmetic inequalities, or  universally or existentially
quantified expressions.  Quantified expressions are in the class 
\texttt{binding-expr} with slots \texttt{bindings} which returns the
bound variables, and \texttt{expression}, which returns the body of
the binding expression.  The other forms are all instances of
the \texttt{application} class consisting of a slot for the
\texttt{operator} and one for the \texttt{argument}.  The first or
only argument of an application \texttt{expr} can be obtained by
\texttt{(args1 expr)}\@.  The second argument, if any, can be obtained
by \texttt{(args2 expr)}\@.   The predicates for recognizing
the different connectives are summarized in the following table.
\begin{center}
\begin{tabular}{|l|l|}
\hline
{\bf Connective} & {\bf Recognizer Form}\\\hline
Negation & \texttt{(negation?\ expr)} \\\hline
Disjunction & \texttt{(disjunction?\ expr)} \\\hline
Conjunction & \texttt{(conjunction?\ expr)} \\\hline
Implication & \texttt{(implication?\ expr)} \\\hline
Equality & \texttt{(equation?\ expr)} \\\hline
Equivalence/Equality & \texttt{(iff?\ expr)} \\\hline
Conditional & \texttt{(branch?\ expr)} \\\hline
Universal Formula & \texttt{(forall-expr?\ expr)} \\\hline
Existential Formula & \texttt{(exists-expr?\ expr)} \\\hline
\end{tabular}
\end{center}


\section{Selecting Sequent Formulas}\label{selection}

 Several Lisp functions  select sequent formulas
given their labels or numbers, or collect the numbers 
of selected sequent formulas.  Given a sequent \texttt{seq}, typically obtained
by \texttt{(s-forms *goal*)} and a list of labels or formula
numbers \texttt{fnums},  the Lisp expression \texttt{(select-seq seq fnums)}
returns the list of sequent formulas in \texttt{seq} corresponding to the
given \texttt{fnums}\@.  The Lisp expression \texttt{(delete-seq seq fnums)}
returns the list of sequent formulas in \texttt{seq} that are not
selected by the given \texttt{fnum}\@.  If we are interested in
selecting the sequent formulas according to some predicate, then
the Lisp expression \texttt{(gather-seq seq yes-fnums no-fnums pred)}
returns the list of sequent formulas in \texttt{seq} that are selected
by \texttt{yes-fnums} but not by \texttt{no-fnums} such that the formula part
of the sequent formula satisfies the unary predicate given by \texttt{pred}\@.  Given a sequent formula \texttt{sf} in \texttt{(s-forms *goal*)} or in the list returned by \texttt{gather-seq}, the Lisp expression
\texttt{(formula sf)} returns the actual PVS term corresponding to the
sequent formula.  Note that the formula numbers input to \texttt{gather-seq}
can also \texttt{be '*} (for all the formulas), \texttt{'+} (for the
consequent formulas), and \texttt{'-} (for the antecedent formulas),
and also formula labels. 

Since many commands take  formula numbers or lists of formula numbers as
arguments, it is useful to select these numbers rather than the formulas
themselves.  The Lisp expression \texttt{(gather-fnums seq yes-fnums no-fnums
pred)} returns the list of all the formula numbers of sequent formulas in
\texttt{seq} corresponding to \texttt{fnums} that satisfy the predicate \texttt{pred}\@.  Note that any reference to the actual PVS term representing
the sequent formula \texttt{sf} in the predicate \texttt{pred} will have to be
of the form \texttt{(formula sf)}\@.  


Thus, the Lisp expression
\begin{alltt}
  (gather-seq (s-forms *goal*)
              '-
              nil
              \#'(lambda (sf) (and (negation? (formula sf))
                               (forall-expr? (args1 (formula sf))))))
\end{alltt}
collects the list of universally quantified
antecedent formulas, and the Lisp expression
\begin{alltt}
  (gather-fnums (s-forms *goal*)
                '-
                nil
              \#'(lambda (sf) (and (negation? (formula sf))
                               (forall-expr? (args1 (formula sf))))))
\end{alltt}
returns the corresponding list of formula numbers.





\section{Strategy Expressions}

We describe the more easily understood aspects of strategies below.  Any
strategy expression can be typed in at the \texttt{Rule?}\ prompt in a
proof.  The syntax for strategy expressions is as follows:

\begin{eqnarray*}
\pair{step} &:= & \pair{\textit{primitive-rule}}\\
           & | & \pair{\textit{defined-rule}}\\
           & | & \pair{\textit{defined-strategy}}\\
           & | & (\indtt{quote}\ \pair{\textit{step}})\\
           & | & (\indtt{try}\ \pair{\textit{step}}\ \pair{\textit{step}}\ \pair{\textit{step}})\\
           & | & \texttt{(if}\ \pair{\textit{lisp-expression}}\ \pair{\textit{step}}\ \pair{\textit{step}}\texttt{)}\\
           & | & \texttt{(let\ (}\{\texttt{(}\pair{\textit{symbol}}\ \pair{\mbox{\emph{lisp-expression}}}\texttt{)}\}^+\texttt{)}\ \pair{\textit{step}}\texttt{)}
\end{eqnarray*}

\section{Defining Strategies}

User-defined strategies should be saved in a file called
\indtt{pvs-strategies}.  PVS loads strategies from files of this name from
both the user's home directory and the current context
directory.\footnote{Note: changing contexts does not remove the strategies
or supporting Lisp forms, so if there is a local \texttt{pvs-strategies}
file it is probably best to quit PVS and restart it in the new context.}
A strategy definition has the form:

\begin{alltt}
     (defstep \emph{name}
              (\emph{required-parameters}
                \&optional \emph{optional-parameters}
                \&rest \emph{parameter})
         \emph{strategy-expression}
         \emph{documentation-string}
         \emph{format-string})
\end{alltt}

This generates both a (blackbox) defined rule \emph{name} and a (glassbox)
strategy \emph{ name\texttt{\char36}}.  There are two other definition
forms that are essentially similar to \indtt{defstep}.  These are
\indtt{defstrat} and \indtt{defhelper}.  The \indtt{defhelper} version is
identical to \indtt{defstep} but is used for defining strategies that are
only meant to be used in the definition of other strategies and are not
likely to be invoked directly by the user, hence they are not shown with
the \texttt{M-x help-pvs-prover} commands, unless invoked with an argument
(e.g., C-u).  The currently defined helper commands are:
\begin{description}
\item[\indtt{tcc}] The different classes of TCCs have their own
strategies, which are assigned by default.  Most of them default to
\indtt{tcc}, which takes an optional \emph{defs} argument (default
\texttt{!}) and simply invokes \texttt{(grind :defs defs)}.  Any of these,
including \texttt{tcc}, may be redefined, usually in a
\indtt{pvs-strategies} file.
\begin{description}
\item[\indtt{subtype-tcc}:] calls \texttt{(tcc explicit)}
\item[\indtt{termination-tcc}:] calls \texttt{(tcc !)}
\item[\indtt{well-founded-tcc}:] calls \texttt{(tcc explicit)}
\item[\indtt{monotonicity-tcc}:] calls \texttt{(tcc explicit)}
\item[\indtt{existence-tcc}:] calls \texttt{(tcc explicit)}
\item[\indtt{assuming-tcc}:] calls \texttt{(tcc explicit)}
\item[\indtt{mapped-axiom-tcc}:] calls \texttt{(tcc explicit)}
\item[\indtt{cases-tcc}:] calls \texttt{(tcc explicit)}
\item[\indtt{cond-coverage-tcc}:] calls \texttt{(tcc explicit)}
\item[\indtt{cond-disjoint-tcc}:] calls \texttt{(tcc explicit)}
\item[\indtt{same-name-tcc}:] calls \texttt{(tcc explicit)}
\end{description}

\item[\indtt{expand1*}:] invoked by \indtt{expand*}
\item[\indtt{label-fnums}:] invoked by \indtt{with-labels}
\item[\indtt{rewrite-directly-with-fnum}:] invoked by \indtt{rewrite-with-fnum}
\item[\indtt{chain-antecedent}:] invoked by \indtt{forward-chain}
\item[\indtt{chain-antecedent*}:] invoked by \indtt{chain-antecedent}
\item[\indtt{detuple-boundvars-in-formulas}:] invoked by \indtt{detuple-boundvars}
\end{description}

The \indtt{defstrat} version defines only a glassbox strategy called
\emph{name} and the rule or blackbox version is not defined.  The
\indtt{defstrat} form does not take the final format-string argument given
to \indtt{defstep}.  The differences between a defined rule and a strategy
are:
\begin{enumerate}

\item A defined rule like a primitive rule is atomic, whereas a strategy
could expand to the application of several atomic rules.

\item Only the expanded form of a strategy is saved to be rerun, whereas
a rule is saved and rerun in its unexpanded form.

\item A defined rule merely returns the unproved subgoals, whereas a
strategy returns the expanded proof tree.  For example, \texttt{prop\char36} and \texttt{ground} are strategies and \indtt{prop} and \texttt{ground} are their corresponding rule versions.  The former are glass
boxes in that their internal behavior is visible to the user, whereas
the latter are black boxes.
\end{enumerate}

Otherwise, defined rules and strategies are very similar.  Both can be
recursive and can involve the application of a number of primitive proof
steps to achieve their effect.  

In the following, we describe some important strategies used for defining
new proof rules.

\section{The Basic Strategies}

\begin{tabular}{|l|l|}\hline
\indtt{if} & Conditional strategy \\\hline
\indtt{let} & Evaluate/bind Lisp expressions/values\\\hline
\indtt{quote} & Identity strategy \\\hline
\indtt{try} & Subgoaling and Backtracking \\\hline
\end{tabular}

\prsubsection{if}{Conditional Selection of Strategies}

\begin{description}

\item[syntax:] \texttt{(if \carg{condition} \carg{step1} \carg{step2})}

\item[effect:] Here \emph{condition} is some Lisp code that is
evaluated against the current goal.  If \emph{condition} evaluates to
\texttt{nil}, then \emph{step2} is applied, else \emph{step1} is applied.

\item[usage:] \texttt{(if (equal (get-goalnum *ps*) 1) (ground) (prop))} :
If the current goal (\texttt{*ps*} is the current proofstate) is the first
subgoal of its parent, the apply \texttt{(ground)}, else apply
\texttt{(prop)}.

\end{description}


\prsubsection{let}{Use Lisp in Strategies}
\begin{description}

\item[syntax:] \texttt{(let ((\carg{var$_1$} \carg{lexpr$_1$}) $\ldots$
(\carg{var$_n$} \carg{lexpr$_n$})) \carg{step})}

\item[effect:] Here \emph{var$_1$} through \emph{var$_n$} are symbols,
and \emph{lexpr$_1$} through \emph{lexpr$_n$} are Lisp expressions.
The \texttt{let}-form allows some values to be computed to be plugged into
strategies.  The scope of each let-binding extends over the later
bindings and the body of the \indtt{let} strategy, so that it is similar
to the \texttt{let*} construct of Lisp.

\item[usage:] A simplified definition of the basic querying strategy is shown
below.  Here, a defined Lisp function \texttt{qread} is used to generate the
\texttt{"Rule?"} query and read the resulting user input.
{\smaller
\begin{alltt}
  (query*) = (let ((input (qread "Rule? ")))
               (try input (query*) (query*)))
\end{alltt}}

The \indtt{then} strategy has the definition shown below:
{\smaller
\begin{alltt}
  (let ((x (when steps (car steps)))
	(y (when steps (cons 'then (cdr steps)))))
    (if steps (if (cdr steps) (try x y y) x) (skip)))
\end{alltt}}
\end{description}

\prsubsection{quote}{The Identity Strategy}

\begin{description}

\item[syntax:] \texttt{(quote \carg{step})}

\item[effect:] The strategy expression \texttt{(quote \emph{step})} simply
evaluates to \emph{step} itself.  This is only useful in support of the
\indtt{let} strategy, in order to ensure that forms are only evaluated
once.

The reason it is needed is that it is quite typical to construct a Lisp
expression corresponding to a strategy but this then needs to be unquoted
to be used as a strategy.  For example the strategy \texttt{(then \rest\
steps)} is given the recursive definition
\begin{alltt}
  (let ((x (when steps (car steps)))
        (y (when steps (cons 'then (cdr steps)))))
    (if steps (if (cdr steps) (try x y y) x) (skip)))
\end{alltt}
In evaluating the body of this definition, the quoted forms of the values
of the bindings for \texttt{x} and \texttt{y} are substituted for these
variables into the body of the let-expression.  When any of these values
are to be evaluated as strategies, the \texttt{quote} simply causes the
underlying steps to be evaluated.

\item[notes:] It is probably a mistake to use this strategy directly.
\end{description}

\prsubsection{try}{Strategy for Subgoaling and Backtracking}

\begin{description}

\item[syntax:] \texttt{(try \carg{step1} \carg{step2} \carg{step3})}

\item[effect:] This is the basic control strategy.  It applies
\emph{step1} to the current goal.  If \emph{step1} succeeds and generates
subgoals, then \emph{step2} is applied to those subgoals.  If \emph{step1}
does nothing, \ie\ behaves as a \texttt{(skip)} step, then \emph{step3} is
applied to the current goal.  The \indtt{try} step thus provides a
backtracking mechanism for proof search that can be controlled by
appropriately signalling failure.  The following identities describe the
behavior of the \indtt{try} strategy (assuming that the \texttt{lemma}
step succeeds):
\begin{enumerate}\label{trypage}
\item \texttt{(try (skip) (assert) (split)) = (split)}
\item \texttt{(try (try (skip) (assert) (fail)) (split) (flatten)) = (flatten)} 
\item \texttt{(try (try (lemma "assoc") (fail) (assert)) (split) (flatten))
= (flatten)}
\item \texttt{(try (try (lemma "assoc") (assert) (fail)) (split) (flatten))
= (then (lemma "assoc") (assert) (split))}
\end{enumerate}

The important thing to note is that if step $A$ succeeds on the current
goal, then applying \texttt{(try $A$ $B$ $C$)} causes the alternative $C$
to be closed.  Then \texttt{(fail)} backtracks to the last open (\ie\ not
closed) alternative.

\begin{usage}{}

\item[\texttt{(try (flatten) (propax) (split))}] : Applies the disjunctive
simplification step to the current goal.  If the goal does disjunctively
simplify, then the \texttt{(propax)} step is applied to the resulting
subgoal.  Otherwise the conjunctive splitting step is applied to the
current goal.

\item[]\texttt{(try (try (flatten) (fail) (skolem 1 ("a" "b"))) (postpone)
(prop))} : If the current goal disjunctively simplifies, then backtrack
and apply \texttt{(prop)}, otherwise introduce skolem constants and if that
fails, try propositional simplification, otherwise postpone.  Notice how
\texttt{(fail)} is used to trigger backtracking from a subgoal.
\end{usage}

\end{description}



\section{Strategies}\label{defined strategies}

We have already briefly discussed the differences between strategies and
defined rules.  In many of the cases, defined rules are analogous to the
\emph{tactics} of LCF, and strategies are like the \emph{tacticals}
which are used to combine rules in various ways.

\prsubsection{branch}{Assign Strategies to Subgoals}
\begin{description}

\item[syntax:] \texttt{(branch \carg{step} \carg{steplist})}

\item[effect:] Just like \indtt{spread} except that when \emph{steplist}
has only $n$ elements and \emph{step} generates more than $n$ subgoals,
the $n$'th element of \texttt{steplist} is also applied to the subgoals
following the $n$'th one (in \indtt{spread}, \texttt{skip} is used).  This
is generally useful when only the first few subgoals generated by
\emph{step} require special attention and the rest of the subgoals yield
to some uniform strategy.
\end{description}

\prsubsection{checkpoint}{Checkpoint Handling}
\begin{description}

\item[syntax:] \texttt{(checkpoint)}

\item[effect:] A synonym for \texttt{query*}: inserting
\texttt{(checkpoint)} into an edited proof and rerunning it causes the
non-checkpointed subproofs to simply be installed (using
\indtt{just-install-proof}) so that the proof quickly run up to the
checkpoint.

\item[notes:] This command is not meant to be used directly; see the User
Guide for details on adding checkpoints to proofs. 
\end{description}


\prsubsection{else}{A Simple Backtracking Strategy}
\begin{description}

\item[syntax:] \texttt{(else \carg{step1} \carg{step2})}

\item[effect:] First applies \emph{step1} and if that does nothing,
then \emph{step2} is applied to the present goal.  The definition of
\texttt{else} is just \texttt{(try \emph{step1} (skip) \emph{step2})}.
\end{description}

\prsubsection{just-install-proof}{Install Proof without Rerunning}
\begin{description}

\item[syntax:] \texttt{(just-install-proof \carg{proof})}

\item[effect:] Installs, the \emph{proof} without actually checking it,
and treats the current subgoal as proved, but marks the proof as
unfinished.  Used in conjunction with \texttt{checkpoint}.

\item[notes:] This command is not meant to be used directly; see the User
Guide for how it is used in editing proofs. 
\end{description}


\prsubsection{query*}{The Basic Interaction Strategy}

\begin{description}

\item[syntax:] \texttt{(query*)}

\item[effect:] This is the strategy that repeatedly queries the user for
the current rule or strategy.  PVS also invokes \indtt{query*} when all
other options have been exhausted and the prover is being used in an
interactive mode.
\end{description}



\prsubsection{repeat}{Iterate Along Main Proof Branch}

\begin{description}

\item[syntax:] \texttt{(repeat \carg{step})}

\item[effect:] First applies \emph{step} to the current goal.  If this
does nothing, then no further steps are indicated.  If the application
of \emph{step} generates subgoals, then \texttt{(repeat \emph{step})} is
recursively applied to the first of those subgoals (the main subgoal).
See \emph{repeat*} below.  The \emph{repeat} strategy must be used
cautiously.  It can easily cause loops since it is only terminated when
\emph{step} does nothing.  Some commands such as \indtt{assert} almost
always take effect even when they seemingly do nothing, and wrapping
these within a \indtt{repeat} can cause a loop.
\end{description}


\prsubsection{repeat*}{Iterate Along all Branches}
\begin{description}

\item[syntax:] \texttt{(repeat* \carg{step})}

\item[effect:] While \emph{repeat} only repeats the step for the main
branch of the proof, \indtt{repeat*} carries out the repetition along all
the subgoals resulting from the first application of \emph{step}.  The
repetition is continued along each branch until an application of \emph{step} has no effect.
\end{description}

\prsubsection{rerun}{Rerun a Proof or Partial Proof}

\begin{description}

\item[syntax:] \texttt{(rerun \optl\ \carg{proof} \carg{recheck?}\
\carg{break?})}

\item[effect:] This step can be used to rerun a partial or completed proof
from a previous attempt or from another branch of the proof.  This step is
largely used automatically by the system when it queries as to whether the
proof should be rerun.  The \emph{proof} argument can also be explicitly
given by the user using either the \emacstt{edit-proof} or
\emacstt{show-proof} commands to generate and edit such inputs.

The \emph{recheck?}\ flag when \texttt{t} is used to rerun an entire proof
expanding all steps so that only primitive proof steps are used.

By default the \texttt{rerun} step simply gives a warning when there is a
mismatch between the number of subgoals and the number of subproofs.  When
\emph{break?}\ is \texttt{t}, an error is produced instead.

This step can be used to:
\begin{enumerate}

\item Restore a partial proof to the state when the proof was
interrupted.

\item Recheck a completed proof.

\item Redo a proof following some changes to the specification.  It is
possible that the old proof only partially works for the changed
specification.  In this case, it is usually possible to clean up and
complete the resulting partial proof.

\item Apply a partial or completed proof from one subgoal to some other
subgoal in the proof attempt.
\end{enumerate}
 
\end{description}

\prsubsection{spread}{Assign Strategies to Subgoals}
\begin{description}

\item[syntax:] \texttt{(spread \carg{step} \carg{steplist})}

\item[effect:] First applies \emph{step1} and then applies the $i$'th
element of \emph{steplist} to the $i$'th subgoal.  If there are more steps
in \emph{steplist} than subgoals the remaining ones are ignored.  If there
are fewer, then \texttt{skip} is applied to the rest.  This is typically
used when \emph{step} splits the proof into multiple branches where a
different strategy is required for each of the branches.  See
\indtt{branch}.
\end{description}

\prsubsection{spread!}{Assigning Strategies to Subgoals with Error Checks}
\index{spread"!@{\texttt{spread"!}}|ii}
\begin{description}

\item[syntax:] \texttt{(spread!\ \carg{step} \carg{steplist})}

\item[effect:] Like \indtt{spread}, applies \emph{step} and then pairs the
steps in \emph{steplist} with the subgoals, but generates an error and
queries the user if the number of subgoals does not match the number of
subproofs.
\end{description}

\prsubsection{spread@}{Assigning Strategies to Subgoals with Warning Checks}
\index{spread"@@{\texttt{spread"@}}|ii}
\begin{description}

\item[syntax:] \texttt{(spread@ \carg{step} \carg{steplist})}

\item[effect:] Like \indtt{spread}, applies \emph{step} and then pairs the
steps in \emph{steplist} with the subgoals, but generates a warning if the
number of subgoals does not match the number of subproofs.
\end{description}


\prsubsection{then}{A Sequencing Strategy}

\begin{description}

\item[syntax:] \texttt{(then \rest\ \carg{steps})}

\item[effect:] Let the list \emph{steps} consist of the first element
\emph{step1} and the remaining steps \emph{rest-steps}.  This strategy
first applies the \emph{step1} to the current goal.  If any subgoals are
generated, then \texttt{(then :steps \emph{rest-steps})} is applied to
each of these subgoals.  If \emph{step1} has no effect, then \texttt{(then
:steps \emph{rest-steps})} is applied to the original goal.  The main body
of \indtt{then} is essentially \texttt{(try \emph{step1} (then
\emph{rest-steps}) (then \emph{rest-steps}))}.
\end{description}


\prsubsection{then@}{Apply Steps in Sequence Along Main Branch}
\index{then"@@{\texttt{then"@}}|ii}
\begin{description}

\item[syntax:] \texttt{(then@ \rest\ \carg{steps})}

\item[effect:] This is a version of \indtt{then} where the given steps are
applied in sequence only along the main branch of the proof, \ie\ if the
the given rule is \texttt{(then@ \emph{step$_1$} \ldots \emph{step$_n$})},
then \emph{step$_{i+1}$} is only applied to the first subgoal of
\emph{step$_i$}.
\end{description}


\prsubsection{time}{Time a Given Strategy}
\begin{description}
\item[syntax:] \texttt{(time \carg{strategy})}

\item[effect:] Executes the given rule or strategy as an atomic step (like
the \indtt{apply} command) while printing out the run times at each of the
leaf nodes.  This command has no other effect on the proof.  It only
prints out timing information when there are leaf nodes generated and
yields no information when the given strategy succeeds in proving the
subgoal.

\item[usage:] \texttt{(time (then (lift-if) (prop) (skolem!)))}

\item[errors:] No error messages other than those generated by the given
strategy.

\end{description}



\prsubsection{try-branch}{Branch or Backtrack}

\begin{description}

\item[syntax:] \texttt{(try-branch \carg{step1} \carg{steplist} \carg{step2})}

\item[effect:] This is a combination of the \indtt{try} and \indtt{branch}
strategies.  It is like branch for \emph{step1} and \emph{steplist}
except that when \emph{step1} has no effect, then \emph{step2} is
attempted on the original subgoal.
\end{description}
\index{Proof Strategies|)}

%\addcontentsline{toc}{chapter}{References}
\bibliographystyle{plain}
\bibliography{../pvs}

%\addcontentsline{toc}{chapter}{Index}   %% Put entry in T-O-C
{\smaller\smaller\printindex}

%\input{prover-r.ind}

\end{document}

% Master File: tutorial.tex
\sloppy

% full list of sections:
%\includeonly{title,intro,informal,pspace,undecide,conclu,ack,rules}

%\pagestyle{myheadings} % page number in upper right corner
%\markboth{Specification and Verification}{}
%\makeindex
%\newcommand{\allttinput}[1]{\hozline{\smaller\smaller\smaller\begin{alltt}\input{#1}\end{alltt}}\hozline}
%\newenvironment{pvsscript}{\hozline\smaller\smaller\smaller\begin{alltt}}{\end{alltt}\hozline}
%\newcommand{\ehdm}{{E{\smaller\smaller HDM}}}
%\newcommand{\Ehdm}{\ehdm}

%\topmargin -10pt
%\textheight 8.5in
%\textwidth 6.0in
%\headheight 15 pt
%\columnwidth \textwidth
%\oddsidemargin 0.5in
%\evensidemargin 0.5in   % fool system for page 0
%\setcounter{topnumber}{9}
%\renewcommand{\topfraction}{.99}
%\setcounter{bottomnumber}{9}
%\renewcommand{\bottomfraction}{.99}
%\setcounter{totalnumber}{10}
%\renewcommand{\textfraction}{.01}
%\renewcommand{\floatpagefraction}{.01}
%\newenvironment{smalltt}{\begin{alltt}\small}{\end{alltt}}
\raggedbottom

\font\largett=cmtt10 scaled\magstep2
\font\hugett=cmtt10 scaled\magstep4
\def\opt{{\smaller\sc {\smaller\smaller \&}optional}}
\def\rest{{\smaller\sc {\smaller\smaller \&}rest}}
\def\default#1{[\,{\tt #1}] }
\def\bkt#1{{$\langle$#1$\rangle$}}
\def\SetFigFont#1#2#3{\smaller\smaller\rm}

%\newenvironment{usage}[1]{\item[usage:\hspace*{-0.175in}]#1\begin{description}\setlength{\itemindent}{-0.2in}\setlength{\itemsep}{0.1in}}{\end{description}}
\renewcommand{\pvstheory}[3]
  {\begin{figure}[htb]\begin{boxedminipage}{\textwidth}%
   {\smaller\smaller\smaller\begin{alltt} \input{#1}\end{alltt}}\end{boxedminipage}%
   \caption{#2}\label{#3}\end{figure}}


\section{Two Hardware Examples}

In this final section, we develop two hardware examples that
illustrate a more sophisticated use of \pvs\ and suggest the
intellectual discipline involved in specifying and proving
industrial-strength applications.  The pipelined microprocessor and
n-bit ripple-carry adder examples provide an opportunity to touch on
modeling issues, specification styles, and hardware proof strategies,
as well as a chance to review many of the \pvs\ language and prover
features described in earlier sections of this tutorial.\footnote{One
point worth noting that may not be apparent in reading these examples
is that the process of specification and verification is an iterative one in
which proof is used not to certify a completed specification, but
as an aid to developing the specification.}



\subsection{A Pipelined Microprocessor}

We first develop a complete proof of a correctness property of the
controller logic of a simple pipelined processor design described at a
register-transfer level.  The design and the property verified are
both based on the processor example given in \cite{Clarke-etal90}.
The example has been used as a benchmark for evaluating how well
finite state-enumeration based tools, such as model checkers, can
handle datapath-oriented circuits with a large number of states by
varying the size of the datapath.  From the perspective of a theorem
prover, the size of the datapath is irrelevant because the
specification and proof are independent of the datapath size.  As a
theorem proving exercise, the challenge is to see if the proof can be
done as automatically as a model checker proof.

\begin{figure}[b]
\begin{center}
\setlength{\unitlength}{0.009375in}%
\begin{picture}(191,262)(112,399)
\thicklines
\put(190,567){\framebox(13,19){}}
\put(190,542){\framebox(13,19){}}
\put(190,514){\framebox(13,19){}}
\put(278,542){\framebox(12,19){}}
\put(278,514){\framebox(12,19){}}
\put(190,461){\framebox(13,38){}}
\put(190,411){\framebox(13,38){}}
\put(228,499){\line( 0,-1){ 38}}
\put(228,449){\line( 0,-1){ 38}}
\put(259,474){\line( 0,-1){ 38}}
\put(228,499){\line( 5,-4){ 31.098}}
\put(228,411){\line( 5, 4){ 31.098}}
\multiput(228,461)(0.66667,-0.33333){19}{\makebox(0.5926,0.8889){\SetFigFont{7}{8.4}{rm}.}}
\multiput(240,455)(-0.66667,-0.33333){19}{\makebox(0.5926,0.8889){\SetFigFont{7}{8.4}{rm}.}}
\put(278,436){\framebox(12,38){}}
\put(203,480){\vector( 1, 0){ 25}}
\put(203,427){\vector( 1, 0){ 25}}
\put(259,455){\vector( 1, 0){ 19}}
\put(265,455){\vector( 0,-1){ 56}}
\put(303,455){\line( 0,-1){ 56}}
\put(303,399){\line(-1, 0){107}}
\put(196,399){\vector( 0, 1){ 12}}
\put(203,524){\vector( 1, 0){ 75}}
\put(203,552){\vector( 1, 0){ 75}}
\put(178,661){\line( 0,-1){ 62}}
\put(165,630){\line( 1, 0){ 13}}
\put(165,599){\framebox(125,62){}}
\put(278,661){\line( 0,-1){ 62}}
\put(165,642){\line(-1, 0){ 25}}
\put(140,642){\line( 0,-1){212}}
\put(140,430){\vector( 1, 0){ 50}}
\put(128,580){\vector( 1, 0){ 62}}
\put(196,461){\line( 0,-1){ 12}}
\multiput(165,511)(8.73333,0.00000){16}{\makebox(0.5926,0.8889){\SetFigFont{10}{12}{rm}.}}
\multiput(165,589)(8.73333,0.00000){16}{\makebox(0.5926,0.8889){\SetFigFont{10}{12}{rm}.}}
\multiput(165,511)(0.00000,8.66667){10}{\makebox(0.5926,0.8889){\SetFigFont{10}{12}{rm}.}}
\multiput(296,511)(0.00000,8.66667){10}{\makebox(0.5926,0.8889){\SetFigFont{10}{12}{rm}.}}
\put(196,499){\vector( 0,-1){0}}
\multiput(196,511)(0.00000,-12.00000){1}{\makebox(0.5926,0.8889){\SetFigFont{10}{12}{rm}.}}
\put(243,486){\vector( 0,-1){0}}
\multiput(243,511)(0.00000,-8.33333){3}{\makebox(0.5926,0.8889){\SetFigFont{10}{12}{rm}.}}
\put(178,524){\vector( 1, 0){ 12}}
\put(128,552){\vector( 1, 0){ 62}}
\put(284,561){\vector( 0, 1){ 38}}
\put(265,455){\vector( 0, 1){ 56}}
\put(290,455){\line( 1, 0){ 13}}
\put(303,455){\line( 0, 1){178}}
\put(303,633){\vector(-1, 0){ 13}}
\put(146,655){\line( 0,-1){125}}
\put(128,614){\line( 1, 0){ 18}}
\put(146,655){\vector( 1, 0){ 19}}
\put(146,605){\vector( 1, 0){ 19}}
\put(146,530){\vector( 1, 0){ 19}}
\put(165,617){\line(-1, 0){ 12}}
\put(153,617){\line( 0,-1){140}}
\put(153,477){\vector( 1, 0){ 37}}
\put(168,518){\makebox(0,0)[lb]{\smash{\SetFigFont{5}{6.0}{bf}stall}}}
\put(209,627){\makebox(0,0)[lb]{\smash{\SetFigFont{5}{6.0}{bf}REGFILE}}}
\put(112,586){\makebox(0,0)[lb]{\smash{\SetFigFont{5}{6.0}{bf}opcode}}}
\put(243,443){\makebox(0,0)[lb]{\smash{\SetFigFont{5}{6.0}{bf}U}}}
\put(243,452){\makebox(0,0)[lb]{\smash{\SetFigFont{5}{6.0}{bf}L}}}
\put(243,461){\makebox(0,0)[lb]{\smash{\SetFigFont{5}{6.0}{bf}A}}}
\put(215,567){\makebox(0,0)[lb]{\smash{\SetFigFont{5}{6.0}{bf}CONTROL}}}
\put(240,542){\makebox(0,0)[lb]{\smash{\SetFigFont{5}{6.0}{bf}dsntdd}}}
\put(165,542){\makebox(0,0)[lb]{\smash{\SetFigFont{5}{6.0}{bf}dstnd}}}
\put(206,527){\makebox(0,0)[lb]{\smash{\SetFigFont{5}{6.0}{bf}stalld}}}
\put(243,514){\makebox(0,0)[lb]{\smash{\SetFigFont{5}{6.0}{bf}stalldd}}}
\put(271,427){\makebox(0,0)[lb]{\smash{\SetFigFont{5}{6.0}{bf}wbreg}}}
\put(153,418){\makebox(0,0)[lb]{\smash{\SetFigFont{5}{6.0}{bf}opreg2}}}
\put(156,461){\makebox(0,0)[lb]{\smash{\SetFigFont{5}{6.0}{bf}opreg1}}}
\put(115,542){\makebox(0,0)[lb]{\smash{\SetFigFont{5}{6.0}{bf}dstn}}}
\put(206,580){\makebox(0,0)[lb]{\smash{\SetFigFont{5}{6.0}{bf}opcoded}}}
\put(112,627){\makebox(0,0)[lb]{\smash{\SetFigFont{5}{6.0}{bf}src2}}}
\put(112,617){\makebox(0,0)[lb]{\smash{\SetFigFont{5}{6.0}{bf}src1}}}
\end{picture}

\end{center}
\caption{A Pipelined Microprocessor}
\label{clarkepipefig}
\end{figure}

\subsubsection{Informal Description}

Figure~\ref{clarkepipefig} shows a block diagram of the pipeline
design.  The processor executes instructions of
the form {\tt (opcode src1 src2 dstn)}, i.e., ``destination register
{\tt dstn} in the register file {\tt REGFILE} becomes some {\tt ALU}
function determined by {\tt opcode} of the contents of source registers
{\tt src1} and {\tt src2}.
Every instruction is executed in three stages (cycles) by the processor:
\begin{enumerate}
\item {\em Read}: Obtain the proper contents of the register file at {\tt src1}
and {\tt src2} and clock them into {\tt opreg1} and {\tt opreg2},
respectively.

\item {\em Compute}: Perform the ALU operation corresponding to the
opcode (remembered in {\tt opcoded}) of the instruction and clock the
result into {\tt wbreg}.

\item {\em Write}: Update the register file at the destination register
(remembered in {\tt dstndd}) of the instruction with the value in
{\tt wbreg}.
\end{enumerate}
The processor uses a three-stage pipeline to simultaneously execute
distinct stages of three successive instructions.  That is, 
the read stage of the current instruction is executed along with the
compute stage of the previous instruction and the write stage
of the previous-to-previous instruction.
Since the {\tt REGFILE} is not updated with the results of the previous and
previous-to-previous instructions while a read is being
performed for the current instruction, the controller
``bypasses'' {\tt REGFILE}, if necessary, to get the correct values for
the read.  The processor can abort, i.e., treat as {\tt NOP},
the instruction in the read stage by asserting the {\tt stall} signal true.
An instruction is aborted by inhibiting its write stage
by remembering the {\tt stall} signal until the write stage via
the registers {\tt stalld} and {\tt stalldd}.
We verify that an instruction entering the pipeline
at any time gets completed correctly, i.e., will write the correct result
into the register file, three cycles later, provided the instruction
is not aborted.

\subsubsection{Formal Specification}

\pvs\ specifications consist of a number of files, each of which
contains one or more theories.
%A theory is a collection of declarations:
%types, constants (including functions), axioms that express properties
%about the constants, and theorems and lemmas to be proved.
%Theories may import other theories.
%Every entity used in a theory must be either declared in an imported theory
%or be part of the prelude (the standard
%collection of theories built-in to \pvs\).
\pvstheory{pipeline-spec}{Microprocessor Specification}{pipeline-spec}
The microprocessor specification is organized into three theories, selected
parts of which are shown in Figures~\ref{pipeline-spec} and
\ref{signal-spec}.
(The complete specification can be found in~\cite{HW-Tutorial:Report}.)
The theory {\tt pipe} (Figure~\ref{pipeline-spec})
contains a specification
of the design and a statement of the correctness property to be
proved.
The theories {\tt signal} and {\tt time}, (Figure~\ref{signal-spec})
imported by {\tt pipe}, declares the types {\tt signal} and {\tt time}
used in {\tt pipe}.

\pvstheory{signal-spec}{Signal Specification}{signal-spec}
%\pvstheory{pipeline-spec}{Microprocessor Specification}{pipeline-spec-part2}

The theory {\tt pipe} is parameterized with respect to the types of the
register address, data, and the opcode field of the instructions.
A theory parameter in PVS can be either a type parameter or
a parameter belonging to a particular type, such as {\tt nat}.
Since {\tt pipe} does not impose any restriction on its parameters,
other than the requirement that they be nonempty, which is stated
in the {\tt ASSUMING} part of the theory,
one can instantiate them with any type.
Every entity declared in a parameterized theory is implicitly parameterized
with respect to the parameters of the theory.
For example, the type {\tt signal} declared in the parameterized
theory {\tt signal} is a parametric type denoting a function that
maps {\tt time} (a synonym for {\tt nat})
to the type parameter {\tt T}.  (The type {\tt signal} is used
to model the wires in our design.)
By importing the theory {\tt signal} uninstantiated in {\tt pipe},
we have the freedom to create any desired instances of the type
{\tt signal}.

In this example, we use a {\em functional} style of specification
to model register-transfer-level digital hardware in logic.
In this style, the inputs to the design and the outputs of every component
in the design are modeled as signals.
Every signal that is an output of a component is
specified as a function of the signals appearing at the inputs to
the component.

This style should be contrasted with
a {\em predicative} style, which is commonly used in most HOL applications.
In the predicative style every hardware component is specified as a
predicate relating the
input and output signals of the component and a design is
specified as a conjunction of the component predicates, with all
the internal signals used to connect the components
hidden by existential quantification.
A proof of correctness for a predicative style specification usually involves
executing a few additional steps at the start of the proof
to essentially transform
the predictative specification into an equivalent functional style.
After that, the proof proceeds similar to that of a proof in
a functional specification.
The additional proof steps required for a predicative specification
essentially unwind the component predicates using
their definitions and then appropriately
instantiate the existentially quantified variables.
An automatic way of performing this translation is discussed in
\cite{HW-Tutorial:Report}, which illustrates more examples
of hardware design verification using PVS.

Returning to our example, the microprocessor specification
in {\tt pipe} consists of two parts.
The first part declares all the signals
used in the design---the inputs
to the design and the internal wires that denote the outputs of components.
The composite state of {\tt REGFILE}, which is represented
as a function from {\tt addr} to {\tt data}, is modeled by the signal
{\tt regfile}.
The signals are declared as uninterpreted constants of appropriate types.
The second part consists of a set of AXIOMs that specify the
the values of the signals over time.
(To conserve space, we have only shown the specification of a subset
of the signals in the design.)
For example, the signal value at the output of the
register {\tt dstnd} at time {\tt t+1} is defined to be that of its
input a cycle earlier.
The output of the ALU, which is a combinational component, is defined
in terms of the inputs at the same time instant.

In PVS, we can use a descriptive style of definition, as illustrated
in this example, by selectively introducing properties of the
constants declared in a theory as AXIOMs.  Alternatively, we can use
the definitional forms provided by the language to define the
constants.  An advantage of using the definitions is that a
specification is guaranteed to be consistent. A disadvantage is that
the resulting specification may be overly specific (i.e.,
overspecified).  An advantage of the descriptive style is that it
gives better control over the degree to which an entity is defined For
example, we could have specified {\tt dstnd} prescriptively, using the
conventional function definition mechanism of PVS, which would have
forced us to specify the value of the signal at time {\tt t = 0} to
ensure that the function is total.  In the descriptive style used, we
have left the value of the signal at {\tt 0} unspecified.

In the present example, the specifications of the signals
{\tt opreg1} and {\tt opreg2} are the most interesting of all.
They have to check for any register collisions that might
exist between the instruction in the
read stage and the instructions in the later stages and bypass reading
from the register file in case of collisions.
The {\tt regfile} signal specification is recursive since the register
file state remains the same as its previous state except,
possibly, at a single register location.
The {\tt WITH} expression is an abbreviation for the result
of updating a function at a given point in the domain value with a new value.
Note that the function {\tt aluop} that denotes the operation ALU performs
for a given {\tt opcode} is left completely unspecified since it
is irrelevant to the controller logic.

The theorem ({\tt correctness}) to be proved states a correctness property
about the execution of the instruction that enters the pipeline at
{\tt t}, provided the instruction is not aborted, i.e., {\tt stall(t)} is
not true.
The equation in the conclusion of the implication compares the
actual value (left hand side) in the destination register three
cycles later, when the result of the instruction would be in place,
with the expected value.
The expected value is the result of applying the {\tt aluop} corresponding
to the opcode of the instruction to the values at the source field
registers in the register file at {\tt t+2}.
We use the state of the register file at {\tt t+2} rather than {\tt t}
to allow for the results of the two previous instructions in the pipeline
to be completed.

\subsubsection{Proof of Correctness}

Once the specification is complete, the next step is to typecheck the
file, which parses and checks for semantic errors, such as undeclared
names and ambiguous types.  As we have already seen, typechecking may
build new files or internal structures such as {\em type correctness
conditions} ({\em \tccs}) that represent {\em proof obligations\/}
that must be discharged before the {\tt pipe} theory can be considered
typechecked.  The typechecker does not generate any \tccs\ in the
present example.  If, for example, one of the assumptions, say for
{\tt addr}, in the {\tt ASSUMING} part of the theory was missing, the
typechecker would generate the following \tcc\ to show that the {\tt
addr} type is nonempty.  The declaration of the signal {\tt src1}
forces generation of this \tcc\ because a function is nonexistent if
its range is empty.

\mbox{}

\noindent
\begin{boxedminipage}{\textwidth}
\begin{alltt}
{\smaller\smaller
% Existence TCC generated (line 17) for src1: signal[addr]
% May need to add an assuming clause to prove this.
  % unproved
src1_TCC1: OBLIGATION (EXISTS (x1: signal[addr]): TRUE);
}
\end{alltt}
\end{boxedminipage}

\mbox{}

%The \pvs\ proof checker runs as a subprocess of Emacs.
%Once invoked on a theorem to be proved, it accepts commands
%directly from the user.

By way of review, the basic objective of developing a proof in \pvs\
as in other subgoal-directed proof checkers (e.g., HOL), is to
generate a {\em proof tree\/} in which all of the leaves are trivially
true.  The nodes of the proof tree are sequents, and in the
prover you are always looking at an unproved leaf of the tree.
The {\em current\/} branch of a proof is the branch leading back to
the root from the current sequent.  When a given branch is complete
(i.e., ends in a true leaf), the prover automatically moves on to the
next unproved branch, or, if there are no more unproven branches,
notifies you that the proof is complete.

%One reason why a proof in \pvs\ differs from a HOL proof is due to
The primitive inference steps in \pvs\ are a lot more
powerful than in HOL; it is not necessary to build complex tactics to
handle tedious lower level proofs in \pvs\@.  A knowledgeable \pvs\
user can typically get proofs to go through mostly automatically by
making a few critical decisions at the start of the proof.  However,
as noted previously, \pvs\ does provide the user with the equivalent
of HOL's tacticals, called {\em strategies}, and other features to
control the desired level of automation in a proof.

The proof of the microprocessor property shown below follows a certain
general pattern that works successfully for most hardware proofs.
This general proof pattern, variants of which have been used
in other verification
exercises \cite{mephisto,HOL:super}, consists of the following
sequence of general proof tasks.
\begin{description}
\item[Quantifier elimination:] Since the decision procedures work on
ground formulas, the user must eliminate the relevant universal
quantifiers by skolemization or selecting variables on which to induct
and existential quantifiers by suitable instantiation.

\item[Unfolding definitions:] The user may have to simplify selected
expressions
and defined function symbols in the goal by rewriting using definitions,
axioms or lemmas.
The user may also have to decide the level to which the function symbols have
to rewritten.

\item[Case analysis:] The user may have to split the proof based on
selected boolean expressions in the current goal and simplify the
resulting goals further.
\end{description}

Each of the above tasks can be accomplished automatically using a short
sequence of primitive PVS proof commands.
The complete proof of the theorem is shown below.
Selected parts of the proof session are reproduced below as we describe
the proof.

\mbox{}

\noindent
\begin{boxedminipage}{\textwidth}
\begin{alltt}
{\smaller\smaller
1: ({\em then*} (skosimp)
2:         (auto-rewrite-theory ``pipe'' :always? t)
3:         (repeat (do-rewrite))
4:         (apply (then* (repeat (lift-if))
5:                       (bddsimp)
6:                       (assert))))
}
\end{alltt}
\end{boxedminipage}

\mbox{}

In the proof, the names of strategies are shown in {\em italics} and
the primitive inference steps in {\tt type-writer font}.
(We have numbered the lines in the proof for reference.)
{\tt Then*} applies
the first command in the list that follows to the current goal;
the rest of the commands in the list are then applied
to each of the subgoals generated by the first command
application.
The {\tt apply} command used in line 5 makes the application of 
a compound proof step implemented by a strategy behave as
an atomic step.

The first goal in the proof session is shown below.
It consists of a
single formula (labeled {\tt \{1\}}) under a dashed line.  This is a
{\em sequent\/}; formulas above the dashed lines are called {\em
antecedents\/} and those below are called {\em succedents\/}.  The
interpretation of a sequent is that the conjunction of the antecedents
implies the disjunction of the succedents.
\comment{
Either or both of the
antecedents and succedents may be empty.\footnote{An empty antecedent
is equivalent to {\tt true}, and an empty succedent is equivalent to
{\tt false}, so if both are empty the sequent is unprovable.}}

\mbox{}

\noindent
\begin{boxedminipage}{\textwidth}
\begin{alltt}
{\smaller\smaller
correctness :   

  |-------
\{1\}   (FORALL t: NOT (stall(t))
                  IMPLIES regfile(t + 3)(dstn(t)) =
                     aluop(opcode(t), regfile(t + 2)(src1(t)),
                           regfile(t + 2)(src2(t))))
}
\end{alltt}
\end{boxedminipage}

\mbox{}

The quantifier elimination task of the proof is accomplished
by the command {\tt skosimp}, which skolemizes all the universally quantified
variables in a formula and flattens the sequent resulting in the following
goal.  Note that {\tt stall(t!1)} has been moved to the
succedent in the sequent because \pvs\ displays every atomic formula in
its positive form.

\mbox{}

\noindent
\begin{boxedminipage}{\textwidth}
\begin{alltt}
{\smaller\smaller
Rule? (skosimp)
Skolemizing and flattening, this simplifies to: 
correctness :   

  |-------
\{1\}   (stall(t!1))
\{2\}   regfile(t!1 + 3)(dstn(t!1))
        =
        aluop(opcode(t!1), regfile(t!1 + 2)(src1(t!1)),
              regfile(t!1 + 2)(src2(t!1)))
}
\end{alltt}
\end{boxedminipage}

\mbox{}

The next task---unfolding definitions---is performed by the commands
in lines 2 through 3.  \pvs\ provides a number of ways of unfolding
definitions ranging from unfolding one step at a time to automatic
rewriting that performs unfolding in a brute-force fashion.
Brute-force rewriting usually results in larger expressions than
controlled unfolding and, hence, potentially larger number of cases to
consider.  If a system provides automatic and efficient rewriting and
case analysis facilities, then the automatic approach is viable,
as illustrated here.  In \pvs\, automatic rewriting is performed
by first entering the definitions and AXIOMs to be used
for unfolding as rewrite rules.  Once entered, the commands that
perform rewriting as part of their repertoire, such as {\tt
do-rewrite} and {\tt assert}, repeatedly apply the rewrite rules until
none of the rules is applicable.  To control the size of the
expression resulting from rewriting and the potential for looping, the
rewriter uses the following restriction for stopping a rewrite: If the
right-hand-side of a rewrite is a conditional expression, then the
rule is applied only if the condition simplifies to true or false.
\comment{ Also, application of a recursive rewrite rule, such as {\tt
regfile\_ax} is inhibited on recursive instances of a function symbol
if the function is inside a conditional expression.}

Here our aim is to unfold every signal in
the sequent so that every signal expression contains only the
start time {\tt t!1}.
So, we make a rewrite rule out of every AXIOM in the theory {\tt pipe}
by means of the command {\tt auto-rewrite-theory} on line 2.
We also force an over-ride of the default restriction for stopping rewriting by
setting the tag\footnote{Tags are one of the ways in which \pvs\ permits the user to modify
the functionality of proof commands.} {\tt always?}\ to true in the {\tt auto-rewrite-theory}
command and embed {\tt do-rewrite} inside a {\tt repeat} loop to force
maximum rewriting.
In the present example, the rewriting
is guaranteed to terminate because every feedback loop is cut by a sequential
component.

At the end of automatic rewriting, the succedent we are trying to prove
is in the form of an equation on two deeply nested conditional expressions
as shown below in an abbreviated fashion.
The various cases in conditional expression shown above
arise as a result of
the different possible conflicts between instructions
in the pipeline.  The equation we are trying to prove contains
two distinct, but equivalent conditional expressions, as in
{\tt IF a THEN b ELSE c ENDIF = IF NOT a THEN c ELSE b ENDIF}, that
can only be proved equal by performing a case-split on one or more of the
conditions.  While {\tt assert} simplifies the leaves of a conditional
expression assuming every condition along the path to the leaves holds,
it does not split propositions.
One way to perform the case-splitting task automatically is
to ``lift'' all the {\tt IF-THEN-ELSE}s to the top so that the equation
is transformed into a propositional formula with unconditional equalities
as atomic predicates.
After performing such a lifting, we can try to reduce the resulting
proposition to true using the propositional simplification command
{\tt bddsimp}.  If {\tt bddsimp} does not simplify the proposition
to true, then it is most likely the case that equations at one or more
of the leaves of the proposition need to be further simplified, e.g., by
{\tt assert}, using the conditions along the path.
If the propositional formula does not reduce to true or false,
{\tt bddsimp} produces a set of subgoals to be proved.
In the present case, each of these goals can be discharged
by {\tt assert}.
The compound proof step appearing on lines 4 through 6 of the proof
accomplishes the case-splitting task.

\mbox{}

\noindent
\begin{boxedminipage}{\textwidth}
\begin{alltt}
{\smaller\smaller
correctness :   

  |-------
[1]   (stall(t!1))
\{2\}   aluop(opcode(t!1),
            IF src1(t!1) = dstnd(t!1) & NOT stalld(t!1)
              THEN aluop(opcoded(t!1), opreg1(t!1), opreg2(t!1))
            ELSIF src1(t!1) = dstndd(t!1) & NOT stalldd(t!1)
            THEN wbreg(t!1)
            ELSE regfile(t!1)(src1(t!1)) ENDIF,
            ....
            ENDIF)
        = aluop(opcode(t!1),
              IF stalld(t!1) THEN IF stalldd(t!1) THEN regfile(t!1)
                ELSE regfile(t!1) WITH [(dstndd(t!1)) := wbreg(t!1)]
                ENDIF
              ELSE ...
              ENDIF(src1(t!1)),
              IF stalld(t!1) THEN IF stalldd(t!1) THEN regfile(t!1)
                ELSE ... ENDIF
              ELSE ...
              ENDIF(src2(t!1)))
}
\end{alltt}
\end{boxedminipage}

\mbox{}

We have found that the sequence of steps shown above works successfully
for proving safety properties of finite state machines that relate
states of the machine that are finite distance apart.  If the strategy
does not succeed, the most likely cause is that either the
property is not true or that a certain property about some of the
functions in the specification unknown to the prover needs to be
proved as a lemma.  In either case, the unproven goals remaining at the
end of the proof provide information about the probable cause.


\subsection{An N-bit Ripple-Carry Adder}

The second example we consider is the verification of a parametrized
N-bit ripple-carry adder circuit.
The theory {\tt adder}, shown in Figure~\ref{adder-spec},
specifies a ripple-carry adder
circuit and a statement of correctness for the circuit.

\pvstheory{adder-spec}{Adder Specification}{adder-spec}

The theory is parameterized with respect to the length of the bit-vectors.
It imports the theories (not shown here)
{\tt full\_adder}, which contains a
specification of a full adder circuit ({\tt fa\_cout} and {\tt fa\_sum}),
and {\tt bv}, which specifies
the bit-vector type ({\tt bvec[N]}) and functions.
An N-bit bit-vector is represented as an array, i.e., a function, from
the the type {\tt below[N]}, a subtype of {\tt nat} ranging from
{\tt 0} through {\tt N-1}, to {\tt bool}; the index {\tt 0} denotes the least
significant bit.
Note that the parameter {\tt N} is constrained to be a {\tt posnat} since
we do not permit bit vectors of length {\tt 0}.
The {\tt adder} theory contains several declarations including a set
of initial variable declarations.
%\pvs\ allows logical variables to be declared globally within a theory
%so that the variables can be used later in function
%definitions and quantified formulas.

The carry bit that ripples through the full adder is specified recursively
by means of the function {\tt nth\_cin}.
%Associated with this definition is a
%{\em measure\/} function, following the {\tt MEASURE} keyword, which
%will be explained below.
The function {\tt bv\_cout} and {\tt bv\_sum} define the carry output
and the bit-vector sum of the adder, respectively.
The theorem {\tt adder\_correct} expresses the conventional correctness
statement of an adder circuit using {\tt bvec2nat}, which returns the
natural number equivalent of an N-bit bit-vector.
Note that variables that are left free in a formula are assumed to be
universally quantified.
We state and prove a more general lemma {\tt adder\_correct\_rec} of which
{\tt adder\_correct} is an instance.
For a given {\tt n < N},
{\tt bvec2nat\_rec} returns the natural number equivalent of
the least significant n-bits of a given bit-vector and {\tt bool2bit}
converts the boolean constants {\tt TRUE} and {\tt FALSE} into the natural
numbers {\tt 1} and {\tt 0}, respectively.

\subsubsection{Typechecking}
\index{typecheck|(}

The typechecker generates several \tccs\
(shown in Figure~\ref{adder-tccs} below) for {\tt adder}.
%These \tccs\ represent
%{\em proof obligations\/} that must be discharged before the {\tt adder}
%theory can be considered typechecked.  The proofs of the \tccs\
%may be postponed until it is convenient to prove them, though it is a good
%idea to view them to see if they are provable.

\pvstheory{adder-tccs}{\tccs\ for Theory {\tt adder}}{adder-tccs}

The first \tcc\ is due to the fact that the first argument to {\tt nth\_cin}
is of type {\tt below[N]}, but the type of the argument ({\tt n-1})
in the recursive
call to {\tt nth\_cin} is integer, since {\tt below[N]} is not closed
under subtraction.
Note that the \tcc\ includes the condition {\tt NOT n = 0}, which holds
in the branch of the {\tt IF-THEN-ELSE} in which the expression
{\tt n - 1} occurs.
A \tcc\ identical to this one is generated for each
of the two other occurrences of the expression {\tt n-1} because
{\tt bv1} and {\tt bv2} also expect arguments of type {\tt below[N]}.
These \tccs\ are not retained because they are subsumed by the first one.

The second \tcc\ is generated by the expression {\tt N-1} in the
definition of the theorem {\tt adder\_correct} because the first
argument to {\tt bv\_cout} is expected to be the subtype {\tt below[N]}.

There is yet another \tcc\ that is internally generated
by \pvs\ but is not even included in the \tccs\ file because
it can be discharged trivially by the typechecker, which calls
the prover to perform simple normalizations of expressions.
This \tcc\ is generated to ensure that the recursive definition
of {\tt nth\_cin} terminates.
\pvs\ does not directly support partial
functions, although its powerful subtyping mechanism allows \pvs\ to
express many operations that are traditionally regarded as partial.
As discussed earlier, the
measure function is used to show that recursive definitions are total by
requiring the measure to decrease with each recursive call.
For the definition of {\tt nth\_cin}, this entails showing
{\tt n-1 < n}, which the typechecker trivially deduces.

In the present case, all the remaining \tccs\ are simple, and in fact can be
discharged automatically
by using the {\tt typecheck-prove} command, which attempts
to prove all \tccs\ that have been generated using a predefined
proof strategy called {\tt tcc}.
\index{TCCs@\tccs|)}\index{typecheck|)}

\subsubsection{Proof of Adder\_correct\_n}
The proof of the lemma uses the same core strategy as in the
microprocessor proof except for the quantifier elimination step.
Since the specification is recursive in the length of the bit-vector,
we need to perform induction on the variable {\tt n}. As we've seen
in earlier proofs,
the user invokes an inductive proof in \pvs\ by means of the command
{\tt induct} with the variable to induct on ({\tt n}) and the induction
scheme to be used ({\tt below\_induction[N]}) as arguments.
The induction used in this case is defined in the \pvs\ prelude and
is parameterized, as is the type {\tt below[N]}, with respect to
the upper limit of the subrange.

This command  generates two subgoals:
the subgoal corresponding to the base case, which is the first goal presented
to prove, is shown in Figure~\ref{base-step}.

\pvstheory{base-step}{Base Step}{base-step}

The goal corresponding to the inductive case is shown below.

\pvstheory{siblings}{Inductive Step}{siblings}

The base and the inductive steps can be proved automatically
using essentially the same strategy used in the microprocessor proof.
A complete proof of {\tt adder\_correct\_n} is shown in~Figure~\ref{siblings}.

\begin{alltt}
{\smaller\smaller
 1: ({\em spread} (induct ``n'' 1 ``below_induction[N]'')
 2:   ( ({\em then*}  (skosimp*)
 3:              (auto-rewrite-defs :always? t)
 4:              (do-rewrite)
 5:              ({\em repeat} (lift-if))
 6:              ({\em apply} ({\em then*} (bddsimp)(assert))))
 7:     ({\em then*} (skosimp*)
 8:             (inst?)
 9:             (auto-rewrite-defs :always? t)
10:             (do-rewrite)
11:             ({\em repeat} (lift-if))
12:             ({\em apply} ({\em then*} (bddsimp)(assert))))))
}
\end{alltt}

The strategy {\em spread} used on line 1 applies the first proof step
({\tt induct})
and then applies the $i^{th}$ element of the list of commands that follow
to the $i^{th}$ subgoal resulting from the application of the first prof step.
Thus, the proof steps listed on lines 2 through 6 prove the base case
of induction, the steps on lines 7 through 12 prove the inductive case, and
the proof step on line 13 takes care of the third \tcc\ subgoal.

%\input{adder-auto-proof-script}

We consider the base case first.
The {\tt induct} command has already instantiated the variable {\tt n}
to {\tt 0}.
The remaining variables are skolemized away by {\tt skosimp*}.
To unfold the definitions in the resulting goal, we use
the command {\tt auto-rewrite-defs}, which makes rewrite rules out
of the definition of every function either directly or indirectly
used in the given formula.
The rest of the proof proceeds exactly as for the microprocessor.

The proof of the inductive step follows exactly the same pattern except
that we need to instantiate the induction hypothesis and use it in
the process of unfolding and case-analysis.
\pvs\ provides a command {\tt inst?}\ that tries to find instantiations
for existential-strength variables in a formula by searching for possible
matches between terms involving these variables with ground terms inside
formulas in the rest of the sequent.  This command finds the desired
instantiations in the present case.  The rest of the proof proceeds
as in the basis case.

Since the inductive proof pattern shown above is applicable to any
iteratively generated hardware designs, we have packaged it into a
general proof strategy called {\tt name-induct-and-bddrewrite}.  The
strategy is parameterized with respect to an induction scheme
and the set of rewrite rules to be used for unfolding.  We have used
the strategy to prove an N-bit ALU~\cite{cantu:alu} that executes 12
microoperations by cascading N 1-bit ALU slices.


\section{Exercises}

\newtheorem{prob}{Problem}

\begin{prob}
Based on the discussion of the specification of stacks, try to specify a
PVS theory formalizing queues.  Can the PVS abstract datatype facility
be used for specifying queues?
\end{prob}

\begin{prob}
Specify binary trees with value type {\tt T} as a parametric abstract
datatype in PVS.
\end{prob}

\begin{prob}
Specify a PVS theory formalizing {\em ordered\/} binary trees with respect to a
type parameter {\tt T} and a given total-ordering relation, \ie\ define
a predicate {\tt ordered?} that checks if a given binary tree is ordered
with respect to the given total ordering.
\end{prob}

\begin{prob}
  Prove the stack axioms for the definitions stated in {\tt newstacks}.
\end{prob}

\begin{prob}
  Prove the theorems in the theory {\tt half} (Page~\pageref{half}).
\end{prob}

\begin{prob}
  Define the operation for carrying out the ordered insertion of a value
into an ordered binary tree.  Prove that the insertion operator applied
to an ordered binary tree returns an ordered binary tree.
\end{prob}

% \section{References}
{\smaller
%\bibliographystyle{modplain}
\addcontentsline{toc}{section}{References}
\bibliography{mybib,/homes/rushby/jmr,/homes/EHDM/pvs/doc/pvs,/homes/krypton/shankar/tex}
\newpage
\pagestyle{empty}
\mbox{ }
\newpage 
\pagestyle{empty}
\mbox{ }
\newpage 
\end{document}
