\section{Library Declarations}
\label{library-decls}
\index{library declaration|(}
\index{declaration!library|(}

Library declarations are used to introduce a new PVS context\index{PVS Context} into a
specification.  Thus a specification may be developed in one context, and
used in many other contexts.  This provides more flexibility, at the cost
of less portability.  Any PVS context other than the current one may be
considered a library.  An example of a library declaration
is\index{LIBRARY@{\tt LIBRARY}}
\begin{pvsex}
  lib: LIBRARY = "~/pvs/protocols"
\end{pvsex}
When encountered, the system verifies that the directory specified within
the quotation marks exists, and that it has a PVS context file
\index{.pvscontext@{\tt .pvscontext}}%
(\texttt{.pvscontext}).  The library declaration is made use of by
including the library id in an importing name:\index{IMPORTING@{\tt IMPORTING}}
\begin{pvsex}
  IMPORTING lib@sliding_window[n]
\end{pvsex}
This has the effect of bringing in the \texttt{sliding\_window} theory,
exactly as if the theory belonged to the current context.

There are several libraries distributed with PVS, in the directory {\tt
lib}.  It is not necessary to give a library declaration for libraries in
this directory, as it will be automatically searched for library
importings.  Also, as described in the PVS System Guide, any libraries
found on the environment variable \texttt{PVS\_LIBRARY\_PATH} do not need
library declarations.  For example, to import the finite sets library over
the natural numbers:
\begin{pvsex}
  IMPORTING finite\_sets@finite\_sets[nat]
\end{pvsex}
An alternative approach (described in the \emph{PVS User
Guide}\cite{PVS:userguide}) is to use the {\tt M-x load-prelude-library},
which augments the PVS prelude with the the theories from a given context.

\index{declaration!library|)}
\index{library declaration|)}
