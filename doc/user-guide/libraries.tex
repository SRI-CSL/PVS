\chapter{PVS Libraries}

The PVS library mechanism provides support for creating library
\emph{packages}, including, in addition to theories, lisp and Emacs lisp
functions that add new features.  We describe here the considerations for
creating and using such packages, in particular, how to structure, 
distribute, and install a package in portable manner.

Here is a typical scenario.  A set of theories are developed and
typechecked, some supporting strategies are defined in the
\texttt{pvs-strategies}\index{pvs-strategies@\texttt{pvs-strategies}}
file, and some Emacs extensions are created in
\texttt{pvs-lib.el}\index{pvs-lib.el@\texttt{pvs-lib.el}} to add new key
bindings to the Emacs prover interface.  These are all in directory
\texttt{<DIR>/lib/mylib/}.  To use the library package, create the
environment variable
\texttt{PVS\_LIBRARY\_PATH}\index{PVS_LIBRARY_PATH@\texttt{PVS\_LIBRARY\_PATH}},
and set it to \texttt{<DIR>/lib}.  Then \texttt{cd} to a new directory,
and start up PVS.  After it comes up, run
\begin{alltt}
  M-x \icmd{load-prelude-library} mylib <RET>
\end{alltt}
and the theories will be loaded as prelude extensions, the pvs-lib.el file
will be loaded into Emacs, and when the prover is invoked next, the
\texttt{pvs-strategies} file will be loaded.  When the session is ended,
the prelude library dependency is stored in the \texttt{.pvscontext}, and
the next time PVS goes to that context (either by starting it up there, or
with the \icmd{change-context} command), the files will all be loaded once
again.

When PVS creates the \texttt{.pvscontext} file, it tries to stay away from
absolute pathnames as much as it can.  In this case, the dependency that
it keeps is simply \texttt{mylib/}.  What this means is that the directory
can be copied to a different site, where the \texttt{mylib} directory is
found in a different directory, but as long as \texttt{mylib} can be found
in the \texttt{PVS\_LIBRARY\_PATH} it will be used properly.

Note that for this to work the supporting files must also be free of any
absolute pathnames.  This includes any library declarations within the
theory files.  This is possible, as the
\texttt{PVS\_LIBRARY\_PATH}\index{PVS_LIBRARY_PATH@\texttt{PVS\_LIBRARY\_PATH}}
may have many paths on it, separated by colons.  Note that a given PVS
library is a subdirectory of an element of the
\texttt{PVS\_LIBRARY\_PATH}, not directly on the path.

PVS uses the \texttt{PVS\_LIBRARY\_PATH} variable in many ways:
\begin{itemize}
\item On startup, the Emacs variable
\texttt{pvs-library-path}\index{pvs-library-path@\texttt{pvs-library-path}}
is set to the corresponding list of pathnames.  These are added to the
Emacs variable \texttt{load-path}\index{load-path@\texttt{load-path}}, so
that the Emacs \texttt{load} command can find them.  Thus if library
\texttt{lib1} contains an Emacs file \texttt{foo.el} that needs the Emacs
file \texttt{bar.el} of \texttt{lib2}, it simply loads \texttt{lib2/bar}.

\item In the lisp image, the variable \texttt{*pvs-library-path*} is
similarly set to the list of pathname components of
\texttt{PVS\_LIBRARY\_PATH}.  

\end{itemize}


